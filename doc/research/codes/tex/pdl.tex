%%%%%%%%%%%%%%%%%%%%%%%%%%%%%%%%%%%%%%%%%%%%%%%%%%%%%%%%%%%%%%%%%%%%%%%%%%%%%%%%%%%%%%%%%%%%%%%%%%%%%%%%%%%%%%%%%%%%%
%%%%%%%%%%%%%%%%%%%%%%%%%%%%%%%%%%%%%%%%%%%%%   Author:Yao Zhang  %%%%%%%%%%%%%%%%%%%%%%%%%%%%%%%%%%%%%%%%%%%%%%%%%%%
%%%%%%%%%%%%%%%%%%%%%%%%%%%%%%%%%%%%%%%%%%%%% Email: jaafar_zhang@163.com %%%%%%%%%%%%%%%%%%%%%%%%%%%%%%%%%%%%%%%%%%%
%%%%%%%%%%%%%%%%%%%%%%%%%%%%%%%%%%%%%%%%%%%%%%%%%%%%%%%%%%%%%%%%%%%%%%%%%%%%%%%%%%%%%%%%%%%%%%%%%%%%%%%%%%%%%%%%%%%%%
\documentclass[11pt]{article}
\usepackage{babel}
\usepackage[utf8]{inputenc} 
\usepackage[table]{xcolor}
\usepackage[most]{tcolorbox}
\usepackage[left=2.50cm, right=1.50cm, top=2.0cm, bottom=2.50cm]{geometry}
\usepackage{xcolor,url}
\usepackage{amsmath,amsthm,amsfonts,amssymb,amscd,multirow,booktabs,fullpage,calc,multicol}
\usepackage{lastpage,enumitem,fancyhdr,mathrsfs,wrapfig,setspace,cancel,amsmath,empheq,framed}
\usepackage[retainorgcmds]{IEEEtrantools}
\usepackage{subfig,graphicx,framed}
\usepackage{ctex}
\usepackage{txfonts}
\usepackage{bbm}
\usepackage{chngcntr}
\usepackage[colorlinks,linkcolor=blue,anchorcolor=green,citecolor=red,urlcolor=blue]{hyperref}
\usepackage{titlesec}
%%%%%%%%%%%%%%%%%%%%%%%%%%%%%%%%%%%%%%%%%%%%%%%%%%%%%%%%%%%%%%%%%%%%%%%%%%%%%%%%%%%%%%%%%%%%%%%%%%%%%%%%%%%%%%%%%%%%%%
\newtheorem{thm}{Theorem}[section]
\newtheorem{defi}{Definition}[subsection]
\newtheorem{exercise}{Exercise}[subsection]
\newtheorem{note}{Note}[subsection]
\newtheorem{notation}{Notation}
\newtheorem{lemma}{Lemma}[subsection]
\newtheorem{proposition}{Proposition}[subsection]
\newtheorem{example}{Example}[subsection]
\newtheorem{problem}{Problem}[section]
\newtheorem{homework}{Homework}[section]
\newtheorem{summary}{Summary}[subsection]
\newtheorem{corollary}{Corollary}[subsection]
\newtheorem{rmk}{Remark}[section]
\usepackage{romannum}
%%%%%%%%%%%%%%%%%%%%%%%%%%%%%%%%%%%%%%%%%%%%%%%%%%%%%%%%%%%%%%%%%%%%%%%%%%%%%%%%%%%%%%%%%%%%%%%%%%%%%%%%%%%%%%%%%%%%%
\newlength{\tabcont}
\setlength{\parindent}{0.0in}
\setlength{\parskip}{0.05in}
\colorlet{shadecolor}{orange!15}
\parindent 0in
\parskip 12pt
\geometry{margin=1in, headsep=0.25in}
%%%%%%%%%%%%%%%%%%%%%%%%%%%%%%%%%%%%%%%%%%%%%%%%%%%%%%%%%%%%%%%%%%%%%%%%%%%%%%%%%%%%%%%%%%%%%%%%%%%%%%%%%%%%%%%%%%%%%
\graphicspath{ {img/EoM/}}
%%%%%%%%%%%%%%%%%%%%%%%%%%%%%%%%%%%%%%%%%%%%%%%%%%%%%%%%%%%%%%%%%%%%%%%%%%%%%%%%%%%%%%%%%%%%%%%%%%%%%%%%%%%%%%%%%%%%%
%\renewcommand{\cite}[1]{[#1]}
\makeatletter
\@addtoreset{equation}{section}
\makeatother
\renewcommand{\theequation}{\arabic{section}.\arabic{equation}}
\renewcommand{\contentsname}{\centering \small \color{blue} Contents}
%\counterwithin{figure}{section}
\renewcommand{\figurename}{\textbf{Fig.}}
%\renewcommand{\refname}{\textbf{\kaishu 参考文献}}
\renewcommand{\refname}{\textbf{Bibliography}}
\setcounter{secnumdepth}{4}
\titleformat{\paragraph}
{\normalfont\normalsize\bfseries}{\theparagraph}{1em}{}
\titlespacing*{\paragraph}{0pt}{3.25ex plus 1ex minus .2ex}{1.5ex plus .2ex}
\def\beginrefs{\begin{list}%
		{[\arabic{equation}]}{\usecounter{equation}
			\setlength{\leftmargin}{0.8truecm}\setlength{\labelsep}{0.4truecm}%
			\setlength{\labelwidth}{1.6truecm}}}
	\def\endrefs{\end{list}}
\def\bibentry#1{\item[\hbox{[#1]}]}
%%%%%%%%%%%%%%%%%%%%%%%%%%%%%%%%%%%%%%%%%%%%%%%%%%%%%%%%%%%%%%%%%%%%%%%%%%%%%%%%%%%%%%%%%%%%%%%%%%%%%%%%%%%%%%%%%%%%%%
%\begin{figure}[!htb]
%	\centering
%	\subfloat[$A \cap B$]{%
%		\includegraphics[width=0.3\linewidth,height=0.2\linewidth]{img001.jpg}}
%	\label{img001}\qquad \qquad %\hfill
%	\subfloat[${A_1} \cap {A_2} \cap {A_3}$]{%
%		\includegraphics[width=0.3\linewidth,height=0.2\linewidth]{img002.jpg}}
%	\label{img002}
	%\caption{ Examples.}
%\end{figure}
%\begin{figure}[!htb]
%	\centering
%	\includegraphics[width=0.4\linewidth,height=0.3\linewidth]{img005.jpg}
%	\label{img005}
	%\caption{ illustration for $ 3 $}
%\end{figure}
%\={a}1 \'{a}2\v{a}3\.{a}4

\usepackage{datetime}
\renewcommand{\today}{\shortmonthname[\the\month] \the \day,  \the\year}
%%%%%%%%%%%%%%%%%%%%%%%%%%%%%%%%%%%%%%%%%%%%%%%%%%%%%%%%%%%%%%%%%%%%%%%%%%%%%%%%%%%%%%%%%%%%%%%%%%%%%%%%%%%%%%%%%%%%%%
\begin{document}
	\kaishu 
	%\thispagestyle{empty}
	\pagenumbering{arabic} 
	\setcounter{section}{0}
	\begin{center}
		{\LARGE  \href{https://www.youtube.com/playlist?list=PLISXH-iEM4Jn3SEi07q8MJmDD6BaMWlJE}{Python for Deep Learning}}
		
		%\vspace{-0.25cm}
		
		{\large \href{https://fin.ntub.edu.tw/p/412-1037-121.php?Lang=en}{Chenghsi Hsieh}}
	\end{center}
%%%%%%%%%%%%%%%%%%%%%%%%%%%%%%%%%%%%%%%%%%%%%%%%%%%%%%%%%%%%%%%%%%%%%%%%%%%%%%%%%%%%%%%%%%%%%%%%%%%%%%%%%%%%%%%%%%%%%%
%%\newpage 
%%\thispagestyle{empty}	
%%%%%%%%%%%%%%%%%%%%%%%%%%%%%%%%%%%%%%%%%%%%%%%%%%%%%%%%%%%%%%%%%%%%%%%%%%%%%%%%%%%%%%%%%%%%%%%%%%%%%%%%%%%%%%%%%%%%%%
%\tableofcontents	
%{\pagestyle{empty}\mbox{}\newpage\pagestyle{empty}}
%\newpage 
%{\pagestyle{empty}\mbox{}\newpage\pagestyle{empty}}
%%%%%%%%%%%%%%%%%%%%%%%%%%%%%%%%%%%%%%%%%%%%%%%%%%%%%%%%%%%%%%%%%%%%%%%%%%%%%%%%%%%%%%%%%%%%%%%%%%%%%%%%%%%%%%%%%%%%%%
%%\newpage 
\setcounter{page}{1}

%\vspace{1.5cm}


\vspace{-0.5cm}

\subsection*{\small \kaishu 1. Outline:}

\vspace{-0.5cm}

\begin{multicols}{2}
	\begin{enumerate}
		\item \href{https://mp.weixin.qq.com/s/k5nhXuALPIh-f2q0Ttk73g}{Prelude}	%1
	\end{enumerate}
\end{multicols}

\subsection*{\small \kaishu 2.  What is Deep Learning:}

\vspace{-0.5cm}

\begin{multicols}{2}
	\begin{enumerate}
		\item \href{https://mp.weixin.qq.com/s/i2oj-hehepM_dMrFPsbuTA}{人工智慧,机器学习与深度学习}	%2
		\item \href{https://mp.weixin.qq.com/s/B7Snaz2d-hn80VszjR4Zaw}{深度学习之前: 机器学习简史 A}	%3
		\item \href{https://mp.weixin.qq.com/s/gttJNbZxRM3rUqDNbpJWdA}{深度学习之前: 机器学习简史 B}	%4
		\item \href{https://mp.weixin.qq.com/s/UIhTzoWchAUXxx3f5lr70Q}{深度学习之前: 机器学习简史 C}	%5
		\item \href{https://mp.weixin.qq.com/s/Yu2FSVFKTxDWbuMRnzucbg}{为什么是深度学习? 为什么是现在?}	%6
	\end{enumerate}
\end{multicols}

\subsection*{\small \kaishu 3. The Mathematical Building Blocks of NN:}

\vspace{-0.5cm}

\begin{multicols}{2}
	\begin{enumerate}
		\item \href{https://mp.weixin.qq.com/s/ZKk_gu__FrkzHovbhUdDwQ}{初识神经网络}	%7
		\item \href{https://mp.weixin.qq.com/s/u6eTUeILRhIsIuPVAVegsw}{神经网络的数据表示}	%8
		\item \href{https://mp.weixin.qq.com/s/HOeED8Um1nz99EEnJ7I_JQ}{神经网络的齿轮: 张量运算}	%9
		\item \href{https://mp.weixin.qq.com/s/9VCT-gYZ4rLtzUIWouqqGA}{神经网络的引擎: 基于梯度的最优化}	%10
	\end{enumerate}
\end{multicols}

\subsection*{\small \kaishu 4. Getting Started with Neural Networks:}

\vspace{-0.5cm}

\begin{multicols}{2}
	\begin{enumerate}
		\item \href{https://mp.weixin.qq.com/s/WlVnJmoQKPPKhLF7weLcrQ}{神经网络剖析}	%11
		\item \href{https://mp.weixin.qq.com/s/qXRHlpKt_X68C95hBBq0vQ}{Keras 简介与建立深度学习工作站}	%12
		\item \href{https://mp.weixin.qq.com/s/vc7O36XSuWspKnnnPQV1yg}{电影评论分类: 二元分类问题 1}	%13
		\item \href{https://mp.weixin.qq.com/s/8ik7dHpse8Dz8FzWo99Amg}{电影评论分类: 二元分类问题 2}	%14
		\item \href{https://mp.weixin.qq.com/s/P0GFe--7gmx6C-VE_K3rZA}{电影评论分类: 二元分类问题 3}	%15
		\item \href{https://mp.weixin.qq.com/s/XV3R-Romc58CPQWsezlG_g}{新闻分类: 多元分类问题}	%16
		\item \href{https://mp.weixin.qq.com/s/NKFtFaekWaD4sN15JDY52A}{预测房价: 回归问题 1}	%17
		\item \href{https://mp.weixin.qq.com/s/8WtHxfCgwCtAgbFlXReJkA}{回归问题 2}	%18
	\end{enumerate}
\end{multicols}

\subsection*{\small \kaishu 5. Fundamentals of Machine Learning:}

\vspace{-0.5cm}

\begin{multicols}{2}
	\begin{enumerate}
		\item \href{https://mp.weixin.qq.com/s/0VY5NjzRlcCUHXe_5ZhCwQ}{机器学习的四个分支与评估机器学习模型}	%19
		\item \href{https://mp.weixin.qq.com/s/dUfmuRUr-U-pYfSTAESRwg}{数据预处理, 特征工程, 过拟合与欠拟合1}	%20
		\item \href{https://mp.weixin.qq.com/s/Pw-Yz2VzSWkE1Wgbn1MbNA}{过拟合与欠拟合 2}	%21
		\item \href{https://mp.weixin.qq.com/s/wpEnkMaYOCplcTTyFiqWPw}{机器学习的通用工作流程}	%22
	\end{enumerate}
\end{multicols}

\subsection*{\small \kaishu 6. Deep Learning for Computer Vision:}

\vspace{-0.5cm}

\begin{multicols}{2}
	\begin{enumerate}
		\item \href{https://mp.weixin.qq.com/s/ALoxlmklv_GCQM5iCu6E8Q}{卷积神经网络简介 1}	%23
		\item \href{https://mp.weixin.qq.com/s/ldn1zDES0uZdjmCFDVXLjA}{卷积神经网络简介 2}	%24
		\item \href{https://mp.weixin.qq.com/s/mbQQVgmCbYPbfBoZYeQpxA}{在小型数据集上从头开始训练 CNN 1}	%25
		\item \href{https://mp.weixin.qq.com/s/PhmmQ1cObPjF9s57qJ79DQ}{在小型数据集上从头开始训练 CNN 2}	%26
		\item \href{https://mp.weixin.qq.com/s/Y3qM_gj5U52-M42uRrk_yg}{在小型数据集上从头开始训练 CNN 3}	%27
		\item \href{https://mp.weixin.qq.com/s/kjzNzFebsa7G57_--ceVug}{使用预训练的卷积神经网络 1}	%28
		\item \href{https://mp.weixin.qq.com/s/nolnfH8TzrDzlxPFokKP5g}{使用预训练的卷积神经网络 2}	%29
		\item \href{https://mp.weixin.qq.com/s/AhjUD4XzdfSi7W79krEyKQ}{使用预训练的卷积神经网络 3}	%30
		\item \href{https://mp.weixin.qq.com/s/kQW0tqPnteTqS7q70cRlJg}{卷积神经网路的可视化 1}	%31
		\item \href{https://mp.weixin.qq.com/s/cMK2avvr-cmH6EIhorbrPA}{卷积神经网路的可视化 2}	%32
		\item \href{https://mp.weixin.qq.com/s/vTp_OF1myN7SMdi1yBcRSQ}{卷积神经网路的可视化 3}	%33
	\end{enumerate}
\end{multicols}

\subsection*{\small \kaishu 7. Deep Learning for Text and Sequences:}

\vspace{-0.5cm}

\begin{multicols}{2}
	\begin{enumerate}
		\item \href{https://mp.weixin.qq.com/s/gcfDr3l6t5ZqEkDMcMqwcQ}{处理文本数据 1}	%34
		\item \href{https://mp.weixin.qq.com/s/9YUaHAICh0Z2OxtakY7CRg}{处理文本数据 2}	%35
		\item \href{https://mp.weixin.qq.com/s/eb724jE74D7IxlEKZdR19w}{处理文本数据 3}	%36
		\item \href{https://mp.weixin.qq.com/s/knR99mOo74mRgrvGYCkoFQ}{处理文本数据 4}	%37
		\item \href{https://mp.weixin.qq.com/s/AJdfDIG5clD-xjee_d4oMA}{理解循环神经网络 1}	%38
		\item \href{https://mp.weixin.qq.com/s/OXihNZshCKdChhXzqDlDFg}{理解循环神经网络 2}	%39
		\item \href{https://mp.weixin.qq.com/s/JewLh7zns2RuQSQQ0WJWhA}{循环神经网络的进阶用法 1}	%40
		\item \href{https://mp.weixin.qq.com/s/nGGuUZg1ld9wPFc9hBtNfA}{循环神经网络的进阶用法 2}	%41
		\item \href{https://mp.weixin.qq.com/s/Ax-suTpyVxhwAEmW-BdiCw}{循环神经网络的进阶用法 3}	%42
		\item \href{https://mp.weixin.qq.com/s/ME1RtxBnfoQRyR3ND_9CfA}{循环神经网络的进阶用法 4}	%43
		\item \href{https://mp.weixin.qq.com/s/zHzB06bL_uSEWhv7qbYa6g}{循环神经网络的进阶用法 5}	%44
		\item \href{https://mp.weixin.qq.com/s/AE2s40A7g0P6KaNQdb-8SQ}{循环神经网络的进阶用法 6}	%45
		\item \href{https://mp.weixin.qq.com/s/g-z-kbY2FxbpQnU2yDV9SA}{用卷积神经网络处理序列资料 1}	%46
		\item \href{https://mp.weixin.qq.com/s/rm2TK1BZ2RhYDdHapHK01g}{用卷积神经网络处理序列资料 2}	%47
	\end{enumerate}
\end{multicols}

\subsection*{\small \kaishu 8. Advanced Deep-Learning Best Practices:}

\vspace{-0.5cm}

\begin{multicols}{2}
	\begin{enumerate}
		\item \href{https://mp.weixin.qq.com/s/_3qS-f0kuwUrp4FB7Hq4bw}{Keras 函数式 API A}	%48
		\item \href{https://mp.weixin.qq.com/s/Qj34FpFW5BFu4VFWFAeiuA}{Keras 函数式 API B}	%49
		\item \href{https://mp.weixin.qq.com/s/ZBffdIjVB_do0OF4YeFe5w}{Keras 函数式 API C}	%50
		\item \href{https://mp.weixin.qq.com/s/JFSLNwHzcHUdy6yINxqK9Q}{Keras 函数式 API D}	%51
		\item \href{https://mp.weixin.qq.com/s/dvxgaTzrezdvjcmlR9xyEA}{Keras 函数式 API E}	%52
		\item \href{https://mp.weixin.qq.com/s/_K7lauHbNRLcl4FFTn-yVg}{Keras 回调函数和 TensorBoard}	%53
		\item \href{https://mp.weixin.qq.com/s/bY6w6AfW-kD-2Lxltov3PQ}{让模型性能发挥到极致 A}	%54
		\item \href{https://mp.weixin.qq.com/s/JyhKWlqfcbKHP7v26-KRqA}{让模型性能发挥到极致 B}	%55
	\end{enumerate}
\end{multicols}

\subsection*{\small \kaishu 9. Generative Deep Learning:}

\vspace{-0.5cm}

\begin{multicols}{2}
	\begin{enumerate}
		\item \href{https://mp.weixin.qq.com/s/PCwTsDdVVftTEWIDFxznFw}{使用LSTM生成文本 A}	%56
		\item \href{https://mp.weixin.qq.com/s/6UILV9UGnSjaLiiz84laDA}{使用LSTM生成文本 B}	%57
		\item \href{https://mp.weixin.qq.com/s/Cso1I8Njc4ifTsqwQ4UUCw}{使用LSTM生成文本 C}	%58
		\item \href{https://mp.weixin.qq.com/s/riA-aYzBdUEE-K6VSOmt6Q}{DeepDream A}	%59
		\item \href{https://mp.weixin.qq.com/s/cWqf3fHK3oQtgvj3CUb-FQ}{DeepDream B}	%60
		\item \href{https://mp.weixin.qq.com/s/TxxqhPaGH4whobWxXkT5hg}{神经网络风格迁移 A}	%61
		\item \href{https://mp.weixin.qq.com/s/MpBgzsyp5PrVCy3s_E5iqA}{神经网络风格迁移 B}	%62
		\item \href{https://mp.weixin.qq.com/s/5VPZx2ncWuQWNS4istq9mw}{用变分自编码器生成图像 A}	%63
		\item \href{https://mp.weixin.qq.com/s/F1L5FtnjwzKW65YvVVIAFw}{用变分自编码器生成图像 B}	%64
		\item \href{https://mp.weixin.qq.com/s/7iLs7o0L8DBYAqYY6qswLA}{用变分自编码器生成图像 C}	%65
		\item \href{https://mp.weixin.qq.com/s/tA38qtSJ4ZJ-kjOZugd5gg}{生成对抗网络简介 A}	%66
		\item \href{https://mp.weixin.qq.com/s/c9J81SsufKd2FzgCgQ3Fog}{生成对抗网络简介 B}	%67
		\item \href{https://mp.weixin.qq.com/s/zTSGydSPQ4CUvoc8JVVygg}{生成对抗网络简介 C}	%68
	\end{enumerate}
\end{multicols}

\subsection*{\small \kaishu 10. Conclusions:}

\vspace{-0.5cm}

\begin{multicols}{2}
	\begin{enumerate}
		\item \href{https://mp.weixin.qq.com/s/q0rSEn19fD8UkZIl_fNKXQ}{重点内容回顾}	%69
		\item \href{https://mp.weixin.qq.com/s/_RGhISMwODlcLlQn-66EBA}{深度学习的局限性和未来}	%70
		%\item \href{url}{Materials}
	\end{enumerate}
\end{multicols}






%%%%%%%%%%%%%%%%%%%%%%%%%%%%%%%%%%%%%%%%%%%%%%%%%%%%%%%%%%%%%%%%%%%%%%%%%%%%%%%%%%%%%%%%%%%%%%%%%%%%%%%%%%%%%%%%%%%%%%%
%\bibliographystyle{ieeetr} % number
%%\bibliographystyle{unsrtnat} % author year
%\bibliography{HeBib}
%%%%%%%%%%%%%%%%%%%%%%%%%%%%%%%%%%%%%%%%%%%%%%%%%%%%%%%%%%%%%%%%%%%%%%%%%%%%%%%%%%%%%%%%%%%%%%%%%%%%%%%%%%%%%%%%%%%%%%%
\begin{flushright}
	\tiny \today 
\end{flushright}
%%%%%%%%%%%%%%%%%%%%%%%%%%%%%%%%%%%%%%%%%%%%%%%%%%%%%%%%%%%%%%%%%%%%%%%%%%%%%%%%%%%%%%%%%%%%%%%%%%%%%%%%%%%%%%%%%%%%%%%
\end{document}
%%%%%%%%%%%%%%%%%%%%%%%%%%%%%%%%%%%%%%%%%%%%%%%%%%%%%%%%%%%%%%%%%%%%%%%%%%%%%%%%%%%%%%%%%%%%%%%%%%%%%%%%%%%%%%%%%%%%%%%
              