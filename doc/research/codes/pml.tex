%%%%%%%%%%%%%%%%%%%%%%%%%%%%%%%%%%%%%%%%%%%%%%%%%%%%%%%%%%%%%%%%%%%%%%%%%%%%%%%%%%%%%%%%%%%%%%%%%%%%%%%%%%%%%%%%%%%%%
%%%%%%%%%%%%%%%%%%%%%%%%%%%%%%%%%%%%%%%%%%%%%   Author:Yao Zhang  %%%%%%%%%%%%%%%%%%%%%%%%%%%%%%%%%%%%%%%%%%%%%%%%%%%
%%%%%%%%%%%%%%%%%%%%%%%%%%%%%%%%%%%%%%%%%%%%% Email: jaafar_zhang@163.com %%%%%%%%%%%%%%%%%%%%%%%%%%%%%%%%%%%%%%%%%%%
%%%%%%%%%%%%%%%%%%%%%%%%%%%%%%%%%%%%%%%%%%%%%%%%%%%%%%%%%%%%%%%%%%%%%%%%%%%%%%%%%%%%%%%%%%%%%%%%%%%%%%%%%%%%%%%%%%%%%
\documentclass[11pt]{article}
\usepackage{babel}
\usepackage[utf8]{inputenc} 
\usepackage[table]{xcolor}
\usepackage[most]{tcolorbox}
\usepackage[left=2.50cm, right=1.50cm, top=2.0cm, bottom=2.50cm]{geometry}
\usepackage{xcolor,url}
\usepackage{amsmath,amsthm,amsfonts,amssymb,amscd,multirow,booktabs,fullpage,calc,multicol}
\usepackage{lastpage,enumitem,fancyhdr,mathrsfs,wrapfig,setspace,cancel,amsmath,empheq,framed}
\usepackage[retainorgcmds]{IEEEtrantools}
\usepackage{subfig,graphicx,framed}
\usepackage{ctex}
\usepackage{txfonts}
\usepackage{bbm}
\usepackage{chngcntr}
\usepackage[colorlinks,linkcolor=blue,anchorcolor=green,citecolor=red,urlcolor=blue]{hyperref}
\usepackage{titlesec}
%%%%%%%%%%%%%%%%%%%%%%%%%%%%%%%%%%%%%%%%%%%%%%%%%%%%%%%%%%%%%%%%%%%%%%%%%%%%%%%%%%%%%%%%%%%%%%%%%%%%%%%%%%%%%%%%%%%%%%
\newtheorem{thm}{Theorem}[section]
\newtheorem{defi}{Definition}[subsection]
\newtheorem{exercise}{Exercise}[subsection]
\newtheorem{note}{Note}[subsection]
\newtheorem{notation}{Notation}
\newtheorem{lemma}{Lemma}[subsection]
\newtheorem{proposition}{Proposition}[subsection]
\newtheorem{example}{Example}[subsection]
\newtheorem{problem}{Problem}[section]
\newtheorem{homework}{Homework}[section]
\newtheorem{summary}{Summary}[subsection]
\newtheorem{corollary}{Corollary}[subsection]
\newtheorem{rmk}{Remark}[section]
\usepackage{romannum}
%%%%%%%%%%%%%%%%%%%%%%%%%%%%%%%%%%%%%%%%%%%%%%%%%%%%%%%%%%%%%%%%%%%%%%%%%%%%%%%%%%%%%%%%%%%%%%%%%%%%%%%%%%%%%%%%%%%%%
\newlength{\tabcont}
\setlength{\parindent}{0.0in}
\setlength{\parskip}{0.05in}
\colorlet{shadecolor}{orange!15}
\parindent 0in
\parskip 12pt
\geometry{margin=1in, headsep=0.25in}
%%%%%%%%%%%%%%%%%%%%%%%%%%%%%%%%%%%%%%%%%%%%%%%%%%%%%%%%%%%%%%%%%%%%%%%%%%%%%%%%%%%%%%%%%%%%%%%%%%%%%%%%%%%%%%%%%%%%%
\graphicspath{ {img/EoM/}}
%%%%%%%%%%%%%%%%%%%%%%%%%%%%%%%%%%%%%%%%%%%%%%%%%%%%%%%%%%%%%%%%%%%%%%%%%%%%%%%%%%%%%%%%%%%%%%%%%%%%%%%%%%%%%%%%%%%%%
%\renewcommand{\cite}[1]{[#1]}
\makeatletter
\@addtoreset{equation}{section}
\makeatother
\renewcommand{\theequation}{\arabic{section}.\arabic{equation}}
\renewcommand{\contentsname}{\centering \small \color{blue} Contents}
%\counterwithin{figure}{section}
\renewcommand{\figurename}{\textbf{Fig.}}
%\renewcommand{\refname}{\textbf{\kaishu 参考文献}}
\renewcommand{\refname}{\textbf{Bibliography}}
\setcounter{secnumdepth}{4}
\titleformat{\paragraph}
{\normalfont\normalsize\bfseries}{\theparagraph}{1em}{}
\titlespacing*{\paragraph}{0pt}{3.25ex plus 1ex minus .2ex}{1.5ex plus .2ex}
\def\beginrefs{\begin{list}%
		{[\arabic{equation}]}{\usecounter{equation}
			\setlength{\leftmargin}{0.8truecm}\setlength{\labelsep}{0.4truecm}%
			\setlength{\labelwidth}{1.6truecm}}}
	\def\endrefs{\end{list}}
\def\bibentry#1{\item[\hbox{[#1]}]}
%%%%%%%%%%%%%%%%%%%%%%%%%%%%%%%%%%%%%%%%%%%%%%%%%%%%%%%%%%%%%%%%%%%%%%%%%%%%%%%%%%%%%%%%%%%%%%%%%%%%%%%%%%%%%%%%%%%%%%
%\begin{figure}[!htb]
%	\centering
%	\subfloat[$A \cap B$]{%
%		\includegraphics[width=0.3\linewidth,height=0.2\linewidth]{img001.jpg}}
%	\label{img001}\qquad \qquad %\hfill
%	\subfloat[${A_1} \cap {A_2} \cap {A_3}$]{%
%		\includegraphics[width=0.3\linewidth,height=0.2\linewidth]{img002.jpg}}
%	\label{img002}
	%\caption{ Examples.}
%\end{figure}
%\begin{figure}[!htb]
%	\centering
%	\includegraphics[width=0.4\linewidth,height=0.3\linewidth]{img005.jpg}
%	\label{img005}
	%\caption{ illustration for $ 3 $}
%\end{figure}
%\={a}1 \'{a}2\v{a}3\.{a}4

\usepackage{datetime}
\renewcommand{\today}{\shortmonthname[\the\month] \the \day,  \the\year}
%%%%%%%%%%%%%%%%%%%%%%%%%%%%%%%%%%%%%%%%%%%%%%%%%%%%%%%%%%%%%%%%%%%%%%%%%%%%%%%%%%%%%%%%%%%%%%%%%%%%%%%%%%%%%%%%%%%%%%
\begin{document}
	\kaishu 
	%\thispagestyle{empty}
	\pagenumbering{arabic} 
	\setcounter{section}{0}
	\begin{center}
		{\LARGE  \href{https://www.youtube.com/playlist?list=PL8xPPUJdubH7jx4wkkX1rvEcp5hmmK-co}{Python Machine Learning}}
		
		%\vspace{-0.25cm}
		
		{\large \href{https://fin.ntub.edu.tw/p/412-1037-121.php?Lang=en}{Chenghsi Hsieh}}
	\end{center}
%%%%%%%%%%%%%%%%%%%%%%%%%%%%%%%%%%%%%%%%%%%%%%%%%%%%%%%%%%%%%%%%%%%%%%%%%%%%%%%%%%%%%%%%%%%%%%%%%%%%%%%%%%%%%%%%%%%%%%
%%\newpage 
%%\thispagestyle{empty}	
%%%%%%%%%%%%%%%%%%%%%%%%%%%%%%%%%%%%%%%%%%%%%%%%%%%%%%%%%%%%%%%%%%%%%%%%%%%%%%%%%%%%%%%%%%%%%%%%%%%%%%%%%%%%%%%%%%%%%%
%\tableofcontents	
%{\pagestyle{empty}\mbox{}\newpage\pagestyle{empty}}
%\newpage 
%{\pagestyle{empty}\mbox{}\newpage\pagestyle{empty}}
%%%%%%%%%%%%%%%%%%%%%%%%%%%%%%%%%%%%%%%%%%%%%%%%%%%%%%%%%%%%%%%%%%%%%%%%%%%%%%%%%%%%%%%%%%%%%%%%%%%%%%%%%%%%%%%%%%%%%%
%%\newpage 
\setcounter{page}{1}

%\vspace{1.5cm}


\vspace{-0.25cm}

\begin{enumerate}
	\item \href{https://mp.weixin.qq.com/s/NWZfDX6aIm7hlAq3qYhBTg}{Coda: 最后的一个拼图}	%1
	\item \href{https://mp.weixin.qq.com/s/kl1LJvakcBHbcx2ESK9JjQ}{Giving Computers the Ability to Learn from Data: 赋予电脑资料学习的能力}	%2
\end{enumerate}

\subsection*{Algorithms for Classification}

\vspace{-0.25cm}

\begin{multicols}{2}
	\begin{enumerate}
		\item \href{https://mp.weixin.qq.com/s/pe9jrh_56vtKjIRZO2t4cw}{感知器的学习规则}	%3
		\item \href{https://mp.weixin.qq.com/s/X2JVorzkJ1VZkEee4cS7EQ}{以物件导向程式实作感知器}	%4
		\item \href{https://mp.weixin.qq.com/s/NjhyB5pvhdpAI6c_9vMRCw}{以鸢尾花资料训练感知器}	%5
		\item \href{https://mp.weixin.qq.com/s/SUat1DedwYHNe6Nh77aTMg}{以物件导向程式实作适应线性神经元}	%6
		\item \href{https://mp.weixin.qq.com/s/S6YvTFY8EUe7zkwymj5Yaw}{资料标准化与随机梯度下降}	%7
	\end{enumerate}
\end{multicols}

\subsection*{Classifiers Using Scikit-Learn}

\vspace{-0.25cm}

\begin{multicols}{2}
	\begin{enumerate}
		\item \href{https://mp.weixin.qq.com/s/hLdDC1hoSVYEww_VfWap0w}{以套件实作感知器}	%8
		\item \href{https://mp.weixin.qq.com/s/uNoPquN9oRi64e1Sl9zr0w}{以物件导向程式实作逻辑回归}	%9
		\item \href{https://mp.weixin.qq.com/s/0VH9HFePdZVMiqWbvaFzLw}{以套件实作逻辑回归}	%10
		\item \href{https://mp.weixin.qq.com/s/EhuWxyETQQKSqLprzB-HYQ}{支持向量机 1}	%11
		\item \href{https://mp.weixin.qq.com/s/o67kQOaKdeN-ZJijU2O7ag}{支持向量机 2}	%12
		\item \href{https://mp.weixin.qq.com/s/W_b-hnmi3ztP1bHznlnf7Q}{实作支持向量机}	%13
		\item \href{https://mp.weixin.qq.com/s/qyyUg_UeRBnGUiqdcRdeAQ}{决策树}	%14
		\item \href{https://mp.weixin.qq.com/s/Vo_qt2rE3LiUjaRYPzkdpw}{随机森林与 K 最近邻分类器}	%15
	\end{enumerate}
\end{multicols}

\subsection*{Data Preprocessing}

\vspace{-0.25cm}

\begin{multicols}{2}
	\begin{enumerate}
		\item \href{https://mp.weixin.qq.com/s/JNmnQSK1zwjzKVaippLCdQ}{处理遗失值及类别资料}	%16
		\item \href{https://mp.weixin.qq.com/s/rfhQxhvFs_i4NM4MclSMBA}{资料切割、特征缩放及选取有意义的特征}	%17
		\item \href{https://mp.weixin.qq.com/s/YnxEMD7ZF10Ep2pHR6hJvQ}{以物件导向程序实作循序特征选择演算法}	%18
		\item \href{https://mp.weixin.qq.com/s/zIDofQXr1HBrtPP4I5WAgQ}{以随机森林评估特征的重要性}	%19
	\end{enumerate}
\end{multicols}

\subsection*{Dimensionality Reduction}

\vspace{-0.25cm}

\begin{multicols}{2}
	\begin{enumerate}
		\item \href{https://mp.weixin.qq.com/s/2-I5HxcBYDTYTNSBEd-z5w}{主成分分析}	%20
		\item \href{https://mp.weixin.qq.com/s/EGDmY4r2QV_kDQDGnxVOcw}{线性判别分析 1}	%21
		\item \href{https://mp.weixin.qq.com/s/CEz1I6dhVpvXD0A4XW_z1Q}{线性判别分析 2}	%22
		\item \href{https://mp.weixin.qq.com/s/WufaoCxegQHAnpdIUY5A5g}{核的概念}	%23
		\item \href{https://mp.weixin.qq.com/s/uo1jOcLzriNvIoymY7Y4oQ}{核主成分分析 1}	%24
		\item \href{https://mp.weixin.qq.com/s/i-H-OMUGMhB8B4ARWJGVZQ}{核主成分分析 2}	%25 
	\end{enumerate}
\end{multicols}

\newpage 

\subsection*{Model Evaluation and Hyperparameter Tuning}

\vspace{-0.25cm}

\begin{multicols}{2}
	\begin{enumerate}
		\item \href{https://mp.weixin.qq.com/s/gkp9O7Jm9-q8-s9mgt7Mpg}{使用管线精简工作流程}	%26
		\item \href{https://mp.weixin.qq.com/s/0zqlxs4ysakjCvheMMzkYA}{K 折交叉验证、学习曲线与验证曲线}	%27
		\item \href{https://mp.weixin.qq.com/s/yOuSAAK91Uha4ip9Avw8xA}{格状搜寻与混肴矩阵}	%28
		\item \href{https://mp.weixin.qq.com/s/sFnvZTjn1GjpIGge2XFSMw}{接收操作特征图(ROC Curve)}	%29
		\item \href{https://mp.weixin.qq.com/s/-ljjb2d3K9hpOb9CcQPAVg}{多元分类计分指标与资料不平衡}	%30 
	\end{enumerate}
\end{multicols}

\subsection*{Ensemble}

\vspace{-0.25cm}

\begin{multicols}{2}
	\begin{enumerate}
		\item \href{https://mp.weixin.qq.com/s/7NRpVfdjBBIh7qF8qFi2AQ}{整体学习简介}	%31
		\item \href{https://mp.weixin.qq.com/s/QUzndlMD9Wl23Rd3Sc8iug}{以物件导向程式实作多数决分类器}	%32
		\item \href{https://mp.weixin.qq.com/s/IaRbx7QxIzEEBH3ID2rc8A}{\small 以多数决原理作预测并评估整体学习分类器}	%33
		\item \href{https://mp.weixin.qq.com/s/-5E7Lvn7PuaaYj1CI8O5QA}{微调整体学习分类器}	%34
		\item \href{https://mp.weixin.qq.com/s/5Te6DwvwIBHYLrETL_qptw}{袋装法}	%35
		\item \href{https://mp.weixin.qq.com/s/cYmRobzrStZeGHfWSHVQ8A}{适应强化法(AdaBoost)}	%36
		\item \href{https://mp.weixin.qq.com/s/sW7u7yxH7VmwPhldUTkPBw}{\small 由依序极小化误差函数看 AdaBoost 与 Gradient Boost 简介}	%37
	\end{enumerate}
\end{multicols}


\subsection*{Regression}

\vspace{-0.25cm}

\begin{multicols}{2}
	\begin{enumerate}
		\item \href{https://mp.weixin.qq.com/s/PoEkNO3lEYsCjVagLeMwEw}{回归简介与探索波士顿房价数据集}	%38
		\item \href{https://mp.weixin.qq.com/s/bgiSRysxTaVLD60NoS6JuQ}{以物件导向程式及套件实作回归模型}	%39
		\item \href{https://mp.weixin.qq.com/s/WEpou9qEp-ZYEgwKK--Qpg}{\small 随机样本共识回归(RANSAC)、脊回归与 LASSO 回归}	%40
		\item \href{https://mp.weixin.qq.com/s/n4goxLw73phLZIJgnpxpfA}{以线性回归配适曲线}	%41
		\item \href{https://mp.weixin.qq.com/s/NmWmFuMIYwiLQcnSck-omg}{决策树回归与随机森林回归}	%42
	\end{enumerate}
\end{multicols}


\subsection*{Clustering Analysis}

\vspace{-0.25cm}

\begin{multicols}{2}
	\begin{enumerate}
		\item \href{https://mp.weixin.qq.com/s/3wPnev_4INVpfupuBavuig}{K-means 与 K-means++}	%43
		\item \href{https://mp.weixin.qq.com/s/aTIxS-_nuunjVbLO59y4PA}{转折判断法与轮廓图}	%44
		\item \href{https://mp.weixin.qq.com/s/FtsA9Jbmq3MuHhz2sRBKzQ}{阶层树分群}	%45
		\item \href{https://mp.weixin.qq.com/s/_guzPqQbCgoFTkwzJug0eA}{以密度为基础的分群 (DBSCAN)}	%46
	\end{enumerate}
\end{multicols}

\subsection*{Supplementary Materials}

\vspace{-0.25cm}

\begin{enumerate}
	\item \href{url}{Materials}
\end{enumerate}




%%%%%%%%%%%%%%%%%%%%%%%%%%%%%%%%%%%%%%%%%%%%%%%%%%%%%%%%%%%%%%%%%%%%%%%%%%%%%%%%%%%%%%%%%%%%%%%%%%%%%%%%%%%%%%%%%%%%%%%
%\bibliographystyle{ieeetr} % number
%%\bibliographystyle{unsrtnat} % author year
%\bibliography{HeBib}
%%%%%%%%%%%%%%%%%%%%%%%%%%%%%%%%%%%%%%%%%%%%%%%%%%%%%%%%%%%%%%%%%%%%%%%%%%%%%%%%%%%%%%%%%%%%%%%%%%%%%%%%%%%%%%%%%%%%%%%
\begin{flushright}
	\tiny \today 
\end{flushright}
%%%%%%%%%%%%%%%%%%%%%%%%%%%%%%%%%%%%%%%%%%%%%%%%%%%%%%%%%%%%%%%%%%%%%%%%%%%%%%%%%%%%%%%%%%%%%%%%%%%%%%%%%%%%%%%%%%%%%%%
\end{document}
%%%%%%%%%%%%%%%%%%%%%%%%%%%%%%%%%%%%%%%%%%%%%%%%%%%%%%%%%%%%%%%%%%%%%%%%%%%%%%%%%%%%%%%%%%%%%%%%%%%%%%%%%%%%%%%%%%%%%%%
              