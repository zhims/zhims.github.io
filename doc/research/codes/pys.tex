%%%%%%%%%%%%%%%%%%%%%%%%%%%%%%%%%%%%%%%%%%%%%%%%%%%%%%%%%%%%%%%%%%%%%%%%%%%%%%%%%%%%%%%%%%%%%%%%%%%%%%%%%%%%%%%%%%%%%
%%%%%%%%%%%%%%%%%%%%%%%%%%%%%%%%%%%%%%%%%%%%%   Author:Yao Zhang  %%%%%%%%%%%%%%%%%%%%%%%%%%%%%%%%%%%%%%%%%%%%%%%%%%%
%%%%%%%%%%%%%%%%%%%%%%%%%%%%%%%%%%%%%%%%%%%%% Email: jaafar_zhang@163.com %%%%%%%%%%%%%%%%%%%%%%%%%%%%%%%%%%%%%%%%%%%
%%%%%%%%%%%%%%%%%%%%%%%%%%%%%%%%%%%%%%%%%%%%%%%%%%%%%%%%%%%%%%%%%%%%%%%%%%%%%%%%%%%%%%%%%%%%%%%%%%%%%%%%%%%%%%%%%%%%%
\documentclass[11pt]{article}
\usepackage{babel}
\usepackage[utf8]{inputenc} 
\usepackage[table]{xcolor}
\usepackage[most]{tcolorbox}
\usepackage[left=2.50cm, right=1.50cm, top=2.0cm, bottom=2.50cm]{geometry}
\usepackage{xcolor,url}
\usepackage{amsmath,amsthm,amsfonts,amssymb,amscd,multirow,booktabs,fullpage,calc,multicol}
\usepackage{lastpage,enumitem,fancyhdr,mathrsfs,wrapfig,setspace,cancel,amsmath,empheq,framed}
\usepackage[retainorgcmds]{IEEEtrantools}
\usepackage{subfig,graphicx,framed}
\usepackage{ctex}
\usepackage{txfonts}
\usepackage{bbm}
\usepackage{chngcntr}
\usepackage[colorlinks,linkcolor=blue,anchorcolor=green,citecolor=red,urlcolor=blue]{hyperref}
\usepackage{titlesec}
%%%%%%%%%%%%%%%%%%%%%%%%%%%%%%%%%%%%%%%%%%%%%%%%%%%%%%%%%%%%%%%%%%%%%%%%%%%%%%%%%%%%%%%%%%%%%%%%%%%%%%%%%%%%%%%%%%%%%%
\newtheorem{thm}{Theorem}[section]
\newtheorem{defi}{Definition}[subsection]
\newtheorem{exercise}{Exercise}[subsection]
\newtheorem{note}{Note}[subsection]
\newtheorem{notation}{Notation}
\newtheorem{lemma}{Lemma}[subsection]
\newtheorem{proposition}{Proposition}[subsection]
\newtheorem{example}{Example}[subsection]
\newtheorem{problem}{Problem}[section]
\newtheorem{homework}{Homework}[section]
\newtheorem{summary}{Summary}[subsection]
\newtheorem{corollary}{Corollary}[subsection]
\newtheorem{rmk}{Remark}[section]
\usepackage{romannum}
%%%%%%%%%%%%%%%%%%%%%%%%%%%%%%%%%%%%%%%%%%%%%%%%%%%%%%%%%%%%%%%%%%%%%%%%%%%%%%%%%%%%%%%%%%%%%%%%%%%%%%%%%%%%%%%%%%%%%
\newlength{\tabcont}
\setlength{\parindent}{0.0in}
\setlength{\parskip}{0.05in}
\colorlet{shadecolor}{orange!15}
\parindent 0in
\parskip 12pt
\geometry{margin=1in, headsep=0.25in}
%%%%%%%%%%%%%%%%%%%%%%%%%%%%%%%%%%%%%%%%%%%%%%%%%%%%%%%%%%%%%%%%%%%%%%%%%%%%%%%%%%%%%%%%%%%%%%%%%%%%%%%%%%%%%%%%%%%%%
\graphicspath{ {img/EoM/}}
%%%%%%%%%%%%%%%%%%%%%%%%%%%%%%%%%%%%%%%%%%%%%%%%%%%%%%%%%%%%%%%%%%%%%%%%%%%%%%%%%%%%%%%%%%%%%%%%%%%%%%%%%%%%%%%%%%%%%
%\renewcommand{\cite}[1]{[#1]}
\makeatletter
\@addtoreset{equation}{section}
\makeatother
\renewcommand{\theequation}{\arabic{section}.\arabic{equation}}
\renewcommand{\contentsname}{\centering \small \color{blue} Contents}
%\counterwithin{figure}{section}
\renewcommand{\figurename}{\textbf{Fig.}}
%\renewcommand{\refname}{\textbf{\kaishu 参考文献}}
\renewcommand{\refname}{\textbf{Bibliography}}
\setcounter{secnumdepth}{4}
\titleformat{\paragraph}
{\normalfont\normalsize\bfseries}{\theparagraph}{1em}{}
\titlespacing*{\paragraph}{0pt}{3.25ex plus 1ex minus .2ex}{1.5ex plus .2ex}
\def\beginrefs{\begin{list}%
		{[\arabic{equation}]}{\usecounter{equation}
			\setlength{\leftmargin}{0.8truecm}\setlength{\labelsep}{0.4truecm}%
			\setlength{\labelwidth}{1.6truecm}}}
	\def\endrefs{\end{list}}
\def\bibentry#1{\item[\hbox{[#1]}]}
%%%%%%%%%%%%%%%%%%%%%%%%%%%%%%%%%%%%%%%%%%%%%%%%%%%%%%%%%%%%%%%%%%%%%%%%%%%%%%%%%%%%%%%%%%%%%%%%%%%%%%%%%%%%%%%%%%%%%%
%\begin{figure}[!htb]
%	\centering
%	\subfloat[$A \cap B$]{%
%		\includegraphics[width=0.3\linewidth,height=0.2\linewidth]{img001.jpg}}
%	\label{img001}\qquad \qquad %\hfill
%	\subfloat[${A_1} \cap {A_2} \cap {A_3}$]{%
%		\includegraphics[width=0.3\linewidth,height=0.2\linewidth]{img002.jpg}}
%	\label{img002}
	%\caption{ Examples.}
%\end{figure}
%\begin{figure}[!htb]
%	\centering
%	\includegraphics[width=0.4\linewidth,height=0.3\linewidth]{img005.jpg}
%	\label{img005}
	%\caption{ illustration for $ 3 $}
%\end{figure}
%\={a}1 \'{a}2\v{a}3\.{a}4

\usepackage{datetime}
\renewcommand{\today}{\shortmonthname[\the\month] \the \day,  \the\year}
%%%%%%%%%%%%%%%%%%%%%%%%%%%%%%%%%%%%%%%%%%%%%%%%%%%%%%%%%%%%%%%%%%%%%%%%%%%%%%%%%%%%%%%%%%%%%%%%%%%%%%%%%%%%%%%%%%%%%%
\begin{document}
	\kaishu 
	%\thispagestyle{empty}
	\pagenumbering{arabic} 
	\setcounter{section}{0}
	\begin{center}
		{\LARGE  \href{https://github.com/jinhualee/datashine}{Python 统计与数据分析}}
		
		%\vspace{-0.25cm}
		
		{\large \href{https://github.com/jinhualee}{Jinhua Lee}}
	\end{center}
%%%%%%%%%%%%%%%%%%%%%%%%%%%%%%%%%%%%%%%%%%%%%%%%%%%%%%%%%%%%%%%%%%%%%%%%%%%%%%%%%%%%%%%%%%%%%%%%%%%%%%%%%%%%%%%%%%%%%%
%%\newpage 
%%\thispagestyle{empty}	
%%%%%%%%%%%%%%%%%%%%%%%%%%%%%%%%%%%%%%%%%%%%%%%%%%%%%%%%%%%%%%%%%%%%%%%%%%%%%%%%%%%%%%%%%%%%%%%%%%%%%%%%%%%%%%%%%%%%%%
%\tableofcontents	
%{\pagestyle{empty}\mbox{}\newpage\pagestyle{empty}}
%\newpage 
%{\pagestyle{empty}\mbox{}\newpage\pagestyle{empty}}
%%%%%%%%%%%%%%%%%%%%%%%%%%%%%%%%%%%%%%%%%%%%%%%%%%%%%%%%%%%%%%%%%%%%%%%%%%%%%%%%%%%%%%%%%%%%%%%%%%%%%%%%%%%%%%%%%%%%%%
%%\newpage 
\setcounter{page}{1}

%\vspace{1.5cm}

\begin{multicols}{2}
	\begin{enumerate}
		\item \href{https://mp.weixin.qq.com/s/z6Y1O0i_hpA1H6dxfXX6_A}{导学}	%1
		\item \href{https://mp.weixin.qq.com/s/U_PFsLxov9d6ae7HIFcIgQ}{位置与分散程度的度量 1}	%2
		\item \href{https://mp.weixin.qq.com/s/pJbunm0GCkA131ZER_U9Gg}{位置与分散程度的度量 2}	%3
		\item \href{https://mp.weixin.qq.com/s/KgKaYDPtGjC7SzfdlCKusw}{位置与分散程度的度量 3}	%4
		\item \href{https://mp.weixin.qq.com/s/1BLMnW-EvSUCjLLeZ2QScA}{关系度量}	%5
		\item \href{https://mp.weixin.qq.com/s/o_FD-HvD15dMuK-WUSCdLg}{分布形状的度量}	%6
		\item \href{https://mp.weixin.qq.com/s/xWGT0SrCcSOH1k4kiAWOOA}{数据特性的总括}	%7
		\item \href{https://mp.weixin.qq.com/s/afp68LQTO4nqM6liN6N2nw}{数据分布的基本概念}	%8
		\item \href{https://mp.weixin.qq.com/s/r2IcgyWzFJmOAW8dRn5L7g}{常见离散型分布}	%9
		\item \href{https://mp.weixin.qq.com/s/ptueUbv_uMgY2TdkidCpdw}{常见连续型分布 1: 正态分布}	%10
		\item \href{https://mp.weixin.qq.com/s/FcCn-W8Irf02Ski5gLEY4g}{常见连续型分布 2: t 分布}	%11
		\item \href{https://mp.weixin.qq.com/s/a5LvPOWxfMKoswZ6WTVfvQ}{常见连续型分布 3: gamma 分布}	%12
		\item \href{https://mp.weixin.qq.com/s/5zuagJgYC5N2GkE2ergBHg}{正态分布的图形}	%13
		\item \href{https://mp.weixin.qq.com/s/UkeruyDb9DYbNTPgwP398g}{卡方分布与 F 分布的图形}	%14
		\item \href{https://mp.weixin.qq.com/s/ve7ImOBw5AuPU_iaHwAHzA}{直方图与核密度估计}	%15
		\item \href{https://mp.weixin.qq.com/s/yF0UyJKkxqTJX5Usk1lwOA}{经验分布函数}	%16
		\item \href{https://mp.weixin.qq.com/s/vGE4-iXxtEhIgRl-gv4GMA}{QQ 图与茎叶图}	%17
		\item \href{https://mp.weixin.qq.com/s/tnDQlPejHi3ocsc5pEG7og}{二元数据的数字特征}	%18
		\item \href{https://mp.weixin.qq.com/s/m_QV4iK0cdl89XghqogYNA}{多元数据的数字特征}	%19
		\item \href{https://mp.weixin.qq.com/s/iXuUYEIZYc3mdiO0Wj9kpw}{多元数据的基本图形表示}	%20
		\item \href{https://mp.weixin.qq.com/s/zO2c0mKLE35spxmBGVDDcQ}{点估计: 极大似然法的概念}	%21
		\item \href{https://mp.weixin.qq.com/s/wtPo6CFWEeLE_-y2LYOElQ}{极大似然估计: 连续函数空间的解析解}	%22
		\item \href{https://mp.weixin.qq.com/s/teyXnzMMS94uSqUGYHnZcQ}{极大似然估计: 对数似然方程的数值解}	%23
		\item \href{https://mp.weixin.qq.com/s/drj8TkgcU6g7Xst414ugQg}{单个正态总体均值的区间估计 1}	%24
		\item \href{https://mp.weixin.qq.com/s/Bg0xb_NN-4td-v5fLibHHw}{单个正态总体均值的区间估计 2}	%25
		\item \href{https://mp.weixin.qq.com/s/itDS7eTdn_nSX0_FPRQVpw}{单个正态总体均值的区间估计 3}	%26
		\item \href{https://mp.weixin.qq.com/s/RZPKDwXN_vjZXJL7DeLG1Q}{单个正态总体的方差的区间估计}	%27
		\item \href{https://mp.weixin.qq.com/s/VTWRsvaTlievKRdl4biMHQ}{两个正态总体均值之差的区间估计 1}	%28
		\item \href{https://mp.weixin.qq.com/s/OlsJjPml-Qlk9Lvy1U2s_A}{两个正态总体均值之差的区间估计 2}	%29
		\item \href{https://mp.weixin.qq.com/s/BlyABZmDR2-F36aEfHz_QA}{两个正态总体的方差比的区间估计}	%30
		\item \href{https://mp.weixin.qq.com/s/GECAEr3hsPSERo-R-CpIxw}{非正态分布总体均值的区间估计}	%31
		\item \href{https://mp.weixin.qq.com/s/6k4OlmOwt3GXbUTNCMABpw}{单侧置信区间估计 1}	%32
		\item \href{https://mp.weixin.qq.com/s/Ppv7qJNEpDQpKeruu9BVYw}{单侧置信区间估计 2}	%33
		\item \href{https://mp.weixin.qq.com/s/4LFB-ugqzALSsOuwPYIxWw}{单侧置信区间估计 3}	%34
		\item \href{https://mp.weixin.qq.com/s/5GUQNpdD2fCanH_QsrEv6A}{单侧置信区间估计 4}	%35
		\item \href{https://mp.weixin.qq.com/s/1s6o5bTvYpkwvmpj1v6byA}{假设检验的基本原理 1}	%36
		\item \href{https://mp.weixin.qq.com/s/uFAevuhlh-IozPaldmIqsA}{假设检验的基本原理 2}	%37
		\item \href{https://mp.weixin.qq.com/s/oKpUapIcUYGTRCBIyP5d2g}{正态总体均值的假设检验 1}	%38
		\item \href{https://mp.weixin.qq.com/s/TVhsUofZKPcSB0pC5GfhcQ}{正态总体均值的假设检验 2}	%39
		\item \href{https://mp.weixin.qq.com/s/krOGj-lWhaktc4XO_IE7ww}{正态总体均值的假设检验 3}	%40
		\item \href{https://mp.weixin.qq.com/s/80Shgqs-tYujcxp0rGjfoQ}{正态总体均值的假设检验 4}	%41
		\item \href{https://mp.weixin.qq.com/s/PLp_HnF09cQb8JREou-iOg}{正态总体方差的假设检验 1}	%42
		\item \href{https://mp.weixin.qq.com/s/rjOPkhriWK8dUpyQfKB0jQ}{正态总体方差的假设检验 2}	%43
		\item \href{https://mp.weixin.qq.com/s/7wRqc64o9RrRCULmLpzsZg}{二项分布总体的假设检验}	%44
		\item \href{https://mp.weixin.qq.com/s/lA0j2HXD8vFzrYceGpNLMA}{回归分析的概念与一元线性回归 1}	%45
		\item \href{https://mp.weixin.qq.com/s/JYRgeun9rIv_eKxGY6ns2g}{回归分析的概念与一元线性回归 2}	%46
		\item \href{https://mp.weixin.qq.com/s/L8AAbA0buMJkktBbsoJ19g}{回归分析的概念与一元线性回归 3}	%47
		\item \href{https://mp.weixin.qq.com/s/G6jlhY2pRShlYVlCzldtNQ}{回归分析的概念与一元线性回归 4}	%48
		\item \href{https://mp.weixin.qq.com/s/FXne817GUt6spMCIMta4TA}{回归分析的概念与一元线性回归 5}	%49
		\item \href{https://mp.weixin.qq.com/s/DPv0phE2cocGX8iO6TDIiQ}{回归分析的概念与一元线性回归 6}	%50
		\item \href{https://mp.weixin.qq.com/s/RiubDcW45fwDizUTV-8r0A}{回归分析的概念与一元线性回归 7}	%51
		\item \href{https://mp.weixin.qq.com/s/xZwNVJyXheYw6j52rdkQWQ}{多元线性回归 1}	%52
		\item \href{https://mp.weixin.qq.com/s/sWqUKZHBeSjZMhENLGobAA}{多元线性回归 2}	%53
		\item \href{https://mp.weixin.qq.com/s/yO4s89OVxYRRSIIDbjTplg}{多元线性回归 3}	%54
		\item \href{https://mp.weixin.qq.com/s/OXiEwynMPZ3OrVhZeYTdhg}{多元线性回归 4}	%55
		\item \href{https://mp.weixin.qq.com/s/IUo7myxrjqoF1qwBCEzyJw}{多元线性回归 5}	%56
		\item \href{https://mp.weixin.qq.com/s/oFwRV3rIPrUr9iI_IygDNQ}{多元线性回归 6}	%57
		\item \href{https://mp.weixin.qq.com/s/piiuvJHy05k-Dc2VLCGfWQ}{多元线性回归 7}	%58
		\item \href{https://mp.weixin.qq.com/s/NBuzHEjqT8azV4jaO1IKfA}{多元线性回归 8 补充 模型修正}	%59
		\item \href{https://mp.weixin.qq.com/s/9q9a3ovEYgZ5_QjqgN8iZg}{逐步回归 1}	%60
		\item \href{https://mp.weixin.qq.com/s/k-UdjNIAntNCnQCBYaWgCQ}{逐步回归 2}	%61
		\item \href{https://mp.weixin.qq.com/s/lhyGgLbjmd--4X-J4rMe4Q}{模型压缩与正则化 1}	%62
		\item \href{https://mp.weixin.qq.com/s/O-KMxreVxx7OEAaVaFrMAg}{模型压缩与正则化 2}	%63
		\item \href{https://mp.weixin.qq.com/s/pl_eoALLNUltoGNxGjv38w}{模型压缩与正则化 3}	%64
		\item \href{https://mp.weixin.qq.com/s/_W4T9qOylZ1Qq8ImrOL58Q}{模型压缩与正则化 4}	%65
		\item \href{https://mp.weixin.qq.com/s/sta0rKuN8aupX-gXjwJfBQ}{模型压缩与正则化 5}	%66
		\item \href{https://mp.weixin.qq.com/s/2TA5lBsmkHtes29GOtObfg}{模型压缩与正则化 6}	%67
		\item \href{https://mp.weixin.qq.com/s/yDRxXuC64JzfgoMILCvO5w}{三种残差}	%68
		\item \href{https://mp.weixin.qq.com/s/nruFdZy89bd1uSwsTa7JXg}{残差图}	%69
		\item \href{https://mp.weixin.qq.com/s/d1WrIuYOYh3Mvzkq1-UIBw}{影响分析}	%70
		\item \href{https://mp.weixin.qq.com/s/RP2Wnot9_BnXBGIVdiJGUQ}{多重共线性}	%71
		\item \href{https://mp.weixin.qq.com/s/Cg1w67-FQTAqAcwVfNjskA}{广义线性模型}	%72
		\item \href{https://mp.weixin.qq.com/s/67EixuRRg3X7_4v4e6DhvA}{逻辑斯蒂回归模型 1}	%73
		\item \href{https://mp.weixin.qq.com/s/kBDgyrsKMa5w8MhjhAsHNg}{逻辑斯蒂回归模型 2}	%74
		\item \href{https://mp.weixin.qq.com/s/D8K8AeB40YAmxQ5oejcq6w}{逻辑斯蒂回归模型 3}	%75
		\item \href{https://mp.weixin.qq.com/s/6rN1NTl1cXg7x08RFozVnA}{逻辑斯蒂回归模型 4}	%76
		\item \href{https://mp.weixin.qq.com/s/4Lt-tedS35UG218qU1Pkcg}{泊松回归模型}	%77
		\item \href{https://mp.weixin.qq.com/s/PhASmnP01_iMkbbrfGMnMw}{多项式回归模型}	%78
		\item \href{https://mp.weixin.qq.com/s/QmT0RzdDCd9JQJqXqfBgaw}{正交多项式回归模型 1}	%79
		\item \href{https://mp.weixin.qq.com/s/XgcudheB53vRmQVo-vJA9A}{正交多项式回归模型 2}	%80
		\item \href{https://mp.weixin.qq.com/s/9uvIzjvdgW6McDQce-DQOA}{内在非线性回归}	%81
		\item \href{https://mp.weixin.qq.com/s/ZIyemFRmiQp0082GuqBWpQ}{阶梯函数}	%82
		\item \href{https://mp.weixin.qq.com/s/odznN76y2DOw6o1i3m0SLA}{回归样条 1}	%83
		\item \href{https://mp.weixin.qq.com/s/lHG9V4lBvqHejyO_2M8JHQ}{回归样条 2}	%84
		\item \href{https://mp.weixin.qq.com/s/ScrbVXlJx_2ZyjZaAgWcdA}{回归样条 3}	%85
		\item \href{https://mp.weixin.qq.com/s/5mfkm7NnsrtQGI3AChKpkA}{广义可加模型}	%86
		\item \href{https://mp.weixin.qq.com/s/yBDbrbFil-qyoJtzGgCY2g}{单因素方差分析}	%87
		\item \href{https://mp.weixin.qq.com/s/tAPOSiLiACp7NSewzy4Zig}{单因素方差分析示例}	%88
		\item \href{https://mp.weixin.qq.com/s/zAwolLEmLZPL4uT7huHfPg}{均值多重比较}	%89
		\item \href{https://mp.weixin.qq.com/s/oza2CZDb_T0oW1eLJpOzpA}{方差齐性检验}	%90
		\item \href{https://mp.weixin.qq.com/s/UF6BhdWP01FKVdHGPl99tA}{Kruskal-Wallis 秩和检验}	%91
		\item \href{https://mp.weixin.qq.com/s/MJ1afMwgKPEnSalm-qeHOA}{Friedman 秩和检验}	%92
		\item \href{https://mp.weixin.qq.com/s/ttl-M_868gIk70q4Rm6j8Q}{不考虑交互作用的双因素方差分析}	%93
		\item \href{https://mp.weixin.qq.com/s/ByA7Tgw7tO3D9JfddF30eQ}{考虑交互作用的双因素方差分析}	%94
		\item \href{https://mp.weixin.qq.com/s/1JhSW9HhSa2m-roSuBMnkQ}{距离判别 1}	%95
		\item \href{https://mp.weixin.qq.com/s/odiOILH-2A0aXNEkNyB6JQ}{距离判别 2}	%96
		\item \href{https://mp.weixin.qq.com/s/A5rGs7IM_swTmYIul80I1A}{距离判别 3}	%97
		\item \href{https://mp.weixin.qq.com/s/eBXdCBwgXBcszxBSE4EePw}{贝叶斯判别 1}	%98
		\item \href{https://mp.weixin.qq.com/s/6x2nQjUNKETcjOgswKvkwQ}{贝叶斯判别 2}	%99
		\item \href{https://mp.weixin.qq.com/s/YAH11ctWCIrgSgMnIK0v8w}{Fisher 判别}	%100
		\item \href{https://mp.weixin.qq.com/s/y_5qnIUQeVeBwe3dsLC8tQ}{线性判别与二次判别分析}	%101
		\item \href{https://mp.weixin.qq.com/s/CmXzgGQzX-dnKsp7_F9-YQ}{线性判别与二次判别分析 补充知识}	%102
		\item \href{https://mp.weixin.qq.com/s/pTtkp4i4EjPA15xGr8bXEA}{聚类分析 1}	%103
		\item \href{https://mp.weixin.qq.com/s/J3ihMUFLRRJ5KkOyOKtiFA}{聚类分析 2}	%104
		\item \href{https://mp.weixin.qq.com/s/dx9GJ9kQ-NVPYYCpqlm6pA}{主成分分析过程 1}	%105
		\item \href{https://mp.weixin.qq.com/s/AdbAE3etIOdXDnmScR4UMQ}{主成分分析过程 2}	%106
		\item \href{https://mp.weixin.qq.com/s/5879fLJKlrF3LrfDHGMhzw}{主成分分析过程 3}	%107
		\item \href{https://mp.weixin.qq.com/s/unHuA9mXDf0aOIKbYTbr4A}{主成分回归}	%108
		\item \href{https://mp.weixin.qq.com/s/C3w_g1j-6nTyQ129zQWZAA}{因子模型}	%109
		\item \href{https://mp.weixin.qq.com/s/8Cq9j8O-c4bslP_N1gBm8Q}{因子分析的参数估计}	%110
		\item \href{https://mp.weixin.qq.com/s/nX6uaqPJtWbgAxG3XFsI-Q}{因子旋转}	%111
		\item \href{https://mp.weixin.qq.com/s/MWurF0RGDhtAuTfsXsvQmg}{因子得分}	%112
		\item \href{https://mp.weixin.qq.com/s/HbdGvDeefApfvO4dsfhD4g}{典型相关分析 1}	%113
		\item \href{https://mp.weixin.qq.com/s/wAa52yRNmIcHhr-YRS1NZQ}{典型相关分析 2}	%114
		\item \href{https://mp.weixin.qq.com/s/PsPXTRWgz6jtj3zHWkwABQ}{典型相关分析 3}	%115
		\item \href{https://mp.weixin.qq.com/s/TRgaufutVZYmxeuhDQohfA}{经验分布}	%116
		\item \href{https://mp.weixin.qq.com/s/JpTPeihiHc3xY2HH0AKTBQ}{生存函数}	%117
		\item \href{https://mp.weixin.qq.com/s/7PoWquHk2n12GqdaonWrqg}{秩检验统计量}	%118
		\item \href{https://mp.weixin.qq.com/s/CsrWq3P2rFlotI7GT2aHXQ}{符号检验}	%119
		\item \href{https://mp.weixin.qq.com/s/m9sNzhNR9ceVNhxcTGikCQ}{分位数检验}	%120
		\item \href{https://mp.weixin.qq.com/s/T9G1uStsdA8tplXqJ89fOg}{Cox-Stuart 趋势存在性检验}	%121
		\item \href{https://mp.weixin.qq.com/s/ud6h02JBG7YjgxxDoN-bdA}{随机游程检验}	%122
		\item \href{https://mp.weixin.qq.com/s/ZaWrv555RxjSlLNGL0g0VQ}{Wilcoxon 符号秩检验}	%123
		\item \href{https://mp.weixin.qq.com/s/QoBHfUtLXswmpv045i7Wtg}{正态记分检验}	%124
		\item \href{https://mp.weixin.qq.com/s/XWgQZz1mlLqMs1pgQFIiFw}{分布一致性检验}	%125
		\item \href{https://mp.weixin.qq.com/s/Ak5Q4W1WrTgeGQuRAS9b3A}{K-S 与 Liliefor 正态性检验}	%126
		\item \href{https://mp.weixin.qq.com/s/ukDR91C9737TCJbOWK0C5g}{Brown-Mood 中位数检验}	%127
		\item \href{https://mp.weixin.qq.com/s/N_KEMx-2gq1PWim8ip_qBA}{W-M-W 秩和检验和 Mood 方差检验}	%128
		\item \href{https://mp.weixin.qq.com/s/ukZMYfHzG1-LfjT8Q7aeaw}{K-W 单因素方差分析}	%129
		\item \href{https://mp.weixin.qq.com/s/KMAgnOXFFI72byPVliA1ow}{Friedman 秩方差分析法}	%130
		\item \href{https://mp.weixin.qq.com/s/2biRultEkp1x36yEWL-mzQ}{Hodges-Lehmann 检验}	%131
		\item \href{https://mp.weixin.qq.com/s/ST1M1dhP8eOxX0ZMqawwKw}{Cochran 检验}	%132
		\item \href{https://mp.weixin.qq.com/s/9U1wJznMwuKfRmAN2VgitQ}{列联表和独立性检验}	%133
		\item \href{https://mp.weixin.qq.com/s/VaGwaiCgoRa5gtSkLBONuQ}{Fisher 精确性检验与 M-H 检验}	%134
		\item \href{https://mp.weixin.qq.com/s/1G4Gbhn4izk15k7SuVufFw}{对数线性模型}	%135
		\item \href{https://mp.weixin.qq.com/s/YoL1lFbCtD0XDwmhtFbdrA}{Spearman 秩相关检验}	%136
		\item \href{https://mp.weixin.qq.com/s/5qqlaLtjsaoTYjqgPKugNg}{Kendall 相关检验}	%137
		\item \href{https://mp.weixin.qq.com/s/m5gcPNTEATxcw3S--7QAVw}{多变量 Kendall 协和系数检验}	%138
		\item \href{https://mp.weixin.qq.com/s/QpWBn7htcLom6vJ52kOHAA}{Kappa 一致性检验}	%139
		\item \href{https://mp.weixin.qq.com/s/8A4Kk0gqBkSvzQUKGcNCHQ}{中位数回归系数估计方法}	%140
		\item \href{https://mp.weixin.qq.com/s/OASb4wPvb7KrfFMuDLdbnQ}{线性分位回归模型}	%141
		\item \href{https://mp.weixin.qq.com/s/a52nR5oArB6NKUsSJnV6Hw}{直方图密度估计}	%142
		\item \href{https://mp.weixin.qq.com/s/Mwigbuo4rFgon_GrV66i5w}{核密度估计}	%143
		\item \href{https://mp.weixin.qq.com/s/-7ZS5hrbXbzVnbN_dY42qg}{核回归光滑模型}	%144
		\item \href{https://mp.weixin.qq.com/s/neXwXL5nbB0GhadkrZ5N0g}{局部多项式回归}	%145
		\item \href{https://mp.weixin.qq.com/s/OtZ4cEzbMB5NyOzuRUECZQ}{LOWESS 稳健回归}	%146
		\item \href{https://mp.weixin.qq.com/s/D7Ha1OxQbxj_ej0ptU0wcQ}{k 近邻回归}	%147
		\item \href{https://mp.weixin.qq.com/s/YnYYk2RLB1kRYM2wRhjNRQ}{正交序列回归与样条回归}	%148	
		%\item \href{url}{Materials}
	\end{enumerate}
\end{multicols}




%%%%%%%%%%%%%%%%%%%%%%%%%%%%%%%%%%%%%%%%%%%%%%%%%%%%%%%%%%%%%%%%%%%%%%%%%%%%%%%%%%%%%%%%%%%%%%%%%%%%%%%%%%%%%%%%%%%%%%%
%\bibliographystyle{ieeetr} % number
%%\bibliographystyle{unsrtnat} % author year
%\bibliography{HeBib}
%%%%%%%%%%%%%%%%%%%%%%%%%%%%%%%%%%%%%%%%%%%%%%%%%%%%%%%%%%%%%%%%%%%%%%%%%%%%%%%%%%%%%%%%%%%%%%%%%%%%%%%%%%%%%%%%%%%%%%%

%\begin{flushright}
%	\tiny \today 
%\end{flushright}
%%%%%%%%%%%%%%%%%%%%%%%%%%%%%%%%%%%%%%%%%%%%%%%%%%%%%%%%%%%%%%%%%%%%%%%%%%%%%%%%%%%%%%%%%%%%%%%%%%%%%%%%%%%%%%%%%%%%%%%
\end{document}
%%%%%%%%%%%%%%%%%%%%%%%%%%%%%%%%%%%%%%%%%%%%%%%%%%%%%%%%%%%%%%%%%%%%%%%%%%%%%%%%%%%%%%%%%%%%%%%%%%%%%%%%%%%%%%%%%%%%%%%
              