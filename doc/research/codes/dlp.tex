%%%%%%%%%%%%%%%%%%%%%%%%%%%%%%%%%%%%%%%%%%%%%%%%%%%%%%%%%%%%%%%%%%%%%%%%%%%%%%%%%%%%%%%%%%%%%%%%%%%%%%%%%%%%%%%%%%%%%
%%%%%%%%%%%%%%%%%%%%%%%%%%%%%%%%%%%%%%%%%%%%%   Author:Yao Zhang  %%%%%%%%%%%%%%%%%%%%%%%%%%%%%%%%%%%%%%%%%%%%%%%%%%%
%%%%%%%%%%%%%%%%%%%%%%%%%%%%%%%%%%%%%%%%%%%%% Email: jaafar_zhang@163.com %%%%%%%%%%%%%%%%%%%%%%%%%%%%%%%%%%%%%%%%%%%
%%%%%%%%%%%%%%%%%%%%%%%%%%%%%%%%%%%%%%%%%%%%%%%%%%%%%%%%%%%%%%%%%%%%%%%%%%%%%%%%%%%%%%%%%%%%%%%%%%%%%%%%%%%%%%%%%%%%%
\documentclass[11pt]{article}
\usepackage{babel}
\usepackage[utf8]{inputenc} 
\usepackage[table]{xcolor}
\usepackage[most]{tcolorbox}
\usepackage[left=2.50cm, right=1.50cm, top=2.0cm, bottom=2.50cm]{geometry}
\usepackage{xcolor,url}
\usepackage{amsmath,amsthm,amsfonts,amssymb,amscd,multirow,booktabs,fullpage,calc,multicol}
\usepackage{lastpage,enumitem,fancyhdr,mathrsfs,wrapfig,setspace,cancel,amsmath,empheq,framed}
\usepackage[retainorgcmds]{IEEEtrantools}
\usepackage{subfig,graphicx,framed}
\usepackage{ctex}
\usepackage{txfonts}
\usepackage{bbm}
\usepackage{chngcntr}
\usepackage[colorlinks,linkcolor=blue,anchorcolor=green,citecolor=red,urlcolor=blue]{hyperref}
\usepackage{titlesec}
%%%%%%%%%%%%%%%%%%%%%%%%%%%%%%%%%%%%%%%%%%%%%%%%%%%%%%%%%%%%%%%%%%%%%%%%%%%%%%%%%%%%%%%%%%%%%%%%%%%%%%%%%%%%%%%%%%%%%%
\newtheorem{thm}{Theorem}[section]
\newtheorem{defi}{Definition}[subsection]
\newtheorem{exercise}{Exercise}[subsection]
\newtheorem{note}{Note}[subsection]
\newtheorem{notation}{Notation}
\newtheorem{lemma}{Lemma}[subsection]
\newtheorem{proposition}{Proposition}[subsection]
\newtheorem{example}{Example}[subsection]
\newtheorem{problem}{Problem}[section]
\newtheorem{homework}{Homework}[section]
\newtheorem{summary}{Summary}[subsection]
\newtheorem{corollary}{Corollary}[subsection]
\newtheorem{rmk}{Remark}[section]
\usepackage{romannum}
%%%%%%%%%%%%%%%%%%%%%%%%%%%%%%%%%%%%%%%%%%%%%%%%%%%%%%%%%%%%%%%%%%%%%%%%%%%%%%%%%%%%%%%%%%%%%%%%%%%%%%%%%%%%%%%%%%%%%
\newlength{\tabcont}
\setlength{\parindent}{0.0in}
\setlength{\parskip}{0.05in}
\colorlet{shadecolor}{orange!15}
\parindent 0in
\parskip 12pt
\geometry{margin=1in, headsep=0.25in}
%%%%%%%%%%%%%%%%%%%%%%%%%%%%%%%%%%%%%%%%%%%%%%%%%%%%%%%%%%%%%%%%%%%%%%%%%%%%%%%%%%%%%%%%%%%%%%%%%%%%%%%%%%%%%%%%%%%%%
\graphicspath{ {img/EoM/}}
%%%%%%%%%%%%%%%%%%%%%%%%%%%%%%%%%%%%%%%%%%%%%%%%%%%%%%%%%%%%%%%%%%%%%%%%%%%%%%%%%%%%%%%%%%%%%%%%%%%%%%%%%%%%%%%%%%%%%
%\renewcommand{\cite}[1]{[#1]}
\makeatletter
\@addtoreset{equation}{section}
\makeatother
\renewcommand{\theequation}{\arabic{section}.\arabic{equation}}
\renewcommand{\contentsname}{\centering \small \color{blue} Contents}
%\counterwithin{figure}{section}
\renewcommand{\figurename}{\textbf{Fig.}}
%\renewcommand{\refname}{\textbf{\kaishu 参考文献}}
\renewcommand{\refname}{\textbf{Bibliography}}
\setcounter{secnumdepth}{4}
\titleformat{\paragraph}
{\normalfont\normalsize\bfseries}{\theparagraph}{1em}{}
\titlespacing*{\paragraph}{0pt}{3.25ex plus 1ex minus .2ex}{1.5ex plus .2ex}
\def\beginrefs{\begin{list}%
		{[\arabic{equation}]}{\usecounter{equation}
			\setlength{\leftmargin}{0.8truecm}\setlength{\labelsep}{0.4truecm}%
			\setlength{\labelwidth}{1.6truecm}}}
	\def\endrefs{\end{list}}
\def\bibentry#1{\item[\hbox{[#1]}]}
%%%%%%%%%%%%%%%%%%%%%%%%%%%%%%%%%%%%%%%%%%%%%%%%%%%%%%%%%%%%%%%%%%%%%%%%%%%%%%%%%%%%%%%%%%%%%%%%%%%%%%%%%%%%%%%%%%%%%%
%\begin{figure}[!htb]
%	\centering
%	\subfloat[$A \cap B$]{%
%		\includegraphics[width=0.3\linewidth,height=0.2\linewidth]{img001.jpg}}
%	\label{img001}\qquad \qquad %\hfill
%	\subfloat[${A_1} \cap {A_2} \cap {A_3}$]{%
%		\includegraphics[width=0.3\linewidth,height=0.2\linewidth]{img002.jpg}}
%	\label{img002}
	%\caption{ Examples.}
%\end{figure}
%\begin{figure}[!htb]
%	\centering
%	\includegraphics[width=0.4\linewidth,height=0.3\linewidth]{img005.jpg}
%	\label{img005}
	%\caption{ illustration for $ 3 $}
%\end{figure}
%\={a}1 \'{a}2\v{a}3\.{a}4

\usepackage{datetime}
\renewcommand{\today}{\shortmonthname[\the\month] \the \day,  \the\year}
%%%%%%%%%%%%%%%%%%%%%%%%%%%%%%%%%%%%%%%%%%%%%%%%%%%%%%%%%%%%%%%%%%%%%%%%%%%%%%%%%%%%%%%%%%%%%%%%%%%%%%%%%%%%%%%%%%%%%%
\begin{document}
	\kaishu 
	%\thispagestyle{empty}
	\pagenumbering{arabic} 
	\setcounter{section}{0}
	\begin{center}
		{\LARGE  \href{https://space.bilibili.com/1274695134/lists/2083514?type=season}{Deep Learning with PyTorch}}
		
		%\vspace{-0.25cm}
		
		{\large \href{https://space.bilibili.com/1274695134}{PhilLee}}
	\end{center}
%%%%%%%%%%%%%%%%%%%%%%%%%%%%%%%%%%%%%%%%%%%%%%%%%%%%%%%%%%%%%%%%%%%%%%%%%%%%%%%%%%%%%%%%%%%%%%%%%%%%%%%%%%%%%%%%%%%%%%
%%\newpage 
%%\thispagestyle{empty}	
%%%%%%%%%%%%%%%%%%%%%%%%%%%%%%%%%%%%%%%%%%%%%%%%%%%%%%%%%%%%%%%%%%%%%%%%%%%%%%%%%%%%%%%%%%%%%%%%%%%%%%%%%%%%%%%%%%%%%%
%\tableofcontents	
%{\pagestyle{empty}\mbox{}\newpage\pagestyle{empty}}
%\newpage 
%{\pagestyle{empty}\mbox{}\newpage\pagestyle{empty}}
%%%%%%%%%%%%%%%%%%%%%%%%%%%%%%%%%%%%%%%%%%%%%%%%%%%%%%%%%%%%%%%%%%%%%%%%%%%%%%%%%%%%%%%%%%%%%%%%%%%%%%%%%%%%%%%%%%%%%%
%%\newpage 
\setcounter{page}{1}

%\vspace{1.5cm}


\vspace{-1cm}

\begin{enumerate}
	\item \href{https://mp.weixin.qq.com/s/cUuTYKWo6SB2EMqSHGo0BQ}{ch0}	%1
	\item \href{https://mp.weixin.qq.com/s/uJAvG8qmZBIKUHUkCzMdDg}{ch1}	%2
	\item \href{https://mp.weixin.qq.com/s/ay522YZNJ15wyjcwLGf3cA}{2.1 Pretrained networks}	%3
	\item \href{https://mp.weixin.qq.com/s/dUHzCriNFep4HvwjeDCy-A}{2.2 Pretrained networks}	%4
	\item \href{https://mp.weixin.qq.com/s/yOL6ihoa01JpU0K3LaPQOg}{2.3 Pretrained networks}	%5
	\item \href{https://mp.weixin.qq.com/s/bbl96mA31YYauQWmXRTW3A}{2.4 Pretrained networks exercise}	%6
	\item \href{https://mp.weixin.qq.com/s/RkoyWy9ul7FMVMJ9G0MZqA}{3.1 一切从张量开始}	%7
	\item \href{https://mp.weixin.qq.com/s/d88KyZeRb84xeu3o9MOF7g}{3.2 三个关于内存的函数}	%8
	\item \href{https://mp.weixin.qq.com/s/Ess4ISXREbBOEhauK5hXlQ}{3.3 数据类型}	%9
	\item \href{https://mp.weixin.qq.com/s/x-C09e4k52Rm_C45KOIhPQ}{3.4 张量的索引引用}	%10
	\item \href{https://mp.weixin.qq.com/s/2VNR9Tkhkpv0QohMH7tw_A}{3.5 读写文件}	%11
	\item \href{https://mp.weixin.qq.com/s/YBjwu2C83BD-nFpTZ6a5iw}{3.6 张量的维度命名}	%12
	\item \href{https://mp.weixin.qq.com/s/PjfZ2NQcBZ1pb6bjU5SFnQ}{3.7 exercise 习题}	%13
	\item \href{https://mp.weixin.qq.com/s/6Zr57a7Y12dTeMSLQHdeVg}{4.1 图像文件读取, 转张量, 标准化}	%14
	\item \href{https://mp.weixin.qq.com/s/BLt4nOj9j7xSHMER2vJNxg}{4.2 读取 3D 图像}	%15
	\item \href{https://mp.weixin.qq.com/s/JEIy-nk4hYw4cLtlKbwrVg}{4.3 表格数据读取,超天真评酒模型}	%16
	\item \href{https://mp.weixin.qq.com/s/FdtFEyBZBv7qE9RRbR0wlg}{4.4 读取时间序列}	%17
	\item \href{https://mp.weixin.qq.com/s/QONqTxujiCUMmsq-AGbtNg}{4.5 读取文本和编码}	%18
	\item \href{https://mp.weixin.qq.com/s/z4F4pyPWw_RAKHVmAMEVGg}{4.6 exercise 习题}	%19
	\item \href{https://mp.weixin.qq.com/s/fPJ5to-eRDArbBEmmYxI1Q}{5.1 手算梯度递降, 解线性回归}	%20
	\item \href{https://mp.weixin.qq.com/s/1araRBCpR6tGiV4cWTM2bA}{5.2 Autograd 自动梯度计算}	%21
	\item \href{https://mp.weixin.qq.com/s/qNmnJ3EGPrczZNNZTH6yJQ}{5.3 Optimizer 优化器和训练流程}	%22
	\item \href{https://mp.weixin.qq.com/s/7G8jrxUq-aMI7K3qTrD03A}{5.4 Exercise 习题}	%23
	\item \href{https://mp.weixin.qq.com/s/SEPqSD9W3HnEeuU6IcxQYQ}{6.1 单层线性神经网络}	%24
	\item \href{https://mp.weixin.qq.com/s/0DHIJdzxWAgtusqxZqeg2g}{6.2 多层 + 激活层的简单神经网络}	%25
	\item \href{https://mp.weixin.qq.com/s/2V68bVVr7Gi05LqOdHgsOw}{6.3 Exercise 习题}	%26
	\item \href{https://mp.weixin.qq.com/s/wcKFTXTyU9szEacaLhGOVw}{7.1 Dataset 数据集 class 的使用}	%27
	\item \href{https://mp.weixin.qq.com/s/tpRiT0r6xMwzXLdo767r3g}{7.2 数据集筛选,softmax,分类结果的输出}	%28
	\item \href{https://mp.weixin.qq.com/s/BMEFM9wVRWp-maQJ7_6yTw}{7.3 分类模型的loss函数 MSE 和 NLL(等价于交叉熵)}	%29
	\item \href{https://mp.weixin.qq.com/s/mZNcLjRIGuRSrwCxOIQYXg}{7.4 分类模型 loss 和激活层的配对,(非卷积)网络的训练}	%30
	\item \href{https://mp.weixin.qq.com/s/DWgyqaFd1L3tHoCsAqo1vQ}{7.5 Exercise 习题(1)}	%31
	\item \href{https://mp.weixin.qq.com/s/DNTQYi2QUf60zEginb5JBw}{7.6 Exercise 习题(2)}	%32
	\item \href{https://mp.weixin.qq.com/s/ofXwMlSUQXdFux9htN7vxw}{8.1 卷积层 Conv2d 和 MaxPool2d}	%33
	\item \href{https://mp.weixin.qq.com/s/PPk6EJce2Yfd_jca-3vDng}{8.2 用 Module subclass 和 Functional as F}	%34
	\item \href{https://mp.weixin.qq.com/s/TObOqNRdj6A88IQ7Oo--3Q}{8.3 用 GPU 训练卷积网络}	%35
	\item \href{https://mp.weixin.qq.com/s/qZUymw_5mMcYokXEhejfhg}{8.4 模型设计: 宽度, 参数规范, Drop out, Batch normluization}	%36
	\item \href{https://mp.weixin.qq.com/s/HI79ymNktyr3l77zzX53Og}{8.5 模型设计: 宽度,参数规范, Drop out, Batch normlization}	%37
	\item \href{https://mp.weixin.qq.com/s/zodvZnERtHC226Dqr1C_dw}{8.6 模型设计: Residual Net 残差网络}	%38
	\item \href{https://mp.weixin.qq.com/s/y2dBbP3hY0uNa_5bHY0XSQ}{8.7 Exercise 习题 (1)}	%39
	\item \href{https://mp.weixin.qq.com/s/HGkmnYuPl36cZTGFYCu8fQ}{8.8 Exercise 习题 (2)}	%40
	\item \href{https://mp.weixin.qq.com/s/Ymay8JW_txqqLZHmR2w5Rw}{9 整个项目的分解}	%41
	\item \href{https://mp.weixin.qq.com/s/2ITdHjl3pE1MbkwrP_0h9A}{10.1 读取和处理 candidate.csv 和 annotatio}	%42
	\item \href{https://mp.weixin.qq.com/s/Th8JZrmaD2LEUKx08YQlvw}{10.2 读取和处理 CT文件}	%43
	\item \href{https://mp.weixin.qq.com/s/lIzTsdNRAcOUwMsjzdne_Q}{10.3 LunaDataset 完成}	%44
	\item \href{https://mp.weixin.qq.com/s/0PRdfoS_JFB1Q1FrKekrrg}{10.4 Exercise}	%45
	\item \href{https://mp.weixin.qq.com/s/nHCuviWnYNLiOa0FfnJOEA}{11.1 拆解缓存准备.py文件}	%46
	\item \href{https://mp.weixin.qq.com/s/FOwuu0XcUOSJ2apWjFVwsg}{11.2 拆解 train.py (1)}	%47
	\item \href{https://mp.weixin.qq.com/s/x7z3bPo4ozUAlyPyYAJbGQ}{11.3 拆解 train.py (2)}	%48
	\item \href{https://mp.weixin.qq.com/s/H4cwFWNLjZb3WZHuTQzuyg}{11.4 训练结果和 TensorBoard}	%49
	\item \href{https://mp.weixin.qq.com/s/xyQoo7qDBxN_Zk8Wt1FElQ}{11.5 Exercise 习题}	%50
	\item \href{https://mp.weixin.qq.com/s/OqJ-vUynfT5xhSqPDtE2Dw}{12.1 Precision, Recall, F1}	%51
	\item \href{https://mp.weixin.qq.com/s/gUEClB605k4iz6YbxgEXhQ}{12.2 平衡阴性阳性数据}	%52
	\item \href{https://mp.weixin.qq.com/s/qqZOqRp-Urn11w8IvmXufw}{12.3 数据增强 Augmentation - 为什么可解决overfitting}	%53
	\item \href{https://mp.weixin.qq.com/s/KTbE3D77ytEA199gVcLuXg}{12.4 Exercise 习题 (1)}	%54
	\item \href{https://mp.weixin.qq.com/s/xs18IxlwNQI1EBXU6E6BmQ}{12.5 Exercise 习题 (2)}	%55
	\item \href{https://mp.weixin.qq.com/s/J1qWWp5EMnlKmGKHd7FOpQ}{13.1 程序没调通, 先读书水一期 U-Net}	%56
	\item \href{https://mp.weixin.qq.com/s/1Ik9lzNetz8-9rkNC214ew}{13.2 Debug 乐趣多}	%57
	\item \href{https://mp.weixin.qq.com/s/br9IxnOfIxSkEHM0FPycsQ}{13.3 explore data debug}	%58
	\item \href{https://mp.weixin.qq.com/s/Ua2lGl55RXBm70UQwTv5eQ}{13.4 CT 类中获得 segmentation mask 的小算法}	%59
	\item \href{https://mp.weixin.qq.com/s/1OjKDD8oJY0sI6kAERE10A}{13.5 Luna2dSegmentationDataset}	%60
	\item \href{https://mp.weixin.qq.com/s/-JjDf_BtB53sCdyLgIvmxg}{13.6 训练准备: Unet 模型, 数据增强, Adam 优化器}	%61
	\item \href{https://mp.weixin.qq.com/s/Fplkh2gBhFT7jqjaGMf-pA}{13.7 Data loader, Dice Loss, 训练!}	%62
	\item \href{https://mp.weixin.qq.com/s/iXBZWaESuz5-RnxLDZmnIg}{13.8 Tensorboard 显示 segmentation 图像结果}	%63
	\item \href{https://mp.weixin.qq.com/s/PxIMdRbtdxER8dASkXd3Hg}{13.9 之前的问题, 本书的抢先看 MEAP 版本}	%64
	\item \href{https://mp.weixin.qq.com/s/OqWj7TkOxfxr-AmBWQgU7w}{14.1 整体 app 还剩的工作, dataset 避免 leak 的修改}	%65
	\item \href{https://mp.weixin.qq.com/s/ncCBnC5OcjYccLQypv-6VQ}{14.2 连接使用分割和分类模型, 模型的读取}	%66
	\item \href{https://mp.weixin.qq.com/s/aUyxMsG5CVsbUGOUGUXAxA}{14.3 nodule 分析主循环, 使用两个模型做预测}	%67
	\item \href{https://mp.weixin.qq.com/s/4QbW1jKzunJRagQHA40usQ}{14.4 结果的混淆矩阵和 ROC AUC}	%68
	\item \href{https://mp.weixin.qq.com/s/BhvLHM3M-IxPSTXmYfIMfQ}{14.5 Fine tune 之前的模型做良性/恶性的分类}	%69
	\item \href{https://mp.weixin.qq.com/s/14XGf6vGCA1_7hyu0I5fmQ}{14.6 TensorBoard 中直方图和 ROC 曲线}	%70
	\item \href{https://mp.weixin.qq.com/s/FAuBkJb-HGrYgvNE63rscw}{14.7 Label smoothing 标签平滑化}	%71
	\item \href{https://mp.weixin.qq.com/s/r42qBNtzC1Yb_EXiORQj1w}{14.8 结尾}	%72	
	%\item \href{url}{Materials}
\end{enumerate}




%%%%%%%%%%%%%%%%%%%%%%%%%%%%%%%%%%%%%%%%%%%%%%%%%%%%%%%%%%%%%%%%%%%%%%%%%%%%%%%%%%%%%%%%%%%%%%%%%%%%%%%%%%%%%%%%%%%%%%%
%\bibliographystyle{ieeetr} % number
%%\bibliographystyle{unsrtnat} % author year
%\bibliography{HeBib}
%%%%%%%%%%%%%%%%%%%%%%%%%%%%%%%%%%%%%%%%%%%%%%%%%%%%%%%%%%%%%%%%%%%%%%%%%%%%%%%%%%%%%%%%%%%%%%%%%%%%%%%%%%%%%%%%%%%%%%%
\begin{flushright}
	\tiny \today 
\end{flushright}
%%%%%%%%%%%%%%%%%%%%%%%%%%%%%%%%%%%%%%%%%%%%%%%%%%%%%%%%%%%%%%%%%%%%%%%%%%%%%%%%%%%%%%%%%%%%%%%%%%%%%%%%%%%%%%%%%%%%%%%
\end{document}
%%%%%%%%%%%%%%%%%%%%%%%%%%%%%%%%%%%%%%%%%%%%%%%%%%%%%%%%%%%%%%%%%%%%%%%%%%%%%%%%%%%%%%%%%%%%%%%%%%%%%%%%%%%%%%%%%%%%%%%
              