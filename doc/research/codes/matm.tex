%%%%%%%%%%%%%%%%%%%%%%%%%%%%%%%%%%%%%%%%%%%%%%%%%%%%%%%%%%%%%%%%%%%%%%%%%%%%%%%%%%%%%%%%%%%%%%%%%%%%%%%%%%%%%%%%%%%%%
%%%%%%%%%%%%%%%%%%%%%%%%%%%%%%%%%%%%%%%%%%%%%   Author:Yao Zhang  %%%%%%%%%%%%%%%%%%%%%%%%%%%%%%%%%%%%%%%%%%%%%%%%%%%
%%%%%%%%%%%%%%%%%%%%%%%%%%%%%%%%%%%%%%%%%%%%% Email: jaafar_zhang@163.com %%%%%%%%%%%%%%%%%%%%%%%%%%%%%%%%%%%%%%%%%%%
%%%%%%%%%%%%%%%%%%%%%%%%%%%%%%%%%%%%%%%%%%%%%%%%%%%%%%%%%%%%%%%%%%%%%%%%%%%%%%%%%%%%%%%%%%%%%%%%%%%%%%%%%%%%%%%%%%%%%
\documentclass[11pt]{article}
\usepackage{babel}
\usepackage[utf8]{inputenc} 
\usepackage[table]{xcolor}
\usepackage[most]{tcolorbox}
\usepackage[left=2.50cm, right=1.50cm, top=2.0cm, bottom=2.50cm]{geometry}
\usepackage{xcolor,url}
\usepackage{amsmath,amsthm,amsfonts,amssymb,amscd,multirow,booktabs,fullpage,calc,multicol}
\usepackage{lastpage,enumitem,fancyhdr,mathrsfs,wrapfig,setspace,cancel,amsmath,empheq,framed}
\usepackage[retainorgcmds]{IEEEtrantools}
\usepackage{subfig,graphicx,framed}
\usepackage{ctex}
\usepackage{txfonts}
\usepackage{bbm}
\usepackage{chngcntr}
\usepackage[colorlinks,linkcolor=blue,anchorcolor=green,citecolor=red,urlcolor=blue]{hyperref}
\usepackage{titlesec}
%%%%%%%%%%%%%%%%%%%%%%%%%%%%%%%%%%%%%%%%%%%%%%%%%%%%%%%%%%%%%%%%%%%%%%%%%%%%%%%%%%%%%%%%%%%%%%%%%%%%%%%%%%%%%%%%%%%%%%
\newtheorem{thm}{Theorem}[section]
\newtheorem{defi}{Definition}[subsection]
\newtheorem{exercise}{Exercise}[subsection]
\newtheorem{note}{Note}[subsection]
\newtheorem{notation}{Notation}
\newtheorem{lemma}{Lemma}[subsection]
\newtheorem{proposition}{Proposition}[subsection]
\newtheorem{example}{Example}[subsection]
\newtheorem{problem}{Problem}[section]
\newtheorem{homework}{Homework}[section]
\newtheorem{summary}{Summary}[subsection]
\newtheorem{corollary}{Corollary}[subsection]
\newtheorem{rmk}{Remark}[section]
\usepackage{romannum}
%%%%%%%%%%%%%%%%%%%%%%%%%%%%%%%%%%%%%%%%%%%%%%%%%%%%%%%%%%%%%%%%%%%%%%%%%%%%%%%%%%%%%%%%%%%%%%%%%%%%%%%%%%%%%%%%%%%%%
\newlength{\tabcont}
\setlength{\parindent}{0.0in}
\setlength{\parskip}{0.05in}
\colorlet{shadecolor}{orange!15}
\parindent 0in
\parskip 12pt
\geometry{margin=1in, headsep=0.25in}
%%%%%%%%%%%%%%%%%%%%%%%%%%%%%%%%%%%%%%%%%%%%%%%%%%%%%%%%%%%%%%%%%%%%%%%%%%%%%%%%%%%%%%%%%%%%%%%%%%%%%%%%%%%%%%%%%%%%%
\graphicspath{ {img/EoM/}}
%%%%%%%%%%%%%%%%%%%%%%%%%%%%%%%%%%%%%%%%%%%%%%%%%%%%%%%%%%%%%%%%%%%%%%%%%%%%%%%%%%%%%%%%%%%%%%%%%%%%%%%%%%%%%%%%%%%%%
%\renewcommand{\cite}[1]{[#1]}
\makeatletter
\@addtoreset{equation}{section}
\makeatother
\renewcommand{\theequation}{\arabic{section}.\arabic{equation}}
\renewcommand{\contentsname}{\centering \small \color{blue} Contents}
%\counterwithin{figure}{section}
\renewcommand{\figurename}{\textbf{Fig.}}
%\renewcommand{\refname}{\textbf{\kaishu 参考文献}}
\renewcommand{\refname}{\textbf{Bibliography}}
\setcounter{secnumdepth}{4}
\titleformat{\paragraph}
{\normalfont\normalsize\bfseries}{\theparagraph}{1em}{}
\titlespacing*{\paragraph}{0pt}{3.25ex plus 1ex minus .2ex}{1.5ex plus .2ex}
\def\beginrefs{\begin{list}%
		{[\arabic{equation}]}{\usecounter{equation}
			\setlength{\leftmargin}{0.8truecm}\setlength{\labelsep}{0.4truecm}%
			\setlength{\labelwidth}{1.6truecm}}}
	\def\endrefs{\end{list}}
\def\bibentry#1{\item[\hbox{[#1]}]}
%%%%%%%%%%%%%%%%%%%%%%%%%%%%%%%%%%%%%%%%%%%%%%%%%%%%%%%%%%%%%%%%%%%%%%%%%%%%%%%%%%%%%%%%%%%%%%%%%%%%%%%%%%%%%%%%%%%%%%
%\begin{figure}[!htb]
%	\centering
%	\subfloat[$A \cap B$]{%
%		\includegraphics[width=0.3\linewidth,height=0.2\linewidth]{img001.jpg}}
%	\label{img001}\qquad \qquad %\hfill
%	\subfloat[${A_1} \cap {A_2} \cap {A_3}$]{%
%		\includegraphics[width=0.3\linewidth,height=0.2\linewidth]{img002.jpg}}
%	\label{img002}
	%\caption{ Examples.}
%\end{figure}
%\begin{figure}[!htb]
%	\centering
%	\includegraphics[width=0.4\linewidth,height=0.3\linewidth]{img005.jpg}
%	\label{img005}
	%\caption{ illustration for $ 3 $}
%\end{figure}
%\={a}1 \'{a}2\v{a}3\.{a}4

\usepackage{datetime}
\renewcommand{\today}{\shortmonthname[\the\month] \the \day,  \the\year}
%%%%%%%%%%%%%%%%%%%%%%%%%%%%%%%%%%%%%%%%%%%%%%%%%%%%%%%%%%%%%%%%%%%%%%%%%%%%%%%%%%%%%%%%%%%%%%%%%%%%%%%%%%%%%%%%%%%%%%
\begin{document}
	\kaishu 
	%\thispagestyle{empty}
	\pagenumbering{arabic} 
	\setcounter{section}{0}
	\begin{center}
		{\LARGE  \href{https://space.bilibili.com/503344993/lists/369799?type=season}{Matlab 数学建模算法与应用}}
		
		%\vspace{-0.25cm}
		
		{\large \href{https://space.bilibili.com/503344993?spm_id_from=333.337.search-card.all.click}{Shoukui Si}}
	\end{center}
%%%%%%%%%%%%%%%%%%%%%%%%%%%%%%%%%%%%%%%%%%%%%%%%%%%%%%%%%%%%%%%%%%%%%%%%%%%%%%%%%%%%%%%%%%%%%%%%%%%%%%%%%%%%%%%%%%%%%%
%%\newpage 
%%\thispagestyle{empty}	
%%%%%%%%%%%%%%%%%%%%%%%%%%%%%%%%%%%%%%%%%%%%%%%%%%%%%%%%%%%%%%%%%%%%%%%%%%%%%%%%%%%%%%%%%%%%%%%%%%%%%%%%%%%%%%%%%%%%%%
%\tableofcontents	
%{\pagestyle{empty}\mbox{}\newpage\pagestyle{empty}}
%\newpage 
%{\pagestyle{empty}\mbox{}\newpage\pagestyle{empty}}
%%%%%%%%%%%%%%%%%%%%%%%%%%%%%%%%%%%%%%%%%%%%%%%%%%%%%%%%%%%%%%%%%%%%%%%%%%%%%%%%%%%%%%%%%%%%%%%%%%%%%%%%%%%%%%%%%%%%%%
%%\newpage 
\setcounter{page}{1}

%\vspace{1.5cm}


\begin{multicols}{2}
	\begin{enumerate}
		\item \href{https://mp.weixin.qq.com/s/ROwL9BqQ1E-KTJ-shX4XVQ}{Matlab 软件入门}	%1
		\item \href{https://mp.weixin.qq.com/s/6k4Wci2bf5J7McDz-MyyMQ}{Matlab 软件入门(续)}	%2
		\item \href{https://mp.weixin.qq.com/s/qJSks7nKWAX-ps6IUe98Jg}{线性规划}	%3
		\item \href{https://mp.weixin.qq.com/s/1j8kLKBG7B6SV2BK4GblSQ}{投资的收益与风险}	%4
		\item \href{https://mp.weixin.qq.com/s/oxafNX6_j666gNiIdPGeOQ}{整数规划}	%5
		\item \href{https://mp.weixin.qq.com/s/6DywMB6qKLPmCctPNSxtAA}{比赛项目排序问题}	%6
		\item \href{https://mp.weixin.qq.com/s/ydKCBteL0Y7boeVmfe1A8g}{非线性规划(一)}	%7
		\item \href{https://mp.weixin.qq.com/s/v5UbUGla3_bHlA_3DFpxig}{非线性规划(二)}	%8
		\item \href{https://mp.weixin.qq.com/s/PYKFN-fnJEOZXKzBVfSPfA}{图与网络模型及方法(一)}	%9
		\item \href{https://mp.weixin.qq.com/s/gtxpv1ayhFzByEjbMBFF1Q}{图与网络模型及方法(二)}	%10
		\item \href{https://mp.weixin.qq.com/s/9v5PgDNsniMtY-2bgyoXmA}{图与网络模型及方法(三)}	%11
		\item \href{https://mp.weixin.qq.com/s/FhBCGQYGxBq12sDo_MkWug}{钢管订购和运输}	%12
		\item \href{https://mp.weixin.qq.com/s/hmHhWaIF7Opmub1b_AEq7g}{插值和拟合(一)}	%13
		\item \href{https://mp.weixin.qq.com/s/W2JNhkUB4fMFhTM1NH3Xkw}{插值和拟合(二)}	%14
		\item \href{https://mp.weixin.qq.com/s/qD2BgQUhql2TQGKyJbPBiw}{插值和拟合(三)}	%15
		\item \href{https://mp.weixin.qq.com/s/z6NA1czRUg5XXkUpldcd3w}{插值和拟合(四)}	%16
		\item \href{https://mp.weixin.qq.com/s/SpnMUv1Xwnr12lBO5dOyww}{微分方程(一)}	%17
		\item \href{https://mp.weixin.qq.com/s/VwPASD4Znk0fK_FrtJq2AA}{微分方程(二)}	%18
		\item \href{https://mp.weixin.qq.com/s/WTtM584myR0sl8yXIfHnEw}{微分方程(三)}	%19
		\item \href{https://mp.weixin.qq.com/s/syMh9enPg38ByNi6o81Y-A}{数理统计(一)}	%20
		\item \href{https://mp.weixin.qq.com/s/MAWWVkacB9wFX1rZpEKQvg}{方差分析}	%21
		\item \href{https://mp.weixin.qq.com/s/l6EuoZvWz-GMjhfekQqKww}{回归分析}	%22
		\item \href{https://mp.weixin.qq.com/s/Mlm2fVnpqnAxvkl2AVMbXQ}{葡萄酒的评价}	%23
		\item \href{https://mp.weixin.qq.com/s/XRjfwEOyapleLhIE9BOt7g}{差分方程(一)}	%24
		\item \href{https://mp.weixin.qq.com/s/2Zn33ZZttZMP0sy3wScX3w}{莱斯利种群模型和最优捕鱼策略}	%25
		\item \href{https://mp.weixin.qq.com/s/FS54kriZHFYK7D3J964zvQ}{支持向量机理论}	%26
		\item \href{https://mp.weixin.qq.com/s/YMFoA8_bxGEQeB7hpW7ITw}{支持向量机的案例及 APP 使用}	%27
		\item \href{https://mp.weixin.qq.com/s/G9vLskepGJpxDm7tmrOzZg}{Q 型聚类}	%28
		\item \href{https://mp.weixin.qq.com/s/G5ie4Jtx-8qITBu2AzdxAQ}{R 型聚类及聚类分析案例}	%29
		\item \href{https://mp.weixin.qq.com/s/uV07NA_aknbZlAIYXP41fg}{主成分分析}	%30
		\item \href{https://mp.weixin.qq.com/s/JivFL-op7SKVTjyts6bGnw}{因子分析(一)}	%31
		\item \href{https://mp.weixin.qq.com/s/_5CeBL4vjGgLM2uMKbx2wQ}{因子分析(二)}	%32
		\item \href{https://mp.weixin.qq.com/s/_bq-YQxVKJPKmElvAO-qPQ}{判别分析}	%33
		\item \href{https://mp.weixin.qq.com/s/J4vuhSe504IsXzXuFbwakQ}{对应分析理论}	%34
		\item \href{https://mp.weixin.qq.com/s/SbdPMtHttbRGaRhQoXAjhw}{对应分析应用案例}	%35
		\item \href{https://mp.weixin.qq.com/s/D6GgMvYVK6TvZIjd6mE7VA}{2021 年 E 题-中药材的鉴别}	%36
		\item \href{https://mp.weixin.qq.com/s/jTy4TI9dZHiPJKYW3o5l3w}{偏最小二乘回归分析}	%37
		\item \href{https://mp.weixin.qq.com/s/3Q41q4L3wxvqfD2qRY-cEA}{模拟退火算法和遗传算法}	%38
		\item \href{https://mp.weixin.qq.com/s/85LcButzNMyJdtIrgPr7xw}{改进的遗传算法与 MATLAB 智能算法简介}	%39
		\item \href{https://mp.weixin.qq.com/s/4AfgKF1WYtvvyju00V5-1A}{数字图像处理(一)}	%40
		\item \href{https://mp.weixin.qq.com/s/yi0gVa0IKxlg33YAbScWwA}{数字图像处理(二)}	%41
		\item \href{https://mp.weixin.qq.com/s/xMRGrzlDTTzA8Ek2quSUiw}{图像水印防伪、加密和隐藏}	%42
		\item \href{https://mp.weixin.qq.com/s/rqa-x42I7c4EO7wtZH5j0Q}{理想解法和模糊综合评判法}	%43
		\item \href{https://mp.weixin.qq.com/s/s1fx-WboCbbGAfVN1lDfeQ}{数据包络分析和灰色关联分析法}	%44
		\item \href{https://mp.weixin.qq.com/s/XaFzc4loBcXUi8TMdOeTag}{PageRank 算法等评价方法}	%45
		\item \href{https://mp.weixin.qq.com/s/GncWotTSPKNxflMOcFKbMA}{2004 年 D 题-公务员招聘}	%46
		\item \href{https://mp.weixin.qq.com/s/cDpUOqocn_r8WkS-e3CWQw}{微分方程预测和灰色预测}	%47
		\item \href{https://mp.weixin.qq.com/s/6YZ8rySBrwh0yc-seqgsTg}{马尔科夫预测与时间序列}	%48
		\item \href{https://mp.weixin.qq.com/s/kaIexZTaeHAGGItE3Jd8aQ}{神经元网络等预测方法}	%49
		\item \href{https://mp.weixin.qq.com/s/YenAIk1IoIC8eUcboU4Jiw}{多目标规划}	%50
		\item \href{https://mp.weixin.qq.com/s/CS515rP1lyVmp0be7YxTiQ}{目标规划}	%51
		\item \href{https://mp.weixin.qq.com/s/eNvcjplWDnowNRCtr-I9Rg}{非线性规划求解(续)}	%52
		\item \href{https://mp.weixin.qq.com/s/vGCKnUWlAos50BeAaJsLVA}{随机模拟}	%53
		\item \href{https://mp.weixin.qq.com/s/bavmERVL-tU8auiEVpwAfQ}{方差分析(续)}	%54
		\item \href{https://mp.weixin.qq.com/s/-CACOvDUoE3TF069g1cqdQ}{偏微分方程用户图形界面解法}	%55
		%\item \href{url}{Materials}
	\end{enumerate}
\end{multicols}






%%%%%%%%%%%%%%%%%%%%%%%%%%%%%%%%%%%%%%%%%%%%%%%%%%%%%%%%%%%%%%%%%%%%%%%%%%%%%%%%%%%%%%%%%%%%%%%%%%%%%%%%%%%%%%%%%%%%%%%
%\bibliographystyle{ieeetr} % number
%%\bibliographystyle{unsrtnat} % author year
%\bibliography{HeBib}
%%%%%%%%%%%%%%%%%%%%%%%%%%%%%%%%%%%%%%%%%%%%%%%%%%%%%%%%%%%%%%%%%%%%%%%%%%%%%%%%%%%%%%%%%%%%%%%%%%%%%%%%%%%%%%%%%%%%%%%
\begin{flushright}
	\tiny \today 
\end{flushright}
%%%%%%%%%%%%%%%%%%%%%%%%%%%%%%%%%%%%%%%%%%%%%%%%%%%%%%%%%%%%%%%%%%%%%%%%%%%%%%%%%%%%%%%%%%%%%%%%%%%%%%%%%%%%%%%%%%%%%%%
\end{document}
%%%%%%%%%%%%%%%%%%%%%%%%%%%%%%%%%%%%%%%%%%%%%%%%%%%%%%%%%%%%%%%%%%%%%%%%%%%%%%%%%%%%%%%%%%%%%%%%%%%%%%%%%%%%%%%%%%%%%%%
              