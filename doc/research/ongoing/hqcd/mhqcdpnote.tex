%%%%%%%%%%%%%%%%%%%%%%%%%%%%%%%%%%%%%%%%%%%%%%%%%%%%%%%%%%%%%%%%%%%%%%%%%%%%%%%%%%%%%%%%%%%%%%%%%%%%%%%%%%%%%%%%%%%%%
%%%%%%%%%%%%%%%%%%%%%%%%%%%%%%%%%%%%%%%%%%%%%   Author:Yao Zhang  %%%%%%%%%%%%%%%%%%%%%%%%%%%%%%%%%%%%%%%%%%%%%%%%%%%
%%%%%%%%%%%%%%%%%%%%%%%%%%%%%%%%%%%%%%%%%%%%% Email: jaafar_zhang@163.com %%%%%%%%%%%%%%%%%%%%%%%%%%%%%%%%%%%%%%%%%%%
%%%%%%%%%%%%%%%%%%%%%%%%%%%%%%%%%%%%%%%%%%%%%%%%%%%%%%%%%%%%%%%%%%%%%%%%%%%%%%%%%%%%%%%%%%%%%%%%%%%%%%%%%%%%%%%%%%%%%
\documentclass[11pt]{article}
\usepackage{babel}
\usepackage[utf8]{inputenc} 
\usepackage[table]{xcolor}
\usepackage[most]{tcolorbox}
\usepackage[left=2.50cm, right=1.50cm, top=2.0cm, bottom=2.50cm]{geometry}
\usepackage{xcolor,url}
\usepackage{amsmath,amsthm,amsfonts,amssymb,amscd,multirow,booktabs,fullpage,calc,multicol}
\usepackage{lastpage,enumitem,fancyhdr,mathrsfs,wrapfig,setspace,cancel,amsmath,empheq,framed}
\usepackage[retainorgcmds]{IEEEtrantools}
\usepackage{subfig,graphicx,framed}
\usepackage{ctex}
\usepackage{txfonts}
\usepackage{bbm}
\usepackage{chngcntr}
\usepackage[colorlinks,linkcolor=blue,anchorcolor=green,citecolor=red,urlcolor=blue]{hyperref}
\usepackage{titlesec}
%%%%%%%%%%%%%%%%%%%%%%%%%%%%%%%%%%%%%%%%%%%%%%%%%%%%%%%%%%%%%%%%%%%%%%%%%%%%%%%%%%%%%%%%%%%%%%%%%%%%%%%%%%%%%%%%%%%%%%
\newtheorem{thm}{Theorem}[section]
\newtheorem{defi}{Definition}[subsection]
\newtheorem{exercise}{Exercise}[subsection]
\newtheorem{note}{Note}[subsection]
\newtheorem{notation}{Notation}
\newtheorem{lemma}{Lemma}[subsection]
\newtheorem{proposition}{Proposition}[subsection]
\newtheorem{example}{Example}[subsection]
\newtheorem{problem}{Problem}[section]
\newtheorem{homework}{Homework}[section]
\newtheorem{summary}{Summary}[subsection]
\newtheorem{corollary}{Corollary}[subsection]
\newtheorem{rmk}{Remark}[section]
%%%%%%%%%%%%%%%%%%%%%%%%%%%%%%%%%%%%%%%%%%%%%%%%%%%%%%%%%%%%%%%%%%%%%%%%%%%%%%%%%%%%%%%%%%%%%%%%%%%%%%%%%%%%%%%%%%%%%
\newlength{\tabcont}
\setlength{\parindent}{0.0in}
\setlength{\parskip}{0.05in}
\colorlet{shadecolor}{orange!15}
\parindent 0in
\parskip 12pt
\geometry{margin=1in, headsep=0.25in}
%%%%%%%%%%%%%%%%%%%%%%%%%%%%%%%%%%%%%%%%%%%%%%%%%%%%%%%%%%%%%%%%%%%%%%%%%%%%%%%%%%%%%%%%%%%%%%%%%%%%%%%%%%%%%%%%%%%%%
\graphicspath{ {img/}}
%%%%%%%%%%%%%%%%%%%%%%%%%%%%%%%%%%%%%%%%%%%%%%%%%%%%%%%%%%%%%%%%%%%%%%%%%%%%%%%%%%%%%%%%%%%%%%%%%%%%%%%%%%%%%%%%%%%%%
%\renewcommand{\cite}[1]{[#1]}
%\makeatletter
%\@addtoreset{equation}{section}
%\makeatother
%\renewcommand{\theequation}{\arabic{section}.\arabic{equation}}
\renewcommand{\contentsname}{\centering \small \color{blue} Contents}
%\counterwithin{figure}{section}
\renewcommand{\figurename}{\textbf{Fig.}}
%\renewcommand{\refname}{\textbf{\kaishu 参考文献}}
\renewcommand{\refname}{\textbf{Bibliography}}
\setcounter{secnumdepth}{4}
\titleformat{\paragraph}
{\normalfont\normalsize\bfseries}{\theparagraph}{1em}{}
\titlespacing*{\paragraph}{0pt}{3.25ex plus 1ex minus .2ex}{1.5ex plus .2ex}
\def\beginrefs{\begin{list}%
		{[\arabic{equation}]}{\usecounter{equation}
			\setlength{\leftmargin}{0.8truecm}\setlength{\labelsep}{0.4truecm}%
			\setlength{\labelwidth}{1.6truecm}}}
	\def\endrefs{\end{list}}
\def\bibentry#1{\item[\hbox{[#1]}]}
%%%%%%%%%%%%%%%%%%%%%%%%%%%%%%%%%%%%%%%%%%%%%%%%%%%%%%%%%%%%%%%%%%%%%%%%%%%%%%%%%%%%%%%%%%%%%%%%%%%%%%%%%%%%%%%%%%%%%%
%\begin{figure}[!htb]
%	\centering
%	\subfloat[$A \cap B$]{%
%		\includegraphics[width=0.3\linewidth,height=0.2\linewidth]{img001.jpg}}
%	\label{img001}\qquad \qquad %\hfill
%	\subfloat[${A_1} \cap {A_2} \cap {A_3}$]{%
%		\includegraphics[width=0.3\linewidth,height=0.2\linewidth]{img002.jpg}}
%	\label{img002}
	%\caption{ Examples.}
%\end{figure}
%\begin{figure}[!htb]
%	\centering
%	\includegraphics[width=0.4\linewidth,height=0.3\linewidth]{img005.jpg}
%	\label{img005}
	%\caption{ illustration for $ 3 $}
%\end{figure}
%\={a}1 \'{a}2\v{a}3\.{a}4

\usepackage{datetime}
\renewcommand{\today}{\shortmonthname[\the\month] \the \day,  \the\year}
%%%%%%%%%%%%%%%%%%%%%%%%%%%%%%%%%%%%%%%%%%%%%%%%%%%%%%%%%%%%%%%%%%%%%%%%%%%%%%%%%%%%%%%%%%%%%%%%%%%%%%%%%%%%%%%%%%%%%%
\begin{document}
	\kaishu 
	\thispagestyle{empty}
	\setcounter{section}{0}
	
	\begin{flushright}
		\tiny Final Version \today 
	\end{flushright}
	
	\begin{center}
		{\Large  Note  for NODE Approach to the MQCD Phase Diagram via Holography }
		
		{\large Yao Zhang}
	\end{center}
%%%%%%%%%%%%%%%%%%%%%%%%%%%%%%%%%%%%%%%%%%%%%%%%%%%%%%%%%%%%%%%%%%%%%%%%%%%%%%%%%%%%%%%%%%%%%%%%%%%%%%%%%%%%%%%%%%%%%%
%%\newpage 
%%\thispagestyle{empty}	
%%%%%%%%%%%%%%%%%%%%%%%%%%%%%%%%%%%%%%%%%%%%%%%%%%%%%%%%%%%%%%%%%%%%%%%%%%%%%%%%%%%%%%%%%%%%%%%%%%%%%%%%%%%%%%%%%%%%%%
%\tableofcontents	
%{\pagestyle{empty}\mbox{}\newpage\pagestyle{empty}}
%\newpage 
%{\pagestyle{empty}\mbox{}\newpage\pagestyle{empty}}
%%%%%%%%%%%%%%%%%%%%%%%%%%%%%%%%%%%%%%%%%%%%%%%%%%%%%%%%%%%%%%%%%%%%%%%%%%%%%%%%%%%%%%%%%%%%%%%%%%%%%%%%%%%%%%%%%%%%%%
%%\newpage 
\setcounter{page}{1}


\section{Deduction}

Assume a vector field representation of the governing equations:

\begin{equation}
	\begin{split}
		\vec \Theta \left( z \right) = \left[ {\begin{matrix}
			{\Phi \left( z \right)}  \\ 
			{F\left( z \right)}  \\ 
			{\Sigma \left( z \right)}  \\ 
			{A\left( z \right)}  \\ 
			{G\left( z \right)}  \\ 
			\end{matrix} } \right] \triangleq \left[ {\begin{matrix}
			{{u_1}\left( z \right)}  \\ 
			{{u_2}\left( z \right)}  \\ 
			{{u_3}\left( z \right)}  \\ 
			{{u_4}\left( z \right)}  \\ 
			{{u_5}\left( z \right)}  \\ 
			\end{matrix} } \right] \triangleq \vec u\left( z \right).
	\end{split}
\end{equation}

Further assume the system dynamics are governed by:

\begin{equation}
	\begin{split}
		\frac{{d\vec \Theta \left( z \right)}}{{dz}}  = \left[ {\begin{matrix}
			{\frac{d}{{dz}}\Phi \left( z \right)}  \\ 
			{\frac{d}{{dz}}F\left( z \right)}  \\ 
			{\frac{d}{{dz}}\Sigma \left( z \right)}  \\ 
			{\frac{d}{{dz}}A\left( z \right)}  \\ 
			{\frac{d}{{dz}}G\left( z \right)}  \\ 
			\end{matrix} } \right] & \triangleq \frac{{d\vec u}}{{dz}} =\left[ {\begin{matrix}
			{\frac{{d{u_1}\left( z \right)}}{{dz}}}  \\ 
			{\frac{{d{u_2}\left( z \right)}}{{dz}}}  \\ 
			{\frac{{d{u_3}\left( z \right)}}{{dz}}}  \\ 
			{\frac{{d{u_4}\left( z \right)}}{{dz}}}  \\ 
			{\frac{{d{u_5}\left( z \right)}}{{dz}}}  \\ 
			\end{matrix} } \right]  = {\vec f}\left( {\vec u;\widehat Z\left( {z\Phi \left( {z;\vec \xi  } \right)} \right),\frac{{\partial \widehat Z\left( {z\Phi \left( {z;\vec \xi  } \right)} \right)}}{{\partial z\Phi \left( {z;\vec \xi  } \right)}}} \right) \\
		& = {\vec f}\left( {\vec u;{u_6}\left( {x;\vec \xi  } \right),{u_7}\left( {x;\vec \xi  } \right)} \right) = \left[ {\begin{matrix}
			{{f_1}\left( {z;\vec u,{u_6}\left( {x;\vec \xi } \right),\frac{{\partial {u_6}\left( {x;\vec \xi } \right)}}{{\partial x}}} \right)}  \\ 
			{{f_2}\left( {z;\vec u,{u_6}\left( {x;\vec \xi } \right),\frac{{\partial {u_6}\left( {x;\vec \xi } \right)}}{{\partial x}}} \right)}  \\ 
			{{f_3}\left( {z;\vec u,{u_6}\left( {x;\vec \xi } \right),\frac{{\partial {u_6}\left( {x;\vec \xi } \right)}}{{\partial x}}} \right)}  \\ 
			{{f_4}\left( {z;\vec u,{u_6}\left( {x;\vec \xi } \right),\frac{{\partial {u_6}\left( {x;\vec \xi } \right)}}{{\partial x}}} \right)}  \\ 
			{{f_5}\left( {z;\vec u,{u_6}\left( {x;\vec \xi } \right),\frac{{\partial {u_6}\left( {x;\vec \xi } \right)}}{{\partial x}}} \right)}  \\ 
		\end{matrix} } \right], 
	\end{split}
\end{equation}

where ${u_6}\left( {x;\vec \xi  } \right) = \widehat Z\left( {z\Phi \left( z \right);\vec \xi  } \right)$ with $ x = z\Phi \left( z \right)$, {\color{blue} under the assumption that $ \frac{{\partial {u_6}\left( {x;\vec \xi } \right)}}{{\partial x}} = {u_7}\left( {x;\vec \xi } \right) $}. 


The constrained optimization problem seeks to minimize:

\begin{equation}
	\mathop {\min }\limits_{\vec \xi } J\left( {\vec u\left( z \right)}; \vec{\xi} \right),\;\;s.t.\;\;\vec f\left( {z,\vec u\left( z \right);{u_6}\left( {x;\vec \xi  } \right),{u_7}\left( {x;\vec \xi  } \right)} \right) - \frac{{d\vec u}}{{dz}} = \vec 0,
\end{equation}

with the loss functional defined as:

\begin{equation}
	J\left( {\vec u} \right) = \int_{{z_0}}^{{z_T}} {g\left( \vec u \right)dz}  + {J_1}\left( {\vec u \left( {{z_T}} \right)} \right) .
\end{equation}


The corresponding Lagrangian functional is constructed as:

\begin{equation}
	\begin{split}
		L\left( {\vec u,\vec \lambda  } \right) = & {J_1}\left( {\vec u\left( {{z_T}} \right)} \right) + \int_{{z_0}}^{{z_T}} {g\left( {\vec u} \right)dz}  + \int_{{z_0}}^{{z_T}} {{{\vec \lambda }^T}\left( {\vec f\left( {z;\vec u,{u_6},{u_7}} \right) - \frac{{d\vec u}}{{dz}}} \right)dz}    \\
		= & {J_1}\left( {\vec u\left( {{z_T}} \right)} \right) + \int_{{z_0}}^{{z_T}} {\left[ {g\left( {\vec u} \right) + {{\vec \lambda }^T}\left( {\vec f\left( {z;\vec u,{u_6},{u_7}} \right) - \frac{{d\vec u}}{{dz}}} \right)} \right]dz} ,
	\end{split}
\end{equation}

Taking the total derivative with respect to the parameter vector  $\vec \xi$ yields:

\begin{equation}
	\frac{{dL}}{{d\vec \xi }} = {\left. {\frac{{\partial {J_1}}}{{\partial \vec u}}\frac{{d\vec u}}{{d\vec \xi }}} \right|_{z = {z_T}}} + \int_{{z_0}}^{{z_T}} {\left[ {\frac{{\partial g}}{{\partial \vec u}}\frac{{d\vec u}}{{d\vec \xi }} + {{\vec \lambda }^T}\left( {\frac{{\partial \vec f}}{{\partial \vec u}}\frac{{d\vec u}}{{d\vec \xi }} + \frac{{\partial \vec f}}{{\partial {u_6}}}\frac{{d{u_6}}}{{d\vec \xi }} + \frac{{\partial \vec f}}{{\partial {u_7}}}\frac{{d{u_7}}}{{d\vec \xi }} - \frac{d}{{d\vec \xi }}\left( {\frac{{d\vec u}}{{dz}}} \right)} \right)} \right]dz},
	\label{eq006}
\end{equation}

where $\frac{{\partial \vec f }}{{\partial \vec u }} = \left[ {\begin{matrix}
		{\frac{{\partial \vec f }}{{\partial {u_1}}}} & {\frac{{\partial \vec f }}{{\partial {u_2}}}} & {\frac{{\partial \vec f }}{{\partial {u_3}}}} & {\frac{{\partial \vec f }}{{\partial {u_4}}}} & {\frac{{\partial \vec f }}{{\partial {u_5}}}}  \\ 
\end{matrix} } \right]$, $\frac{{\partial {u_6}}}{{\partial \vec \xi  }} = {\left[ {\begin{matrix}
			{\frac{{\partial {u_6}}}{{\partial {\xi _1}}}} & {\frac{{\partial {u_6}}}{{\partial {\xi _2}}}} &  \cdots  &  \cdots  & {\frac{{\partial {u_6}}}{{\partial {\xi _p}}}}  \\ 
	\end{matrix} } \right]^T}$ and other terms follow analogously.

Applying integration by parts to the term ${\frac{d}{{d\vec \xi }}\left( {\frac{{d\vec u}}{{dz}}} \right)}$ :

\begin{equation}
	\begin{split}
		&\int_{{z_0}}^{{z_T}} { - {{\vec \lambda }^T}\frac{d}{{d\overrightarrow \xi  }}\left( {\frac{{d\vec u}}{{dz}}} \right)dz}  = \int_{{z_0}}^{{z_T}} { - {{\vec \lambda }^T}\frac{d}{{dz}}\left( {\frac{{d\vec u}}{{d\overrightarrow \xi  }}} \right)dz}  =  - \left. {\left[ {{{\vec \lambda }^T}\frac{{d\vec u}}{{d\overrightarrow \xi  }}} \right]} \right|_{{z_0}}^{{z_T}} + \int_{{z_0}}^{{z_T}} {{{\left( {\frac{{d\vec \lambda }}{{dz}}} \right)}^T}\frac{{d\vec u}}{{d\overrightarrow \xi  }}dz} \\
		= & {\left( {\vec \lambda \left( {{z_0}} \right)} \right)^T}{\left. {\frac{{d\vec u}}{{d\vec \xi }}} \right|_{z = {z_0}}} - {\left( {\vec \lambda \left( {{z_T}} \right)} \right)^T}{\left. {\frac{{d\vec u}}{{d\vec \xi }}} \right|_{z = {z_T}}} + \int_{{z_0}}^{{z_T}} {{{\left( {\frac{{d\vec \lambda }}{{dz}}} \right)}^T}\frac{{d\vec u}}{{d\vec \xi }}dz} =  - {\left( {\vec \lambda \left( {{z_T}} \right)} \right)^T}{\left. {\frac{{d\vec u}}{{d\vec \xi }}} \right|_{z = {z_T}}} + \int_{{z_0}}^{{z_T}} {{{\left( {\frac{{d\vec \lambda }}{{dz}}} \right)}^T}\frac{{d\vec u}}{{d\vec \xi }}dz} .
	\end{split}
	\label{eq007}
\end{equation}

where we assume ${\left. {\frac{{d\vec u}}{{d\vec \xi }}} \right|_{z = {z_0}}} = \vec 0$ due to initial condition independence. 

Substituting equation \eqref{eq007} into \eqref{eq006} yields:

\begin{equation}
	\begin{split}
		& \frac{{dL}}{{d\vec \xi }} = {\left. {\frac{{\partial {J_1}}}{{\partial \vec u}}\frac{{d\vec u}}{{d\vec \xi }}} \right|_{z = {z_T}}} + \int_{{z_0}}^{{z_T}} {\left[ {\frac{{\partial g}}{{\partial \vec u}}\frac{{d\vec u}}{{d\vec \xi }} + {{\vec \lambda }^T}\left( {\frac{{\partial \vec f}}{{\partial \vec u}}\frac{{d\vec u}}{{d\vec \xi }} + \frac{{\partial \vec f}}{{\partial {u_6}}}\frac{{\partial {u_6}}}{{\partial \vec \xi }} + \frac{{\partial \vec f}}{{\partial {u_7}}}\frac{{\partial {u_7}}}{{\partial \vec \xi }}} \right) + {{\left( {\frac{{d\vec \lambda }}{{dz}}} \right)}^T}\frac{{d\vec u}}{{d\vec \xi }}} \right]dz}  - {\left( {\vec \lambda \left( {{z_T}} \right)} \right)^T}{\left. {\frac{{d\vec u}}{{d\vec \xi }}} \right|_{z = {z_T}}}\\
		& = \int_{{z_0}}^{{z_T}} {\left[ {\left[ {\frac{{\partial g}}{{\partial \vec u}} + {{\vec \lambda }^T}\frac{{\partial \vec f}}{{\partial \vec u}} + {{\left( {\frac{{d\vec \lambda }}{{dz}}} \right)}^T}} \right]\frac{{d\vec u}}{{d\vec \xi }} + {{\vec \lambda }^T}\left( {\frac{{\partial \vec f}}{{\partial {u_6}}}\frac{{\partial {u_6}}}{{\partial \vec \xi }} + \frac{{\partial \vec f}}{{\partial {u_7}}}\frac{{\partial {u_7}}}{{\partial \vec \xi }}} \right)} \right]dz}  + \left( {\frac{{\partial {J_1}}}{{\partial \vec u}} - {{\left( {\vec \lambda \left( {{z_T}} \right)} \right)}^T}} \right){\left. {\frac{{d\vec u}}{{d\vec \xi }}} \right|_{z = {z_T}}},
	\end{split}
	\label{eq008}
\end{equation}

To address the computational challenges posed by $\frac{{d\vec u}}{{d\vec \xi }}$ and ${\left. {\frac{{d\vec u}}{{d\vec \xi }}} \right|_{z = {z_T}}}$ in \eqref{eq008}, we enforce the adjoint system:

\begin{equation}
	\left\{ \begin{matrix} 
		\frac{{\partial g}}{{\partial \vec u}} + {{\vec \lambda }^T}\frac{{\partial \vec f}}{{\partial \vec u}} + {\left( {\frac{{d\vec \lambda }}{{dz}}} \right)^T} = \vec 0 , \hfill \cr 
		\vec \lambda \left( {{z_T}} \right) = {\frac{{\partial {J_1}}}{{\partial \vec u}}} \left( {{z_T}} \right) . \hfill \cr 
	\end{matrix}  \right.
\end{equation}


It yields the simplified sensitivity expression through the equivalence $\frac{{dL}}{{d\vec \xi }} = \frac{{dJ}}{{d\vec \xi }}$:

\begin{equation}
	\frac{{dJ}}{{d\vec \xi }} = \int_{{z_0}}^{{z_T}} {{{\vec \lambda }^T}\left( {\frac{{\partial \vec f}}{{\partial {u_6}}}\frac{{\partial {u_6}}}{{\partial \vec \xi }} + \frac{{\partial \vec f}}{{\partial {u_7}}}\frac{{\partial {u_7}}}{{\partial \vec \xi }}} \right)dz} . 
\end{equation}

In summary:

\begin{enumerate}
	\item  Initial value problem: Solve forward 
	\begin{equation}
		\left\{ \begin{matrix} 
			\frac{{d{\vec u}}}{{dz}} = {f}\left( {z;\vec u,{u_6}\left( {x;\vec \xi } \right), {u_7}\left( {x;\vec \xi } \right)}\right) \hfill \cr 
			{\vec u}\left( {z = 1} \right) = \vec u_0 \hfill \cr 
		\end{matrix}  \right.
		\label{eq0011}
	\end{equation}
		
	This system can be expanded as:
		
	\begin{equation}
		\left\{
		\begin{aligned}
			& \begin{bmatrix}
				\frac{{du_1}}{{dz}} \\
				\frac{{du_2}}{{dz}} \\
				\frac{{du_3}}{{dz}} \\
				\frac{{du_4}}{{dz}} \\
				\frac{{du_5}}{{dz}} \\
			\end{bmatrix} = \begin{bmatrix}
				f_1\left( {z;\vec u,{u_6}\left( {x;\vec \xi } \right), {u_7}\left( {x;\vec \xi } \right)} \right) \\
				f_2\left( {z;\vec u,{u_6}\left( {x;\vec \xi } \right), {u_7}\left( {x;\vec \xi } \right)} \right) \\
				f_3\left( {z;\vec u,{u_6}\left( {x;\vec \xi } \right), {u_7}\left( {x;\vec \xi } \right)} \right) \\
				f_4\left( {z;\vec u,{u_6}\left( {x;\vec \xi } \right), {u_7}\left( {x;\vec \xi } \right)} \right) \\
				f_5\left( {z;\vec u,{u_6}\left( {x;\vec \xi } \right), {u_7}\left( {x;\vec \xi } \right)} \right) \\
			\end{bmatrix}, \\
			& \begin{bmatrix}
					u_1\left( {z = 1} \right) \\
				u_2\left( {z = 1} \right) \\
				u_3\left( {z = 1} \right) \\
				u_4\left( {z = 1} \right) \\
				u_5\left( {z = 1} \right) \\
			\end{bmatrix} = \begin{bmatrix}
				u_{10} \\
				u_{20} \\
				u_{30} \\
				u_{40} \\
				u_{50} \\
			\end{bmatrix}.
		\end{aligned}
		\right.
	\end{equation}
	This system is solved by determining $u_{1}$ through $u_{5}$ via NDSolve, with $u_{6}$ and $u_{7}$ are provided by the neural network.
	
	\item Terminal value problem: Solve backward
	\begin{equation}
		\left\{ {\begin{matrix}
				\begin{matrix} 
					\frac{{\partial g}}{{\partial \vec u}} + {{\vec \lambda }^T}\frac{{\partial \vec f}}{{\partial \vec u}} + {\left( {\frac{{d\vec \lambda }}{{dz}}} \right)^T} = \vec 0, \hfill \cr 
					\vec \lambda \left( {{z= z_T = 0}} \right) = \frac{{\partial {J_1}}}{{\partial \vec u}}\left( {{z_T}} \right), \hfill \cr 
				\end{matrix}   \\ 
		\end{matrix} } \right.
		\label{eq0012}
	\end{equation}
	for $\vec \lambda \left( z \right)$.
	
	Now, in detail, we have the following system of equations:
	
	\begin{equation}
		\begin{split}
			& \left[ {\begin{matrix}
					{\frac{{\partial g}}{{\partial {u_1}}}} & {\frac{{\partial g}}{{\partial {u_2}}}} & {\frac{{\partial g}}{{\partial {u_3}}}} & {\frac{{\partial g}}{{\partial {u_4}}}} & {\frac{{\partial g}}{{\partial {u_5}}}}  \\ 
			\end{matrix} } \right] + \left[ {\begin{matrix}
					{{\lambda _1}} & {{\lambda _2}} & {{\lambda _3}} & {{\lambda _4}} & {{\lambda _5}}   \\ 
			\end{matrix} } \right]\left[ {\begin{matrix}
					{\frac{{\partial {f_1}}}{{\partial {u_1}}}} & {\frac{{\partial {f_1}}}{{\partial {u_2}}}} & {\frac{{\partial {f_1}}}{{\partial {u_3}}}} & {\frac{{\partial {f_1}}}{{\partial {u_4}}}} & {\frac{{\partial {f_1}}}{{\partial {u_5}}}} \\ 
					{\frac{{\partial {f_2}}}{{\partial {u_1}}}} & {\frac{{\partial {f_2}}}{{\partial {u_2}}}} & {\frac{{\partial {f_2}}}{{\partial {u_3}}}} & {\frac{{\partial {f_2}}}{{\partial {u_4}}}} & {\frac{{\partial {f_2}}}{{\partial {u_5}}}}  \\ 
					{\frac{{\partial {f_3}}}{{\partial {u_1}}}} & {\frac{{\partial {f_3}}}{{\partial {u_2}}}} & {\frac{{\partial {f_3}}}{{\partial {u_3}}}} & {\frac{{\partial {f_3}}}{{\partial {u_4}}}} & {\frac{{\partial {f_3}}}{{\partial {u_5}}}}  \\ 
					{\frac{{\partial {f_4}}}{{\partial {u_1}}}} & {\frac{{\partial {f_4}}}{{\partial {u_2}}}} & {\frac{{\partial {f_4}}}{{\partial {u_3}}}} & {\frac{{\partial {f_4}}}{{\partial {u_4}}}} & {\frac{{\partial {f_4}}}{{\partial {u_5}}}}  \\ 
					{\frac{{\partial {f_5}}}{{\partial {u_1}}}} & {\frac{{\partial {f_5}}}{{\partial {u_2}}}} & {\frac{{\partial {f_5}}}{{\partial {u_3}}}} & {\frac{{\partial {f_5}}}{{\partial {u_4}}}} & {\frac{{\partial {f_5}}}{{\partial {u_5}}}}  \\  
			\end{matrix} } \right] + \cdots \\
			& \cdots + \left[ {\begin{matrix}
					{\frac{{\partial {\lambda _1}}}{{\partial z}}} & {\frac{{\partial {\lambda _2}}}{{\partial z}}} & {\frac{{\partial {\lambda _3}}}{{\partial z}}} & {\frac{{\partial {\lambda _4}}}{{\partial z}}} & {\frac{{\partial {\lambda _5}}}{{\partial z}}}  \\ 
			\end{matrix} } \right]= \left[ {\begin{matrix}
					0 & 0 & 0 & 0 & 0  \\ 
			\end{matrix} } \right]
		\end{split}
	\end{equation}
	
	That is,  
	
	\begin{equation}
		\begin{split}
			& \frac{{\partial g}}{{\partial {u_1}}} + \left[ {{\lambda _1}\frac{{\partial {f_1}}}{{\partial {u_1}}} + {\lambda _2}\frac{{\partial {f_2}}}{{\partial {u_1}}} +  \cdots  + {\lambda _5}\frac{{\partial {f_5}}}{{\partial {u_1}}}} \right] + \frac{{\partial {\lambda _1}}}{{\partial z}} = 0,\\
			& \frac{{\partial g}}{{\partial {u_2}}} + \left[ {{\lambda _1}\frac{{\partial {f_1}}}{{\partial {u_2}}} + {\lambda _2}\frac{{\partial {f_2}}}{{\partial {u_2}}} +  \cdots  + {\lambda _5}\frac{{\partial {f_5}}}{{\partial {u_2}}}} \right] + \frac{{\partial {\lambda _2}}}{{\partial z}} = 0,\\
			& \vdots \\
			& \frac{{\partial g}}{{\partial {u_7}}} + \left[ {{\lambda _1}\frac{{\partial {f_1}}}{{\partial {u_5}}} + {\lambda _2}\frac{{\partial {f_2}}}{{\partial {u_5}}} +  \cdots  + {\lambda _7}\frac{{\partial {f_7}}}{{\partial {u_5}}}} \right] + \frac{{\partial {\lambda _5}}}{{\partial z}} = 0.
		\end{split}
	\end{equation}
	
	In other words, we can express the system as:
	
	\begin{equation}
		\frac{{\partial g}}{{\partial {u_i}}} + \sum\limits_{j = 1}^5 {{\lambda _j}\frac{{\partial {f_j}}}{{\partial {u_i}}}}  + \frac{{\partial {\lambda _i}}}{{\partial z}} = 0,\ \ i \in \left[ {1,5} \right].
	\end{equation}
	
	Finally, the system in \eqref{eq0012} becomes:
	
	\begin{equation}
		\left\{
		\begin{aligned}
			& \frac{{\partial g}}{{\partial {u_i}}} + \sum\limits_{j = 1}^7 {{\lambda _j}\frac{{\partial {f_j}}}{{\partial {u_i}}}}  + \frac{{\partial {\lambda _i}}}{{\partial z}} = 0, \\
			& {\lambda _i}\left( {z = 0} \right) = \frac{{\partial {J_1}}}{{\partial {u_i}}}, \\
		\end{aligned}
		\right.
		\quad i \in \left[ {1,7} \right].
	\end{equation}
	
	\item Evaluate 
	\begin{equation}
		\begin{split}
			\frac{{dJ}}{{d\vec \xi }} & = \int_{{z_0}}^{{z_T}} {{{\vec \lambda }^T}\left( {\frac{{\partial \vec f}}{{\partial {u_6}}}\frac{{\partial {u_6}}}{{\partial \vec \xi }} + \frac{{\partial \vec f}}{{\partial {u_7}}}\frac{{\partial {u_7}}}{{\partial \vec \xi }}} \right)dz} \\
			& = \int_1^0 {\left[ {\begin{matrix}
						{{\lambda _1}} & {{\lambda _2}} & {{\lambda _3}} & {{\lambda _4}} & {{\lambda _5}} \\ 
				\end{matrix} } \right]\left( {\left[ {\begin{matrix}
							{\frac{{\partial {f_1}}}{{\partial {u_6}}}}  \\ 
							{\frac{{\partial {f_2}}}{{\partial {u_6}}}}  \\ 
							{\frac{{\partial {f_3}}}{{\partial {u_6}}}}  \\ 
							{\frac{{\partial {f_4}}}{{\partial {u_6}}}}  \\ 
							{\frac{{\partial {f_5}}}{{\partial {u_6}}}}  \\ 
					\end{matrix} } \right]\frac{{\partial {u_6}}}{{\partial \vec \xi }} + \left[ {\begin{matrix}
					{\frac{{\partial {f_1}}}{{\partial {u_7}\left( {x;\vec \xi } \right)}}}  \\ 
					{\frac{{\partial {f_2}}}{{\partial {u_7}\left( {x;\vec \xi } \right)}}}  \\ 
					{\frac{{\partial {f_3}}}{{\partial {u_7}\left( {x;\vec \xi } \right)}}}  \\ 
					{\frac{{\partial {f_4}}}{{\partial {u_7}\left( {x;\vec \xi } \right)}}}  \\ 
					{\frac{{\partial {f_5}}}{{\partial {u_7}\left( {x;\vec \xi } \right)}}}  \\ 
					\end{matrix} } \right]\frac{{\partial {u_7}\left( {x;\vec \xi } \right)}}{{\partial \vec \xi }}} \right)dz} \\
			& = \int_1^0 {\left[ {\left( {\sum\limits_{i = 1}^5 {{\lambda _i}\frac{{\partial {f_i}}}{{\partial {u_6}}}} } \right)\frac{{\partial {u_6}}}{{\partial \vec \xi }} + \left( {\sum\limits_{i = 1}^5 {{\lambda _i}\frac{{\partial {f_i}}}{{\partial {u_7}\left( {x;\vec \xi } \right)}}} } \right)\frac{{\partial {u_7}\left( {x;\vec \xi } \right)}}{{\partial \vec \xi }}} \right]dz} \\
			& = \int_1^0 {\left[ {\left( {\sum\limits_{j = 1}^5 {{\lambda _j}\frac{{\partial {f_j}}}{{\partial {u_6}}}} } \right)\frac{{\partial {u_6}}}{{\partial \vec \xi }} + \left( {\sum\limits_{j = 1}^5 {{\lambda _j}\frac{{\partial {f_j}}}{{\partial {u_7}\left( {x;\vec \xi } \right)}}} } \right)\frac{{\partial {u_7}\left( {x;\vec \xi } \right)}}{{\partial \vec \xi }}} \right]dz} .
		\end{split}
		\label{eq0013} 
	\end{equation}
\end{enumerate}


\section{Implement}
%
%\begin{figure}[!htb]
%	\centering
%	\includegraphics[width=0.99\linewidth]{derivation.png}
%	\label{img005}
%\caption{Derivation}
%\end{figure}
%
To solve this problem using a system of first-order differential equations, we introduce auxiliary variables $\Phi_2(z)$ and $G_2(z)$, defined as $\Phi_2(z) = \Phi^{\prime}(z)$ and $G_2(z) = G^{\prime}(z)$, respectively. Consequently, $\vec{\Theta} \left( {z} \right)$ can be expressed as:

\begin{equation}
	\begin{split}
		\vec \Theta \left( z \right) = \left[ {\begin{matrix}
				{{\Phi _2}\left( z \right)}  \\ 
				{{G_2}\left( z \right)}  \\ 
				{\Phi \left( z \right)}  \\ 
				{G\left( z \right)}  \\ 
				{\Sigma \left( z \right)}  \\ 
				{F\left( z \right)}  \\  
		\end{matrix} } \right] = \left[ {\begin{matrix}
				{{u_1}\left( z \right)}  \\ 
				{{u_2}\left( z \right)}  \\ 
				{{u_3}\left( z \right)}  \\ 
				{{u_4}\left( z \right)}  \\ 
				{{u_5}\left( z \right)}  \\ 
				{{u_6}\left( z \right)}  \\ 
		\end{matrix} } \right] = \vec u\left( z\right),
	\end{split}
\end{equation}

with the system dynamics: 

\begin{equation}
	\begin{split}
		\frac{{d\vec \Theta \left( z \right)}}{{dz}} = \left[ {\begin{matrix}
			{\frac{d}{{dz}}{\Phi _2}\left( z \right)}  \\ 
			{\frac{d}{{dz}}{G_2}\left( z \right)}  \\ 
			{\frac{d}{{dz}}\Phi \left( z \right)}  \\ 
			{\frac{d}{{dz}}G\left( z \right)}  \\ 
			{\frac{d}{{dz}}\Sigma \left( z \right)}  \\ 
			{\frac{d}{{dz}}F\left( z \right)}  \\ 
			\end{matrix} } \right]{\text{ }} = \left[ {\begin{matrix}
			{\frac{{d{u_1}\left( z \right)}}{{dz}}}  \\ 
			{\frac{{d{u_2}\left( z \right)}}{{dz}}}  \\ 
			{\frac{{d{u_3}\left( z \right)}}{{dz}}}  \\ 
			{\frac{{d{u_4}\left( z \right)}}{{dz}}}  \\ 
			{\frac{{d{u_5}\left( z \right)}}{{dz}}}  \\ 
			{\frac{{d{u_6}\left( z \right)}}{{dz}}}  \\ 
			\end{matrix} } \right] = \vec f\left( {\vec u;{u_7}\left( {x;\vec \xi } \right),{u_8}\left( {x;\vec \xi } \right)} \right) = \left[ {\begin{matrix}
			{{f_1}\left( {z;\vec u,{u_7}\left( {x;\vec \xi } \right),{u_8}\left( {x;\vec \xi } \right)} \right)}  \\ 
			{{f_2}\left( {z;\vec u,{u_7}\left( {x;\vec \xi } \right),{u_8}\left( {x;\vec \xi } \right)} \right)}  \\ 
			{{f_3}\left( {z;\vec u,{u_7}\left( {x;\vec \xi } \right),{u_8}\left( {x;\vec \xi } \right)} \right)}  \\ 
			{{f_4}\left( {z;\vec u,{u_7}\left( {x;\vec \xi } \right),{u_8}\left( {x;\vec \xi } \right)} \right)}  \\ 
			{{f_5}\left( {z;\vec u,{u_7}\left( {x;\vec \xi } \right),{u_8}\left( {x;\vec \xi } \right)} \right)}  \\ 
			{{f_6}\left( {z;\vec u,{u_7}\left( {x;\vec \xi } \right),{u_8}\left( {x;\vec \xi } \right)} \right)}  \\ 
			\end{matrix} } \right],
	\end{split}
\end{equation}

where ${u_7}\left( {x;\vec \xi  } \right) = \widehat Z\left( {z\Phi \left( z \right);\vec \xi  } \right)$ with $ x = z\Phi \left( z \right)$, and {\color{blue} assume that $ \frac{{\partial {u_7}\left( {x;\vec \xi } \right)}}{{\partial x}} = {u_8}\left( {x;\vec \xi } \right) $}. 

\begin{enumerate}
	\item  Solve forward 
		\begin{equation}
			\begin{bmatrix}
			\frac{{du_1}}{{dz}} \\
			\frac{{du_2}}{{dz}} \\
			\frac{{du_3}}{{dz}} \\
			\frac{{du_4}}{{dz}} \\
			\frac{{du_5}}{{dz}} \\
			\frac{{du_6}}{{dz}} \\
		    \end{bmatrix} = \begin{bmatrix}
				f_1\left( {z;\vec u,{u_7}\left( {x;\vec \xi } \right), {u_8}\left( {x;\vec \xi } \right) }\right)\\
				f_2\left( {z;\vec u,{u_7}\left( {x;\vec \xi } \right), {u_8}\left( {x;\vec \xi } \right) }\right)\\
				f_3\left( {z;\vec u,{u_7}\left( {x;\vec \xi } \right), {u_8}\left( {x;\vec \xi } \right) }\right)\\
				f_4\left( {z;\vec u,{u_7}\left( {x;\vec \xi } \right), {u_8}\left( {x;\vec \xi } \right) }\right)\\
				f_5\left( {z;\vec u,{u_7}\left( {x;\vec \xi } \right), {u_8}\left( {x;\vec \xi } \right) }\right)\\
				f_6\left( {z;\vec u,{u_7}\left( {x;\vec \xi } \right), {u_8}\left( {x;\vec \xi } \right) }\right) \\
			\end{bmatrix},
		\label{eq21}
	\end{equation}
	with the initial values:
	\begin{equation}
		\begin{bmatrix}
		u_1\left( {z = 1} \right) \\
		u_2\left( {z = 1} \right) \\
		u_3\left( {z = 1} \right) \\
		u_4\left( {z = 1} \right) \\
		u_5\left( {z = 1} \right) \\
		u_6\left( {z = 1} \right)
		\end{bmatrix} = \begin{bmatrix}
			u_{10} \\
			u_{20} \\
			u_{30} \\
			u_{40} \\
			u_{50} \\
			u_{60}
		\end{bmatrix}.
	\end{equation}
	
	The system will be solved using NDSolve, with $u_{7}$ and $u_{8}$ are provided by the neural network.
	
	
	\item Solve backward
	
	\begin{equation}
		\left\{
		\begin{aligned}
			& \frac{{\partial {\lambda _i}}}{{\partial z}} + \sum\limits_{j = 1}^6 {{\lambda _j}\frac{{\partial {f_j}}}{{\partial {u_i}}}}  + \frac{{\partial g}}{{\partial {u_i}}} = 0, \\
			& {\lambda _i}\left( {z = 0} \right) = \frac{{\partial {J_1}}}{{\partial {u_i}}}, \\
		\end{aligned}
		\right.
		\quad i \in \left[ {1,6} \right].
		\label{eq23}
	\end{equation}
	
	for $\lambda_{i}$, $i \in \left[ {1,6} \right]$.
	
	\item Evaluate 
	\begin{equation}
		\begin{split}
			\frac{{dJ}}{{d\vec \xi }} = \int_1^0 {\left[ {\left( {\sum\limits_{j = 1}^6 {{\lambda _j}\frac{{\partial {f_j}}}{{\partial {u_7}\left( {x;\vec \xi } \right)}}} } \right)\frac{{\partial {u_7}\left( {x;\vec \xi } \right)}}{{\partial \vec \xi }} + \left( {\sum\limits_{j = 1}^6 {{\lambda _j}\frac{{\partial {f_j}}}{{\partial {u_8}\left( {x;\vec \xi } \right)}}} } \right)\frac{{\partial {u_8}\left( {x;\vec \xi } \right)}}{{\partial \vec \xi }}} \right]dz}. 
		\end{split}
		\label{eq24}
	\end{equation}

\end{enumerate}
	
	\begin{figure}[!htb]
		\centering
		\includegraphics[width=0.99\linewidth]{eq11.png}
		\caption{Incomplete codes for \eqref{eq21}.}
	\end{figure}
%
\begin{figure}[!htb]
	\centering
	\includegraphics[width=0.99\linewidth]{eq12.png}
	\caption{Incomplete codes for \eqref{eq23}.}
\end{figure}
%
\begin{figure}[!htb]
	\centering
	\includegraphics[width=0.99\linewidth]{eq13.png}
	\caption{Incomplete codes for \eqref{eq24}.}
\end{figure}

%%%%%%%%%%%%%%%%%%%%%%%%%%%%%%%%%%%%%%%%%%%%%%%%%%%%%%%%%%%%%%%%%%%%%%%%%%%%%%%%%%%%%%%%%%%%%%%%%%%%%%%%%%%%%%%%%%%%%%%
%\bibliographystyle{ieeetr} % number
%%\bibliographystyle{unsrtnat} % author year
%\bibliography{HeBib}
%%%%%%%%%%%%%%%%%%%%%%%%%%%%%%%%%%%%%%%%%%%%%%%%%%%%%%%%%%%%%%%%%%%%%%%%%%%%%%%%%%%%%%%%%%%%%%%%%%%%%%%%%%%%%%%%%%%%%%%
%\begin{flushright}
%	\tiny \today 
%\end{flushright}
%%%%%%%%%%%%%%%%%%%%%%%%%%%%%%%%%%%%%%%%%%%%%%%%%%%%%%%%%%%%%%%%%%%%%%%%%%%%%%%%%%%%%%%%%%%%%%%%%%%%%%%%%%%%%%%%%%%%%%%
\end{document}
%%%%%%%%%%%%%%%%%%%%%%%%%%%%%%%%%%%%%%%%%%%%%%%%%%%%%%%%%%%%%%%%%%%%%%%%%%%%%%%%%%%%%%%%%%%%%%%%%%%%%%%%%%%%%%%%%%%%%%%
              