%%%%%%%%%%%%%%%%%%%%%%%%%%%%%%%%%%%%%%%%%%%%%%%%%%%%%%%%%%%%%%%%%%%%%%%%%%%%%%%%%%%%%%%%%%%%%%%%%%%%%%%%%%%%%%%%%%%%%
%%%%%%%%%%%%%%%%%%%%%%%%%%%%%%%%%%%%%%%%%%%%%   Author:Yao Zhang  %%%%%%%%%%%%%%%%%%%%%%%%%%%%%%%%%%%%%%%%%%%%%%%%%%%
%%%%%%%%%%%%%%%%%%%%%%%%%%%%%%%%%%%%%%%%%%%%% Email: jaafar_zhang@163.com %%%%%%%%%%%%%%%%%%%%%%%%%%%%%%%%%%%%%%%%%%%
%%%%%%%%%%%%%%%%%%%%%%%%%%%%%%%%%%%%%%%%%%%%%%%%%%%%%%%%%%%%%%%%%%%%%%%%%%%%%%%%%%%%%%%%%%%%%%%%%%%%%%%%%%%%%%%%%%%%%
\documentclass[11pt]{article}
\usepackage{babel}
\usepackage[utf8]{inputenc} 
\usepackage[table]{xcolor}
\usepackage[most]{tcolorbox}
\usepackage[left=2.50cm, right=1.50cm, top=2.0cm, bottom=2.50cm]{geometry}
\usepackage{xcolor,url}
\usepackage{amsmath,amsthm,amsfonts,amssymb,amscd,multirow,booktabs,fullpage,calc,multicol}
\usepackage{lastpage,enumitem,fancyhdr,mathrsfs,wrapfig,setspace,cancel,amsmath,empheq,framed}
\usepackage[retainorgcmds]{IEEEtrantools}
\usepackage{subfig,graphicx,framed}
\usepackage{ctex}
\usepackage{txfonts}
\usepackage{bbm}
\usepackage{chngcntr}
\usepackage[colorlinks,linkcolor=blue,anchorcolor=green,citecolor=red,urlcolor=blue]{hyperref}
\usepackage{titlesec}
%%%%%%%%%%%%%%%%%%%%%%%%%%%%%%%%%%%%%%%%%%%%%%%%%%%%%%%%%%%%%%%%%%%%%%%%%%%%%%%%%%%%%%%%%%%%%%%%%%%%%%%%%%%%%%%%%%%%%%
\newtheorem{thm}{Theorem}[section]
\newtheorem{defi}{Definition}[subsection]
\newtheorem{exercise}{Exercise}[subsection]
\newtheorem{note}{Note}[subsection]
\newtheorem{notation}{Notation}
\newtheorem{lemma}{Lemma}[subsection]
\newtheorem{proposition}{Proposition}[subsection]
\newtheorem{example}{Example}[subsection]
\newtheorem{problem}{Problem}[section]
\newtheorem{homework}{Homework}[section]
\newtheorem{summary}{Summary}[subsection]
\newtheorem{corollary}{Corollary}[subsection]
\newtheorem{rmk}{Remark}[section]
\usepackage{romannum}
%%%%%%%%%%%%%%%%%%%%%%%%%%%%%%%%%%%%%%%%%%%%%%%%%%%%%%%%%%%%%%%%%%%%%%%%%%%%%%%%%%%%%%%%%%%%%%%%%%%%%%%%%%%%%%%%%%%%%
\newlength{\tabcont}
\setlength{\parindent}{0.0in}
\setlength{\parskip}{0.05in}
\colorlet{shadecolor}{orange!15}
\parindent 0in
\parskip 12pt
\geometry{margin=1in, headsep=0.25in}
%%%%%%%%%%%%%%%%%%%%%%%%%%%%%%%%%%%%%%%%%%%%%%%%%%%%%%%%%%%%%%%%%%%%%%%%%%%%%%%%%%%%%%%%%%%%%%%%%%%%%%%%%%%%%%%%%%%%%
\graphicspath{ {img/EoM/}}
%%%%%%%%%%%%%%%%%%%%%%%%%%%%%%%%%%%%%%%%%%%%%%%%%%%%%%%%%%%%%%%%%%%%%%%%%%%%%%%%%%%%%%%%%%%%%%%%%%%%%%%%%%%%%%%%%%%%%
%\renewcommand{\cite}[1]{[#1]}
\makeatletter
\@addtoreset{equation}{section}
\makeatother
\renewcommand{\theequation}{\arabic{section}.\arabic{equation}}
\renewcommand{\contentsname}{\centering \small \color{blue} Contents}
%\counterwithin{figure}{section}
\renewcommand{\figurename}{\textbf{Fig.}}
%\renewcommand{\refname}{\textbf{\kaishu 参考文献}}
\renewcommand{\refname}{\textbf{Bibliography}}
\setcounter{secnumdepth}{4}
\titleformat{\paragraph}
{\normalfont\normalsize\bfseries}{\theparagraph}{1em}{}
\titlespacing*{\paragraph}{0pt}{3.25ex plus 1ex minus .2ex}{1.5ex plus .2ex}
\def\beginrefs{\begin{list}%
		{[\arabic{equation}]}{\usecounter{equation}
			\setlength{\leftmargin}{0.8truecm}\setlength{\labelsep}{0.4truecm}%
			\setlength{\labelwidth}{1.6truecm}}}
	\def\endrefs{\end{list}}
\def\bibentry#1{\item[\hbox{[#1]}]}
%%%%%%%%%%%%%%%%%%%%%%%%%%%%%%%%%%%%%%%%%%%%%%%%%%%%%%%%%%%%%%%%%%%%%%%%%%%%%%%%%%%%%%%%%%%%%%%%%%%%%%%%%%%%%%%%%%%%%%
%\begin{figure}[!htb]
%	\centering
%	\subfloat[$A \cap B$]{%
%		\includegraphics[width=0.3\linewidth,height=0.2\linewidth]{img001.jpg}}
%	\label{img001}\qquad \qquad %\hfill
%	\subfloat[${A_1} \cap {A_2} \cap {A_3}$]{%
%		\includegraphics[width=0.3\linewidth,height=0.2\linewidth]{img002.jpg}}
%	\label{img002}
	%\caption{ Examples.}
%\end{figure}
%\begin{figure}[!htb]
%	\centering
%	\includegraphics[width=0.4\linewidth,height=0.3\linewidth]{img005.jpg}
%	\label{img005}
	%\caption{ illustration for $ 3 $}
%\end{figure}
%\={a}1 \'{a}2\v{a}3\.{a}4

\usepackage{datetime}
\renewcommand{\today}{\shortmonthname[\the\month] \the \day,  \the\year}
%%%%%%%%%%%%%%%%%%%%%%%%%%%%%%%%%%%%%%%%%%%%%%%%%%%%%%%%%%%%%%%%%%%%%%%%%%%%%%%%%%%%%%%%%%%%%%%%%%%%%%%%%%%%%%%%%%%%%%
\begin{document}
	\kaishu 
	%\thispagestyle{empty}
	\pagenumbering{arabic} 
	\setcounter{section}{0}
	\begin{center}
		{\LARGE  \href{https://lerman.web.illinois.edu/535/f20/535f20.html}{General Topology}}
		
		%\vspace{-0.25cm}
		
		{\large \href{https://lerman.web.illinois.edu/}{Eugene Lerman}}
	\end{center}
%%%%%%%%%%%%%%%%%%%%%%%%%%%%%%%%%%%%%%%%%%%%%%%%%%%%%%%%%%%%%%%%%%%%%%%%%%%%%%%%%%%%%%%%%%%%%%%%%%%%%%%%%%%%%%%%%%%%%%
%%\newpage 
%%\thispagestyle{empty}	
%%%%%%%%%%%%%%%%%%%%%%%%%%%%%%%%%%%%%%%%%%%%%%%%%%%%%%%%%%%%%%%%%%%%%%%%%%%%%%%%%%%%%%%%%%%%%%%%%%%%%%%%%%%%%%%%%%%%%%
%\tableofcontents	
%{\pagestyle{empty}\mbox{}\newpage\pagestyle{empty}}
%\newpage 
%{\pagestyle{empty}\mbox{}\newpage\pagestyle{empty}}
%%%%%%%%%%%%%%%%%%%%%%%%%%%%%%%%%%%%%%%%%%%%%%%%%%%%%%%%%%%%%%%%%%%%%%%%%%%%%%%%%%%%%%%%%%%%%%%%%%%%%%%%%%%%%%%%%%%%%%
%%\newpage 
\setcounter{page}{1}

%\vspace{1.5cm}


\vspace{-1cm}

\begin{enumerate}
	\item \href{https://mp.weixin.qq.com/s/e_pmSIvauYPGcgV2_EvVFw}{metric spaces, open balls, open sets, notion of a topology, differentmetrics may define the same topology, some topologies cannot come from metrics}	%1
	\item \href{https://mp.weixin.qq.com/s/L6OgzKDD_blNz5FwskpOvw}{closed sets, epsilon-delta definition of continuity in metric spaces, continuity of functions between topological spaces, subspace topology, bases}	%2
	\item \href{https://mp.weixin.qq.com/s/bWjzi_gcGr7sYpb-YIa-eg}{ Composites of continuous maps are continuous, subbases, topology generated by a subset of the power set, Cartesian products of sets}	%3
	\item \href{https://mp.weixin.qq.com/s/sFO53AyDxIvYCVLAtoezfA}{ Products of pairs of topological spaces, homeomorphism, not every continuous bijection is a homeomorphism, uniqueness of product topology}	%4
	\item \href{https://mp.weixin.qq.com/s/_GPtW3xIJm5IUk94xCa5lA}{Products of families of topological spaces and their universal properties, box topology, open and closed maps, coproducts}	%5
	\item \href{https://mp.weixin.qq.com/s/dYNtDxUOyhse8sCc0tMY3A}{Quotient topology, limit points}	%6
	\item \href{https://mp.weixin.qq.com/s/Hcqz3hAf9aO-sx9uNLdqtg}{Closure of a subset, limit points of a subset, convergence of sequences, limits of sequences lie in the closure but not conversely, limits need not be unique}	%7
	\item \href{https://mp.weixin.qq.com/s/T2M8zFK7ikwLGcVZei2Z3g}{ Interior of a subset, boundary. Interaction of closure, interior, boundary and complements. First countable topological spaces}	%8
	\item \href{https://mp.weixin.qq.com/s/uB-yj-nlkjzX0mryoMOgSQ}{Preorders, directed sets, nets, convergence of nets, points in the closure and limits of nets, continuity and convergence of nets, being Hausdorff and uniqueness of limits of nets}	%9
	\item \href{https://mp.weixin.qq.com/s/XJ5Y2ZRm8zaOx-Q5wIq98A}{ Subnets, subnet of a convergent net. Compactness: images of compact sets are compact, closed subsets of compact sets are compact, compact subsets of Hausdorff spaces are closed, if f: $X \rightarrow Y$ is a continuous bijection, X compact, Y Hausdorff, then f is a homeomorphism}	%10
	\item \href{https://mp.weixin.qq.com/s/zhIaVKHrCH9sb3EHq0MCOA}{Bolzano-Weirstrass, tube lemma, products of two compact spaces are compact, a subset of $\mathbb{R}^n$ is compact iff it is closed and bounded; X compact, $X \rightarrow \mathbb{R}$ continuous $\Rightarrow$ f achieves max and min on X}	%11
	\item \href{https://mp.weixin.qq.com/s/M-0XSv7twgSzrO7hqihwgg}{Finite Intersection Property (FIP) and compactness in terms of FIP, cluster/limit/accumilation point of a net, a net has a cluster point iff it has a convergent subnet, a space is compact iff a net has a cluster point iff a net has a convergent subnet}	%12
	\item \href{https://mp.weixin.qq.com/s/cgIR3Q8Z3io88MEpKZGz_w}{Tychonoff's theorem: a product of compact spaces is compact. Lebesque lemma}	%13
	\item \href{https://mp.weixin.qq.com/s/GFdQoU1Tv0QZ0xYhm_P43Q}{ A metric space is compact iff every sequence has a convergent subsequence iff the space is complete and totally bounded. Separation axioms}	%14
	\item \href{https://mp.weixin.qq.com/s/a8DhbB4Ewouvy2Uw3J9Zhg}{A Hausdorff space X is regular iff for every x in X any nbd N of x contains a closed nbd of x. There is a Hausdorff space which is not regular. There are also regular spaces that are not normal. Metric spaces are normal}	%15
	\item \href{https://mp.weixin.qq.com/s/Xi1OTPlVjzUaIFWxSVNnUg}{Compact Hausdorff spaces are normal, completely regular and Tychonoff spaces. Urysohn's lemma}	%16
	\item \href{https://mp.weixin.qq.com/s/g4UmuykkDk8i2bJMRb-H2w}{ $[0,1]^{N}$ is metrizable. Urysohn's metrization theorem: second countable completely regular $T_{1}$ spaces embed in $[0,1]^{N}$ hence are metrizable}	%17
	\item \href{https://mp.weixin.qq.com/s/EJZkPgmo-UedLEtlPfI_7w}{2nd countable + regular $\Rightarrow$ metrizable, Lindelof spaces, 2nd countable $\Rightarrow$ Lindelof. Tietze extension theorem}	%18
	\item \href{https://mp.weixin.qq.com/s/z2LowbgV9tOxOtpFs_gFSg}{Moore plane is not normal. Local compactness. Manifolds. LCH (locally compact Hausdorff). 2nd countable LCH space is normal and metrizable. Compactifications and 1 point compactifications}	%19
	\item \href{https://mp.weixin.qq.com/s/4MUD1kGBMEaxR6YNPG7DAg}{X has a 1 point compactification $\Leftrightarrow$ X is LCH and noncompact. 1 point compactifications are unique. Proper maps. Continuous maps need not extend to continuous maps on 1 point compactifications}	%20
	\item \href{https://mp.weixin.qq.com/s/AXHH_4TEnp1RI6CMa5ThCw}{a map f extends to a continuos map of 1-point compactifications $\Leftrightarrow$ f is proper. Proper continuous maps between LCH spaces are closed. Topological groups, continuous group actions, orbit spaces, proper group actoins}	%21
	\item \href{https://mp.weixin.qq.com/s/uDg064NRu44Li6R28FVc7g}{Quotients of LCH groups acting on LCH spaces are Hausdorff. Notion of connectedness. [0,1] is connected. X is connected $\Leftrightarrow$ any continuous map from X to any discrete space is constant}	%22
	\item \href{https://mp.weixin.qq.com/s/RoclA358UdCYue4nbT71pA}{Connected components of a space are connected and closed. A in X connected and E sits between A and the closure of A then E is connected. Path connected $\Rightarrow$ connected but there are connected spaces that are not path connected}	%23
	\item \href{https://mp.weixin.qq.com/s/W2omCNXIDSlkv3SjoqXV-g}{Path components. Connected and locally path connected spaces are connected. Manifolds are locally path connected. Notions of partition of unity and paracompactness. A compact Hausdorff manifold may be embedded in some $\mathbb{R}^N$}	%24
	\item \href{https://mp.weixin.qq.com/s/aFdVoPlhNMPYKP7nUxS2ow}{sigma-compactness. Locally compact sigma-compact Hausdorff spaces is paracompact. Paracompact spaces are normal}	%25
	\item \href{https://mp.weixin.qq.com/s/03IoaMrFbslVR1FNxnUhww}{Existence of partitions of 1 on a paracompact space. A manifold M is paracompact iff M is a disjoint union of Hausdorff second countable manifolds}	%26
	\item \href{https://mp.weixin.qq.com/s/8jJIs_LhE_UPCym0I-TbFQ}{Homotopy; homotopy classes of maps compose. Notion of a categroy}	%27
	\item \href{https://mp.weixin.qq.com/s/37GlOC0ff5_jXJAvb0LSRw}{Isomorphisms in a category. Homotopy equivalence of spaces. Functors. Groupoids}	%28
	\item \href{https://mp.weixin.qq.com/s/4XQbPq5UWjKf2oD3yMAPVQ}{Construction of the fundamental groupoid of a space. Fundamental groups. Fundamental groupoid of a convex subset of $\mathbb{R}^{n}$ . Pair groupoid}	%29
	\item \href{https://mp.weixin.qq.com/s/5OshKGMWrBzqU2RBjEJ1DQ}{The functor $\Pi$ from spaces to groupoids. Natural transformations. Homotopies give rise to natural transformations}	%30
	\item \href{https://mp.weixin.qq.com/s/wboTxmkWpOYL6gSbEyQ4TQ}{Natural isomorphisms. Equivalent categories. Full, faithful and essentially surjective functors. Equivalences of categories are full, faithful and essentially surjective}	%31
	\item \href{https://mp.weixin.qq.com/s/pmz6I04CYzk1m8z4Qq-mQw}{A fully faithful and essentially surjective functor is part of the equivalences of categories. Pushouts}	%32
	\item \href{https://mp.weixin.qq.com/s/lqoX-KpQ-gzSSxem6gNTMA}{Uniqueness of pushouts. A space is a pushout of its cover. Statement of Brown-Seifert-van Kampen: fundamental groupoid functor takes pushouts in Top to pushouts in Groupoid}	%33
	\item \href{https://mp.weixin.qq.com/s/YT4krGWMkAYgfBoOtHYyFg}{Computation of the fundamental groupoid of the circle and the fundamental group of the circle}	%34
	\item \href{https://mp.weixin.qq.com/s/3SoNZ3pUF5mUY_T4Wg7Y2Q}{Proof of B-S-v K theorem. Free products of groups}	%35
	\item \href{https://mp.weixin.qq.com/s/9O7I8mpcbcWjHDxs8-XcTA}{Pushouts in the category of groups = amalgamated free products. Proof of Seifert - van Kampen from B.-S. v. K}	%36
	\item \href{https://mp.weixin.qq.com/s/TrKmVo0PtMxNBmNLlsDuMg}{Degree of a map from the circle to the circle. Fundamental Theorem of Algebra. Definition of a compact-open topology}	%37
	\item \href{https://mp.weixin.qq.com/s/q1OoRqVMOpQAsaC6jJikIg}{compact-open topology}	%38
	\item \href{https://mp.weixin.qq.com/s/3jOl4G36j-zir5uFdaGfiw}{uniform convergence on compact sets and compact-open topology}	%39
	\item \href{https://mp.weixin.qq.com/s/78e7FVucqSvMq3IQtfQH6w}{Stone-\v{C}ech compactification}	%40
	%\item \href{url}{Materials}
\end{enumerate}

%. [https://pan.baidu.com/s/1fq4i7uCjnB6KbYuqAzk7Rw?pwd=s26k Notes]




%%%%%%%%%%%%%%%%%%%%%%%%%%%%%%%%%%%%%%%%%%%%%%%%%%%%%%%%%%%%%%%%%%%%%%%%%%%%%%%%%%%%%%%%%%%%%%%%%%%%%%%%%%%%%%%%%%%%%%%
%\bibliographystyle{ieeetr} % number
%%\bibliographystyle{unsrtnat} % author year
%\bibliography{HeBib}
%%%%%%%%%%%%%%%%%%%%%%%%%%%%%%%%%%%%%%%%%%%%%%%%%%%%%%%%%%%%%%%%%%%%%%%%%%%%%%%%%%%%%%%%%%%%%%%%%%%%%%%%%%%%%%%%%%%%%%%
\begin{flushright}
	\tiny \today 
\end{flushright}
%%%%%%%%%%%%%%%%%%%%%%%%%%%%%%%%%%%%%%%%%%%%%%%%%%%%%%%%%%%%%%%%%%%%%%%%%%%%%%%%%%%%%%%%%%%%%%%%%%%%%%%%%%%%%%%%%%%%%%%
\end{document}
%%%%%%%%%%%%%%%%%%%%%%%%%%%%%%%%%%%%%%%%%%%%%%%%%%%%%%%%%%%%%%%%%%%%%%%%%%%%%%%%%%%%%%%%%%%%%%%%%%%%%%%%%%%%%%%%%%%%%%%
              