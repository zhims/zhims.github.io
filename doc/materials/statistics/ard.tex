%%%%%%%%%%%%%%%%%%%%%%%%%%%%%%%%%%%%%%%%%%%%%%%%%%%%%%%%%%%%%%%%%%%%%%%%%%%%%%%%%%%%%%%%%%%%%%%%%%%%%%%%%%%%%%%%%%%%%
%%%%%%%%%%%%%%%%%%%%%%%%%%%%%%%%%%%%%%%%%%%%%   Author:Yao Zhang  %%%%%%%%%%%%%%%%%%%%%%%%%%%%%%%%%%%%%%%%%%%%%%%%%%%
%%%%%%%%%%%%%%%%%%%%%%%%%%%%%%%%%%%%%%%%%%%%% Email: jaafar_zhang@163.com %%%%%%%%%%%%%%%%%%%%%%%%%%%%%%%%%%%%%%%%%%%
%%%%%%%%%%%%%%%%%%%%%%%%%%%%%%%%%%%%%%%%%%%%%%%%%%%%%%%%%%%%%%%%%%%%%%%%%%%%%%%%%%%%%%%%%%%%%%%%%%%%%%%%%%%%%%%%%%%%%
\documentclass[11pt]{article}
\usepackage{babel}
\usepackage[utf8]{inputenc} 
\usepackage[table]{xcolor}
\usepackage[most]{tcolorbox}
\usepackage[left=2.50cm, right=1.50cm, top=2.0cm, bottom=2.50cm]{geometry}
\usepackage{xcolor,url}
\usepackage{amsmath,amsthm,amsfonts,amssymb,amscd,multirow,booktabs,fullpage,calc,multicol}
\usepackage{lastpage,enumitem,fancyhdr,mathrsfs,wrapfig,setspace,cancel,amsmath,empheq,framed}
\usepackage[retainorgcmds]{IEEEtrantools}
\usepackage{subfig,graphicx,framed}
\usepackage{ctex}
\usepackage{txfonts}
\usepackage{bbm}
\usepackage{chngcntr}
\usepackage[colorlinks,linkcolor=blue,anchorcolor=green,citecolor=red,urlcolor=blue]{hyperref}
\usepackage{titlesec}
%%%%%%%%%%%%%%%%%%%%%%%%%%%%%%%%%%%%%%%%%%%%%%%%%%%%%%%%%%%%%%%%%%%%%%%%%%%%%%%%%%%%%%%%%%%%%%%%%%%%%%%%%%%%%%%%%%%%%%
\newtheorem{thm}{Theorem}[section]
\newtheorem{defi}{Definition}[subsection]
\newtheorem{exercise}{Exercise}[subsection]
\newtheorem{note}{Note}[subsection]
\newtheorem{notation}{Notation}
\newtheorem{lemma}{Lemma}[subsection]
\newtheorem{proposition}{Proposition}[subsection]
\newtheorem{example}{Example}[subsection]
\newtheorem{problem}{Problem}[section]
\newtheorem{homework}{Homework}[section]
\newtheorem{summary}{Summary}[subsection]
\newtheorem{corollary}{Corollary}[subsection]
\newtheorem{rmk}{Remark}[section]
\usepackage{romannum}
%%%%%%%%%%%%%%%%%%%%%%%%%%%%%%%%%%%%%%%%%%%%%%%%%%%%%%%%%%%%%%%%%%%%%%%%%%%%%%%%%%%%%%%%%%%%%%%%%%%%%%%%%%%%%%%%%%%%%
\newlength{\tabcont}
\setlength{\parindent}{0.0in}
\setlength{\parskip}{0.05in}
\colorlet{shadecolor}{orange!15}
\parindent 0in
\parskip 12pt
\geometry{margin=1in, headsep=0.25in}
%%%%%%%%%%%%%%%%%%%%%%%%%%%%%%%%%%%%%%%%%%%%%%%%%%%%%%%%%%%%%%%%%%%%%%%%%%%%%%%%%%%%%%%%%%%%%%%%%%%%%%%%%%%%%%%%%%%%%
\graphicspath{ {img/EoM/}}
%%%%%%%%%%%%%%%%%%%%%%%%%%%%%%%%%%%%%%%%%%%%%%%%%%%%%%%%%%%%%%%%%%%%%%%%%%%%%%%%%%%%%%%%%%%%%%%%%%%%%%%%%%%%%%%%%%%%%
%\renewcommand{\cite}[1]{[#1]}
\makeatletter
\@addtoreset{equation}{section}
\makeatother
\renewcommand{\theequation}{\arabic{section}.\arabic{equation}}
\renewcommand{\contentsname}{\centering \small \color{blue} Contents}
%\counterwithin{figure}{section}
\renewcommand{\figurename}{\textbf{Fig.}}
%\renewcommand{\refname}{\textbf{\kaishu 参考文献}}
\renewcommand{\refname}{\textbf{Bibliography}}
\setcounter{secnumdepth}{4}
\titleformat{\paragraph}
{\normalfont\normalsize\bfseries}{\theparagraph}{1em}{}
\titlespacing*{\paragraph}{0pt}{3.25ex plus 1ex minus .2ex}{1.5ex plus .2ex}
\def\beginrefs{\begin{list}%
		{[\arabic{equation}]}{\usecounter{equation}
			\setlength{\leftmargin}{0.8truecm}\setlength{\labelsep}{0.4truecm}%
			\setlength{\labelwidth}{1.6truecm}}}
	\def\endrefs{\end{list}}
\def\bibentry#1{\item[\hbox{[#1]}]}
%%%%%%%%%%%%%%%%%%%%%%%%%%%%%%%%%%%%%%%%%%%%%%%%%%%%%%%%%%%%%%%%%%%%%%%%%%%%%%%%%%%%%%%%%%%%%%%%%%%%%%%%%%%%%%%%%%%%%%
%\begin{figure}[!htb]
%	\centering
%	\subfloat[$A \cap B$]{%
%		\includegraphics[width=0.3\linewidth,height=0.2\linewidth]{img001.jpg}}
%	\label{img001}\qquad \qquad %\hfill
%	\subfloat[${A_1} \cap {A_2} \cap {A_3}$]{%
%		\includegraphics[width=0.3\linewidth,height=0.2\linewidth]{img002.jpg}}
%	\label{img002}
	%\caption{ Examples.}
%\end{figure}
%\begin{figure}[!htb]
%	\centering
%	\includegraphics[width=0.4\linewidth,height=0.3\linewidth]{img005.jpg}
%	\label{img005}
	%\caption{ illustration for $ 3 $}
%\end{figure}
%\={a}1 \'{a}2\v{a}3\.{a}4

\usepackage{datetime}
\renewcommand{\today}{\shortmonthname[\the\month] \the \day,  \the\year}
%%%%%%%%%%%%%%%%%%%%%%%%%%%%%%%%%%%%%%%%%%%%%%%%%%%%%%%%%%%%%%%%%%%%%%%%%%%%%%%%%%%%%%%%%%%%%%%%%%%%%%%%%%%%%%%%%%%%%%
\begin{document}
	\kaishu 
	%\thispagestyle{empty}
	\pagenumbering{arabic} 
	\setcounter{section}{0}
	\begin{center}
		{\LARGE  \href{https://mediaspace.illinois.edu/createdby/eyJpdiI6Imp1RGVjNDlKZkFaYVplNlVLVm9kYWc9PSIsInZhbHVlIjoiakUzeXJkbEpydkFBaVV5Yko3T3AxZz09IiwibWFjIjoiOGM0MzIwZWEzYjE2ODFmOTYxNjM1MmNiMjAxYzY5MjMxZWQ2ZjcwZWVkNGRiZDFmMWMwMDNmNTYyNzk4MjJlYyJ9}{Applied Regression and Design}}
		
		%\vspace{-0.25cm}
		
		{\large \href{https://www.lelysbravo.com/}{Lelys Bravo De Guenni}}
	\end{center}
%%%%%%%%%%%%%%%%%%%%%%%%%%%%%%%%%%%%%%%%%%%%%%%%%%%%%%%%%%%%%%%%%%%%%%%%%%%%%%%%%%%%%%%%%%%%%%%%%%%%%%%%%%%%%%%%%%%%%%
%%\newpage 
%%\thispagestyle{empty}	
%%%%%%%%%%%%%%%%%%%%%%%%%%%%%%%%%%%%%%%%%%%%%%%%%%%%%%%%%%%%%%%%%%%%%%%%%%%%%%%%%%%%%%%%%%%%%%%%%%%%%%%%%%%%%%%%%%%%%%
%\tableofcontents	
%{\pagestyle{empty}\mbox{}\newpage\pagestyle{empty}}
%\newpage 
%{\pagestyle{empty}\mbox{}\newpage\pagestyle{empty}}
%%%%%%%%%%%%%%%%%%%%%%%%%%%%%%%%%%%%%%%%%%%%%%%%%%%%%%%%%%%%%%%%%%%%%%%%%%%%%%%%%%%%%%%%%%%%%%%%%%%%%%%%%%%%%%%%%%%%%%
%%\newpage 
\setcounter{page}{1}

%\vspace{1.5cm}


\vspace{-1cm}

\begin{enumerate}
	\item \href{https://mp.weixin.qq.com/s/yUeFyOxG3cUc2goY26DFCg}{Introduction to Regression Analysis: In this video we talk about the fundamental steps of Data Analysis and the origin of Regression Analysis}	%1
	\item \href{https://mp.weixin.qq.com/s/jp2E23_TAq3VJGKZ4Gn_-Q}{Simple Linear Regression 1: In this video we talk about the Least Square methods to find the Simple Linear Regression estimates, and we talk about the R-squared (Goodness of Fit) in SLR}	%2
	\item \href{https://mp.weixin.qq.com/s/qTHAQGuV4OtyqeKVVO3ceA}{Simple Linear Regression 2: In This video we discuss the properties of the LS estimators, and their probability  distribution}	%3
	\item \href{https://mp.weixin.qq.com/s/Di7xXeR3jX62Al9qdSvqzg}{Simple Linear Regression 3: In this video we discuss the F-test and the ANOVA table, and explain the difference between estimation and prediction.}	%4
	\item \href{https://mp.weixin.qq.com/s/I7npJx93v2ZdqTHnkdgb2A}{Multiple Linear Regression 1: In this video we discuss the Multiple Linear Regression; the LS estimation method and the geometrical interpretation of the LS estimates}	%5
	\item \href{https://mp.weixin.qq.com/s/Xuqb-kd7hk1sUuzyv3Ve8g}{Multiple Linear Regression 2: In this video we talk about the properties of the LS estimates for MLR, and the importance of the Gauss -Markov theorem}	%6
	\item \href{https://mp.weixin.qq.com/s/fMI2ZhPIDFQgWkv0qyLfMQ}{Multiple Linear Regression 3: In this video we talk about hypothesis testing in MLR, the F-test and the ANOVA table and how to do nested model comparisons.}	%7
	\item \href{https://mp.weixin.qq.com/s/e4kItE_7S-z0I45mj5QF4w}{Multiple Linear Regression 4: In this video we talk about the Monte Carlo Method and the Permutation test to do hypothesis testing when the Normality assumption does not hold}	%8
	\item \href{https://mp.weixin.qq.com/s/zFm7FAPcdJU_Ygob3Th7lQ}{Multiple Linear Regression 5: In this video we discuss how to calculate confidence intervals for the beta\_i's and confidence regions for a vector or sub-vector of the regression coefficients. We also discuss how to build confidence intervals and prediction intervals for the mean or future prediction of the response at a new observation.}	%9
	\item \href{https://mp.weixin.qq.com/s/aAnAytJTkUK-PhfirlAdcg}{Regression Diagnostics 1: In this video we talk about unusual observations. In particular how to detect high-leverage points. We also discuss the difference between the true residuals and the estimated residuals.}	%10
	\item \href{https://mp.weixin.qq.com/s/N3DIFzT7eOVbMt0uPQqLKA}{Regression Diagnostics 2: In this video we discuss the difference between standardized residuals and studentized residuals. We discuss a t-test for detecting outliers using the Bonferroni correction, and the Cook's Distance measure to detect highly influential observations}	%11
	\item \href{https://mp.weixin.qq.com/s/psVaHQF9JyYyU5GsmBgnGw}{Regression Diagnostics 3: In this videos we discuss how to use the residuals plots to do model diagnostics. We discuss the problem of non-constant variance or heterokedasticity, and how to do formal testing. We also discuss Variance Stabilizing transformations and how to detect departures from normality}	%12
	\item \href{https://mp.weixin.qq.com/s/U3Go8ScPAWUaslVAkkZRcQ}{Regression Diagnostics 4: In this video we discuss how to check the linear model structure using the partial regression plots. We discuss possible linearizing transformations and the Box-Cox as a normalizing transformation.}	%13
	\item \href{https://mp.weixin.qq.com/s/VKRN-Zy7EBezRsIm0AIv-A}{Collinearity: We discuss exact collinearity and approximate collinearity  or multicollinearity. We also explain how to detect approximate multicollinearity, the symptoms and remedies for this problem.}	%14
	\item \href{https://mp.weixin.qq.com/s/MUoc9VhuGCr8VI7PSox56w}{Generalized Least Squares: In this video we  discussed the GLS method in two cases: When the Variance-Covariance matrix is known, and  when the Variance-Covariance matrix is unknown. We also discuss the Weighted Least Squares problem as a particular case.}	%15
	\item \href{https://mp.weixin.qq.com/s/vzhpRJTKTZ_4Ca7h3WfaXw}{Lack of Fit test: In this video we talk about the Lack of Fit test in two cases: When the variance is known, and when the variance is unknown.}	%16
	\item \href{https://mp.weixin.qq.com/s/cuM4OouJhRi8lE-rF8sxTQ}{Polynomial Regression: In this video we talk about polynomial regression and piece-wise polynomials. We also define Cubic Splines and Natural Cubic Splines.}	%17
	\item \href{https://mp.weixin.qq.com/s/4uLYtZnD3TQ-Oi7aF2BGug}{Regression Splines: In this video we talk about regression splines and how to set the number of knots or degrees of freedom  (df)to define the design matrix. We also discuss how to select the number of knots or df using k-fold Cross-Validation.}	%18
	\item \href{https://mp.weixin.qq.com/s/MTY8Eyavgnd8c2npyh1Tkw}{Non-parametric Regression: In this video we discuss Non-parametric regression methods: kernel estimators, smoothing splines and local polynomials}	%19
	\item \href{https://mp.weixin.qq.com/s/aq4OgekCqkyrpdShuakUFw}{AnCova Models 1: We discuss AnCova models including a categorical variable with two-levels. We also discuss the different models resulting form the general model with interaction.}	%20
	\item \href{https://mp.weixin.qq.com/s/fydJsiF4AvlQAo9F_eY59Q}{AnCova Models 2: We discuss ANCOVA models when the categorical variable has k-levels. We also discuss the sequential F-test for model comparison}	%21
	\item \href{https://mp.weixin.qq.com/s/8mDK4IzTxbvssrVxRTfkVQ}{Variable Selection: In this video we discuss how to do variable selection in linear regression using testing-based methods and criteria-based methods. We also discuss different searching methods to come up with the best model selection.}	%22
	\item \href{https://mp.weixin.qq.com/s/BrPASb00jklNy0506RBLtw}{Shrinkage Methods 1: In this video we talk about Principal Components Regression. This method is useful to reduce the dimensionality of the predictors space and  when the high number of predictors  can cause collinearity problems.}	%23
	\item \href{https://mp.weixin.qq.com/s/gie_Ndz74VJPEgB60rs8Pw}{Shrinkage Methods 2: In this video we discuss Ridge Regression and Lasso Regression. These two methods are also called penalized regression methods. They are useful when just a subset of the regression coefficients is important for prediction.}	%24
	\item \href{https://mp.weixin.qq.com/s/M8ntjxVx4CHMo_NZmAWdjw}{ A/B testing: In this video we give a brief introduction of A/B testing}	%25
	\item \href{https://mp.weixin.qq.com/s/AvWnVaxS985WxOC9fBqjAA}{One-way ANOVA models 1: In this video we explain the One-way ANOVA model and different versions of the model, including the mean model and the Factors effect model.}	%26
	\item \href{https://mp.weixin.qq.com/s/iIk1csBKWV8vKIz7GDJgzg}{One-way ANOVA models 2: In this video we talk about the properties of the One-way ANOVA model, model estimation and model Inference. We also discuss model diagnostics including the Levene's test for equality of variances.}	%27
	\item \href{https://mp.weixin.qq.com/s/ljze-kWn1NG2B8qjnTHy6g}{One-way ANOVA models 3: In this video we explain different ways of comparing the level means of a factor in a one-way ANOVA model, including contrasts. We discuss how to calculate the confidence intervals for the difference between two-means and different types of contrasts.}	%28
	\item \href{https://mp.weixin.qq.com/s/Ieh3JwQoZS-savFa6dgV-g}{One-way ANOVA models 4: In this video we continue discussing different methods to test one-factor level means including the Tukey's test and the Scheffe's test.}	%29
	\item \href{https://mp.weixin.qq.com/s/yFyfTGox2ZUxLdXx80C7SQ}{Two-way ANOVA 1: In this video we describe the two-facto ANOVA models and discuss the difference between the cell means model and the factor effects model. We all discuss estimation in the two-factor ANOVA model.}	%30
	\item \href{https://mp.weixin.qq.com/s/OkR8MV5gNZdqFEGQPXsg_A}{Two-way ANOVA 2: In this video we discuss the ANOVA table and the F-tests for the two-way ANOVA model. We also discuss the estimation of factor level means when the interaction is not significant and the treatment means estimation when the interaction is significant.}	%31
	\item \href{https://mp.weixin.qq.com/s/A2PSRjlhJ7gQQD1JCZ4qeg}{Two-way ANOVA 3: In this video we discuss the two-way ANOVA model for the unbalanced case. We also discuss the balance case with 1 replication and the Tukey's additivity test for the interaction.}	%32
	\item \href{https://mp.weixin.qq.com/s/8zY8Ez-iXy3RCoy-U9cICg}{R examples: Two-way ANOVA models 4: additional cases, In this video we discuss two-way anova models when we have an unbalance case and we need to analyze the model factors using the Type III Sum of Squares. We also discuss the balanced case with n=1 (a single replication) and the Tukey's additivity test.}	%33
	\item \href{https://mp.weixin.qq.com/s/U8rizPEskkZOpHZhKMqKCA}{Experimental Design 1: In this video we introduce the components and steps of experimental design, randomization and introduce randomization tests. We also discussed matched pairs comparisons}	%34
	\item \href{https://mp.weixin.qq.com/s/f45uvnm6FEYtMkAtGUj2jQ}{Experimental Design 2: In this video we discuss cross-over experiments as a special case of matched pairs, and provide another example of randomization tests.}	%35
	\item \href{https://mp.weixin.qq.com/s/6bugSfbFOa-q0qByo3UNpg}{R Examples: Experimental Design 1 (Paired t-test and blocking factors): In this video we re-visit the t-test for two independent and dependent samples and use the lm function with a blocking factor to calculate the t-test. We also discuss the randomized t-test.}	%36
	\item \href{https://mp.weixin.qq.com/s/FiysXIb8Lua13k77LOFnqg}{Experimental Design 3: In this video we discuss the Completely Randomized Block Designs and the Latin squares design. We also discuss how to calculate the relative efficiency between two designs.}	%37
	\item \href{https://mp.weixin.qq.com/s/GcjqOh-ppezqWudnCm2Ksg}{Random Effects Model 1: In this video we discuss the meaning of random factors vs. fixed factors. We also introduced random and mixed effects models and how to estimate the model parameters using maximum likelihood and restricted maximum likelihood estimation.}	%38
	\item \href{https://mp.weixin.qq.com/s/bv4KRjX6g5hejMB9mMAY0Q}{Random effects model 2: In this video we discuss how to do inference in the random effects model using the Likelihood Ratio test and parametric bootstrap methods.}	%39
	%\item \href{url}{Materials}
\end{enumerate}




%%%%%%%%%%%%%%%%%%%%%%%%%%%%%%%%%%%%%%%%%%%%%%%%%%%%%%%%%%%%%%%%%%%%%%%%%%%%%%%%%%%%%%%%%%%%%%%%%%%%%%%%%%%%%%%%%%%%%%%
%\bibliographystyle{ieeetr} % number
%%\bibliographystyle{unsrtnat} % author year
%\bibliography{HeBib}
%%%%%%%%%%%%%%%%%%%%%%%%%%%%%%%%%%%%%%%%%%%%%%%%%%%%%%%%%%%%%%%%%%%%%%%%%%%%%%%%%%%%%%%%%%%%%%%%%%%%%%%%%%%%%%%%%%%%%%%
\begin{flushright}
	\tiny \today 
\end{flushright}
%%%%%%%%%%%%%%%%%%%%%%%%%%%%%%%%%%%%%%%%%%%%%%%%%%%%%%%%%%%%%%%%%%%%%%%%%%%%%%%%%%%%%%%%%%%%%%%%%%%%%%%%%%%%%%%%%%%%%%%
\end{document}
%%%%%%%%%%%%%%%%%%%%%%%%%%%%%%%%%%%%%%%%%%%%%%%%%%%%%%%%%%%%%%%%%%%%%%%%%%%%%%%%%%%%%%%%%%%%%%%%%%%%%%%%%%%%%%%%%%%%%%%
              