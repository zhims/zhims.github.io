%%%%%%%%%%%%%%%%%%%%%%%%%%%%%%%%%%%%%%%%%%%%%%%%%%%%%%%%%%%%%%%%%%%%%%%%%%%%%%%%%%%%%%%%%%%%%%%%%%%%%%%%%%%%%%%%%%%%%
%%%%%%%%%%%%%%%%%%%%%%%%%%%%%%%%%%%%%%%%%%%%%   Author:Yao Zhang  %%%%%%%%%%%%%%%%%%%%%%%%%%%%%%%%%%%%%%%%%%%%%%%%%%%
%%%%%%%%%%%%%%%%%%%%%%%%%%%%%%%%%%%%%%%%%%%%% Email: jaafar_zhang@163.com %%%%%%%%%%%%%%%%%%%%%%%%%%%%%%%%%%%%%%%%%%%
%%%%%%%%%%%%%%%%%%%%%%%%%%%%%%%%%%%%%%%%%%%%%%%%%%%%%%%%%%%%%%%%%%%%%%%%%%%%%%%%%%%%%%%%%%%%%%%%%%%%%%%%%%%%%%%%%%%%%
\documentclass[11pt]{article}
\usepackage{babel}
\usepackage[utf8]{inputenc} 
\usepackage[table]{xcolor}
\usepackage[most]{tcolorbox}
\usepackage[left=2.50cm, right=1.50cm, top=2.0cm, bottom=2.50cm]{geometry}
\usepackage{xcolor,url}
\usepackage{amsmath,amsthm,amsfonts,amssymb,amscd,multirow,booktabs,fullpage,calc,multicol}
\usepackage{lastpage,enumitem,fancyhdr,mathrsfs,wrapfig,setspace,cancel,amsmath,empheq,framed}
\usepackage[retainorgcmds]{IEEEtrantools}
\usepackage{subfig,graphicx,framed}
\usepackage{ctex}
\usepackage{txfonts}
\usepackage{bbm}
\usepackage{chngcntr}
\usepackage[colorlinks,linkcolor=blue,anchorcolor=green,citecolor=red,urlcolor=blue]{hyperref}
\usepackage{titlesec}
%%%%%%%%%%%%%%%%%%%%%%%%%%%%%%%%%%%%%%%%%%%%%%%%%%%%%%%%%%%%%%%%%%%%%%%%%%%%%%%%%%%%%%%%%%%%%%%%%%%%%%%%%%%%%%%%%%%%%%
\newtheorem{thm}{Theorem}[section]
\newtheorem{defi}{Definition}[subsection]
\newtheorem{exercise}{Exercise}[subsection]
\newtheorem{note}{Note}[subsection]
\newtheorem{notation}{Notation}
\newtheorem{lemma}{Lemma}[subsection]
\newtheorem{proposition}{Proposition}[subsection]
\newtheorem{example}{Example}[subsection]
\newtheorem{problem}{Problem}[section]
\newtheorem{homework}{Homework}[section]
\newtheorem{summary}{Summary}[subsection]
\newtheorem{corollary}{Corollary}[subsection]
\newtheorem{rmk}{Remark}[section]
\usepackage{romannum}
%%%%%%%%%%%%%%%%%%%%%%%%%%%%%%%%%%%%%%%%%%%%%%%%%%%%%%%%%%%%%%%%%%%%%%%%%%%%%%%%%%%%%%%%%%%%%%%%%%%%%%%%%%%%%%%%%%%%%
\newlength{\tabcont}
\setlength{\parindent}{0.0in}
\setlength{\parskip}{0.05in}
\colorlet{shadecolor}{orange!15}
\parindent 0in
\parskip 12pt
\geometry{margin=1in, headsep=0.25in}
%%%%%%%%%%%%%%%%%%%%%%%%%%%%%%%%%%%%%%%%%%%%%%%%%%%%%%%%%%%%%%%%%%%%%%%%%%%%%%%%%%%%%%%%%%%%%%%%%%%%%%%%%%%%%%%%%%%%%
\graphicspath{ {img/EoM/}}
%%%%%%%%%%%%%%%%%%%%%%%%%%%%%%%%%%%%%%%%%%%%%%%%%%%%%%%%%%%%%%%%%%%%%%%%%%%%%%%%%%%%%%%%%%%%%%%%%%%%%%%%%%%%%%%%%%%%%
%\renewcommand{\cite}[1]{[#1]}
\makeatletter
\@addtoreset{equation}{section}
\makeatother
\renewcommand{\theequation}{\arabic{section}.\arabic{equation}}
\renewcommand{\contentsname}{\centering \small \color{blue} Contents}
%\counterwithin{figure}{section}
\renewcommand{\figurename}{\textbf{Fig.}}
%\renewcommand{\refname}{\textbf{\kaishu 参考文献}}
\renewcommand{\refname}{\textbf{Bibliography}}
\setcounter{secnumdepth}{4}
\titleformat{\paragraph}
{\normalfont\normalsize\bfseries}{\theparagraph}{1em}{}
\titlespacing*{\paragraph}{0pt}{3.25ex plus 1ex minus .2ex}{1.5ex plus .2ex}
\def\beginrefs{\begin{list}%
		{[\arabic{equation}]}{\usecounter{equation}
			\setlength{\leftmargin}{0.8truecm}\setlength{\labelsep}{0.4truecm}%
			\setlength{\labelwidth}{1.6truecm}}}
	\def\endrefs{\end{list}}
\def\bibentry#1{\item[\hbox{[#1]}]}
%%%%%%%%%%%%%%%%%%%%%%%%%%%%%%%%%%%%%%%%%%%%%%%%%%%%%%%%%%%%%%%%%%%%%%%%%%%%%%%%%%%%%%%%%%%%%%%%%%%%%%%%%%%%%%%%%%%%%%
%\begin{figure}[!htb]
%	\centering
%	\subfloat[$A \cap B$]{%
%		\includegraphics[width=0.3\linewidth,height=0.2\linewidth]{img001.jpg}}
%	\label{img001}\qquad \qquad %\hfill
%	\subfloat[${A_1} \cap {A_2} \cap {A_3}$]{%
%		\includegraphics[width=0.3\linewidth,height=0.2\linewidth]{img002.jpg}}
%	\label{img002}
	%\caption{ Examples.}
%\end{figure}
%\begin{figure}[!htb]
%	\centering
%	\includegraphics[width=0.4\linewidth,height=0.3\linewidth]{img005.jpg}
%	\label{img005}
	%\caption{ illustration for $ 3 $}
%\end{figure}
%\={a}1 \'{a}2\v{a}3\.{a}4

\usepackage{datetime}
\renewcommand{\today}{\shortmonthname[\the\month] \the \day,  \the\year}
%%%%%%%%%%%%%%%%%%%%%%%%%%%%%%%%%%%%%%%%%%%%%%%%%%%%%%%%%%%%%%%%%%%%%%%%%%%%%%%%%%%%%%%%%%%%%%%%%%%%%%%%%%%%%%%%%%%%%%
\begin{document}
	\kaishu 
	%\thispagestyle{empty}
	\pagenumbering{arabic} 
	\setcounter{section}{0}
	\begin{center}
		{\LARGE  \href{https://shenghsuanlin.web.nycu.edu.tw/wp-content/uploads/sites/349/2022/09/Syllabus.pdf}{Causal Inference}}
		
		%\vspace{-0.25cm}
		
		{\large \href{https://shenghsuanlin.web.nycu.edu.tw/}{Sheng-Hsuan Lin}}
	\end{center}
%%%%%%%%%%%%%%%%%%%%%%%%%%%%%%%%%%%%%%%%%%%%%%%%%%%%%%%%%%%%%%%%%%%%%%%%%%%%%%%%%%%%%%%%%%%%%%%%%%%%%%%%%%%%%%%%%%%%%%
%%\newpage 
%%\thispagestyle{empty}	
%%%%%%%%%%%%%%%%%%%%%%%%%%%%%%%%%%%%%%%%%%%%%%%%%%%%%%%%%%%%%%%%%%%%%%%%%%%%%%%%%%%%%%%%%%%%%%%%%%%%%%%%%%%%%%%%%%%%%%
%\tableofcontents	
%{\pagestyle{empty}\mbox{}\newpage\pagestyle{empty}}
%\newpage 
%{\pagestyle{empty}\mbox{}\newpage\pagestyle{empty}}
%%%%%%%%%%%%%%%%%%%%%%%%%%%%%%%%%%%%%%%%%%%%%%%%%%%%%%%%%%%%%%%%%%%%%%%%%%%%%%%%%%%%%%%%%%%%%%%%%%%%%%%%%%%%%%%%%%%%%%
%%\newpage 
\setcounter{page}{1}

%\vspace{1.5cm}


\vspace{-1cm}

\subsubsection*{\kaishu 1. 导读}

\vspace{-0.5cm}

\begin{enumerate}
	\item \href{https://mp.weixin.qq.com/s/GRRr4QLAyD6dl-DqSXUppA}{因果经济学导论}	%1
\end{enumerate}

\vspace{-1cm}

\subsubsection*{\kaishu 2. 必备知识}

\vspace{-0.5cm}

\begin{multicols}{2}
	\begin{enumerate}
		\item \href{https://mp.weixin.qq.com/s/_4459wttNLq4Ug4iPMNuyQ}{因果推论分类}	%1
		\item \href{https://mp.weixin.qq.com/s/uNK3o0KsPv4NoVp1vDiZvA}{统计模型分类}	%2
		\item \href{https://mp.weixin.qq.com/s/vFPrBpP47TYifWgaqA0QLg}{流行病学的基本定义与测量}	%3
		\item \href{https://mp.weixin.qq.com/s/y40XDRGrVoIgcCCYtAn80A}{辛普森悖论}	%4
		\item \href{https://mp.weixin.qq.com/s/RTbZ2oePbaL9Lx2UhosjAw}{干扰因子}	%5
		\item \href{https://mp.weixin.qq.com/s/s2HTAYpsZt-JRUaMreC-yg}{分层分析与模型}	%6
	\end{enumerate}
\end{multicols}

\vspace{-1cm}

\subsubsection*{\kaishu 3. 因果推论}

\vspace{-0.5cm}

\begin{multicols}{2}
	\begin{enumerate}
		\item \href{https://mp.weixin.qq.com/s/W92ZurwB0431bOjolErMcQ}{Introduction of causal inference 1}	%1
		\item \href{https://mp.weixin.qq.com/s/8R9Tew3wshkK_95TIfNQeQ}{Introduction of causal inference 2}	%2
		\item \href{https://mp.weixin.qq.com/s/2LjJU_2HnIQuVikL2nJGXQ}{Framework of causal inference}	%3
		\item \href{https://mp.weixin.qq.com/s/9-aAIHYw7cIScdiZkmyDjg}{Causal structure: DAGs}	%4
		\item \href{https://mp.weixin.qq.com/s/3gdqkn7SgzBf1jzQX-cfxw}{Variable in causal inference}	%5
		\item \href{https://mp.weixin.qq.com/s/anf-w03I-eiUCfSy7jaYDg}{Causal structure: D-separation rule}	%6
		\item \href{https://mp.weixin.qq.com/s/3ko1L3dg_2_z7cvpI02Gsw}{Causal structure: NPSEM}	%7
		\item \href{https://mp.weixin.qq.com/s/Mld8h9U1qgeg_myPnbYTbQ}{\small Causal structure: NPSEM for d-separation 1}	%8
		\item \href{https://mp.weixin.qq.com/s/ZooQicl91QtUglL867KPsw}{\small Causal structure: NPSEM for d-separation 2}	%9
		\item \href{https://mp.weixin.qq.com/s/1s1W5jLFTNFEBXbC1pMNqw}{Counterfactual outcome model}	%10
		\item \href{https://mp.weixin.qq.com/s/L0rovGlXJF78IuaPsQBb4g}{Identification: exchangeability and RCT}	%11
		\item \href{https://mp.weixin.qq.com/s/czrPh4QeGfloK0C8HwkDoA}{Estimation: g-formula, IPW, and MSM}	%12
		\item \href{https://mp.weixin.qq.com/s/VhO4AD5lIPQtHEbFsDBKSw}{Estimation: g-estimation and SNM}	%13
		\item \href{https://mp.weixin.qq.com/s/byaU2Ja7p2fipSJBPZhFhw}{Estimation: doubly robust estimator}	%14
		\item \href{https://mp.weixin.qq.com/s/K68QFvy2kBbAZg7P0mipeA}{Efficiency theorem: Why EIF}	%15
		\item \href{https://mp.weixin.qq.com/s/tVgr9wOWmP8Uuami5H_PkA}{Efficiency theorem: What is EIF}	%16
		\item \href{https://mp.weixin.qq.com/s/VtDcsaEEdxR1yZmdz9eB-g}{Efficiency theorem: How to derive EIF 1}	%17
		\item \href{https://mp.weixin.qq.com/s/r7wr1zonO-GGEGyVyLxWQA}{Efficiency theorem: How to derive EIF 2,3}	%18
	\end{enumerate}
\end{multicols}



\vspace{-1cm}

\subsubsection*{\kaishu 4. 因果推论架构与概念}

\vspace{-0.5cm}

\begin{multicols}{2}
	\begin{enumerate}
		\item \href{https://mp.weixin.qq.com/s/tBHwHv3gQdgX2lKMJfHpcA}{常见名词}	%1
		\item \href{https://mp.weixin.qq.com/s/6F7J_KaNsKEnnQF3M1peJQ}{绪论}	%2
		\item \href{https://mp.weixin.qq.com/s/gsIw1zhrvkF2dZnxNKzoEA}{Causal Diagram}	%3
		\item \href{https://mp.weixin.qq.com/s/FC3qJ9acrJZbV208v17UTA}{counterfactual model}	%4
		\item \href{https://mp.weixin.qq.com/s/q-xD8pj2bM2RfD67PlEPmA}{常见 causal effects}	%5
		\item \href{https://mp.weixin.qq.com/s/yiZLG7CQsUCsek88vNz39A}{positivity and consistency}	%6
		\item \href{https://mp.weixin.qq.com/s/BgCtclwJsd_vJlFuppHiIQ}{exchangeability}	%7
		\item \href{https://mp.weixin.qq.com/s/aeflhUV9ooP80hSoPtd4Ew}{其它两种 assumptions}	%8
		\item \href{https://mp.weixin.qq.com/s/tjGhDj2vW7btp9_mpH7fsQ}{因果研究中 assumption 的重点}	%9
		\item \href{https://mp.weixin.qq.com/s/ygs4MYKA3s-eZnrO7q6GSA}{sensitivity analysis}	%10
		\item \href{https://mp.weixin.qq.com/s/Lwy0ZgNU9E-m7xtYjEwZYA}{Identification}	%11
		\item \href{https://mp.weixin.qq.com/s/6XFNSn2uEQhJVoqj2zfbvQ}{Statistical Inference}	%12
	\end{enumerate}
\end{multicols}

\vspace{-1cm}

\subsubsection*{\kaishu 5. Data analysis DEMO}

\vspace{-0.5cm}

\begin{multicols}{3}
	\begin{enumerate}
		\item \href{https://mp.weixin.qq.com/s/6GagqDA5KoD1ITFghalsHA}{Demo 1}	%1
		\item \href{https://mp.weixin.qq.com/s/gW1Ho58AGP0manJkTh7uRw}{Demo 2}	%2
		\item \href{https://mp.weixin.qq.com/s/nc8LYP3OUiwaeXKXUIL3GA}{Demo 3}	%3
	\end{enumerate}
\end{multicols}



\vspace{-1cm}

\subsubsection*{\kaishu 6. NDE and NIE}

\vspace{-0.5cm}

\begin{enumerate}
	\item \href{https://mp.weixin.qq.com/s/Cw1v-UKBb9yigKF8q-yidg}{Introduction of causal mediation analysis}	%1
	\item \href{https://mp.weixin.qq.com/s/F8WKmhrzXEyyiVd54RqZqA}{Definition for causal mediation analysis (NDE and NIE)}	%2
	\item \href{https://mp.weixin.qq.com/s/DZnhE-gtiEh2iP0C9JmxeA}{Identification for NDE and NIE 1}	%3
	\item \href{https://mp.weixin.qq.com/s/oXRnkyD_0JK9tQuipgnOWw}{Identification for NDE and NIE 2}	%4
	\item \href{https://mp.weixin.qq.com/s/s5yrgAO913wcXmhOg-ysoQ}{Identification for NDE and NIE 3 recanting witness and NPSEM}	%5
	\item \href{https://mp.weixin.qq.com/s/98U94FvD3D1oHK-HbdttPA}{Review framework}	%6
	\item \href{https://mp.weixin.qq.com/s/UWTRO0gSPiswEDsIxtvPsA}{Identification for NDE and NIE 4 when M is continuous}	%7
	\item \href{https://mp.weixin.qq.com/s/vBVU-ppWwzmZyUKgg0Kl4w}{Estimation for NDE and NIE 1}	%8
	\item \href{https://mp.weixin.qq.com/s/bt0p9SKevVP9M4avaJctPw}{Estimation for NDE and NIE 2}	%9
	\item \href{https://mp.weixin.qq.com/s/vAWnR_0fuI1AOJ0KQ_isTA}{Estimation for NDE and NIE 3}	%10
	\item \href{https://mp.weixin.qq.com/s/SnOdeYG1GWulNHkrp1feCQ}{Estimation for NDE and NIE 4}	%11
	\item \href{https://mp.weixin.qq.com/s/dEhVZrK9HF6A-c2Vxw5DTw}{Estimation for NDE and NIE 5: regression with interaction term}	%12
	\item \href{https://mp.weixin.qq.com/s/Ryw632CtN_UNRRkzaYOXxw}{Estimation for NDE and NIE 6: Monte-Carlo simulation and software (mediation.R)}	%13
	\item \href{https://mp.weixin.qq.com/s/-g18ym1rwA9dcfN4E0WEcg}{Estimation for NDE and NIE 7: IPW}	%14
\end{enumerate}

\vspace{-1cm}

\subsubsection*{\kaishu 7. Binary outcome}

\vspace{-0.5cm}

\begin{enumerate}
	\item \href{https://mp.weixin.qq.com/s/Ik32rRrlkNIB-AlhBy6NZQ}{Causal mediation analysis with binary outcome 1}	%1
	\item \href{https://mp.weixin.qq.com/s/niCB01OT_NscVGvP6I_Fvw}{Causal mediation analysis with binary outcome 2: Counterfactual OR and PM}	%2
	\item \href{https://mp.weixin.qq.com/s/yQg9Vr9YinbQCZW_8fVwxQ}{Causal mediation analysis with binary outcome 3: Estimation A}	%3
	\item \href{https://mp.weixin.qq.com/s/oniTcLXzqKiho7i3uDsv0A}{Causal mediation analysis with binary outcome 4: Estimation B}	%4
\end{enumerate}

\vspace{-1cm}

\subsubsection*{\kaishu 8. Alternative definition}

\vspace{-0.5cm}

\begin{multicols}{2}
	\begin{enumerate}
		\item \href{https://mp.weixin.qq.com/s/ULs-LgvmoyZg9Q00kLoOcQ}{3 diff. def. for causal mediation analysis}	%1
		\item \href{https://mp.weixin.qq.com/s/pti3oszhsNJY6r5aZOUYhA}{CDE 1}	%2
		\item \href{https://mp.weixin.qq.com/s/xLrNXJFeMtjvd3Wnivc06g}{CDE 2}	%3
		\item \href{https://mp.weixin.qq.com/s/vEsX4FY3R-y-8ulHakQifw}{Interventional approach 1: Introduction}	%4
		\item \href{https://mp.weixin.qq.com/s/qUxU9v45u5qgJ9WWlkY2JA}{Interventional approach 2: Definition}	%5
		\item \href{https://mp.weixin.qq.com/s/nj-NmXpdj8qhT03ect3F2w}{Interventional approach 3: Identification}	%6
	\end{enumerate}
\end{multicols}

\vspace{-1cm}

\subsubsection*{\kaishu 9. 中介效应分析}

\vspace{-0.5cm}

\begin{enumerate}
	\item \href{https://mp.weixin.qq.com/s/QVo-71WwLWt99ljKQOtXtg}{Mediation analysis 基本概念常见的错误解读}	%1
	\item \href{https://mp.weixin.qq.com/s/tzJnVwOSfnCNbJtr9ksGUQ}{传统 Mediation analysis 的做法以及 causal mediation}	%2
	\item \href{https://mp.weixin.qq.com/s/KPbpaspmIonZpCDdAFJbpQ}{Mediation analysis 三种常见的定义及 interpretation}	%3
	\item \href{https://mp.weixin.qq.com/s/AoNnTWP3dMCCcmskwfvJ0g}{Mediation analysis 三种常见的定义在辨识中需要的假设及意义}	%4
	\item \href{https://mp.weixin.qq.com/s/9hUGxYCy_a8UWkwXy78tAw}{Mediation analysis identification and estimation}	%5
\end{enumerate}

\vspace{-1cm}

\subsubsection*{\kaishu 10. 交互作用分析}

\vspace{-0.5cm}

\begin{multicols}{2}
	\begin{enumerate}
		\item \href{https://mp.weixin.qq.com/s/E7txm3jtXFuGbGS_b9yMFw}{Introduction}	%1
		\item \href{https://mp.weixin.qq.com/s/t51Bs8d73iR8unozout9yA}{三种 interaction}	%2
		\item \href{https://mp.weixin.qq.com/s/zcVLsoAjGCEkPOF5eASvjA}{additive vs multiplcative interaction}	%3
		\item \href{https://mp.weixin.qq.com/s/OVingr4LK-R979zHzAG3NA}{RERI 1}	%4
		\item \href{https://mp.weixin.qq.com/s/RfIYqjLL90LQLCB0eryeLA}{RERI 2}	%5
		\item \href{https://mp.weixin.qq.com/s/VW9h_Bztrchcg3tMyBM8zA}{Effect modification}	%6
	\end{enumerate}
\end{multicols}

\vspace{-1cm}

\subsubsection*{\kaishu 11. 多重中介效应}

\vspace{-0.5cm}

\begin{multicols}{2}
	\begin{enumerate}
		\item \href{https://mp.weixin.qq.com/s/6-TvJrRUcluUYCzGwgluKA}{DAGs}	%1
		\item \href{https://mp.weixin.qq.com/s/RMfV5z9WIxsswops8HLKng}{counterfactual 1}	%2
		\item \href{https://mp.weixin.qq.com/s/EKEqsaAx_RmQDJlFy9vsmw}{counterfactual 2}	%3
		\item \href{https://mp.weixin.qq.com/s/5k5HY6pT6tnrAW4gFVRVKw}{Difficulty of identification}	%4
		\item \href{https://mp.weixin.qq.com/s/n-mpLQk9pysPWUrSz26TVg}{Introduction of different strategy}	%5
	\end{enumerate}
\end{multicols}



\vspace{-1cm}

\subsubsection*{\kaishu 12. 时变系统的因果推论}

\vspace{-0.5cm}

\begin{multicols}{2}
	\begin{enumerate}
		\item \href{https://mp.weixin.qq.com/s/RmfZv7W4_e6vWohTkAYGjg}{Introduction of time-varying system}	%1
		\item \href{https://mp.weixin.qq.com/s/YJnn2hG_jsK-7BxGfgobTQ}{DAGs of time-varying system}	%2
		\item \href{https://mp.weixin.qq.com/s/W7W-MO-VCpU92tYF73aqRQ}{\small Why regression fail and g-formula in simple case}	%3
		\item \href{https://mp.weixin.qq.com/s/B3rAP8NRpz60hfU1PpgJ4A}{g-formula 1}	%4
		\item \href{https://mp.weixin.qq.com/s/1UlKBo28nTVETec53cIn3Q}{g-formula 2}	%5
		\item \href{https://mp.weixin.qq.com/s/nuRefZw7A7BsL10giGz__w}{g-formula 3}	%6
		\item \href{https://mp.weixin.qq.com/s/mJsF27jYbdXUu9gAlkhIFw}{g-formula 4}	%7
		\item \href{https://mp.weixin.qq.com/s/oe3ygsTkzY3s9sw3CgVDXw}{g-formula 5}	%8
		\item \href{https://mp.weixin.qq.com/s/GOce-aNrvnZz-zpeZJxTrg}{g-formula 6}	%9
		\item \href{https://mp.weixin.qq.com/s/ne_rbSLWPXV5hoGrySikrw}{g-formula 7}	%10
		\item \href{https://mp.weixin.qq.com/s/FUYsX3IZQLBjW-7yoI35Lw}{g-formula 8}	%11
	\end{enumerate}
\end{multicols}




%%%%%%%%%%%%%%%%%%%%%%%%%%%%%%%%%%%%%%%%%%%%%%%%%%%%%%%%%%%%%%%%%%%%%%%%%%%%%%%%%%%%%%%%%%%%%%%%%%%%%%%%%%%%%%%%%%%%%%%
%\bibliographystyle{ieeetr} % number
%%\bibliographystyle{unsrtnat} % author year
%\bibliography{HeBib}
%%%%%%%%%%%%%%%%%%%%%%%%%%%%%%%%%%%%%%%%%%%%%%%%%%%%%%%%%%%%%%%%%%%%%%%%%%%%%%%%%%%%%%%%%%%%%%%%%%%%%%%%%%%%%%%%%%%%%%%
\begin{flushright}
	\tiny \today 
\end{flushright}
%%%%%%%%%%%%%%%%%%%%%%%%%%%%%%%%%%%%%%%%%%%%%%%%%%%%%%%%%%%%%%%%%%%%%%%%%%%%%%%%%%%%%%%%%%%%%%%%%%%%%%%%%%%%%%%%%%%%%%%
\end{document}
%%%%%%%%%%%%%%%%%%%%%%%%%%%%%%%%%%%%%%%%%%%%%%%%%%%%%%%%%%%%%%%%%%%%%%%%%%%%%%%%%%%%%%%%%%%%%%%%%%%%%%%%%%%%%%%%%%%%%%%
              