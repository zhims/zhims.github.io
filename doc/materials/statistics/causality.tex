%%%%%%%%%%%%%%%%%%%%%%%%%%%%%%%%%%%%%%%%%%%%%%%%%%%%%%%%%%%%%%%%%%%%%%%%%%%%%%%%%%%%%%%%%%%%%%%%%%%%%%%%%%%%%%%%%%%%%
%%%%%%%%%%%%%%%%%%%%%%%%%%%%%%%%%%%%%%%%%%%%%   Author:Yao Zhang  %%%%%%%%%%%%%%%%%%%%%%%%%%%%%%%%%%%%%%%%%%%%%%%%%%%
%%%%%%%%%%%%%%%%%%%%%%%%%%%%%%%%%%%%%%%%%%%%% Email: jaafar_zhang@163.com %%%%%%%%%%%%%%%%%%%%%%%%%%%%%%%%%%%%%%%%%%%
%%%%%%%%%%%%%%%%%%%%%%%%%%%%%%%%%%%%%%%%%%%%%%%%%%%%%%%%%%%%%%%%%%%%%%%%%%%%%%%%%%%%%%%%%%%%%%%%%%%%%%%%%%%%%%%%%%%%%
\documentclass[11pt]{article}
\usepackage{babel}
\usepackage[utf8]{inputenc} 
\usepackage[table]{xcolor}
\usepackage[most]{tcolorbox}
\usepackage[left=2.50cm, right=1.50cm, top=2.0cm, bottom=2.50cm]{geometry}
\usepackage{xcolor,url}
\usepackage{amsmath,amsthm,amsfonts,amssymb,amscd,multirow,booktabs,fullpage,calc,multicol}
\usepackage{lastpage,enumitem,fancyhdr,mathrsfs,wrapfig,setspace,cancel,amsmath,empheq,framed}
\usepackage[retainorgcmds]{IEEEtrantools}
\usepackage{subfig,graphicx,framed}
\usepackage{ctex}
\usepackage{txfonts}
\usepackage{bbm}
\usepackage{chngcntr}
\usepackage[colorlinks,linkcolor=blue,anchorcolor=green,citecolor=red,urlcolor=blue]{hyperref}
\usepackage{titlesec}
%%%%%%%%%%%%%%%%%%%%%%%%%%%%%%%%%%%%%%%%%%%%%%%%%%%%%%%%%%%%%%%%%%%%%%%%%%%%%%%%%%%%%%%%%%%%%%%%%%%%%%%%%%%%%%%%%%%%%%
\newtheorem{thm}{Theorem}[section]
\newtheorem{defi}{Definition}[subsection]
\newtheorem{exercise}{Exercise}[subsection]
\newtheorem{note}{Note}[subsection]
\newtheorem{notation}{Notation}
\newtheorem{lemma}{Lemma}[subsection]
\newtheorem{proposition}{Proposition}[subsection]
\newtheorem{example}{Example}[subsection]
\newtheorem{problem}{Problem}[section]
\newtheorem{homework}{Homework}[section]
\newtheorem{summary}{Summary}[subsection]
\newtheorem{corollary}{Corollary}[subsection]
\newtheorem{rmk}{Remark}[section]
\usepackage{romannum}
%%%%%%%%%%%%%%%%%%%%%%%%%%%%%%%%%%%%%%%%%%%%%%%%%%%%%%%%%%%%%%%%%%%%%%%%%%%%%%%%%%%%%%%%%%%%%%%%%%%%%%%%%%%%%%%%%%%%%
\newlength{\tabcont}
\setlength{\parindent}{0.0in}
\setlength{\parskip}{0.05in}
\colorlet{shadecolor}{orange!15}
\parindent 0in
\parskip 12pt
\geometry{margin=1in, headsep=0.25in}
%%%%%%%%%%%%%%%%%%%%%%%%%%%%%%%%%%%%%%%%%%%%%%%%%%%%%%%%%%%%%%%%%%%%%%%%%%%%%%%%%%%%%%%%%%%%%%%%%%%%%%%%%%%%%%%%%%%%%
\graphicspath{ {img/EoM/}}
%%%%%%%%%%%%%%%%%%%%%%%%%%%%%%%%%%%%%%%%%%%%%%%%%%%%%%%%%%%%%%%%%%%%%%%%%%%%%%%%%%%%%%%%%%%%%%%%%%%%%%%%%%%%%%%%%%%%%
%\renewcommand{\cite}[1]{[#1]}
\makeatletter
\@addtoreset{equation}{section}
\makeatother
\renewcommand{\theequation}{\arabic{section}.\arabic{equation}}
\renewcommand{\contentsname}{\centering \small \color{blue} Contents}
%\counterwithin{figure}{section}
\renewcommand{\figurename}{\textbf{Fig.}}
%\renewcommand{\refname}{\textbf{\kaishu 参考文献}}
\renewcommand{\refname}{\textbf{Bibliography}}
\setcounter{secnumdepth}{4}
\titleformat{\paragraph}
{\normalfont\normalsize\bfseries}{\theparagraph}{1em}{}
\titlespacing*{\paragraph}{0pt}{3.25ex plus 1ex minus .2ex}{1.5ex plus .2ex}
\def\beginrefs{\begin{list}%
		{[\arabic{equation}]}{\usecounter{equation}
			\setlength{\leftmargin}{0.8truecm}\setlength{\labelsep}{0.4truecm}%
			\setlength{\labelwidth}{1.6truecm}}}
	\def\endrefs{\end{list}}
\def\bibentry#1{\item[\hbox{[#1]}]}
%%%%%%%%%%%%%%%%%%%%%%%%%%%%%%%%%%%%%%%%%%%%%%%%%%%%%%%%%%%%%%%%%%%%%%%%%%%%%%%%%%%%%%%%%%%%%%%%%%%%%%%%%%%%%%%%%%%%%%
%\begin{figure}[!htb]
%	\centering
%	\subfloat[$A \cap B$]{%
%		\includegraphics[width=0.3\linewidth,height=0.2\linewidth]{img001.jpg}}
%	\label{img001}\qquad \qquad %\hfill
%	\subfloat[${A_1} \cap {A_2} \cap {A_3}$]{%
%		\includegraphics[width=0.3\linewidth,height=0.2\linewidth]{img002.jpg}}
%	\label{img002}
	%\caption{ Examples.}
%\end{figure}
%\begin{figure}[!htb]
%	\centering
%	\includegraphics[width=0.4\linewidth,height=0.3\linewidth]{img005.jpg}
%	\label{img005}
	%\caption{ illustration for $ 3 $}
%\end{figure}
%\={a}1 \'{a}2\v{a}3\.{a}4

\usepackage{datetime}
\renewcommand{\today}{\shortmonthname[\the\month] \the \day,  \the\year}
%%%%%%%%%%%%%%%%%%%%%%%%%%%%%%%%%%%%%%%%%%%%%%%%%%%%%%%%%%%%%%%%%%%%%%%%%%%%%%%%%%%%%%%%%%%%%%%%%%%%%%%%%%%%%%%%%%%%%%
\begin{document}
	\kaishu 
	%\thispagestyle{empty}
	\pagenumbering{arabic} 
	\setcounter{section}{0}
	\begin{center}
		{\LARGE  \href{https://stat.ethz.ch/lectures/ss21/causality.php}{Causality}}
		
		%\vspace{-0.25cm}
		
		{\large \href{https://christinaheinze.github.io/}{Christina Heinze}}
	\end{center}
%%%%%%%%%%%%%%%%%%%%%%%%%%%%%%%%%%%%%%%%%%%%%%%%%%%%%%%%%%%%%%%%%%%%%%%%%%%%%%%%%%%%%%%%%%%%%%%%%%%%%%%%%%%%%%%%%%%%%%
%%\newpage 
%%\thispagestyle{empty}	
%%%%%%%%%%%%%%%%%%%%%%%%%%%%%%%%%%%%%%%%%%%%%%%%%%%%%%%%%%%%%%%%%%%%%%%%%%%%%%%%%%%%%%%%%%%%%%%%%%%%%%%%%%%%%%%%%%%%%%
%\tableofcontents	
%{\pagestyle{empty}\mbox{}\newpage\pagestyle{empty}}
%\newpage 
%{\pagestyle{empty}\mbox{}\newpage\pagestyle{empty}}
%%%%%%%%%%%%%%%%%%%%%%%%%%%%%%%%%%%%%%%%%%%%%%%%%%%%%%%%%%%%%%%%%%%%%%%%%%%%%%%%%%%%%%%%%%%%%%%%%%%%%%%%%%%%%%%%%%%%%%
%%\newpage 
\setcounter{page}{1}

%\vspace{1.5cm}


\vspace{-1cm}

\begin{enumerate}
	\item \href{https://mp.weixin.qq.com/s/BCAkzOx5AKgpBVByO1Vp5Q}{Introduction}	%1
	\item \href{https://mp.weixin.qq.com/s/RAsgQyFDkVZ53qrk9DaIsw}{Graphical models}	%2
	\item \href{https://mp.weixin.qq.com/s/S-vzFYGQB_U9B8bL618OYA}{Causal graphical models}	%3
	\item \href{https://mp.weixin.qq.com/s/1unXNgWZZxSqKy17KdIibA}{Causal models and covariate adjustment}	%4
	\item \href{https://mp.weixin.qq.com/s/ZQqRiXs4DvKazoli1UOYWg}{Covariate adjustment}	%5
	\item \href{https://mp.weixin.qq.com/s/tnSzVeko11E7aP1cCW2xCg}{Frontdoor criterion, instrumental variables and transportability}	%6
	\item \href{https://mp.weixin.qq.com/s/Z4drzGFfblVFRTsBw9wBXA}{Counterfactuals, potential outcomes and estimation}	%7
	\item \href{https://mp.weixin.qq.com/s/8j9XwhUKVy1rCka_WVXOhg}{Towards structure learning}	%8
	\item \href{https://mp.weixin.qq.com/s/uYBHcXnYzg-bc2Zn7fVcLA}{Constraint-based causal structure learning}	%9
	\item \href{https://mp.weixin.qq.com/s/eaSDnSGKHxNVoWlhXVngfw}{Score-based causal structure learning and restricted SEMs}	%10
	\item \href{https://mp.weixin.qq.com/s/DZUgyzsQA3vwR15I0ZTQkQ}{LiNGAM and Invariant Causal Prediction}	%11
	%\item \href{url}{Materials}
\end{enumerate}

\subsubsection*{\href{https://space.bilibili.com/491707363/lists/25679?type=season}{\kaishu 因果推断入门:}}

\vspace{-0.5cm}

\begin{multicols}{3}
	\begin{enumerate}
		\item \href{https://mp.weixin.qq.com/s/qroVyHDwNjOEkoLWPTFDwg}{简介}	%1
		\item \href{https://mp.weixin.qq.com/s/d4PVqAFXsdeMN-ZWz5aATA}{辛普森悖论 1}	%2
		\item \href{https://mp.weixin.qq.com/s/BbGhJG6zVFxo_xPRUMuE9A}{辛普森悖论 2}	%3
		\item \href{https://mp.weixin.qq.com/s/nbCWMl-eaTzCjYqGrHSjSA}{概率统计基本工具 1}	%4
		\item \href{https://mp.weixin.qq.com/s/_OALIAraHh1kLxz4WM4SDQ}{概率统计基本工具 2}	%5
		\item \href{https://mp.weixin.qq.com/s/dJcCR8LJlUKa80lmB8M8eg}{图模型}	%6
		\item \href{https://mp.weixin.qq.com/s/ioHaB_aSR5hzQuqjZB_VZQ}{结构因果模型}	%7
		\item \href{https://mp.weixin.qq.com/s/D55qKD7w5yLKTjP7_S3wug}{Intransitive Case}	%8
		\item \href{https://mp.weixin.qq.com/s/oacDo6ydqp6YERJC-_bQ0Q}{链状结构}	%9
		\item \href{https://mp.weixin.qq.com/s/tTwGhY6cnCWgsM9StlBj0A}{叉状结构}	%10
		\item \href{https://mp.weixin.qq.com/s/6Uw-RI8G5pmvH71kER392A}{对撞结构}	%11
		\item \href{https://mp.weixin.qq.com/s/Q2Um2f-cy_D5s6nEhtFUUA}{D-分隔}	%12
		\item \href{https://mp.weixin.qq.com/s/f_DPek3v2I7h2JfsXheCRw}{模型检验和等价类}	%13
		\item \href{https://mp.weixin.qq.com/s/Uf2Gvs7y9u7A8STtQQHeCg}{乘积分解法则}	%14
		\item \href{https://mp.weixin.qq.com/s/mvpAu4WOkg6PULIncn8MmA}{混淆变量}	%15
		\item \href{https://mp.weixin.qq.com/s/EybWLdgnl4cAGdHCFHihUg}{习题 1.3.2 和 1.4.1}	%16
		\item \href{https://mp.weixin.qq.com/s/8mBCmaYlAdL-AMISakK3kw}{观测数据和试验数据 1}	%17
		\item \href{https://mp.weixin.qq.com/s/o-CXebk_5XEgsNOFgPFELQ}{观测数据和试验数据 2}	%18
		\item \href{https://mp.weixin.qq.com/s/vwlbqxVd0VccyC8d9560VA}{干预}	%19
		\item \href{https://mp.weixin.qq.com/s/yRsmm3m4-Gml4omoORn52Q}{do 算子}	%20
		\item \href{https://mp.weixin.qq.com/s/-qLTS8iGjqj6IKS3S5GxWg}{调整公式}	%21
		\item \href{https://mp.weixin.qq.com/s/SFulXDLBui7nJSZJemwYkg}{支线任务:调整公式实例}	%22
		\item \href{https://mp.weixin.qq.com/s/u2cRH1uGTJoPweWrLHuEdA}{支线任务:干预调整公式}	%23
		\item \href{https://mp.weixin.qq.com/s/nYpB7nC794hEPtlY8eY9WQ}{结果模型}	%24
		%\item \href{url}{Materials}
	\end{enumerate}
\end{multicols}

\subsubsection*{\href{https://space.bilibili.com/421438815/lists/813891?type=season}{\kaishu 倾向得分匹配:}}

\vspace{-0.25cm}

\begin{enumerate}
	\item \href{URLhttps://mp.weixin.qq.com/s/nqHhY8w6u1Od2Mfl6e97Fg}{简介, 鲁宾因果模型}
	\item \href{https://mp.weixin.qq.com/s/Uu3KEcNG3ECdbC8bL_NtKA}{潜在结果, 随机实验, 观测数据, 条件独立, 鲁宾因果模型, 假想随机实验}
	\item \href{https://mp.weixin.qq.com/s/gLDTBrMqyW93nOzYDJy6IQ}{倾向得分定理, 假想随机实验, Rosenbaum and Robin(1983), 近邻匹配, 半径匹配, 核匹配}
	\item \href{https://mp.weixin.qq.com/s/v0Kq34XXf0jGO0gtS-BTng}{Stata 操作详解: 计算倾向得分, 倾向得分匹配指令}
	\item \href{https://mp.weixin.qq.com/s/uUXJzggQCx--5i5vs_-_hw}{Stata 操作详解: 倾向得分匹配指令, 匹配对象读取, 检验匹配效果}
	\item \href{https://mp.weixin.qq.com/s/NvAopch22BI4-IJc07E0eg}{Stata 操作详解: 倾向得分结果解读, 匹配数据回归}
\end{enumerate}


\subsubsection*{\href{https://space.bilibili.com/421438815/lists/1043569?type=season}{\kaishu 工具变量:}}

\vspace{-0.25cm}

\begin{enumerate}
	\item \href{https://mp.weixin.qq.com/s/ZT9hly47Fa3-Ag8YiwjS1A}{工具变量 instrumental variables, 两阶段最小二乘法 2SLS,相关性, 排他性, 联立方程, 简约式}
	\item \href{https://mp.weixin.qq.com/s/iMTV2k1a_cnAFDHwz8b04Q}{弱工具变量, F 检验, 偏 R 方, 过度识别检验, 相关性, 排他性, 可识别}
	\item \href{https://mp.weixin.qq.com/s/c0IGE8_P0R_E0BrJd9_GRw}{2SLS 还是OLS? 豪斯曼检验,内生性检验}
	\item \href{https://mp.weixin.qq.com/s/ymBux95BhKxKsXMrj7dGpg}{被禁止的回归(比赛), 虚拟内生变量, 01内生变量, 二元内生变量, 非线性内生变量}
	\item \href{https://mp.weixin.qq.com/s/-Qtk2aBJajDju1zYjHyRcg}{Stata操作详解: ivregress, first, firststage, overid, endog}
\end{enumerate}


\subsubsection*{\href{https://space.bilibili.com/421438815/lists?sid=339394&spm_id_from=333.788.0.0}{\kaishu Logit回归模型:}}

\vspace{-0.25cm}

\begin{enumerate}
	\item \href{https://mp.weixin.qq.com/s/1pclvq5JJV-Xg5InK0hK5g}{logit vs probit, logit vs logistic, 发生比 odds, 比值比 odds ratio}
	\item \href{https://mp.weixin.qq.com/s/ehJ9mesEokfN2hIWNcJTHg}{最大似然估计法, 似然函数, log-likelihood function, 连接函数}
	\item \href{https://mp.weixin.qq.com/s/EtMHuAzlLjGxnX8G4IUTEQ}{McFadden's 伪R方, 似然比检验, 对数似然函数验算}
	\item \href{https://mp.weixin.qq.com/s/x32OBDqqvSABPAzCRcejOA}{logit 回归系数的解读, Z 统计量, 标准误和置信区间}
\end{enumerate}



%%%%%%%%%%%%%%%%%%%%%%%%%%%%%%%%%%%%%%%%%%%%%%%%%%%%%%%%%%%%%%%%%%%%%%%%%%%%%%%%%%%%%%%%%%%%%%%%%%%%%%%%%%%%%%%%%%%%%%%
%\bibliographystyle{ieeetr} % number
%%\bibliographystyle{unsrtnat} % author year
%\bibliography{HeBib}
%%%%%%%%%%%%%%%%%%%%%%%%%%%%%%%%%%%%%%%%%%%%%%%%%%%%%%%%%%%%%%%%%%%%%%%%%%%%%%%%%%%%%%%%%%%%%%%%%%%%%%%%%%%%%%%%%%%%%%%
\begin{flushright}
	\tiny \today 
\end{flushright}
%%%%%%%%%%%%%%%%%%%%%%%%%%%%%%%%%%%%%%%%%%%%%%%%%%%%%%%%%%%%%%%%%%%%%%%%%%%%%%%%%%%%%%%%%%%%%%%%%%%%%%%%%%%%%%%%%%%%%%%
\end{document}
%%%%%%%%%%%%%%%%%%%%%%%%%%%%%%%%%%%%%%%%%%%%%%%%%%%%%%%%%%%%%%%%%%%%%%%%%%%%%%%%%%%%%%%%%%%%%%%%%%%%%%%%%%%%%%%%%%%%%%%
              