%%%%%%%%%%%%%%%%%%%%%%%%%%%%%%%%%%%%%%%%%%%%%%%%%%%%%%%%%%%%%%%%%%%%%%%%%%%%%%%%%%%%%%%%%%%%%%%%%%%%%%%%%%%%%%%%%%%%%
%%%%%%%%%%%%%%%%%%%%%%%%%%%%%%%%%%%%%%%%%%%%%   Author:Yao Zhang  %%%%%%%%%%%%%%%%%%%%%%%%%%%%%%%%%%%%%%%%%%%%%%%%%%%
%%%%%%%%%%%%%%%%%%%%%%%%%%%%%%%%%%%%%%%%%%%%% Email: jaafar_zhang@163.com %%%%%%%%%%%%%%%%%%%%%%%%%%%%%%%%%%%%%%%%%%%
%%%%%%%%%%%%%%%%%%%%%%%%%%%%%%%%%%%%%%%%%%%%%%%%%%%%%%%%%%%%%%%%%%%%%%%%%%%%%%%%%%%%%%%%%%%%%%%%%%%%%%%%%%%%%%%%%%%%%
\documentclass[11pt]{article}
\usepackage{babel}
\usepackage[utf8]{inputenc} 
\usepackage[table]{xcolor}
\usepackage[most]{tcolorbox}
\usepackage[left=2.50cm, right=1.50cm, top=2.0cm, bottom=2.50cm]{geometry}
\usepackage{xcolor,url}
\usepackage{amsmath,amsthm,amsfonts,amssymb,amscd,multirow,booktabs,fullpage,calc,multicol}
\usepackage{lastpage,enumitem,fancyhdr,mathrsfs,wrapfig,setspace,cancel,amsmath,empheq,framed}
\usepackage[retainorgcmds]{IEEEtrantools}
\usepackage{subfig,graphicx,framed}
\usepackage{ctex}
\usepackage{txfonts}
\usepackage{bbm}
\usepackage{chngcntr}
\usepackage[colorlinks,linkcolor=blue,anchorcolor=green,citecolor=red,urlcolor=blue]{hyperref}
\usepackage{titlesec}
%%%%%%%%%%%%%%%%%%%%%%%%%%%%%%%%%%%%%%%%%%%%%%%%%%%%%%%%%%%%%%%%%%%%%%%%%%%%%%%%%%%%%%%%%%%%%%%%%%%%%%%%%%%%%%%%%%%%%%
\newtheorem{thm}{Theorem}[section]
\newtheorem{defi}{Definition}[subsection]
\newtheorem{exercise}{Exercise}[subsection]
\newtheorem{note}{Note}[subsection]
\newtheorem{notation}{Notation}
\newtheorem{lemma}{Lemma}[subsection]
\newtheorem{proposition}{Proposition}[subsection]
\newtheorem{example}{Example}[subsection]
\newtheorem{problem}{Problem}[section]
\newtheorem{homework}{Homework}[section]
\newtheorem{summary}{Summary}[subsection]
\newtheorem{corollary}{Corollary}[subsection]
\newtheorem{rmk}{Remark}[section]
%%%%%%%%%%%%%%%%%%%%%%%%%%%%%%%%%%%%%%%%%%%%%%%%%%%%%%%%%%%%%%%%%%%%%%%%%%%%%%%%%%%%%%%%%%%%%%%%%%%%%%%%%%%%%%%%%%%%%
\newlength{\tabcont}
\setlength{\parindent}{0.0in}
\setlength{\parskip}{0.05in}
\colorlet{shadecolor}{orange!15}
\parindent 0in
\parskip 12pt
\geometry{margin=1in, headsep=0.25in}
%%%%%%%%%%%%%%%%%%%%%%%%%%%%%%%%%%%%%%%%%%%%%%%%%%%%%%%%%%%%%%%%%%%%%%%%%%%%%%%%%%%%%%%%%%%%%%%%%%%%%%%%%%%%%%%%%%%%%
\graphicspath{ {img/EoM/}}
%%%%%%%%%%%%%%%%%%%%%%%%%%%%%%%%%%%%%%%%%%%%%%%%%%%%%%%%%%%%%%%%%%%%%%%%%%%%%%%%%%%%%%%%%%%%%%%%%%%%%%%%%%%%%%%%%%%%%
%\renewcommand{\cite}[1]{[#1]}
\makeatletter
\@addtoreset{equation}{section}
\makeatother
\renewcommand{\theequation}{\arabic{section}.\arabic{equation}}
\renewcommand{\contentsname}{\centering \small \color{blue} Contents}
%\counterwithin{figure}{section}
\renewcommand{\figurename}{\textbf{Fig.}}
%\renewcommand{\refname}{\textbf{\kaishu 参考文献}}
\renewcommand{\refname}{\textbf{Bibliography}}
\setcounter{secnumdepth}{4}
\titleformat{\paragraph}
{\normalfont\normalsize\bfseries}{\theparagraph}{1em}{}
\titlespacing*{\paragraph}{0pt}{3.25ex plus 1ex minus .2ex}{1.5ex plus .2ex}
\def\beginrefs{\begin{list}%
		{[\arabic{equation}]}{\usecounter{equation}
			\setlength{\leftmargin}{0.8truecm}\setlength{\labelsep}{0.4truecm}%
			\setlength{\labelwidth}{1.6truecm}}}
	\def\endrefs{\end{list}}
\def\bibentry#1{\item[\hbox{[#1]}]}
%%%%%%%%%%%%%%%%%%%%%%%%%%%%%%%%%%%%%%%%%%%%%%%%%%%%%%%%%%%%%%%%%%%%%%%%%%%%%%%%%%%%%%%%%%%%%%%%%%%%%%%%%%%%%%%%%%%%%%
%\begin{figure}[!htb]
%	\centering
%	\subfloat[$A \cap B$]{%
%		\includegraphics[width=0.3\linewidth,height=0.2\linewidth]{img001.jpg}}
%	\label{img001}\qquad \qquad %\hfill
%	\subfloat[${A_1} \cap {A_2} \cap {A_3}$]{%
%		\includegraphics[width=0.3\linewidth,height=0.2\linewidth]{img002.jpg}}
%	\label{img002}
	%\caption{ Examples.}
%\end{figure}
%\begin{figure}[!htb]
%	\centering
%	\includegraphics[width=0.4\linewidth,height=0.3\linewidth]{img005.jpg}
%	\label{img005}
	%\caption{ illustration for $ 3 $}
%\end{figure}
%\={a}1 \'{a}2\v{a}3\.{a}4

\usepackage{datetime}
\renewcommand{\today}{\shortmonthname[\the\month] \the \day,  \the\year}
%%%%%%%%%%%%%%%%%%%%%%%%%%%%%%%%%%%%%%%%%%%%%%%%%%%%%%%%%%%%%%%%%%%%%%%%%%%%%%%%%%%%%%%%%%%%%%%%%%%%%%%%%%%%%%%%%%%%%%
\begin{document}
	\kaishu 
	%\thispagestyle{empty}
	\setcounter{section}{0}
	\begin{center}
		{\LARGE  \href{https://www.youtube.com/playlist?list=PLISXH-iEM4Jk27AmSvISooRRKH4WtlWKP}{Automatic Differentiation, Adjoints, and Sensitivities}}
		
		%\vspace{-0.25cm}
		
		{\large \href{https://fkoehler.site/}{Felix Matteo K\"{o}hler}}
	\end{center}
%%%%%%%%%%%%%%%%%%%%%%%%%%%%%%%%%%%%%%%%%%%%%%%%%%%%%%%%%%%%%%%%%%%%%%%%%%%%%%%%%%%%%%%%%%%%%%%%%%%%%%%%%%%%%%%%%%%%%%
%%\newpage 
%%\thispagestyle{empty}	
%%%%%%%%%%%%%%%%%%%%%%%%%%%%%%%%%%%%%%%%%%%%%%%%%%%%%%%%%%%%%%%%%%%%%%%%%%%%%%%%%%%%%%%%%%%%%%%%%%%%%%%%%%%%%%%%%%%%%%
%\tableofcontents	
%{\pagestyle{empty}\mbox{}\newpage\pagestyle{empty}}
%\newpage 
%{\pagestyle{empty}\mbox{}\newpage\pagestyle{empty}}
%%%%%%%%%%%%%%%%%%%%%%%%%%%%%%%%%%%%%%%%%%%%%%%%%%%%%%%%%%%%%%%%%%%%%%%%%%%%%%%%%%%%%%%%%%%%%%%%%%%%%%%%%%%%%%%%%%%%%%
%%\newpage 
\setcounter{page}{1}

%\vspace{1.5cm}


%\vspace{-1cm}

\begin{enumerate}
	\item  \href{https://mp.weixin.qq.com/s/zFCMeqcESJfZCJjfu92sjg}{Adjoint Equation of a Linear System of Equations - by implicit derivative} %1
	\item  \href{https://mp.weixin.qq.com/s/b9yHy1WEKca1qda24E3aRw}{Adjoint Sensitivities of a Linear System of Equations - derived using the Lagrangian} %2
	\item  \href{https://mp.weixin.qq.com/s/L1wNHylcA3h7Qm9Hx0lcxw}{Python Example for the Adjoint Sensitivities of a Linear System  Full Details \& Timings} %3
	\item  \href{https://mp.weixin.qq.com/s/4OILR7Xf64M-rDkLNH4s5g}{Adjoint Sensitivities of a Non-Linear system of equations  Full Derivation} %4
	\item  \href{https://mp.weixin.qq.com/s/HNPeePbm_ctDeGclViHi6g}{Lagrangian Perspective on the Derivation of Adjoint Sensitivities of Nonlinear Systems} %5
	\item  \href{https://mp.weixin.qq.com/s/WtHAOIb27wdFaRImCLWUYw}{Python Example for Adjoint Sensitivities of Nonlinear Equation} %6
	\item  \href{https://mp.weixin.qq.com/s/3okjGSJq5GDRC6cQscSDxw}{Adjoint State Method for an ODE  Adjoint Sensitivity Analysis} %7
	\item  \href{https://mp.weixin.qq.com/s/b1QZQSVmr07hIXjNewDyYA}{Adjoint Sensitivities over nonlinear equation with JAX Automatic Differentiation} %8
	\item  \href{https://mp.weixin.qq.com/s/CZBsfhYkLenCRGnMju0KRA}{Adjoint Sensitivities in Julia with Zygote \& ChainRules} %9
	\item  \href{https://mp.weixin.qq.com/s/i_4D0CPukDh7f4x33VNgng}{Python Example Adjoint Sensitivities over nonlinear SYSTEMS of equations} %10
	\item  \href{https://mp.weixin.qq.com/s/e5H0Z6RkdjjAk_Dq6Laqbw}{Using JAX Jacobians for Adjoint Sensitivities over Nonlinear Systems of Equations} %11
	\item  \href{https://mp.weixin.qq.com/s/GTiBXck7mhtdXxv7k1XNQA}{What is a Jacobian-Vector product (jvp) in JAX} %12
	\item  \href{https://mp.weixin.qq.com/s/K8HVYGsHW9sgkNcJDL0d8g}{Naive Jacobian-vector product vs. JAX.jvp  Benchmark comparison} %13
	\item  \href{https://mp.weixin.qq.com/s/DOz5Uc9RsfcyfNwROYKHpg}{What is a vector-Jacobian product (vjp) in JAX} %14
	\item  \href{https://mp.weixin.qq.com/s/dWnLDP7_iWhxnsEEfm0bZA}{Naive vector-Jacobian product vs JAX.vjp  Benchmark comparison} %15
	\item  \href{https://mp.weixin.qq.com/s/jMFEBU2sJe-k_USHUsRqbQ}{Jacobian-vector product (Jvp) with ForwardDiff.jl in Julia} %16
	\item  \href{https://mp.weixin.qq.com/s/Bgd7OYQdDVkOGf4Syf_veQ}{What is a Pullback in Zygote.jl  vector-Jacobian products in Julia} %17
	\item  \href{https://mp.weixin.qq.com/s/ZJTDVi14YLw-HiE4zRoLvg}{Full Jacobian Matrix using forward-mode AD in JAX} %18
	\item  \href{https://mp.weixin.qq.com/s/BGN4wbaS9bEIS0H-7eJ-pQ}{Full Jacobian using reverse-mode AD in JAX} %19
	\item \href{https://pan.baidu.com/s/1reoxKyen6z699E7lKxB5xw}{Materials} 
\end{enumerate}

%%%%%%%%%%%%%%%%%%%%%%%%%%%%%%%%%%%%%%%%%%%%%%%%%%%%%%%%%%%%%%%%%%%%%%%%%%%%%%%%%%%%%%%%%%%%%%%%%%%%%%%%%%%%%%%%%%%%%%%
%\bibliographystyle{ieeetr} % number
%%\bibliographystyle{unsrtnat} % author year
%\bibliography{HeBib}
%%%%%%%%%%%%%%%%%%%%%%%%%%%%%%%%%%%%%%%%%%%%%%%%%%%%%%%%%%%%%%%%%%%%%%%%%%%%%%%%%%%%%%%%%%%%%%%%%%%%%%%%%%%%%%%%%%%%%%%
\begin{flushright}
	\tiny \today 
\end{flushright}
%%%%%%%%%%%%%%%%%%%%%%%%%%%%%%%%%%%%%%%%%%%%%%%%%%%%%%%%%%%%%%%%%%%%%%%%%%%%%%%%%%%%%%%%%%%%%%%%%%%%%%%%%%%%%%%%%%%%%%%
\end{document}
%%%%%%%%%%%%%%%%%%%%%%%%%%%%%%%%%%%%%%%%%%%%%%%%%%%%%%%%%%%%%%%%%%%%%%%%%%%%%%%%%%%%%%%%%%%%%%%%%%%%%%%%%%%%%%%%%%%%%%%
              