%%%%%%%%%%%%%%%%%%%%%%%%%%%%%%%%%%%%%%%%%%%%%%%%%%%%%%%%%%%%%%%%%%%%%%%%%%%%%%%%%%%%%%%%%%%%%%%%%%%%%%%%%%%%%%%%%%%%%
%%%%%%%%%%%%%%%%%%%%%%%%%%%%%%%%%%%%%%%%%%%%%   Author:Yao Zhang  %%%%%%%%%%%%%%%%%%%%%%%%%%%%%%%%%%%%%%%%%%%%%%%%%%%
%%%%%%%%%%%%%%%%%%%%%%%%%%%%%%%%%%%%%%%%%%%%% Email: jaafar_zhang@163.com %%%%%%%%%%%%%%%%%%%%%%%%%%%%%%%%%%%%%%%%%%%
%%%%%%%%%%%%%%%%%%%%%%%%%%%%%%%%%%%%%%%%%%%%%%%%%%%%%%%%%%%%%%%%%%%%%%%%%%%%%%%%%%%%%%%%%%%%%%%%%%%%%%%%%%%%%%%%%%%%%
\documentclass[11pt]{article}
\usepackage{babel}
\usepackage[utf8]{inputenc} 
\usepackage[table]{xcolor}
\usepackage[most]{tcolorbox}
\usepackage[left=2.50cm, right=1.50cm, top=2.0cm, bottom=2.50cm]{geometry}
\usepackage{xcolor,url}
\usepackage{amsmath,amsthm,amsfonts,amssymb,amscd,multirow,booktabs,fullpage,calc,multicol}
\usepackage{lastpage,enumitem,fancyhdr,mathrsfs,wrapfig,setspace,cancel,amsmath,empheq,framed}
\usepackage[retainorgcmds]{IEEEtrantools}
\usepackage{subfig,graphicx,framed}
\usepackage{ctex}
\usepackage{txfonts}
\usepackage{bbm}
\usepackage{chngcntr}
\usepackage[colorlinks,linkcolor=blue,anchorcolor=green,citecolor=red,urlcolor=blue]{hyperref}
\usepackage{titlesec}
%%%%%%%%%%%%%%%%%%%%%%%%%%%%%%%%%%%%%%%%%%%%%%%%%%%%%%%%%%%%%%%%%%%%%%%%%%%%%%%%%%%%%%%%%%%%%%%%%%%%%%%%%%%%%%%%%%%%%%
\newtheorem{thm}{Theorem}[section]
\newtheorem{defi}{Definition}[subsection]
\newtheorem{exercise}{Exercise}[subsection]
\newtheorem{note}{Note}[subsection]
\newtheorem{notation}{Notation}
\newtheorem{lemma}{Lemma}[subsection]
\newtheorem{proposition}{Proposition}[subsection]
\newtheorem{example}{Example}[subsection]
\newtheorem{problem}{Problem}[section]
\newtheorem{homework}{Homework}[section]
\newtheorem{summary}{Summary}[subsection]
\newtheorem{corollary}{Corollary}[subsection]
\newtheorem{rmk}{Remark}[section]
%%%%%%%%%%%%%%%%%%%%%%%%%%%%%%%%%%%%%%%%%%%%%%%%%%%%%%%%%%%%%%%%%%%%%%%%%%%%%%%%%%%%%%%%%%%%%%%%%%%%%%%%%%%%%%%%%%%%%
\newlength{\tabcont}
\setlength{\parindent}{0.0in}
\setlength{\parskip}{0.05in}
\colorlet{shadecolor}{orange!15}
\parindent 0in
\parskip 12pt
\geometry{margin=1in, headsep=0.25in}
%%%%%%%%%%%%%%%%%%%%%%%%%%%%%%%%%%%%%%%%%%%%%%%%%%%%%%%%%%%%%%%%%%%%%%%%%%%%%%%%%%%%%%%%%%%%%%%%%%%%%%%%%%%%%%%%%%%%%
\graphicspath{ {img/EoM/}}
%%%%%%%%%%%%%%%%%%%%%%%%%%%%%%%%%%%%%%%%%%%%%%%%%%%%%%%%%%%%%%%%%%%%%%%%%%%%%%%%%%%%%%%%%%%%%%%%%%%%%%%%%%%%%%%%%%%%%
%\renewcommand{\cite}[1]{[#1]}
\makeatletter
\@addtoreset{equation}{section}
\makeatother
\renewcommand{\theequation}{\arabic{section}.\arabic{equation}}
\renewcommand{\contentsname}{\centering \small \color{blue} Contents}
%\counterwithin{figure}{section}
\renewcommand{\figurename}{\textbf{Fig.}}
%\renewcommand{\refname}{\textbf{\kaishu 参考文献}}
\renewcommand{\refname}{\textbf{Bibliography}}
\setcounter{secnumdepth}{4}
\titleformat{\paragraph}
{\normalfont\normalsize\bfseries}{\theparagraph}{1em}{}
\titlespacing*{\paragraph}{0pt}{3.25ex plus 1ex minus .2ex}{1.5ex plus .2ex}
\def\beginrefs{\begin{list}%
		{[\arabic{equation}]}{\usecounter{equation}
			\setlength{\leftmargin}{0.8truecm}\setlength{\labelsep}{0.4truecm}%
			\setlength{\labelwidth}{1.6truecm}}}
	\def\endrefs{\end{list}}
\def\bibentry#1{\item[\hbox{[#1]}]}
%%%%%%%%%%%%%%%%%%%%%%%%%%%%%%%%%%%%%%%%%%%%%%%%%%%%%%%%%%%%%%%%%%%%%%%%%%%%%%%%%%%%%%%%%%%%%%%%%%%%%%%%%%%%%%%%%%%%%%
%\begin{figure}[!htb]
%	\centering
%	\subfloat[$A \cap B$]{%
%		\includegraphics[width=0.3\linewidth,height=0.2\linewidth]{img001.jpg}}
%	\label{img001}\qquad \qquad %\hfill
%	\subfloat[${A_1} \cap {A_2} \cap {A_3}$]{%
%		\includegraphics[width=0.3\linewidth,height=0.2\linewidth]{img002.jpg}}
%	\label{img002}
	%\caption{ Examples.}
%\end{figure}
%\begin{figure}[!htb]
%	\centering
%	\includegraphics[width=0.4\linewidth,height=0.3\linewidth]{img005.jpg}
%	\label{img005}
	%\caption{ illustration for $ 3 $}
%\end{figure}
%\={a}1 \'{a}2\v{a}3\.{a}4

\usepackage{datetime}
\renewcommand{\today}{\shortmonthname[\the\month] \the \day,  \the\year}
%%%%%%%%%%%%%%%%%%%%%%%%%%%%%%%%%%%%%%%%%%%%%%%%%%%%%%%%%%%%%%%%%%%%%%%%%%%%%%%%%%%%%%%%%%%%%%%%%%%%%%%%%%%%%%%%%%%%%%
\begin{document}
	\kaishu 
	\thispagestyle{empty}
	\setcounter{section}{1}
	\begin{center}
		{\LARGE  \href{https://www.youtube.com/playlist?list=PLemsnf33Vij4-kv-JTjDthaGUYUnQbbws}{Advanced Machine Learning}}
		
		\vspace{0.25cm}
		
		{\large \href{https://www.mpg.de/10655393/science-of-light-marquardt}{Florian Marquardt}}
	\end{center}
%%%%%%%%%%%%%%%%%%%%%%%%%%%%%%%%%%%%%%%%%%%%%%%%%%%%%%%%%%%%%%%%%%%%%%%%%%%%%%%%%%%%%%%%%%%%%%%%%%%%%%%%%%%%%%%%%%%%%%
%%\newpage 
%%\thispagestyle{empty}	
%%%%%%%%%%%%%%%%%%%%%%%%%%%%%%%%%%%%%%%%%%%%%%%%%%%%%%%%%%%%%%%%%%%%%%%%%%%%%%%%%%%%%%%%%%%%%%%%%%%%%%%%%%%%%%%%%%%%%%
%\tableofcontents	
%{\pagestyle{empty}\mbox{}\newpage\pagestyle{empty}}
%\newpage 
%{\pagestyle{empty}\mbox{}\newpage\pagestyle{empty}}
%%%%%%%%%%%%%%%%%%%%%%%%%%%%%%%%%%%%%%%%%%%%%%%%%%%%%%%%%%%%%%%%%%%%%%%%%%%%%%%%%%%%%%%%%%%%%%%%%%%%%%%%%%%%%%%%%%%%%%
%%\newpage 
\setcounter{page}{1}

%\vspace{1.5cm}


\begin{enumerate}
	\item \href{https://mp.weixin.qq.com/s/mzK9xnrxmOKGNyVy4fHocQ}{Lecture 1: Introduction. Artificial Scientific Discovery.} %1
	\item \href{https://mp.weixin.qq.com/s/2ulUaFS__dZurnQ70kmWWQ}{Lecture 2: Basic neural network structure.} %2
	\item \href{https://mp.weixin.qq.com/s/94F0Q32cobXLfBuKeAsaog}{Lecture 3: Stochastic Gradient Descent. Backpropagation} %3
	\item \href{https://mp.weixin.qq.com/s/0gZzScsgtwV1WHjSSkziUw}{Lecture 4: Loss functions. Overfitting. Dropout. Adaptive Gradient Descent. Convolutional networks.}%4
	\item \href{https://mp.weixin.qq.com/s/ggM9hb298_Mxe8JKjRVI_Q}{Lecture 5: Representation Learning: Goals. Principal Component Analysis.}%5
	\item \href{https://mp.weixin.qq.com/s/H61taZpRSTCBQqykXs-70g}{Lecture 6: Basic Autoencoders.}%6
	\item \href{https://mp.weixin.qq.com/s/vyJukgqDgnBByaqotPtedA}{Lecture 7: Contractive Autoencoder. Shannon's Information Theory: Compression and Information.}%7
	\item \href{https://mp.weixin.qq.com/s/WjIuCGRhNVE-BlhrHEbGWA}{Lecture 8: Entropy. Bayes formula.}%8
	\item \href{https://mp.weixin.qq.com/s/CBjRzYZOudjrSp1UTZlI4Q}{Lecture 9: Bayes. Gaussian Random Processes.}%9
	\item \href{https://mp.weixin.qq.com/s/uan1k-91A49GFpcgM4F6kQ}{Lecture 10: Inductive Bias. Fisher Information. Information Geometry.}%10
	\item \href{https://mp.weixin.qq.com/s/I6Aa-ZAFrye0qlfhZquOVQ}{Lecture 11: Natural Gradient. Kullback-Leibler Divergence. Mutual Information.}%11
	\item \href{https://mp.weixin.qq.com/s/xZs7W0xHQQY_EpSMonqkWQ}{Lecture 12: Mutual Information. Learning Probability Distributions. Normalizing Flows.}%12
	\item \href{https://mp.weixin.qq.com/s/XA5j3-sfTkUyrwEe0iF2fA}{Lecture 13: Invertible Neural Networks. Convolutional and Conditional Invertible Networks.}%13
	\item \href{https://mp.weixin.qq.com/s/uENPzwLW7mau1l-jN8qsVA}{Lecture 14: Boltzmann Machines (General Theory).}%14
	\item \href{https://mp.weixin.qq.com/s/gELMbha7Xw8SAXZGKmsH4w}{Lecture 15: Restricted Boltzmann Machines. Conditional Sampling. Variational Autoencoder.}%15
	\item \href{https://mp.weixin.qq.com/s/i_QjCpibflejgeUql1qojQ}{Lecture 16: Variational Autoencoder. Generative Adversarial Networks.}%16
	\item \href{https://mp.weixin.qq.com/s/RbPXU8Es-ipy9bOt1zeKcA}{Lecture 17: Generative Adversarial Networks. Recurrent Neural Networks.}%17
	\item \href{https://mp.weixin.qq.com/s/4VGsSCHsf2Jsp6WmaRivAw}{Lecture 18: Recurrent Neural Networks. Graph Neural Networks.}%18
	\item \href{https://mp.weixin.qq.com/s/8kGlkNaP9rnTIcBE6d46aA}{Lecture 19: Graph Neural Networks. Attention Mechanisms (Basics).}%19
	\item \href{https://mp.weixin.qq.com/s/GvGFZBGskYBRd89KH1MvCg}{Lecture 20: Attention. Differentiable Neural Computer. Transformers.}%20
	\item \href{https://mp.weixin.qq.com/s/jI3TAmtOXYaqy3U2nRN-gg}{Lecture 21: Transformers (and examples). Implicit Layers.}%21
	\item \href{https://mp.weixin.qq.com/s/t0yu1kgJZplIHeoe_gzmeA}{Lecture 22: Implicit Layers. Hamiltonian and Lagrangian Networks. Reinforcement Learning Overview.}%22
	\item \href{https://mp.weixin.qq.com/s/nnghVZXxebGAPPRU61wEXw}{Lecture 23: Reinforcement Learning - Policy Gradient and Q-Learning.}%23
	\item \href{https://mp.weixin.qq.com/s/SK0vRxXBOLFTYe-75DL35g}{Lecture 24: Advantage Actor-Critic. Trust Regions. Proximal Policy Optimization.}%24
	\item \href{https://mp.weixin.qq.com/s/XejVx34Xeu9MTbIENMuJJw}{Lecture 25: Reinforcement Learning: Continuous actions. Model-based. Monte Carlo Tree Search.}%25
	\item \href{https://mp.weixin.qq.com/s/nG5CdFTxCK48HwxPLAKlUA}{Lecture 26: Active Learning for Network Training: Uncertainty Sampling and other approaches.}%26
	\item \href{https://mp.weixin.qq.com/s/jjAWohCS942dOftl4-RO5A}{ Lecture 27: Bayesian Optimal Experimental Design. Active Learning: Gaussian Processes and Networks.}%27
	\item \href{https://mp.weixin.qq.com/s/lBQL5RmAbNq2QuiHWl4YPA}{Lecture 28: Turing Machines. Algorithmic (Kolomogoroff) Complexity. Universal (Levin) Search.}%28
	\item \href{https://mp.weixin.qq.com/s/aWXLGFTHQqYc0wGxARo2kA}{Lecture 29: Solomonoff's Algorithmic Probability and Theory of Induction. Conclusions/Outlook.}%29
	\item \href{https://mp.weixin.qq.com/s/x7Ll4Jz3mXs14K1qYtcGxg}{Animation: Generative Adversarial Network}%30
	\item \href{https://mp.weixin.qq.com/s/EeytQR4GIti8wApsEFEnaw}{Animation: Graph Neural Network predicting Quantum Ground States}%31
	\item \href{https://mp.weixin.qq.com/s/OVYGO7-Kfq0ogsHxEPItpw}{Animation: Normalizing Flow (Invertible Neural Network)}%32
	\item \href{https://mp.weixin.qq.com/s/9mDR44X6quFyEXVedwbaSQ}{Animation: Variational Autoencoder}%33
	\item \href{url}{Materials}
\end{enumerate}


%%%%%%%%%%%%%%%%%%%%%%%%%%%%%%%%%%%%%%%%%%%%%%%%%%%%%%%%%%%%%%%%%%%%%%%%%%%%%%%%%%%%%%%%%%%%%%%%%%%%%%%%%%%%%%%%%%%%%%%
%\bibliographystyle{ieeetr} % number
%%\bibliographystyle{unsrtnat} % author year
%\bibliography{HeBib}
%%%%%%%%%%%%%%%%%%%%%%%%%%%%%%%%%%%%%%%%%%%%%%%%%%%%%%%%%%%%%%%%%%%%%%%%%%%%%%%%%%%%%%%%%%%%%%%%%%%%%%%%%%%%%%%%%%%%%%%
\begin{flushright}
	\tiny \today 
\end{flushright}
%%%%%%%%%%%%%%%%%%%%%%%%%%%%%%%%%%%%%%%%%%%%%%%%%%%%%%%%%%%%%%%%%%%%%%%%%%%%%%%%%%%%%%%%%%%%%%%%%%%%%%%%%%%%%%%%%%%%%%%
\end{document}
%%%%%%%%%%%%%%%%%%%%%%%%%%%%%%%%%%%%%%%%%%%%%%%%%%%%%%%%%%%%%%%%%%%%%%%%%%%%%%%%%%%%%%%%%%%%%%%%%%%%%%%%%%%%%%%%%%%%%%%
              