%%%%%%%%%%%%%%%%%%%%%%%%%%%%%%%%%%%%%%%%%%%%%%%%%%%%%%%%%%%%%%%%%%%%%%%%%%%%%%%%%%%%%%%%%%%%%%%%%%%%%%%%%%%%%%%%%%%%%
%%%%%%%%%%%%%%%%%%%%%%%%%%%%%%%%%%%%%%%%%%%%%   Author:Yao Zhang  %%%%%%%%%%%%%%%%%%%%%%%%%%%%%%%%%%%%%%%%%%%%%%%%%%%
%%%%%%%%%%%%%%%%%%%%%%%%%%%%%%%%%%%%%%%%%%%%% Email: jaafar_zhang@163.com %%%%%%%%%%%%%%%%%%%%%%%%%%%%%%%%%%%%%%%%%%%
%%%%%%%%%%%%%%%%%%%%%%%%%%%%%%%%%%%%%%%%%%%%%%%%%%%%%%%%%%%%%%%%%%%%%%%%%%%%%%%%%%%%%%%%%%%%%%%%%%%%%%%%%%%%%%%%%%%%%
\documentclass[11pt]{article}
\usepackage{babel}
\usepackage[utf8]{inputenc} 
\usepackage[table]{xcolor}
\usepackage[most]{tcolorbox}
\usepackage[left=2.50cm, right=1.50cm, top=2.0cm, bottom=2.50cm]{geometry}
\usepackage{xcolor,url}
\usepackage{amsmath,amsthm,amsfonts,amssymb,amscd,multirow,booktabs,fullpage,calc,multicol}
\usepackage{lastpage,enumitem,fancyhdr,mathrsfs,wrapfig,setspace,cancel,amsmath,empheq,framed}
\usepackage[retainorgcmds]{IEEEtrantools}
\usepackage{subfig,graphicx,framed}
\usepackage{ctex}
\usepackage{txfonts}
\usepackage{bbm}
\usepackage{chngcntr}
\usepackage[colorlinks,linkcolor=blue,anchorcolor=green,citecolor=red,urlcolor=blue]{hyperref}
\usepackage{titlesec}
%%%%%%%%%%%%%%%%%%%%%%%%%%%%%%%%%%%%%%%%%%%%%%%%%%%%%%%%%%%%%%%%%%%%%%%%%%%%%%%%%%%%%%%%%%%%%%%%%%%%%%%%%%%%%%%%%%%%%%
\newtheorem{thm}{Theorem}[section]
\newtheorem{defi}{Definition}[subsection]
\newtheorem{exercise}{Exercise}[subsection]
\newtheorem{note}{Note}[subsection]
\newtheorem{notation}{Notation}
\newtheorem{lemma}{Lemma}[subsection]
\newtheorem{proposition}{Proposition}[subsection]
\newtheorem{example}{Example}[subsection]
\newtheorem{problem}{Problem}[section]
\newtheorem{homework}{Homework}[section]
\newtheorem{summary}{Summary}[subsection]
\newtheorem{corollary}{Corollary}[subsection]
\newtheorem{rmk}{Remark}[section]
\usepackage{romannum}
%%%%%%%%%%%%%%%%%%%%%%%%%%%%%%%%%%%%%%%%%%%%%%%%%%%%%%%%%%%%%%%%%%%%%%%%%%%%%%%%%%%%%%%%%%%%%%%%%%%%%%%%%%%%%%%%%%%%%
\newlength{\tabcont}
\setlength{\parindent}{0.0in}
\setlength{\parskip}{0.05in}
\colorlet{shadecolor}{orange!15}
\parindent 0in
\parskip 12pt
\geometry{margin=1in, headsep=0.25in}
%%%%%%%%%%%%%%%%%%%%%%%%%%%%%%%%%%%%%%%%%%%%%%%%%%%%%%%%%%%%%%%%%%%%%%%%%%%%%%%%%%%%%%%%%%%%%%%%%%%%%%%%%%%%%%%%%%%%%
\graphicspath{ {img/EoM/}}
%%%%%%%%%%%%%%%%%%%%%%%%%%%%%%%%%%%%%%%%%%%%%%%%%%%%%%%%%%%%%%%%%%%%%%%%%%%%%%%%%%%%%%%%%%%%%%%%%%%%%%%%%%%%%%%%%%%%%
%\renewcommand{\cite}[1]{[#1]}
\makeatletter
\@addtoreset{equation}{section}
\makeatother
\renewcommand{\theequation}{\arabic{section}.\arabic{equation}}
\renewcommand{\contentsname}{\centering \small \color{blue} Contents}
%\counterwithin{figure}{section}
\renewcommand{\figurename}{\textbf{Fig.}}
%\renewcommand{\refname}{\textbf{\kaishu 参考文献}}
\renewcommand{\refname}{\textbf{Bibliography}}
\setcounter{secnumdepth}{4}
\titleformat{\paragraph}
{\normalfont\normalsize\bfseries}{\theparagraph}{1em}{}
\titlespacing*{\paragraph}{0pt}{3.25ex plus 1ex minus .2ex}{1.5ex plus .2ex}
\def\beginrefs{\begin{list}%
		{[\arabic{equation}]}{\usecounter{equation}
			\setlength{\leftmargin}{0.8truecm}\setlength{\labelsep}{0.4truecm}%
			\setlength{\labelwidth}{1.6truecm}}}
	\def\endrefs{\end{list}}
\def\bibentry#1{\item[\hbox{[#1]}]}
%%%%%%%%%%%%%%%%%%%%%%%%%%%%%%%%%%%%%%%%%%%%%%%%%%%%%%%%%%%%%%%%%%%%%%%%%%%%%%%%%%%%%%%%%%%%%%%%%%%%%%%%%%%%%%%%%%%%%%
%\begin{figure}[!htb]
%	\centering
%	\subfloat[$A \cap B$]{%
%		\includegraphics[width=0.3\linewidth,height=0.2\linewidth]{img001.jpg}}
%	\label{img001}\qquad \qquad %\hfill
%	\subfloat[${A_1} \cap {A_2} \cap {A_3}$]{%
%		\includegraphics[width=0.3\linewidth,height=0.2\linewidth]{img002.jpg}}
%	\label{img002}
	%\caption{ Examples.}
%\end{figure}
%\begin{figure}[!htb]
%	\centering
%	\includegraphics[width=0.4\linewidth,height=0.3\linewidth]{img005.jpg}
%	\label{img005}
	%\caption{ illustration for $ 3 $}
%\end{figure}
%\={a}1 \'{a}2\v{a}3\.{a}4

\usepackage{datetime}
\renewcommand{\today}{\shortmonthname[\the\month] \the \day,  \the\year}
%%%%%%%%%%%%%%%%%%%%%%%%%%%%%%%%%%%%%%%%%%%%%%%%%%%%%%%%%%%%%%%%%%%%%%%%%%%%%%%%%%%%%%%%%%%%%%%%%%%%%%%%%%%%%%%%%%%%%%
\begin{document}
	\kaishu 
	%\thispagestyle{empty}
	\pagenumbering{arabic} 
	\setcounter{section}{0}
	\begin{center}
		{\LARGE  \href{https://www.youtube.com/playlist?list=PL5EH0ZJ7V0jV7kMYvPcZ7F9oaf_YAlfbI}{Asymptotics and Perturbation Methods}}
		
		%\vspace{-0.25cm}
		
		{\large \href{https://www.stevenstrogatz.com/}{Steven Strogatz}}
	\end{center}
%%%%%%%%%%%%%%%%%%%%%%%%%%%%%%%%%%%%%%%%%%%%%%%%%%%%%%%%%%%%%%%%%%%%%%%%%%%%%%%%%%%%%%%%%%%%%%%%%%%%%%%%%%%%%%%%%%%%%%
%%\newpage 
%%\thispagestyle{empty}	
%%%%%%%%%%%%%%%%%%%%%%%%%%%%%%%%%%%%%%%%%%%%%%%%%%%%%%%%%%%%%%%%%%%%%%%%%%%%%%%%%%%%%%%%%%%%%%%%%%%%%%%%%%%%%%%%%%%%%%
%\tableofcontents	
%{\pagestyle{empty}\mbox{}\newpage\pagestyle{empty}}
%\newpage 
%{\pagestyle{empty}\mbox{}\newpage\pagestyle{empty}}
%%%%%%%%%%%%%%%%%%%%%%%%%%%%%%%%%%%%%%%%%%%%%%%%%%%%%%%%%%%%%%%%%%%%%%%%%%%%%%%%%%%%%%%%%%%%%%%%%%%%%%%%%%%%%%%%%%%%%%
%%\newpage 
\setcounter{page}{1}

%\vspace{1.5cm}


\vspace{-1cm}

\begin{enumerate}
	\item \href{https://mp.weixin.qq.com/s/sq71mMisInVJuyHRySzlAw}{Asymptotic expansions}	%1
	\item \href{https://mp.weixin.qq.com/s/KzIk0Oj7zLyB_Puo9COHuw}{Properties of asymptotic expansions}	%2
	\item \href{https://mp.weixin.qq.com/s/qb6rrIVIlWOLocC_BRW_4w}{Integration by parts}	%3
	\item \href{https://mp.weixin.qq.com/s/uIfjRhImaUFtXWRVjMQRAQ}{Laplace's method}	%4
	\item \href{https://mp.weixin.qq.com/s/MZjEuLRphjS0iNl-caLZJg}{Stationary phase}	%5
	\item \href{https://mp.weixin.qq.com/s/ByfE7JW91TmJDuQV8R0YFw}{Steepest descent}	%6
	\item \href{https://mp.weixin.qq.com/s/fXvVQ6xx_moz8HlHRENjGA}{Saddle points}	%7
	\item \href{https://mp.weixin.qq.com/s/f5UyTJVPoBfHx25iw0YViA}{Integral representations and an introduction to dominant balance}	%8
	\item \href{https://mp.weixin.qq.com/s/gO9_j8554yTHZfZta38cDw}{Dominant balance}	%9
	\item \href{https://mp.weixin.qq.com/s/my-M5878d6mktzOqrsRExA}{Perturbation methods for algebraic equations}	%10
	\item \href{https://mp.weixin.qq.com/s/91bRTdyfNgZILGsDo5r7sg}{Regular perturbation methods for ODEs}	%11
	\item \href{https://mp.weixin.qq.com/s/zA8542-rzl0dnJ1UClxRrw}{Introduction to boundary layer theory}	%12
	\item \href{https://mp.weixin.qq.com/s/l1NGXmq7wrcXHcT_9JQUHw}{Higher-order matching in boundary layer theory}	%13
	\item \href{https://mp.weixin.qq.com/s/k_tlsdXpBujNoGdtE2sIzw}{Location and thickness of boundary layers}	%14
	\item \href{https://mp.weixin.qq.com/s/xdEtLg-j58nDsgEJyByq9g}{Corner layers}	%15
	\item \href{https://mp.weixin.qq.com/s/F8ObieL75Ahj80lWuZ9ABA}{A tricky nonlinear boundary-value problem}	%16
	\item \href{https://mp.weixin.qq.com/s/0Xs45-jAe63lXWbkloMLZA}{An application to systems biology: the Michaelis-Menten model}	%17
	\item \href{https://mp.weixin.qq.com/s/eRKm6Xak3qxKTH1pAzr4gQ}{Introduction to WKB theory}	%18
	\item \href{https://mp.weixin.qq.com/s/Z1Ls3uK0W_qV2s3o0nLv9w}{Turning points and Airy functions}	%19
	\item \href{https://mp.weixin.qq.com/s/HSYv_akuqiwpRzNNt3biAg}{WKB for eigenvalue problems}	%20
	\item \href{https://mp.weixin.qq.com/s/5pdjtd2sD281hJnhv7px0A}{Delayed bifurcation}	%21
	\item \href{https://mp.weixin.qq.com/s/rd1yvdzwger5BW3UfpNrlg}{Introduction to the method of multiple scales}	%22
	\item \href{https://mp.weixin.qq.com/s/9pnUFfxiHVSllfAe6i2b6w}{Two-timing}	%23
	\item \href{https://mp.weixin.qq.com/s/R_0Q_xOt137YUCV7nMraOw}{Aging spring and adiabatic invariants}	%24
	\item \href{https://mp.weixin.qq.com/s/IJe10L-_RWfEauvYZUl_MA}{Difference equations and multiple scales}	%25
	\item \href{https://mp.weixin.qq.com/s/oXJFjp5bDARCNzMnvQ5nPg}{PDEs and boundary layers}	%26
	\item \href{https://mp.weixin.qq.com/s/UAE3iMFgfNQCncWnuv61pg}{Renormalization and envelopes}	%27
	%\item \href{url}{Materials}
\end{enumerate}




%%%%%%%%%%%%%%%%%%%%%%%%%%%%%%%%%%%%%%%%%%%%%%%%%%%%%%%%%%%%%%%%%%%%%%%%%%%%%%%%%%%%%%%%%%%%%%%%%%%%%%%%%%%%%%%%%%%%%%%
%\bibliographystyle{ieeetr} % number
%%\bibliographystyle{unsrtnat} % author year
%\bibliography{HeBib}
%%%%%%%%%%%%%%%%%%%%%%%%%%%%%%%%%%%%%%%%%%%%%%%%%%%%%%%%%%%%%%%%%%%%%%%%%%%%%%%%%%%%%%%%%%%%%%%%%%%%%%%%%%%%%%%%%%%%%%%
\begin{flushright}
	\tiny \today 
\end{flushright}
%%%%%%%%%%%%%%%%%%%%%%%%%%%%%%%%%%%%%%%%%%%%%%%%%%%%%%%%%%%%%%%%%%%%%%%%%%%%%%%%%%%%%%%%%%%%%%%%%%%%%%%%%%%%%%%%%%%%%%%
\end{document}
%%%%%%%%%%%%%%%%%%%%%%%%%%%%%%%%%%%%%%%%%%%%%%%%%%%%%%%%%%%%%%%%%%%%%%%%%%%%%%%%%%%%%%%%%%%%%%%%%%%%%%%%%%%%%%%%%%%%%%%
              