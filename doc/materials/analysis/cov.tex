%%%%%%%%%%%%%%%%%%%%%%%%%%%%%%%%%%%%%%%%%%%%%%%%%%%%%%%%%%%%%%%%%%%%%%%%%%%%%%%%%%%%%%%%%%%%%%%%%%%%%%%%%%%%%%%%%%%%%
%%%%%%%%%%%%%%%%%%%%%%%%%%%%%%%%%%%%%%%%%%%%%   Author:Yao Zhang  %%%%%%%%%%%%%%%%%%%%%%%%%%%%%%%%%%%%%%%%%%%%%%%%%%%
%%%%%%%%%%%%%%%%%%%%%%%%%%%%%%%%%%%%%%%%%%%%% Email: jaafar_zhang@163.com %%%%%%%%%%%%%%%%%%%%%%%%%%%%%%%%%%%%%%%%%%%
%%%%%%%%%%%%%%%%%%%%%%%%%%%%%%%%%%%%%%%%%%%%%%%%%%%%%%%%%%%%%%%%%%%%%%%%%%%%%%%%%%%%%%%%%%%%%%%%%%%%%%%%%%%%%%%%%%%%%
\documentclass[11pt]{article}
\usepackage{babel}
\usepackage[utf8]{inputenc} 
\usepackage[table]{xcolor}
\usepackage[most]{tcolorbox}
\usepackage[left=2.50cm, right=1.50cm, top=2.0cm, bottom=2.50cm]{geometry}
\usepackage{xcolor,url}
\usepackage{amsmath,amsthm,amsfonts,amssymb,amscd,multirow,booktabs,fullpage,calc,multicol}
\usepackage{lastpage,enumitem,fancyhdr,mathrsfs,wrapfig,setspace,cancel,amsmath,empheq,framed}
\usepackage[retainorgcmds]{IEEEtrantools}
\usepackage{subfig,graphicx,framed}
\usepackage{ctex}
\usepackage{txfonts}
\usepackage{bbm}
\usepackage{chngcntr}
\usepackage[colorlinks,linkcolor=blue,anchorcolor=green,citecolor=red,urlcolor=blue]{hyperref}
\usepackage{titlesec}
%%%%%%%%%%%%%%%%%%%%%%%%%%%%%%%%%%%%%%%%%%%%%%%%%%%%%%%%%%%%%%%%%%%%%%%%%%%%%%%%%%%%%%%%%%%%%%%%%%%%%%%%%%%%%%%%%%%%%%
\newtheorem{thm}{Theorem}[section]
\newtheorem{defi}{Definition}[subsection]
\newtheorem{exercise}{Exercise}[subsection]
\newtheorem{note}{Note}[subsection]
\newtheorem{notation}{Notation}
\newtheorem{lemma}{Lemma}[subsection]
\newtheorem{proposition}{Proposition}[subsection]
\newtheorem{example}{Example}[subsection]
\newtheorem{problem}{Problem}[section]
\newtheorem{homework}{Homework}[section]
\newtheorem{summary}{Summary}[subsection]
\newtheorem{corollary}{Corollary}[subsection]
\newtheorem{rmk}{Remark}[section]
%%%%%%%%%%%%%%%%%%%%%%%%%%%%%%%%%%%%%%%%%%%%%%%%%%%%%%%%%%%%%%%%%%%%%%%%%%%%%%%%%%%%%%%%%%%%%%%%%%%%%%%%%%%%%%%%%%%%%
\newlength{\tabcont}
\setlength{\parindent}{0.0in}
\setlength{\parskip}{0.05in}
\colorlet{shadecolor}{orange!15}
\parindent 0in
\parskip 12pt
\geometry{margin=1in, headsep=0.25in}
%%%%%%%%%%%%%%%%%%%%%%%%%%%%%%%%%%%%%%%%%%%%%%%%%%%%%%%%%%%%%%%%%%%%%%%%%%%%%%%%%%%%%%%%%%%%%%%%%%%%%%%%%%%%%%%%%%%%%
\graphicspath{ {img/EoM/}}
%%%%%%%%%%%%%%%%%%%%%%%%%%%%%%%%%%%%%%%%%%%%%%%%%%%%%%%%%%%%%%%%%%%%%%%%%%%%%%%%%%%%%%%%%%%%%%%%%%%%%%%%%%%%%%%%%%%%%
%\renewcommand{\cite}[1]{[#1]}
\makeatletter
\@addtoreset{equation}{section}
\makeatother
\renewcommand{\theequation}{\arabic{section}.\arabic{equation}}
\renewcommand{\contentsname}{\centering \small \color{blue} Contents}
%\counterwithin{figure}{section}
\renewcommand{\figurename}{\textbf{Fig.}}
%\renewcommand{\refname}{\textbf{\kaishu 参考文献}}
\renewcommand{\refname}{\textbf{Bibliography}}
\setcounter{secnumdepth}{4}
\titleformat{\paragraph}
{\normalfont\normalsize\bfseries}{\theparagraph}{1em}{}
\titlespacing*{\paragraph}{0pt}{3.25ex plus 1ex minus .2ex}{1.5ex plus .2ex}
\def\beginrefs{\begin{list}%
		{[\arabic{equation}]}{\usecounter{equation}
			\setlength{\leftmargin}{0.8truecm}\setlength{\labelsep}{0.4truecm}%
			\setlength{\labelwidth}{1.6truecm}}}
	\def\endrefs{\end{list}}
\def\bibentry#1{\item[\hbox{[#1]}]}
%%%%%%%%%%%%%%%%%%%%%%%%%%%%%%%%%%%%%%%%%%%%%%%%%%%%%%%%%%%%%%%%%%%%%%%%%%%%%%%%%%%%%%%%%%%%%%%%%%%%%%%%%%%%%%%%%%%%%%
%\begin{figure}[!htb]
%	\centering
%	\subfloat[$A \cap B$]{%
%		\includegraphics[width=0.3\linewidth,height=0.2\linewidth]{img001.jpg}}
%	\label{img001}\qquad \qquad %\hfill
%	\subfloat[${A_1} \cap {A_2} \cap {A_3}$]{%
%		\includegraphics[width=0.3\linewidth,height=0.2\linewidth]{img002.jpg}}
%	\label{img002}
	%\caption{ Examples.}
%\end{figure}
%\begin{figure}[!htb]
%	\centering
%	\includegraphics[width=0.4\linewidth,height=0.3\linewidth]{img005.jpg}
%	\label{img005}
	%\caption{ illustration for $ 3 $}
%\end{figure}
%\={a}1 \'{a}2\v{a}3\.{a}4

\usepackage{datetime}
\renewcommand{\today}{\shortmonthname[\the\month] \the \day,  \the\year}
%%%%%%%%%%%%%%%%%%%%%%%%%%%%%%%%%%%%%%%%%%%%%%%%%%%%%%%%%%%%%%%%%%%%%%%%%%%%%%%%%%%%%%%%%%%%%%%%%%%%%%%%%%%%%%%%%%%%%%
\begin{document}
	\kaishu 
	\thispagestyle{empty}
	\setcounter{section}{0}
	\begin{center}
		{\LARGE  \href{https://en.wikipedia.org/wiki/Calculus_of_variations}{Calculus of Variations}}
		
		%\vspace{0.25cm}
		
		%{\large \href{https://www.mpg.de/10655393/science-of-light-marquardt}{Florian Marquardt}}
	\end{center}
%%%%%%%%%%%%%%%%%%%%%%%%%%%%%%%%%%%%%%%%%%%%%%%%%%%%%%%%%%%%%%%%%%%%%%%%%%%%%%%%%%%%%%%%%%%%%%%%%%%%%%%%%%%%%%%%%%%%%%
%%\newpage 
%%\thispagestyle{empty}	
%%%%%%%%%%%%%%%%%%%%%%%%%%%%%%%%%%%%%%%%%%%%%%%%%%%%%%%%%%%%%%%%%%%%%%%%%%%%%%%%%%%%%%%%%%%%%%%%%%%%%%%%%%%%%%%%%%%%%%
%\tableofcontents	
%{\pagestyle{empty}\mbox{}\newpage\pagestyle{empty}}
%\newpage 
%{\pagestyle{empty}\mbox{}\newpage\pagestyle{empty}}
%%%%%%%%%%%%%%%%%%%%%%%%%%%%%%%%%%%%%%%%%%%%%%%%%%%%%%%%%%%%%%%%%%%%%%%%%%%%%%%%%%%%%%%%%%%%%%%%%%%%%%%%%%%%%%%%%%%%%%
%%\newpage 
\setcounter{page}{1}

\vspace{-1cm}

\section*{ \normalfont \large \href{https://www.bilibili.com/video/BV1f7411b74T/?spm_id_from=333.1387.collection.video_card.click&vd_source=d0661ef3fa56ddc850a3d1afc8c571e5}{\kaishu 变分学}}

\vspace{-0.5cm}

\begin{enumerate}
	\item  \href{https://mp.weixin.qq.com/s/CMXY7anEs6M3Nr2UqQIOBg}{前言与变分问题 A} %1
	\item  \href{https://mp.weixin.qq.com/s/KGb4YkxCHowgFszqiPo9mQ}{前言与变分问题 B} %2
	\item  \href{https://mp.weixin.qq.com/s/tkLrdZAb2TqX8GLMl9RByw}{Euler-Lagrange方程 A} %3
	\item  \href{https://mp.weixin.qq.com/s/j3PZJUV3N6PNaUavw9mBhw}{Euler-Lagrange方程 B} %4
	\item  \href{https://mp.weixin.qq.com/s/_RNK0fwfExx_sx-CT0y2gQ}{极小点的必要条件和充分条件 A} %5
	\item  \href{https://mp.weixin.qq.com/s/hC2ZITjK1_QCjSRNPD8nBw}{极小点的必要条件和充分条件 B} %6
	\item  \href{https://mp.weixin.qq.com/s/0V2-WjHAjQcaWQaMbcKUsQ}{强极小与临界场 A} %7
	\item  \href{https://mp.weixin.qq.com/s/_-ZJzGpi8n15C6pUymI2xA}{强极小与临界场 B} %8
	\item  \href{https://mp.weixin.qq.com/s/C4bcRqBi4DYLsOfQ_6K4kQ}{强极小与临界场 C} %9
	\item  \href{https://mp.weixin.qq.com/s/XPcekfHeVQ-aC8GuXVR4og}{Hamilton Jacobi 理论 A} %10
	\item  \href{https://mp.weixin.qq.com/s/O2OiEaxOZ46JHNwqf0ugfA}{Hamilton Jacobi 理论 B} %11
	\item  \href{https://mp.weixin.qq.com/s/0_wg66FeanlIPXoaJ2Av4A}{含多重积分的变分问题} %12
	\item  \href{https://mp.weixin.qq.com/s/-SCMHvQoZPIcVMo-bSIQ0A}{约束变分问题 A} %13
	\item  \href{https://mp.weixin.qq.com/s/VkBuGSKMIeReAUUwNXevQQ}{约束变分问题 B} %14
	\item  \href{https://mp.weixin.qq.com/s/1TbphFql7fT-_aXzPgcStA}{守恒律与 Noether 定理 A} %15
	\item  \href{https://mp.weixin.qq.com/s/WluaUwv0P0U5I7k_s-jKJA}{守恒律与 Noether 定理 B} %16
	\item  \href{https://mp.weixin.qq.com/s/-WcVEZHhgHJIOy1s5xLcbQ}{直接方法 A} %17
	\item  \href{https://mp.weixin.qq.com/s/Lk3-goRw1ChjcICA3s3A4Q}{直接方法 B} %18
	\item  \href{https://mp.weixin.qq.com/s/YP9Zd5r31N1fH2tEPgnGRg}{Sobolev 空间 A} %19
	\item  \href{https://mp.weixin.qq.com/s/utiSRR3NmnhqM4nWMikAQQ}{Sobolev 空间 B} %20
	\item  \href{https://mp.weixin.qq.com/s/sMS7nJn2RPrjSw7ZsK5mRw}{弱下半连续性 A} %21
	\item  \href{https://mp.weixin.qq.com/s/jF3RTOD3nSp9i5HPQ3fVkw}{弱下半连续性 B} %22
	\item  \href{https://mp.weixin.qq.com/s/3DBfHz_O7Iw5mpwjDNJDMA}{存在性与正则性 A} %23
	\item  \href{https://mp.weixin.qq.com/s/p5VRGnKwloZeQp6CdRUHrw}{存在性与正则性 B} %24
	\item  \href{https://mp.weixin.qq.com/s/lI1tqSgLZEv_VlCMKw61ug}{正交投影方法} %25
	\item  \href{https://mp.weixin.qq.com/s/J4-O5z_BhqJh4HuskbSnkQ}{特征值问题} %26
	\item  \href{https://mp.weixin.qq.com/s/-xP_qOKpD1KrLlnIgJjwiA}{变分问题的数值方法 A} %27
	\item  \href{https://mp.weixin.qq.com/s/L6KNb3Wjwna__O4-T-CrGw}{变分问题的数值方法 B} %28
	\item  \href{https://mp.weixin.qq.com/s/Y5TT9Egd2z_dxMuzBHW8dw}{松弛泛函与图象处理 A} %29
	\item  \href{https://mp.weixin.qq.com/s/R21wUSIpfYFm_6K3yAZ3kg}{松弛泛函与图象处理 B} %30
	\item  \href{https://mp.weixin.qq.com/s/DcrTZelPbjB82yiG94yPjg}{ 最优控制问题 A} %31
	\item  \href{https://mp.weixin.qq.com/s/jS9wYb1IofK5Zbz96bF-MQ}{ 最优控制问题 B} %32
	\item  \href{https://mp.weixin.qq.com/s/P5QY1hJktDyG2VrdEU2d4Q}{Ekeland 变分原理与山路定理 A} %33
	\item  \href{https://mp.weixin.qq.com/s/jnDiRCPRhdzbPP-Cn23GMg}{Ekeland 变分原理与山路定理 B} %34
	%\item \href{url}{Materials}
\end{enumerate}

\section*{ \normalfont \large \href{https://www.silviofanzon.com/blog/2021/Calculus-of-Variations/}{Calculus of Variations}}

\vspace{-0.5cm}

\begin{enumerate}
	\item \href{https://mp.weixin.qq.com/s/A0Nn95o2yqF5vqGeG6Bhkg}{Introduction. Basic examples. Functional analysis  revision. Part 1}  \quad \href{https://mp.weixin.qq.com/s/PxGcikR3U59UH8pFilmAdQ}{Part 2} %1 
	\item \href{https://mp.weixin.qq.com/s/2lGKir-zolq4q77RAkzvKg}{Functional Analysis Revision. Calculus in Normed Spaces. Part1} \quad \href{https://mp.weixin.qq.com/s/-r-1wN6hfq0rfFdFWXb5Fw}{Part2} %2
	\item \href{https://mp.weixin.qq.com/s/6EFFPQNBgrt6TlmoWP-VKg}{Calculus in Normed Spaces. Indirect Method. Part1} \quad \href{https://mp.weixin.qq.com/s/x3-0awsAMsinItFgnwyF7g}{Part2} %3
	\item \href{https://mp.weixin.qq.com/s/XvVVyOvx00HBHa2Gkt-Q_A}{Fundamental Lemmas. Boundary conditions. Part1} \quad \href{https://mp.weixin.qq.com/s/TbJycXsjTNSZv2mYN0cvCg}{Part2} %4
	\item \href{https://mp.weixin.qq.com/s/UkBV_MDS8k5bvHeM1pAtSQ}{Euler-Lagrange Equation. Revision of $L^{p}$ Spaces. Part1} \quad \href{https://mp.weixin.qq.com/s/OeMerMePwpczXTmNv1SBOg}{Part2} \quad \href{https://mp.weixin.qq.com/s/zGPYm_OG4OmekGsGauM8YA}{Revision} %5
	\item  \href{https://mp.weixin.qq.com/s/AGJ7LAdLBNxm4LMaa4C8YA}{Sufficient Conditions: Convexity, Trivial Lemma. Convolutions. Part1} \quad \href{https://mp.weixin.qq.com/s/gV76EBci2jua3hdxRL0wjw}{Part2} %6
	\item \href{https://mp.weixin.qq.com/s/U519vGvCKjJjIt-lRs1j8w}{FLCV and DBR Lemma. Sobolev Spaces. Part1} \quad \href{https://mp.weixin.qq.com/s/KG63b67pEnliDFXEc2OUIQ}{Part2} %7
	\item \href{https://mp.weixin.qq.com/s/yJMq5NXmXGNYB6_PuBH--A}{Sobolev Spaces: Regularity and Density Results. Part1} \quad \href{https://mp.weixin.qq.com/s/iRvZ5YMDvUgD4EOObP7kgQ}{Part2} %8
	\item  \href{https://mp.weixin.qq.com/s/mDfF2IzrRaNkWHiCmxo_5A}{Sobolev embedding. Ascoli-Arzel\`{a} Theorem. Part1} \quad \href{https://mp.weixin.qq.com/s/1Ov-FTbZSCX_cI6dkHJp0g}{Part2} %9
	\item  \href{https://mp.weixin.qq.com/s/u9A8c0H1K066h3_cr9MFkw}{Higher Order Sobolev Spaces. Traces. Euler-Lagrange Equation. Part1} \quad \href{https://mp.weixin.qq.com/s/F9_mA1TG65n0QVTPkr9vcA}{Part2} %10
	\item  \href{https://mp.weixin.qq.com/s/j-JvJ8KeTgbYn186wct3eg}{Boundary Conditions. Sufficient Conditions. Direct Method. Part1} \quad \href{https://mp.weixin.qq.com/s/upXcmc1HHbyn1TS7A-sfuA}{Part2} %11
	\item  \href{https://mp.weixin.qq.com/s/vyWOgnlj0V12fC3LTSj7pg}{Direct method: Example. General Existence Theorem. Part1} \quad \href{https://mp.weixin.qq.com/s/3pOGFKjtdrQSdWemt453Ew}{Part2} %12
	\item  \href{https://mp.weixin.qq.com/s/YI_Xje7ZaywfC61tAKZtGw}{LSC Envelope. Relaxation and its Computation. Part1} \quad \href{https://mp.weixin.qq.com/s/BXWPSYLB0RWQqdcJSUUY8A}{Part2} %13
	\item  \href{https://mp.weixin.qq.com/s/mng-S1TkWvOE-k8m8NNwQQ}{Relaxation of Integral Functionals. $\Gamma$-Convergence. Part1} \quad \href{https://mp.weixin.qq.com/s/YXRdmIl0jyM-UX9Vf-EErQ}{Part2} %14
	\item  \href{https://mp.weixin.qq.com/s/h7ShM-M-j_plHZpxxcTSnQ}{Examples of $\Gamma$-Convergence. Homogenization Problems. Part1} \quad \href{https://mp.weixin.qq.com/s/RYN3DERJapgNnpvL9ygh-w}{Part2} %15
	\item \href{https://pan.baidu.com/s/1NI71kBUtCYHTCu51Ses-0A?pwd=1121}{Materials}
\end{enumerate}
%%%%%%%%%%%%%%%%%%%%%%%%%%%%%%%%%%%%%%%%%%%%%%%%%%%%%%%%%%%%%%%%%%%%%%%%%%%%%%%%%%%%%%%%%%%%%%%%%%%%%%%%%%%%%%%%%%%%%%%
%\bibliographystyle{ieeetr} % number
%%\bibliographystyle{unsrtnat} % author year
%\bibliography{HeBib}
%%%%%%%%%%%%%%%%%%%%%%%%%%%%%%%%%%%%%%%%%%%%%%%%%%%%%%%%%%%%%%%%%%%%%%%%%%%%%%%%%%%%%%%%%%%%%%%%%%%%%%%%%%%%%%%%%%%%%%%
\begin{flushright}
	\tiny \today 
\end{flushright}
%%%%%%%%%%%%%%%%%%%%%%%%%%%%%%%%%%%%%%%%%%%%%%%%%%%%%%%%%%%%%%%%%%%%%%%%%%%%%%%%%%%%%%%%%%%%%%%%%%%%%%%%%%%%%%%%%%%%%%%
\end{document}
%%%%%%%%%%%%%%%%%%%%%%%%%%%%%%%%%%%%%%%%%%%%%%%%%%%%%%%%%%%%%%%%%%%%%%%%%%%%%%%%%%%%%%%%%%%%%%%%%%%%%%%%%%%%%%%%%%%%%%%
              