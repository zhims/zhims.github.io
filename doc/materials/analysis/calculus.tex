%%%%%%%%%%%%%%%%%%%%%%%%%%%%%%%%%%%%%%%%%%%%%%%%%%%%%%%%%%%%%%%%%%%%%%%%%%%%%%%%%%%%%%%%%%%%%%%%%%%%%%%%%%%%%%%%%%%%%
%%%%%%%%%%%%%%%%%%%%%%%%%%%%%%%%%%%%%%%%%%%%%   Author:Yao Zhang  %%%%%%%%%%%%%%%%%%%%%%%%%%%%%%%%%%%%%%%%%%%%%%%%%%%
%%%%%%%%%%%%%%%%%%%%%%%%%%%%%%%%%%%%%%%%%%%%% Email: jaafar_zhang@163.com %%%%%%%%%%%%%%%%%%%%%%%%%%%%%%%%%%%%%%%%%%%
%%%%%%%%%%%%%%%%%%%%%%%%%%%%%%%%%%%%%%%%%%%%%%%%%%%%%%%%%%%%%%%%%%%%%%%%%%%%%%%%%%%%%%%%%%%%%%%%%%%%%%%%%%%%%%%%%%%%%
\documentclass[11pt]{article}
\usepackage{babel}
\usepackage[utf8]{inputenc} 
\usepackage[table]{xcolor}
\usepackage[most]{tcolorbox}
\usepackage[left=2.50cm, right=1.50cm, top=2.0cm, bottom=2.50cm]{geometry}
\usepackage{xcolor,url}
\usepackage{amsmath,amsthm,amsfonts,amssymb,amscd,multirow,booktabs,fullpage,calc,multicol}
\usepackage{lastpage,enumitem,fancyhdr,mathrsfs,wrapfig,setspace,cancel,amsmath,empheq,framed}
\usepackage[retainorgcmds]{IEEEtrantools}
\usepackage{subfig,graphicx,framed}
\usepackage{ctex}
\usepackage{txfonts}
\usepackage{bbm}
\usepackage{chngcntr}
\usepackage[colorlinks,linkcolor=blue,anchorcolor=green,citecolor=red,urlcolor=blue]{hyperref}
\usepackage{titlesec}
%%%%%%%%%%%%%%%%%%%%%%%%%%%%%%%%%%%%%%%%%%%%%%%%%%%%%%%%%%%%%%%%%%%%%%%%%%%%%%%%%%%%%%%%%%%%%%%%%%%%%%%%%%%%%%%%%%%%%%
\newtheorem{thm}{Theorem}[section]
\newtheorem{defi}{Definition}[subsection]
\newtheorem{exercise}{Exercise}[subsection]
\newtheorem{note}{Note}[subsection]
\newtheorem{notation}{Notation}
\newtheorem{lemma}{Lemma}[subsection]
\newtheorem{proposition}{Proposition}[subsection]
\newtheorem{example}{Example}[subsection]
\newtheorem{problem}{Problem}[section]
\newtheorem{homework}{Homework}[section]
\newtheorem{summary}{Summary}[subsection]
\newtheorem{corollary}{Corollary}[subsection]
\newtheorem{rmk}{Remark}[section]
\usepackage{romannum}
%%%%%%%%%%%%%%%%%%%%%%%%%%%%%%%%%%%%%%%%%%%%%%%%%%%%%%%%%%%%%%%%%%%%%%%%%%%%%%%%%%%%%%%%%%%%%%%%%%%%%%%%%%%%%%%%%%%%%
\newlength{\tabcont}
\setlength{\parindent}{0.0in}
\setlength{\parskip}{0.05in}
\colorlet{shadecolor}{orange!15}
\parindent 0in
\parskip 12pt
\geometry{margin=1in, headsep=0.25in}
%%%%%%%%%%%%%%%%%%%%%%%%%%%%%%%%%%%%%%%%%%%%%%%%%%%%%%%%%%%%%%%%%%%%%%%%%%%%%%%%%%%%%%%%%%%%%%%%%%%%%%%%%%%%%%%%%%%%%
\graphicspath{ {img/EoM/}}
%%%%%%%%%%%%%%%%%%%%%%%%%%%%%%%%%%%%%%%%%%%%%%%%%%%%%%%%%%%%%%%%%%%%%%%%%%%%%%%%%%%%%%%%%%%%%%%%%%%%%%%%%%%%%%%%%%%%%
%\renewcommand{\cite}[1]{[#1]}
\makeatletter
\@addtoreset{equation}{section}
\makeatother
\renewcommand{\theequation}{\arabic{section}.\arabic{equation}}
\renewcommand{\contentsname}{\centering \small \color{blue} Contents}
%\counterwithin{figure}{section}
\renewcommand{\figurename}{\textbf{Fig.}}
%\renewcommand{\refname}{\textbf{\kaishu 参考文献}}
\renewcommand{\refname}{\textbf{Bibliography}}
\setcounter{secnumdepth}{4}
\titleformat{\paragraph}
{\normalfont\normalsize\bfseries}{\theparagraph}{1em}{}
\titlespacing*{\paragraph}{0pt}{3.25ex plus 1ex minus .2ex}{1.5ex plus .2ex}
\def\beginrefs{\begin{list}%
		{[\arabic{equation}]}{\usecounter{equation}
			\setlength{\leftmargin}{0.8truecm}\setlength{\labelsep}{0.4truecm}%
			\setlength{\labelwidth}{1.6truecm}}}
	\def\endrefs{\end{list}}
\def\bibentry#1{\item[\hbox{[#1]}]}
%%%%%%%%%%%%%%%%%%%%%%%%%%%%%%%%%%%%%%%%%%%%%%%%%%%%%%%%%%%%%%%%%%%%%%%%%%%%%%%%%%%%%%%%%%%%%%%%%%%%%%%%%%%%%%%%%%%%%%
%\begin{figure}[!htb]
%	\centering
%	\subfloat[$A \cap B$]{%
%		\includegraphics[width=0.3\linewidth,height=0.2\linewidth]{img001.jpg}}
%	\label{img001}\qquad \qquad %\hfill
%	\subfloat[${A_1} \cap {A_2} \cap {A_3}$]{%
%		\includegraphics[width=0.3\linewidth,height=0.2\linewidth]{img002.jpg}}
%	\label{img002}
	%\caption{ Examples.}
%\end{figure}
%\begin{figure}[!htb]
%	\centering
%	\includegraphics[width=0.4\linewidth,height=0.3\linewidth]{img005.jpg}
%	\label{img005}
	%\caption{ illustration for $ 3 $}
%\end{figure}
%\={a}1 \'{a}2\v{a}3\.{a}4

\usepackage{datetime}
\renewcommand{\today}{\shortmonthname[\the\month] \the \day,  \the\year}
%%%%%%%%%%%%%%%%%%%%%%%%%%%%%%%%%%%%%%%%%%%%%%%%%%%%%%%%%%%%%%%%%%%%%%%%%%%%%%%%%%%%%%%%%%%%%%%%%%%%%%%%%%%%%%%%%%%%%%
\begin{document}
	\kaishu 
	\setcounter{page}{1}
	%\thispagestyle{empty}
	\pagenumbering{arabic} 
	\setcounter{section}{0}
	\begin{center}
		{\LARGE  \href{https://en.wikipedia.org/wiki/Calculus}{Calculus}}
		
		%\vspace{-0.25cm}
		
		\href{https://sites.google.com/site/shencmath/}{Chun-Yen Shen}
	\end{center}
%%%%%%%%%%%%%%%%%%%%%%%%%%%%%%%%%%%%%%%%%%%%%%%%%%%%%%%%%%%%%%%%%%%%%%%%%%%%%%%%%%%%%%%%%%%%%%%%%%%%%%%%%%%%%%%%%%%%%%
%%\newpage 
%%\thispagestyle{empty}	
%%%%%%%%%%%%%%%%%%%%%%%%%%%%%%%%%%%%%%%%%%%%%%%%%%%%%%%%%%%%%%%%%%%%%%%%%%%%%%%%%%%%%%%%%%%%%%%%%%%%%%%%%%%%%%%%%%%%%%

%\tableofcontents	
%{\pagestyle{empty}\mbox{}\newpage\pagestyle{empty}}
%\newpage 
%{\pagestyle{empty}\mbox{}\newpage\pagestyle{empty}}
%%%%%%%%%%%%%%%%%%%%%%%%%%%%%%%%%%%%%%%%%%%%%%%%%%%%%%%%%%%%%%%%%%%%%%%%%%%%%%%%%%%%%%%%%%%%%%%%%%%%%%%%%%%%%%%%%%%%%%
%%\newpage

\vspace{-0.5cm}

\large {\href{https://www.youtube.com/playlist?list=PLVJXJebpO4PjZwfkPwKs8jb1VKCnRtNd1&si=BjCAEuQK3MjtkRIv}{1.}} 

%\begin{center}
%	\large {\href{https://www.youtube.com/playlist?list=PLVJXJebpO4PhhPaG5r3Sxp0w89hqCfi0a}{Calculus \Romannum{1}}}
%\end{center}

\vspace{-0.75cm}

\begin{enumerate}
	\item \href{https://mp.weixin.qq.com/s/aRWrlvisSPUyQY6G8bt8XA}{System of real number 1}	%1
	\item \href{https://mp.weixin.qq.com/s/henNU4zPR_yXjaqzXPTk6w}{System of real number 2}	%2
	\item \href{https://mp.weixin.qq.com/s/ZMfZ3gmRMELibbHYZT31Bw}{System of real number 3}	%3
	\item \href{https://mp.weixin.qq.com/s/dAzIXujM2Ztv5dhAic4pjg}{Completeness of Real Number, Weierstrass Theorem 1}	%4
	\item \href{https://mp.weixin.qq.com/s/cJoPRLFBSuXa1SGsen0duQ}{Weierstrass Theorem 2, define subsequence, Cauchy Sequence}	%5
	\item \href{https://mp.weixin.qq.com/s/f35OfWIHRAcmMfie0F7lFg}{Least Upper Bound \& Greatest Lower Bound}	%6
	\item \href{https://mp.weixin.qq.com/s/9rx4aAQrB1vRyIwAIg1KXQ}{Continuous function, uniformly continuous, Lipchitz continuous}	%7
	\item \href{https://mp.weixin.qq.com/s/ATm88605is00bQxQLUZ-PA}{Intermediate value theorem, Extreme Value Theory}	%8
	\item \href{https://mp.weixin.qq.com/s/ytuoZYjY2xCNi9Dsa65Y5g}{Rieman sum}	%9
	\item \href{https://mp.weixin.qq.com/s/uWfljPPgj0LGRM7iyk1mNg}{Properties of integral, MVT/general MVT for integral, logarithm}	%10
	\item \href{https://mp.weixin.qq.com/s/KsTIZjU5m_lbgQkrcE0zdA}{Logarithm and exponential 1}	%11
	\item \href{https://mp.weixin.qq.com/s/AX-xsj2DspX2CO_8VvFOPA}{Logarithm and exponential 2}	%12
	\item \href{https://mp.weixin.qq.com/s/ZFbnBA_79XyjBBL21OuW5A}{Fundamental Theorem of Calculus 1, Rolle's Theorem, MVT for derivative 1}	%13
	\item \href{https://mp.weixin.qq.com/s/aTRJWwRRFsJHvkcteVyruw}{MVT for derivative 2, approximation by linear function, FTC 2}	%14
	\item \href{https://mp.weixin.qq.com/s/wKSPlTi0H9EP1Y2RPweSVQ}{Properties of derivative, composite functions, chain rule}	%15
	\item \href{https://mp.weixin.qq.com/s/pGkX3-pMSFTGLiDeGMyTAw}{First/second derivative test}	%16
	\item \href{https://mp.weixin.qq.com/s/cxEM3aY-GiW_MsM3eeY-TA}{Order of magnitude, introduce an smooth function}	%17
	\item \href{https://mp.weixin.qq.com/s/fvxg2mLtSSM8Gyo4wRrRfQ}{Some example of oscillating functions, chain rule, method of substitution}	%18
	\item \href{https://mp.weixin.qq.com/s/1_BBnHcpBOr5QwzhCMS2Jw}{Integration by parts}	%19
	\item \href{https://mp.weixin.qq.com/s/Ei7SnZWqDcaStALEiaigBA}{Integration of rational functions 1}	%20
	\item \href{https://mp.weixin.qq.com/s/18XRgIGhs7i-qkrp3OZTyA}{Integration of rational functions 2, Improper integral 1}	%21
	\item \href{https://mp.weixin.qq.com/s/1HyKxNFDCccuQr07KC-UXA}{Improper integral 2}	%22
	\item \href{https://mp.weixin.qq.com/s/8XoyA4tGn1KwGNxWmuNbYw}{Taylor's series: Gamma function, power series, expansion of the logarithm}	%23
	\item \href{https://mp.weixin.qq.com/s/ho7_hB0jR7bNepTmeIjNdQ}{Taylor's series: Taylor's theorem, Cauchy's \& Lagrange's remainder}	%24
	\item \href{https://mp.weixin.qq.com/s/9E9eGleHOtv7ovy1lUcLrA}{Taylor's series: estimate remainder term, expansion of elementary}	%25
	\item \href{https://mp.weixin.qq.com/s/P4GEHqjFht-kRQl4qeW0VA}{Taylor's series: examples}	%26
	\item \href{https://mp.weixin.qq.com/s/Eufsx2R4xi-2LIb218vkSw}{Taylor's polynomial: interpolation 1}	%27
	\item \href{https://mp.weixin.qq.com/s/nCIjtbgzz893SbNFfR-NTw}{Taylor's polynomial: interpolation 2, approximation in first order}	%28
	\item \href{https://mp.weixin.qq.com/s/ua3ZGUI0M8ChNP_hpEQ6hg}{Taylor's polynomial: approximation in second order}	%29
	\item \href{https://mp.weixin.qq.com/s/C5su0-glrDMnWzU9VdTsEw}{Taylor's polynomial: fixed point approximation, Stirling's formula}	%30
	\item \href{https://mp.weixin.qq.com/s/MFMIa3EkvjOm7V70XUG37A}{Series: concepts of convergence and divergence 1}	%31
	\item \href{https://mp.weixin.qq.com/s/AVwjWW5NBDDHgkRsnOmoNQ}{Series: rearrangement, tests for absolute convergence and divergence 1}	%32
	\item \href{https://mp.weixin.qq.com/s/Tjw9T4Bcqb0LB7VKhAItDA}{Series: tests for abs. convergence and divergence 2, sequences of functions}	%33
	\item \href{https://mp.weixin.qq.com/s/h2Gsu-uxRFgqLk34sHdqGw}{Series: pointwise/uniform convergence 1}	%34
	\item \href{https://mp.weixin.qq.com/s/em69T457kgN8DCNOi3MqNw}{Series: pointwise/uniform convergence 2}	%35
	\item \href{https://mp.weixin.qq.com/s/C_qWwY9r7Wqquk0Dpq_r5A}{Series: power series, interval of convergence}	%36
	\item \href{https://mp.weixin.qq.com/s/FPru75acJ7toOM06Mkq0MQ}{Series: product of two power series}	%37
	\item \href{https://mp.weixin.qq.com/s/7D57J6Qic6hNnAjvPPM44w}{Series: expansion of given power series, infinite product}	%38
	\item \href{https://mp.weixin.qq.com/s/7UXE-cub3pKSqixYXnsrgQ}{Fourier series: periodic function, complex notation 1}	%39
	\item \href{https://mp.weixin.qq.com/s/okK6U1OAHyqWDo9RuUPLPA}{Fourier series: complex form 2, trigonometric formula, Riemann Lebesgue Lemma}	%40
	\item \href{https://mp.weixin.qq.com/s/dmb6iMfwGthvqdx37UaLFQ}{Fourier series: examples of Fourier expansion}	%41
	\item \href{https://mp.weixin.qq.com/s/SSNRnAP2Ixx-SmtZsjy2jA}{Fourier series: main theorem on Fourier expansion}	%42
	\item \href{https://mp.weixin.qq.com/s/s736cBxFxHhcteNPzCochQ}{Fourier series: examples of Fourier series 1}	%43
	\item \href{https://mp.weixin.qq.com/s/wtLeeApB45OSev34uW7lGA}{Fourier series: examples of Fourier series 2, Bessel's Inequality}	%44
	\item \href{https://mp.weixin.qq.com/s/tfS_J6gQ-1XWo5SVPCNKwA}{Approximation by trigonometric and rational polynomial 1}	%45
	\item \href{https://mp.weixin.qq.com/s/rKuvvXPphPKXL1ByORd4cA}{Approximation by trigonometric and rational polynomial 2}	%46
	\item \href{https://mp.weixin.qq.com/s/WyY2kuqoDXI0rj4tiVGF9w}{Approximation by trigonometric and rational polynomial 3}	%47
	\item \href{https://mp.weixin.qq.com/s/Mt-QnrU7IvwvyLj8MluLLA}{Inner product}	%48
	\item \href{https://mp.weixin.qq.com/s/UwOwRipaO1DbHEY-Xl0jLg}{Bernoulli polynomial and their applications 1}	%49
	\item \href{https://mp.weixin.qq.com/s/SNKjyVc5iyEtFnqHTbIKSQ}{Bernoulli polynomial and their applications 2}	%50
	%\item \href{url}{Materials}
\end{enumerate}

\subsection*{\href{https://www.youtube.com/playlist?list=PLVJXJebpO4Pi_4cETi8EL19qJLdcLEW2a&si=YwWo1F4e-o_KZ5_s}{2.}}

\vspace{-0.75cm}

\begin{enumerate}
	\item \href{https://mp.weixin.qq.com/s/bDhkUEZvMhreRqtfUtJLig}{Functions of multiple variables \& partial derivative}	%1
	\item \href{https://mp.weixin.qq.com/s/kUzP7n_yHdNj4tn9Fty6-g}{Continuity}	%2
	\item \href{https://mp.weixin.qq.com/s/AZ5YUhzyy--ws_s9xAGWeg}{Differentiability \& directional derivative (1)}	%3
	\item \href{https://mp.weixin.qq.com/s/tOC1mEdZZttDdqHmah9Zqg}{Directional derivative (2) \& tangent plane}	%4
	\item \href{https://mp.weixin.qq.com/s/mFU23KZzCUmfhEvfXL7_Rg}{Change of variables \& Taylor series}	%5
	\item \href{https://mp.weixin.qq.com/s/JgYEAfJC-5psD7MIrfqgvQ}{MVT \& Taylor expansion \& Integral}	%6
	\item \href{https://mp.weixin.qq.com/s/N1_45wIPGIKQRxs9qTMC4A}{Double integral \& length of curve}	%7
	\item \href{https://mp.weixin.qq.com/s/Q2oLh8mb3pF3DAJkWp9tqA}{Curvature \& linear differential one form}	%8
	\item \href{https://mp.weixin.qq.com/s/11utINvjKpiv_-qE25NnsQ}{Line integral}	%9
	\item \href{https://mp.weixin.qq.com/s/pDquAXZJGuAhIiO5f5yn2A}{Heine-Borel theorem}	%10
	\item \href{https://mp.weixin.qq.com/s/dar5eiTVv9u0cnkM3nfmLA}{Compact subset \& Implicit Function theorem (1)}	%11
	\item \href{https://mp.weixin.qq.com/s/_JQb_vYCM8oyltC8NYS69A}{Implicit Function theorem (2)}	%12
	\item \href{https://mp.weixin.qq.com/s/C1ruVdkOUj0u68DExiO9bw}{Inverse Function theorem (1)}	%13
	\item \href{https://mp.weixin.qq.com/s/vuSw-YiioZSat0vNSduJzg}{Inverse Function theorem (2) \& extreme value}	%14
	\item \href{https://mp.weixin.qq.com/s/FnIgRSTFcAOGlAtICzfXvA}{Lagrange Multiplier method}	%15
	\item \href{https://mp.weixin.qq.com/s/YqrE_d7mOysPTEWr23AktQ}{Examples \& H{\"o}lder Inequality}	%16
	\item \href{https://mp.weixin.qq.com/s/dvNVrNc2HOcP8KVPeCmgsQ}{sufficient condition for local max/min}	%17
	\item \href{https://mp.weixin.qq.com/s/LTz6AxCJV6FeksnC4ymr4w}{Jordan measurable}	%18
	\item \href{https://mp.weixin.qq.com/s/sL0OeOhB-uerhPBGkjvz8g}{Jordan area (1)}	%19
	\item \href{https://mp.weixin.qq.com/s/3lNhydWx_bMVwgs0m8sb7g}{Jordan area (2)}	%20
	\item \href{https://mp.weixin.qq.com/s/MN9H1f2uwMYlUEp26NqE6w}{Double integral (1)}	%21
	\item \href{https://mp.weixin.qq.com/s/w0_3AgvhZfBsGPhNRXFP1g}{Double integral (2)}	%22
	\item \href{https://mp.weixin.qq.com/s/SAdVbMScJP8I3rQ3XmVwOw}{Transformation of multiple integrals}	%23
	\item \href{https://mp.weixin.qq.com/s/kob2IvplxCmKFTtFpbNA_g}{Improper multiple integrals (1)}	%24
	\item \href{https://mp.weixin.qq.com/s/ZdR6BpqL_kikazr7yh6gGg}{Improper multiple integrals (2) \& volumes}	%25
	\item \href{https://mp.weixin.qq.com/s/Abi6Lf2Xjw5a1oDwvW5usg}{Surface area}	%26
	\item \href{https://mp.weixin.qq.com/s/R4mn26dgB2_MeQqIOgg-OA}{Surface area formula}	%27
	\item \href{https://mp.weixin.qq.com/s/rhMCKrdEWJqkRj0kafOJ8Q}{Multiple integral in curved coordinate}	%28
	\item \href{https://mp.weixin.qq.com/s/FnoVINT35kZ8b9a7IL310Q}{Extend to $\mathbb{R}^n$}	%29
	\item \href{https://mp.weixin.qq.com/s/ylmcO7WRa39DSbEk7KysxQ}{Integral of unbounded set}	%30
	\item \href{https://mp.weixin.qq.com/s/0a73qNWFZvH7FKEb685vXQ}{Fourier integral theorem (1)}	%31
	\item \href{https://mp.weixin.qq.com/s/FFCguuWwUYZTmuw0sGv2Gg}{Fourier integral theorem (2)}	%32
	\item \href{https://mp.weixin.qq.com/s/C1RIlDACuV_jDDrVUP1sqg}{Fourier transform - decay property \& Parseval's identity}	%33
	\item \href{https://mp.weixin.qq.com/s/j3bD1spNfFUMvLIBohfteg}{Fourier transform of several variables}	%34
	\item \href{https://mp.weixin.qq.com/s/hWZqkCh-U-84YFBAlxCxfw}{Green theorem}	%35
	\item \href{https://mp.weixin.qq.com/s/B9rRDwt1c_qbpvo4HAfr7Q}{Divergence theorem}	%36
	\item \href{https://mp.weixin.qq.com/s/jl5Dm-h0FOb-LnGLHCnoWw}{Prove Inverse Function theorem by Green theorem}	%37
	\item \href{https://mp.weixin.qq.com/s/ZhvRDHR2Qr1fAesaBAMjZQ}{Orientation of surface in $\mathbb{R}^3$ \Romannum{1}}	%38
	\item \href{https://mp.weixin.qq.com/s/lH_4M5iuhdUYWTifm2XJ8w}{Orientation of surface in $\mathbb{R}^3$ \Romannum{2}}	%39
	\item \href{https://mp.weixin.qq.com/s/2ndGfaeBQaUMIG_yTid1tg}{Gauss theorem in $\mathbb{R}^3$ \Romannum{1}}	%40
	\item \href{https://mp.weixin.qq.com/s/zqPCUM7cgoBiJM8JPcGegg}{Gauss theorem in $\mathbb{R}^3$ \Romannum{2}}	%41
	\item \href{https://mp.weixin.qq.com/s/z4d_PLC2h_3fjAVzNaubqw}{Application of fluid}	%42
	\item \href{https://mp.weixin.qq.com/s/PYlTJj0QBEbQhaI7PiGoog}{Stoke's theorem}	%43
	\item \href{https://mp.weixin.qq.com/s/oWj6BULQX2Gl1G7Pfpu-pg}{General surface (partition of unit) (1)}	%44
	\item \href{https://mp.weixin.qq.com/s/kIFyjgRkmLdiMT_Mcid9Qg}{General surface (partition of unit) (2)}	%45
	\item \href{https://mp.weixin.qq.com/s/ZVYIUHoouqtsnmmmY0PQkA}{General surface (partition of unit) (3) \& functions of one complex variable}	%46
	\item \href{https://mp.weixin.qq.com/s/W7HT590gZWbkwAJmrF04Qw}{Complex function - power series}	%47
	\item \href{https://mp.weixin.qq.com/s/9ZjDipFBq9plPhU3C2ZI8A}{Complex function - differentiability}	%48
	\item \href{https://mp.weixin.qq.com/s/9-AEGn0ACllBCmeZ7ydI0w}{Conformal map \& integration of analytic functions - Cauchy's theorem}	%49
	\item \href{https://mp.weixin.qq.com/s/5lW4yeB81NNozAqobEFMuw}{Complex integral theorem \& Cauchy integral formula}	%50
	\item \href{https://mp.weixin.qq.com/s/RrXbgYPcq6Ou8zivAkJM-Q}{Zeros, poles and residues}	%51
	%\item \href{url}{Materials}
\end{enumerate}


%%%%%%%%%%%%%%%%%%%%%%%%%%%%%%%%%%%%%%%%%%%%%%%%%%%%%%%%%%%%%%%%%%%%%%%%%%%%%%%%%%%%%%%%%%%%%%%%%%%%%%%%%%%%%%%%%%%%%%%
%\bibliographystyle{ieeetr} % number
%%\bibliographystyle{unsrtnat} % author year
%\bibliography{HeBib}
%%%%%%%%%%%%%%%%%%%%%%%%%%%%%%%%%%%%%%%%%%%%%%%%%%%%%%%%%%%%%%%%%%%%%%%%%%%%%%%%%%%%%%%%%%%%%%%%%%%%%%%%%%%%%%%%%%%%%%%
\begin{flushright}
	\tiny \today 
\end{flushright}

%%%%%%%%%%%%%%%%%%%%%%%%%%%%%%%%%%%%%%%%%%%%%%%%%%%%%%%%%%%%%%%%%%%%%%%%%%%%%%%%%%%%%%%%%%%%%%%%%%%%%%%%%%%%%%%%%%%%%%%
\end{document}
%%%%%%%%%%%%%%%%%%%%%%%%%%%%%%%%%%%%%%%%%%%%%%%%%%%%%%%%%%%%%%%%%%%%%%%%%%%%%%%%%%%%%%%%%%%%%%%%%%%%%%%%%%%%%%%%%%%%%%%
              