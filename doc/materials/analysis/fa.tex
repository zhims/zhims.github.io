%%%%%%%%%%%%%%%%%%%%%%%%%%%%%%%%%%%%%%%%%%%%%%%%%%%%%%%%%%%%%%%%%%%%%%%%%%%%%%%%%%%%%%%%%%%%%%%%%%%%%%%%%%%%%%%%%%%%%
%%%%%%%%%%%%%%%%%%%%%%%%%%%%%%%%%%%%%%%%%%%%%   Author:Yao Zhang  %%%%%%%%%%%%%%%%%%%%%%%%%%%%%%%%%%%%%%%%%%%%%%%%%%%
%%%%%%%%%%%%%%%%%%%%%%%%%%%%%%%%%%%%%%%%%%%%% Email: jaafar_zhang@163.com %%%%%%%%%%%%%%%%%%%%%%%%%%%%%%%%%%%%%%%%%%%
%%%%%%%%%%%%%%%%%%%%%%%%%%%%%%%%%%%%%%%%%%%%%%%%%%%%%%%%%%%%%%%%%%%%%%%%%%%%%%%%%%%%%%%%%%%%%%%%%%%%%%%%%%%%%%%%%%%%%
\documentclass[11pt]{article}
\usepackage{babel}
\usepackage[utf8]{inputenc} 
\usepackage[table]{xcolor}
\usepackage[most]{tcolorbox}
\usepackage[left=2.50cm, right=1.50cm, top=2.0cm, bottom=2.50cm]{geometry}
\usepackage{xcolor,url}
\usepackage{amsmath,amsthm,amsfonts,amssymb,amscd,multirow,booktabs,fullpage,calc,multicol}
\usepackage{lastpage,enumitem,fancyhdr,mathrsfs,wrapfig,setspace,cancel,amsmath,empheq,framed}
\usepackage[retainorgcmds]{IEEEtrantools}
\usepackage{subfig,graphicx,framed}
\usepackage{ctex}
\usepackage{txfonts}
\usepackage{bbm}
\usepackage{chngcntr}
\usepackage[colorlinks,linkcolor=blue,anchorcolor=green,citecolor=red,urlcolor=blue]{hyperref}
\usepackage{titlesec}
%%%%%%%%%%%%%%%%%%%%%%%%%%%%%%%%%%%%%%%%%%%%%%%%%%%%%%%%%%%%%%%%%%%%%%%%%%%%%%%%%%%%%%%%%%%%%%%%%%%%%%%%%%%%%%%%%%%%%%
\newtheorem{thm}{Theorem}[section]
\newtheorem{defi}{Definition}[subsection]
\newtheorem{exercise}{Exercise}[subsection]
\newtheorem{note}{Note}[subsection]
\newtheorem{notation}{Notation}
\newtheorem{lemma}{Lemma}[subsection]
\newtheorem{proposition}{Proposition}[subsection]
\newtheorem{example}{Example}[subsection]
\newtheorem{problem}{Problem}[section]
\newtheorem{homework}{Homework}[section]
\newtheorem{summary}{Summary}[subsection]
\newtheorem{corollary}{Corollary}[subsection]
\newtheorem{rmk}{Remark}[section]
%%%%%%%%%%%%%%%%%%%%%%%%%%%%%%%%%%%%%%%%%%%%%%%%%%%%%%%%%%%%%%%%%%%%%%%%%%%%%%%%%%%%%%%%%%%%%%%%%%%%%%%%%%%%%%%%%%%%%
\newlength{\tabcont}
\setlength{\parindent}{0.0in}
\setlength{\parskip}{0.05in}
\colorlet{shadecolor}{orange!15}
\parindent 0in
\parskip 12pt
\geometry{margin=1in, headsep=0.25in}
%%%%%%%%%%%%%%%%%%%%%%%%%%%%%%%%%%%%%%%%%%%%%%%%%%%%%%%%%%%%%%%%%%%%%%%%%%%%%%%%%%%%%%%%%%%%%%%%%%%%%%%%%%%%%%%%%%%%%
\graphicspath{ {img/EoM/}}
%%%%%%%%%%%%%%%%%%%%%%%%%%%%%%%%%%%%%%%%%%%%%%%%%%%%%%%%%%%%%%%%%%%%%%%%%%%%%%%%%%%%%%%%%%%%%%%%%%%%%%%%%%%%%%%%%%%%%
%\renewcommand{\cite}[1]{[#1]}
\makeatletter
\@addtoreset{equation}{section}
\makeatother
\renewcommand{\theequation}{\arabic{section}.\arabic{equation}}
\renewcommand{\contentsname}{\centering \small \color{blue} Contents}
%\counterwithin{figure}{section}
\renewcommand{\figurename}{\textbf{Fig.}}
%\renewcommand{\refname}{\textbf{\kaishu 参考文献}}
\renewcommand{\refname}{\textbf{Bibliography}}
\setcounter{secnumdepth}{4}
\titleformat{\paragraph}
{\normalfont\normalsize\bfseries}{\theparagraph}{1em}{}
\titlespacing*{\paragraph}{0pt}{3.25ex plus 1ex minus .2ex}{1.5ex plus .2ex}
\def\beginrefs{\begin{list}%
		{[\arabic{equation}]}{\usecounter{equation}
			\setlength{\leftmargin}{0.8truecm}\setlength{\labelsep}{0.4truecm}%
			\setlength{\labelwidth}{1.6truecm}}}
	\def\endrefs{\end{list}}
\def\bibentry#1{\item[\hbox{[#1]}]}
%%%%%%%%%%%%%%%%%%%%%%%%%%%%%%%%%%%%%%%%%%%%%%%%%%%%%%%%%%%%%%%%%%%%%%%%%%%%%%%%%%%%%%%%%%%%%%%%%%%%%%%%%%%%%%%%%%%%%%
%\begin{figure}[!htb]
%	\centering
%	\subfloat[$A \cap B$]{%
%		\includegraphics[width=0.3\linewidth,height=0.2\linewidth]{img001.jpg}}
%	\label{img001}\qquad \qquad %\hfill
%	\subfloat[${A_1} \cap {A_2} \cap {A_3}$]{%
%		\includegraphics[width=0.3\linewidth,height=0.2\linewidth]{img002.jpg}}
%	\label{img002}
	%\caption{ Examples.}
%\end{figure}
%\begin{figure}[!htb]
%	\centering
%	\includegraphics[width=0.4\linewidth,height=0.3\linewidth]{img005.jpg}
%	\label{img005}
	%\caption{ illustration for $ 3 $}
%\end{figure}
%\={a}1 \'{a}2\v{a}3\.{a}4

\usepackage{datetime}
\renewcommand{\today}{\shortmonthname[\the\month] \the \day,  \the\year}
%%%%%%%%%%%%%%%%%%%%%%%%%%%%%%%%%%%%%%%%%%%%%%%%%%%%%%%%%%%%%%%%%%%%%%%%%%%%%%%%%%%%%%%%%%%%%%%%%%%%%%%%%%%%%%%%%%%%%%
\begin{document}
	\kaishu 
	\thispagestyle{empty}
	\setcounter{section}{0}
	\begin{center}
		{\LARGE  \href{https://www.youtube.com/playlist?list=PLo4jXE-LdDTTIIIRwqK35CbFJieSJEcVR}{Functional Analysis}}
		
		%\vspace{-0.25cm}
		
		{\large \href{https://w3.impa.br/~landim/}{Claudio Landim}}
	\end{center}
%%%%%%%%%%%%%%%%%%%%%%%%%%%%%%%%%%%%%%%%%%%%%%%%%%%%%%%%%%%%%%%%%%%%%%%%%%%%%%%%%%%%%%%%%%%%%%%%%%%%%%%%%%%%%%%%%%%%%%
%%\newpage 
%%\thispagestyle{empty}	
%%%%%%%%%%%%%%%%%%%%%%%%%%%%%%%%%%%%%%%%%%%%%%%%%%%%%%%%%%%%%%%%%%%%%%%%%%%%%%%%%%%%%%%%%%%%%%%%%%%%%%%%%%%%%%%%%%%%%%
%\tableofcontents	
%{\pagestyle{empty}\mbox{}\newpage\pagestyle{empty}}
%\newpage 
%{\pagestyle{empty}\mbox{}\newpage\pagestyle{empty}}
%%%%%%%%%%%%%%%%%%%%%%%%%%%%%%%%%%%%%%%%%%%%%%%%%%%%%%%%%%%%%%%%%%%%%%%%%%%%%%%%%%%%%%%%%%%%%%%%%%%%%%%%%%%%%%%%%%%%%%
%%\newpage 
\setcounter{page}{1}

%\vspace{1.5cm}


\vspace{-1cm}

\begin{enumerate}
	\item  \href{https://mp.weixin.qq.com/s/31noYsQCVBUwL4gon-xEXw}{Linear Spaces: Definition, Examples and Linear Span.} %1
	\item  \href{https://mp.weixin.qq.com/s/VwvCvoFwbf1WEC0qFkcgOw}{Linear Spaces: Quotient Spaces and Convex Sets.} %2
	\item  \href{https://mp.weixin.qq.com/s/YAjwN-rNBZ1rcaEj2FHRWg}{Normed Linear Spaces: Definition and Basic Properties.} %3
	\item  \href{https://mp.weixin.qq.com/s/NYydI1Z_jMx6jawXUU0cJQ}{Completing a Normed Linear Space.} %4
	\item  \href{https://mp.weixin.qq.com/s/WduGLBOqcXibbUKE_ffTQw}{Finite Dimensional Linear Spaces.} %5
	\item  \href{https://mp.weixin.qq.com/s/9imaUe39l6Fw_eJzPEtV6A}{Examples of Normed Linear Spaces.} %6
	\item  \href{https://mp.weixin.qq.com/s/jJ083Tx9l3kQ866A1ykfMQ}{In Infinite Dimensions the Unit Ball is not Compact.} %7
	\item  \href{https://mp.weixin.qq.com/s/_rXF0h5vp3J7SA6z1elhNg}{Zorn's Lemma.} %8
	\item  \href{https://mp.weixin.qq.com/s/PwtKSUgPS4fK841ascRN3A}{The Hahn-Banach Theorem.} %9
	\item  \href{https://mp.weixin.qq.com/s/NkORGJ0XK8olbsbC9FOPhQ}{Convex Sets and Gauge Functions.} %10
	\item  \href{https://mp.weixin.qq.com/s/BT8w36jLSLtgvBhsl5JliA}{Geometric Hanh-Banach Theorems.} %11
	\item  \href{https://mp.weixin.qq.com/s/czKghXVYsL4L3zbjj0Y6qQ}{Dual o a Normed Linear Space.} %12
	\item  \href{https://mp.weixin.qq.com/s/bDVf8DLW-ZOC8bHJ45Swjw}{Extension of Bounded Linear Functionals, Closed Linear Spans.} %13
	\item  \href{https://mp.weixin.qq.com/s/BVhVTgMhJrDmEhGbUoPBqg}{Reflexive Spaces.} %14
	\begin{enumerate}
		\item \href{https://mp.weixin.qq.com/s/qC45agvaGUygRifKgUPhkw}{The Dual Space of C([a,b]).}
		\item \href{https://mp.weixin.qq.com/s/bnbzCYhMj-wfkwr5o6MpLA}{An Application of the Hahn-Banach Theorem: the Moment Problem and Chebyshev Approximation.}
		\item \href{https://mp.weixin.qq.com/s/FJl9skrqchkOFYmTIDDHOw}{A Dual Variational Problem in Optimal Control.}
		\item \href{https://mp.weixin.qq.com/s/lwCOIuUz7nUMy5gIJ-VVTQ}{An Application of the Hahn-Banach Theorem: the Existence of a Green Function.}
	\end{enumerate}
	\item  \href{https://mp.weixin.qq.com/s/suNiTCRfo0QVdOgdJjUV8w}{Hilbert Spaces.} %15
	\item  \href{https://mp.weixin.qq.com/s/4WtVNAURMgnHWEh3E1-A5w}{Closed Convex Subsets of a Hilbert Space.} %16
	\item  \href{https://mp.weixin.qq.com/s/tGqgKIQtZKPd6wrIeiDT7A}{Riesz and Lax-Milgram Representation Theorems.} %17
	\item  \href{https://mp.weixin.qq.com/s/hawZA03UB1LZVOisHtVSKQ}{Orthonormal Sets and Closed Linear Spans.} %18
	\item  \href{https://mp.weixin.qq.com/s/39Q3WJ6eU54Py_4FPX3SMg}{Orthonormal Bases.} %19
	\begin{enumerate}
		\item \href{https://mp.weixin.qq.com/s/7Of_lPt7Q4J5MAIXr9BwKQ}{A Quadratic Variational Problem.}
		\item \href{https://mp.weixin.qq.com/s/nFmQfYgAs4QWG75CmxesVw}{The Dirichlet Principle.}
		\item \href{https://mp.weixin.qq.com/s/pNR4OwdJpb3hU4jCwTBmaQ}{Generalized Derivatives and Sobolev Spaces.}
	\end{enumerate}
	\item  \href{https://mp.weixin.qq.com/s/70PNERNM_6nbTDwRqVuxOQ}{Uniform Boundedness Principle.} %20
	\item  \href{https://mp.weixin.qq.com/s/h39rMcHqKCWtMjasDihOtw}{Weak Convergence.} %21
	\item  \href{https://mp.weixin.qq.com/s/FgxaH8jHvSRsWYSjYR4jEg}{Uniform Boundedness of Weak Converging Sequences.} %22
	\item  \href{https://mp.weixin.qq.com/s/EX5dyhXGruK5UMFxhaALJQ}{Weak Sequentially Compactness.} %23
	\item  \href{https://mp.weixin.qq.com/s/rPCaEkj_DRckKKEaVoUSdg}{Weak* Topology.} %24
	\item  \href{https://mp.weixin.qq.com/s/ocrmhjX_rKUINq9ikw6agA}{Applications of Weak Convergence.} %25
	\item  \href{https://mp.weixin.qq.com/s/AHFTU9uyxE2qhzuVyS_cOQ}{Bounded Linear Operators.} %26
	\item  \href{https://mp.weixin.qq.com/s/c79wuGVd9G0nOto5xkw_cQ}{Transpose of Bounded Linear Operators.} %27
	\item  \href{https://mp.weixin.qq.com/s/yUu3Fl2UZ1RNgCg0HWHTsw}{Strong and Weak Convergence of Operators.} %28
	\item  \href{https://mp.weixin.qq.com/s/6azd_l8QIOJ54mmO41HI3w}{Principle of Uniform Boundedness for Maps and Compositions.} %29
	\item  \href{https://mp.weixin.qq.com/s/ETet6wS-udN_0PCyq9L4wg}{Open Map Principle.} %30
	\item  \href{https://mp.weixin.qq.com/s/2eWeNU41MiO9TeBi-iujeg}{The Closed Graph Theorem.} %31
	\item  \href{https://mp.weixin.qq.com/s/Ex3vqAsfDePlvf-MNXpTeQ}{Examples of Bounded Linear Maps: Integral Operators.} %32
	\item  \href{https://mp.weixin.qq.com/s/y4QnNRMBuSvpjsUrgvQXBg}{Symmetric Operators.} %33
	\item  \href{https://mp.weixin.qq.com/s/3KafTQA_-lHNFpqBAqbtpg}{Eigenvalues of Compact Symmetric Operators.} %34
	\item  \href{https://mp.weixin.qq.com/s/zWsrqCXvefTB2LR3J6fXjg}{The Fredholm Alternative.} %35
	\item  \href{https://mp.weixin.qq.com/s/Jxu8WofU5jjbppWJwGW19g}{An Application to Integral Operators.} %36
	\item \href{https://pan.baidu.com/s/1MrqEmxLAqhar2GiFoDc91g?pwd=1121}{Materials}
\end{enumerate}

%%%%%%%%%%%%%%%%%%%%%%%%%%%%%%%%%%%%%%%%%%%%%%%%%%%%%%%%%%%%%%%%%%%%%%%%%%%%%%%%%%%%%%%%%%%%%%%%%%%%%%%%%%%%%%%%%%%%%%%
%\bibliographystyle{ieeetr} % number
%%\bibliographystyle{unsrtnat} % author year
%\bibliography{HeBib}
%%%%%%%%%%%%%%%%%%%%%%%%%%%%%%%%%%%%%%%%%%%%%%%%%%%%%%%%%%%%%%%%%%%%%%%%%%%%%%%%%%%%%%%%%%%%%%%%%%%%%%%%%%%%%%%%%%%%%%%
\begin{flushright}
	\tiny \today 
\end{flushright}
%%%%%%%%%%%%%%%%%%%%%%%%%%%%%%%%%%%%%%%%%%%%%%%%%%%%%%%%%%%%%%%%%%%%%%%%%%%%%%%%%%%%%%%%%%%%%%%%%%%%%%%%%%%%%%%%%%%%%%%
\end{document}
%%%%%%%%%%%%%%%%%%%%%%%%%%%%%%%%%%%%%%%%%%%%%%%%%%%%%%%%%%%%%%%%%%%%%%%%%%%%%%%%%%%%%%%%%%%%%%%%%%%%%%%%%%%%%%%%%%%%%%%
              