%%%%%%%%%%%%%%%%%%%%%%%%%%%%%%%%%%%%%%%%%%%%%%%%%%%%%%%%%%%%%%%%%%%%%%%%%%%%%%%%%%%%%%%%%%%%%%%%%%%%%%%%%%%%%%%%%%%%%
%%%%%%%%%%%%%%%%%%%%%%%%%%%%%%%%%%%%%%%%%%%%%   Author:Yao Zhang  %%%%%%%%%%%%%%%%%%%%%%%%%%%%%%%%%%%%%%%%%%%%%%%%%%%
%%%%%%%%%%%%%%%%%%%%%%%%%%%%%%%%%%%%%%%%%%%%% Email: jaafar_zhang@163.com %%%%%%%%%%%%%%%%%%%%%%%%%%%%%%%%%%%%%%%%%%%
%%%%%%%%%%%%%%%%%%%%%%%%%%%%%%%%%%%%%%%%%%%%%%%%%%%%%%%%%%%%%%%%%%%%%%%%%%%%%%%%%%%%%%%%%%%%%%%%%%%%%%%%%%%%%%%%%%%%%
\documentclass[11pt]{article}
\usepackage{babel}
\usepackage[utf8]{inputenc} 
\usepackage[table]{xcolor}
\usepackage[most]{tcolorbox}
\usepackage[left=2.50cm, right=1.50cm, top=2.0cm, bottom=2.50cm]{geometry}
\usepackage{xcolor,url}
\usepackage{amsmath,amsthm,amsfonts,amssymb,amscd,multirow,booktabs,fullpage,calc,multicol}
\usepackage{lastpage,enumitem,fancyhdr,mathrsfs,wrapfig,setspace,cancel,amsmath,empheq,framed}
\usepackage[retainorgcmds]{IEEEtrantools}
\usepackage{subfig,graphicx,framed}
\usepackage{ctex}
\usepackage{txfonts}
\usepackage{bbm}
\usepackage{chngcntr}
\usepackage[colorlinks,linkcolor=blue,anchorcolor=green,citecolor=red,urlcolor=blue]{hyperref}
\usepackage{titlesec}
%%%%%%%%%%%%%%%%%%%%%%%%%%%%%%%%%%%%%%%%%%%%%%%%%%%%%%%%%%%%%%%%%%%%%%%%%%%%%%%%%%%%%%%%%%%%%%%%%%%%%%%%%%%%%%%%%%%%%%
\newtheorem{thm}{Theorem}[section]
\newtheorem{defi}{Definition}[subsection]
\newtheorem{exercise}{Exercise}[subsection]
\newtheorem{note}{Note}[subsection]
\newtheorem{notation}{Notation}
\newtheorem{lemma}{Lemma}[subsection]
\newtheorem{proposition}{Proposition}[subsection]
\newtheorem{example}{Example}[subsection]
\newtheorem{problem}{Problem}[section]
\newtheorem{homework}{Homework}[section]
\newtheorem{summary}{Summary}[subsection]
\newtheorem{corollary}{Corollary}[subsection]
\newtheorem{rmk}{Remark}[section]
\usepackage{romannum}
%%%%%%%%%%%%%%%%%%%%%%%%%%%%%%%%%%%%%%%%%%%%%%%%%%%%%%%%%%%%%%%%%%%%%%%%%%%%%%%%%%%%%%%%%%%%%%%%%%%%%%%%%%%%%%%%%%%%%
\newlength{\tabcont}
\setlength{\parindent}{0.0in}
\setlength{\parskip}{0.05in}
\colorlet{shadecolor}{orange!15}
\parindent 0in
\parskip 12pt
\geometry{margin=1in, headsep=0.25in}
%%%%%%%%%%%%%%%%%%%%%%%%%%%%%%%%%%%%%%%%%%%%%%%%%%%%%%%%%%%%%%%%%%%%%%%%%%%%%%%%%%%%%%%%%%%%%%%%%%%%%%%%%%%%%%%%%%%%%
\graphicspath{ {img/EoM/}}
%%%%%%%%%%%%%%%%%%%%%%%%%%%%%%%%%%%%%%%%%%%%%%%%%%%%%%%%%%%%%%%%%%%%%%%%%%%%%%%%%%%%%%%%%%%%%%%%%%%%%%%%%%%%%%%%%%%%%
%\renewcommand{\cite}[1]{[#1]}
\makeatletter
\@addtoreset{equation}{section}
\makeatother
\renewcommand{\theequation}{\arabic{section}.\arabic{equation}}
\renewcommand{\contentsname}{\centering \small \color{blue} Contents}
%\counterwithin{figure}{section}
\renewcommand{\figurename}{\textbf{Fig.}}
%\renewcommand{\refname}{\textbf{\kaishu 参考文献}}
\renewcommand{\refname}{\textbf{Bibliography}}
\setcounter{secnumdepth}{4}
\titleformat{\paragraph}
{\normalfont\normalsize\bfseries}{\theparagraph}{1em}{}
\titlespacing*{\paragraph}{0pt}{3.25ex plus 1ex minus .2ex}{1.5ex plus .2ex}
\def\beginrefs{\begin{list}%
		{[\arabic{equation}]}{\usecounter{equation}
			\setlength{\leftmargin}{0.8truecm}\setlength{\labelsep}{0.4truecm}%
			\setlength{\labelwidth}{1.6truecm}}}
	\def\endrefs{\end{list}}
\def\bibentry#1{\item[\hbox{[#1]}]}
%%%%%%%%%%%%%%%%%%%%%%%%%%%%%%%%%%%%%%%%%%%%%%%%%%%%%%%%%%%%%%%%%%%%%%%%%%%%%%%%%%%%%%%%%%%%%%%%%%%%%%%%%%%%%%%%%%%%%%
%\begin{figure}[!htb]
%	\centering
%	\subfloat[$A \cap B$]{%
%		\includegraphics[width=0.3\linewidth,height=0.2\linewidth]{img001.jpg}}
%	\label{img001}\qquad \qquad %\hfill
%	\subfloat[${A_1} \cap {A_2} \cap {A_3}$]{%
%		\includegraphics[width=0.3\linewidth,height=0.2\linewidth]{img002.jpg}}
%	\label{img002}
	%\caption{ Examples.}
%\end{figure}
%\begin{figure}[!htb]
%	\centering
%	\includegraphics[width=0.4\linewidth,height=0.3\linewidth]{img005.jpg}
%	\label{img005}
	%\caption{ illustration for $ 3 $}
%\end{figure}
%\={a}1 \'{a}2\v{a}3\.{a}4

\usepackage{datetime}
\renewcommand{\today}{\shortmonthname[\the\month] \the \day,  \the\year}
%%%%%%%%%%%%%%%%%%%%%%%%%%%%%%%%%%%%%%%%%%%%%%%%%%%%%%%%%%%%%%%%%%%%%%%%%%%%%%%%%%%%%%%%%%%%%%%%%%%%%%%%%%%%%%%%%%%%%%
\begin{document}
	\kaishu 
	%\thispagestyle{empty}
	\pagenumbering{arabic} 
	\setcounter{section}{0}
	\begin{center}
		{\LARGE  Numerical Analysis}
		
		%\vspace{-0.25cm}
		
		%{\large \href{https://sites.psu.edu/wxs27/}{Wen Shen}}
	\end{center}
%%%%%%%%%%%%%%%%%%%%%%%%%%%%%%%%%%%%%%%%%%%%%%%%%%%%%%%%%%%%%%%%%%%%%%%%%%%%%%%%%%%%%%%%%%%%%%%%%%%%%%%%%%%%%%%%%%%%%%
%%\newpage 
%%\thispagestyle{empty}	
%%%%%%%%%%%%%%%%%%%%%%%%%%%%%%%%%%%%%%%%%%%%%%%%%%%%%%%%%%%%%%%%%%%%%%%%%%%%%%%%%%%%%%%%%%%%%%%%%%%%%%%%%%%%%%%%%%%%%%
%\tableofcontents	
%{\pagestyle{empty}\mbox{}\newpage\pagestyle{empty}}
%\newpage 
%{\pagestyle{empty}\mbox{}\newpage\pagestyle{empty}}
%%%%%%%%%%%%%%%%%%%%%%%%%%%%%%%%%%%%%%%%%%%%%%%%%%%%%%%%%%%%%%%%%%%%%%%%%%%%%%%%%%%%%%%%%%%%%%%%%%%%%%%%%%%%%%%%%%%%%%
%%\newpage 
\setcounter{page}{1}

%\vspace{1.5cm}

\begin{center}
	{\Large \href{https://www.youtube.com/playlist?list=PLbxFfU5GKZz3D4NPYvvY7dvXiZ0awd4zn}{A. Numerical Computation}} {\large by \href{https://sites.psu.edu/wxs27/}{Wen Shen}}
\end{center}

\vspace{-1cm}

\section*{1. Computer Arithmetic}

\vspace{-0.5cm}

\begin{multicols}{2}
	\begin{enumerate}
		\item \href{https://mp.weixin.qq.com/s/cgsyPIr7w-fEbRACNE1ezg}{Introduction to Numerical Computation}	%1
		\item \href{https://mp.weixin.qq.com/s/QCssJJjJ8xno2kND1BAYaA}{Repre. of numbers in different bases}	%2
		\item \href{https://mp.weixin.qq.com/s/iWqrQSD0Gqr8DKG2GydUbQ}{Floating point representation}	%3
		\item \href{https://mp.weixin.qq.com/s/9Tj0u32iptv76BmdE0pMOA}{Introduction to Matlab}	%4
		\item \href{https://mp.weixin.qq.com/s/U4CjxHjiI_Xng5sTcorRVQ}{Loss of significance}	%5
		\item \href{https://mp.weixin.qq.com/s/hny_RovdEhvd1qBYzPRr8g}{More examples on lost of significant digits}	%6
		\item \href{https://mp.weixin.qq.com/s/VjbfnKxb1SXusT-z9vVFYw}{Review of Taylor Series}	%7
		\item \href{https://mp.weixin.qq.com/s/sM24GvqIp8F-Q6NAfz0iGQ}{Finite Difference Approximation}	%8
	\end{enumerate}
\end{multicols}

\vspace{-1cm}

\section*{2. Polynomial Interpolation}

\vspace{-0.5cm}

\begin{multicols}{2}
	\begin{enumerate}
		\item \href{https://mp.weixin.qq.com/s/9yUIeWYNzgn2FgCt8AGujQ}{Poly. inter., Van der Monde matrix}	%9
		\item \href{https://mp.weixin.qq.com/s/9ML5eqtNbdMjpY5uYRQaSg}{Polynomial interpolation, Lagrange form}	%10
		\item \href{https://mp.weixin.qq.com/s/se_xxFTL8u--rEhU2ydl1A}{Barycentric forms of Lagrange polynomials}	%11
		\item \href{https://mp.weixin.qq.com/s/Bq_CHw7hC2YKOJlEoKOXbg}{Poly. inter., Newton's divided diff. 1}	%12
		\item \href{https://mp.weixin.qq.com/s/8f7N-_PPe0TTwwQC8kvFLw}{Poly. inter., Newton's divided diff. 2}	%13
		\item \href{https://mp.weixin.qq.com/s/6bWSlqWqDOJ9Ju06fdTOtg}{Poly. inter., Newton's divided diff. 3}	%14
		\item \href{https://mp.weixin.qq.com/s/cCADIfro1dgt1ilxQgAnEQ}{Poly. inter., existence and uniqueness Thm}	%15
		\item \href{https://mp.weixin.qq.com/s/k00gBj9Sod_2YdxdX-pZqA}{Error Thm for Polynomial Interpolation}	%16
		\item \href{https://mp.weixin.qq.com/s/HmGJt5Fb1LGDkwZW1MQibg}{Examples for Error Theorem}	%17
		\item \href{https://mp.weixin.qq.com/s/rNiDjcs1ZCZVLgzX8i4tPA}{Uniform Grid}	%18
		\item \href{https://mp.weixin.qq.com/s/XoZ4JQd7TGLjr81kCegHgQ}{Chebyshev nodes}	%19
		\item \href{https://mp.weixin.qq.com/s/tj3uPgaqMjDSXXfUAte4rQ}{Matlab simulation for VdM matrix}	%20
		\item \href{https://mp.weixin.qq.com/s/0CyWNhXiXUwuF3jzr193Lg}{Matlab Simulation for Chebyshev nodes}	%21
		\item \href{https://mp.weixin.qq.com/s/GfiEj_0CgIvv-uiLXS3frQ}{Matlab simulation with uniform grid}	%22
		\item \href{https://mp.weixin.qq.com/s/d9y1vepHoLCxwcbZnSAtxg}{Aitken-Neville Interpolation}	%23
		\item \href{https://mp.weixin.qq.com/s/RFVVuY_8-E4KUYCL5LjXdg}{Hermite interpolation}	%24
	\end{enumerate}
\end{multicols}

\vspace{-1cm}

\section*{3. Piecewise Polynomial Interpolation: Splines}

\vspace{-0.5cm}

\begin{multicols}{2}
	\begin{enumerate}
		\item \href{https://mp.weixin.qq.com/s/SKpBAnBLArmTadU5MLYrcw}{Splines}	%25
		\item \href{https://mp.weixin.qq.com/s/wT0jvmxhTavpe5M1MSay1w}{Examples of Spline Functions}	%26
		\item \href{https://mp.weixin.qq.com/s/OcFv0FlErxEPgyQUeFa24Q}{Linear Splines}	%27
		\item \href{https://mp.weixin.qq.com/s/pXGhthhxw57NhTG1lCNWhg}{Quadratic Splines}	%28
		\item \href{https://mp.weixin.qq.com/s/RwgTND4ozRQFIZOrQTSb1A}{Natural cubic spline}	%29
		\item \href{https://mp.weixin.qq.com/s/jC8xpEFhZc-dcZbLsGqjVg}{Natural Cubic Splines, Derivation of Algo.}	%30
		\item \href{https://mp.weixin.qq.com/s/HKCImmgLYY_QdKPrMJXBpw}{Smoothness Thm for Natural Cubic Spline}	%31
		\item \href{https://mp.weixin.qq.com/s/nITJmQghAwXAtEFS5orRAQ}{Matlab Simulation}	%32
		\item \href{https://mp.weixin.qq.com/s/4bpvHd4JuAQ4Nq89SnQ65A}{Other Types of BCs for Cubic Splines}	%33
		\item \href{https://mp.weixin.qq.com/s/2NsOIcISB9OHfhv8BMKjeg}{Cubic Hermite Spline}	%34
		\item \href{https://mp.weixin.qq.com/s/zWWRpdb7sFGYqTZd8sRScg}{B{\'e}zier curves}	%35
	\end{enumerate}
\end{multicols}

\newpage

\section*{4. Numerical Integration}

\vspace{-0.5cm}

\begin{multicols}{2}
	\begin{enumerate}
		\item \href{https://mp.weixin.qq.com/s/n0nvErfrI6dky09jLMUrQw}{Numerical integration: Trapezoid Rule}	%36
		\item \href{https://mp.weixin.qq.com/s/OZSXY4VVAWWYm0ZGQwIfSw}{Example and sample codes for TR}	%37
		\item \href{https://mp.weixin.qq.com/s/HWJass--DbZm6p0k-g4L1w}{Error estimate for trapezoid rule}	%38
		\item \href{https://mp.weixin.qq.com/s/27i-0YAAzpgSc4Mm9z_1kQ}{Simpson's rule, derivation}	%39
		\item \href{https://mp.weixin.qq.com/s/pz0EkDTPP1SIh_ghAqp0wQ}{Ex. and sample code for Simpson's rule}	%40
		\item \href{https://mp.weixin.qq.com/s/xxvjM123pQ8Krr2NnEMxoA}{Error estimate for Simpson's Rule}	%41
		\item \href{https://mp.weixin.qq.com/s/hMED_ZgGRiBKk6JEov4eAA}{Recursive trapezoid, composite schemes}	%42
		\item \href{https://mp.weixin.qq.com/s/5cQiy-CkeRIJ67hpuY_8XA}{Richardson extrapolation}	%43
		\item \href{https://mp.weixin.qq.com/s/_zdSdK2JI_ZLCkutSbpR1A}{Romberg algorithm}	%44
		\item \href{https://mp.weixin.qq.com/s/xV5PZp8vDpXRYFsuR46KbQ}{Adaptive Simpson's Quadrature}	%45
		\item \href{https://mp.weixin.qq.com/s/jNlAtU4pNMvYs-ixW1TE6Q}{Gaussian quadrature 1}	%46
		\item \href{https://mp.weixin.qq.com/s/xWZgQmc-pVxl8fWgi3iHUQ}{Gaussian quadrature 2}	%47
		\item \href{https://mp.weixin.qq.com/s/5eDxwwn8L5M7nmOmEyk16w}{Matlab}	%48
		\item \href{https://mp.weixin.qq.com/s/FYmE0A8-vO9wXnzK0GP_uQ}{Numerical integration rules in a more abstract setting}	%49
		\item \href{https://mp.weixin.qq.com/s/f2HuxK0JO5lKGQuvFRnq2A}{Integrals over Infinite Intervals, Gauss Laguerre, Gauss Hermite}	%50
		\item \href{https://mp.weixin.qq.com/s/ggjZNv1d9WSFa31VOaWdyg}{Monte Carlo Integration}	%51
	\end{enumerate}
\end{multicols}

\vspace{-1cm}

\section*{5. Numerical Solutions of Non-linear equations}

\vspace{-0.5cm}

\begin{multicols}{2}
	\begin{enumerate}
		\item \href{https://mp.weixin.qq.com/s/yr5EHHwZVtPGon3EdBoDzA}{Numerical Solutions of nonlinear equations}	%52
		\item \href{https://mp.weixin.qq.com/s/Ph10nb81tuMZ47YzqciLVA}{Bisection method}	%53
		\item \href{https://mp.weixin.qq.com/s/11ak7oZXlkzshrAGEEN2KA}{Fixed point iteration, algorithm}	%54
		\item \href{https://mp.weixin.qq.com/s/QrKBuI9ZjwAFwDfW-_Nbbw}{Fixed Point iteration, convergence}	%55
		\item \href{https://mp.weixin.qq.com/s/s4O2FA2BVJkFaT8_JxMeTA}{Fixed point iteration, error analysis}	%56
		\item \href{https://mp.weixin.qq.com/s/cTfmwA3OUkn4altTOktS4Q}{Newton's iteration}	%57
		\item \href{https://mp.weixin.qq.com/s/sdMxmu7SPBb_pO6pfLH4Jw}{Newton's iteration, convergence}	%58
		\item \href{https://mp.weixin.qq.com/s/0uuR1jTBduU4bDrUjUFcvA}{Newton's iteration, example, code}	%59
		\item \href{https://mp.weixin.qq.com/s/vdyiZ3I6bRUOoDIiOQJO1w}{Secant method}	%60
		\item \href{https://mp.weixin.qq.com/s/KXqvDSYHKoAjBZpUB_IjKA}{Aitken Method and Acceleration}	%61
		\item \href{https://mp.weixin.qq.com/s/jrY_M_J5DfQAyZxXCf1Q8A}{Halley's Method: an improved version of Newton's method}	%62
		\item \href{https://mp.weixin.qq.com/s/lBUY6TME_k3m3MCwKZuflQ}{Roots of Polynomials,  Horner's Algorithm}	%63
		\item \href{https://mp.weixin.qq.com/s/Z5gQYKfK1C8mfl8LysA6Dg}{Continuation Method}	%64
	\end{enumerate}
\end{multicols}

\vspace{-1cm}

\section*{6. Direct Methods for Systems of Linear Equations}

\vspace{-0.5cm}

\begin{multicols}{2}
	\begin{enumerate}
		\item \href{https://mp.weixin.qq.com/s/Z_13kehDMeB4_i88-9_GUw}{System of linear eq.: Gaussian Elimination}	%65
		\item \href{https://mp.weixin.qq.com/s/8WBh4ilKGracDy3tYRUsDA}{LU-Factorization}	%66
		\item \href{https://mp.weixin.qq.com/s/hcgXql0waFNvYluhlB5uDw}{Cholesky Factorization}	%67
		\item \href{https://mp.weixin.qq.com/s/NqoT8ERfz-LFmm7wfHGUfg}{Vector Norms}	%68
		\item \href{https://mp.weixin.qq.com/s/6M0sQoxl8mHelkVNf5yu7A}{Matrix norms}	%69
		\item \href{https://mp.weixin.qq.com/s/f0eZ2Q5cJsCjYRBCw89jkQ}{Condition number of a matrix}	%70
		\item \href{https://mp.weixin.qq.com/s/iT0RD2HiMk5WthwTX46-qQ}{Overdetermined Systems and QR fact.}	%71
		\item \href{https://mp.weixin.qq.com/s/-z3cDzIGcDXNzKJHdkQQWQ}{SVD and Image compression}	%72
	\end{enumerate}
\end{multicols}

\newpage

\section*{7. Fixed Point Iterative Solvers for Linear and Non-linear Systems}

\vspace{-0.5cm}

\begin{multicols}{2}
	\begin{enumerate}
		\item \href{https://mp.weixin.qq.com/s/svbd9RpB8fl26FuIunXtPA}{Iterative Solvers: Jacobi Iterations}	%73
		\item \href{https://mp.weixin.qq.com/s/3a-ppJKyB8SzioYF7DVsWQ}{Example}	%74
		\item \href{https://mp.weixin.qq.com/s/4z5IDIaFvsZRnA3pOP6EGA}{Gauss-Seidal iterations}	%75
		\item \href{https://mp.weixin.qq.com/s/nD_8BHYLl6-1rpTvGOMjmw}{SOR iterations}	%76
		\item \href{https://mp.weixin.qq.com/s/XTJHqi26WIYJTYxJCT9EWA}{Linear Fixed Point Iteration for systems}	%77
		\item \href{https://mp.weixin.qq.com/s/A75sYJaEJRkahnw0Z-uwnA}{Convergence Analysis}	%78
		\item \href{https://mp.weixin.qq.com/s/wwOVdVvId5faciM8pdKaDg}{Matlab}	%79
		\item \href{https://mp.weixin.qq.com/s/wMT0eHEjKI7soRb0QPXo3A}{Systems of Non-linear Equations, Fixed Point iterations}	%80
		\item \href{https://mp.weixin.qq.com/s/Ermc6MMpN3loAUBOLWnJuA}{Systems of Non-linear Equations, Newton iterations}	%81
	\end{enumerate}
\end{multicols}

\vspace{-1cm}

\section*{8. The Method of Least Squares}

\vspace{-0.5cm}

\begin{multicols}{2}
	\begin{enumerate}
		\item \href{https://mp.weixin.qq.com/s/3PQaIc9mpxbaWeGLWRzXDg}{Least Squares Method: Linear Regression}	%82
		\item \href{https://mp.weixin.qq.com/s/0uE2K6WzW8zt5Oz-1jlSvA}{Linear Least Squares with three functions}	%83
		\item \href{https://mp.weixin.qq.com/s/z1CLA5QYT2zhR1wUizg1Eg}{General Linear Squares Method}	%84
		\item \href{https://mp.weixin.qq.com/s/VzQYg4ogJSljxvyGjp_Drw}{Nonlinear Least Squares Method}	%85
		\item \href{https://mp.weixin.qq.com/s/e09xN6GGu37ZJvPHRsm8Tw}{Least Squares Method for continuous functions}	%86
		\item \href{https://mp.weixin.qq.com/s/Uph6THvP0ZLu99ZvhtZ8YQ}{Examples of orthogonal basis functions}	%87
		\item \href{https://mp.weixin.qq.com/s/uAKAPJHYUyojZdPfAVuorw}{Matlab Examples on Least Squares Method}	%88
	\end{enumerate}
\end{multicols}

\vspace{-1cm}

\section*{9. Numerical Solutions for ODEs}

\vspace{-0.5cm}

\begin{multicols}{2}
	\begin{enumerate}
		\item \href{https://mp.weixin.qq.com/s/edOjNMh_l0dGaZ9ANC6qdA}{Numerical solutions for ODEs}	%89
		\item \href{https://mp.weixin.qq.com/s/cyub8b9hvutn3mw2MdBUbQ}{Taylor series methods for ODEs}	%90
		\item \href{https://mp.weixin.qq.com/s/hZjW_0rV5wAp7Iq0I9szXg}{Examples of Taylor Series Method}	%91
		\item \href{https://mp.weixin.qq.com/s/ravom15hYpGNlA0PWoFcQQ}{Error analysis for Taylor Series Methods}	%92
		\item \href{https://mp.weixin.qq.com/s/KR9Gm611HYpDnaBj0N2ldw}{RK Methods, Euler step and Heun step}	%93
		\item \href{https://mp.weixin.qq.com/s/ta5WltWKFz8EFutuUZnliQ}{The classical 4th order RK method}	%94
		\item \href{https://mp.weixin.qq.com/s/-X-J-piAF8J3xZJdeRMhZA}{Numerical Simulations of RK methods}	%95
		\item \href{https://mp.weixin.qq.com/s/kZRaqvgMPdwmoUh1vyHE8Q}{Adaptive RKF method}	%96
		\item \href{https://mp.weixin.qq.com/s/_55GlLIueMUn5VdZOc-I5g}{Explicit AB method for ODEs}	%97
		\item \href{https://mp.weixin.qq.com/s/H9mNqIlLHbd2lbUAtMxgtA}{Examples of explicit AB methods}	%98
		\item \href{https://mp.weixin.qq.com/s/OjdvaUB1qSAGPgN_gUYOIQ}{Implicit ABM methods}	%99
		\item \href{https://mp.weixin.qq.com/s/GQFcWYDYtJcWdTlLwnHzAQ}{Multistep ABM methods for ODEs}	%100
		\item \href{https://mp.weixin.qq.com/s/evKNoMBeTj-ffGBdfiIHqA}{First order systems of ODEs}	%101
		\item \href{https://mp.weixin.qq.com/s/rAaky-w2djQD-LCkqbnnug}{Higher order ODEs and systems}	%102
		\item \href{https://mp.weixin.qq.com/s/-TM3NlotxD3JP45Bpmt3Vg}{Stiffness of ODEs, Scalar ODEs}	%103
		\item \href{https://mp.weixin.qq.com/s/yDlqxbww-g6jABb_QGKdQg}{Systems of ODEs}	%104
		\item \href{https://mp.weixin.qq.com/s/BffTGTEjotU-zntWU3mAmQ}{Stiff system, Implicit method}	%105
		\item \href{https://mp.weixin.qq.com/s/Q0wxbw4_49__3xyCyfCQTA}{Geometric Int.: Symplectic, Hamiltonian preserving method}	%106
	\end{enumerate}
\end{multicols}

\newpage 

\section*{10. Numerical Methods for Two-point Boundary Value Problems}

\vspace{-0.5cm}

\begin{multicols}{2}
	\begin{enumerate}
		\item \href{https://mp.weixin.qq.com/s/rQJCn_4LOho9bQBjVE11LQ}{Two-point Boundary value problems}	%107
		\item \href{https://mp.weixin.qq.com/s/m6VF2jXMJzcBlwPA1PdHIA}{Shooting method}	%108
		\item \href{https://mp.weixin.qq.com/s/Pv47HQkofQ71zdduYCYbTw}{Linear Shooting method, extensions}	%109
		\item \href{https://mp.weixin.qq.com/s/8-vQbUMZCYjAAiiECNfzXw}{Nonlinear shooting method}	%110
		\item \href{https://mp.weixin.qq.com/s/CNcZPxMqd_Si46o8aDoZdw}{FDM for two-point BVP}	%111
		\item \href{https://mp.weixin.qq.com/s/0jHm4X9vRXTRN7CzLtzRGg}{Finite Difference Methods in 1D}	%112
		\item \href{https://mp.weixin.qq.com/s/suoWOEbjIqgXuloTSAzNIA}{Neumann BC, Poisson's equation}	%113
		\item \href{https://mp.weixin.qq.com/s/gPpcQKQ2EYzQUNqIpouq4A}{Robin BC for Poisson Equation}	%114
	\end{enumerate}
\end{multicols}

\vspace{-1cm}

\section*{11. Finite Difference Methods for Some Partial Differential Equations}

\vspace{-0.5cm}

\begin{multicols}{2}
	\begin{enumerate}
		\item \href{https://mp.weixin.qq.com/s/GYxDOVdBZ-yz0AldbRzKfw}{FDM for Laplace Equation in 2D}	%115
		\item \href{https://mp.weixin.qq.com/s/3dHQIBRT1jz5puezMADdUw}{System of Linear Equations for Discrete Laplace Equation with FDM}	%116
		\item \href{https://mp.weixin.qq.com/s/CmM-GYuzoWVd5rCHPgEC7A}{Laplace equation with non-homogeneous Dirichlet BCs}	%117
		\item \href{https://mp.weixin.qq.com/s/n3VHX-ZthlABzNen_NnAfQ}{Poisson equation on a unit square}	%118
		\item \href{https://mp.weixin.qq.com/s/8DiZQA7OD156V2Q4qxdVag}{Laplace equation with Neumann BC}	%119
		\item \href{https://mp.weixin.qq.com/s/JoE1-6NE2cNPj0MJFWq5GQ}{Heat equation in 1D, forward Euler method}	%120
		\item \href{https://mp.weixin.qq.com/s/ef-e0SnuAB4cjJ_lJC84QA}{Heat equation, CFL stability condition for explicit forward Euler method}	%121
		\item \href{https://mp.weixin.qq.com/s/k_O9jjISbrII1ur7B9lYoQ}{Heat equation, implicit backward Euler step, unconditionally stable}	%122
		\item \href{https://mp.weixin.qq.com/s/r9P8c6EtdQxUzLoaOeO1xA}{Heat equation, Crank-Nicholson scheme}	%123
		\item \href{https://mp.weixin.qq.com/s/8HBzNrEH18BdJ-dgltxWtA}{Heat equation with Neumann BC}	%124
		%\item \href{url}{Materials}
	\end{enumerate}
\end{multicols}


\newpage 

\begin{center}
	{\Large \href{http://www.math.ncu.edu.tw/~cchsiao/Course/Numerical_Analysis_082/index.html}{B. Numerical Analysis}} {\large by \href{http://www.math.ncu.edu.tw/~cchsiao/}{Ching-hsiao Cheng}}
\end{center}

\vspace{-1cm}

\section*{\href{https://www.youtube.com/playlist?list=PLGwoNNTgFGehjEtt_1jijzMQslt7NV-Oh}{1. Mathematical Preliminaries}} %1

\vspace{-0.5cm}

\begin{enumerate}
	\item \href{https://mp.weixin.qq.com/s/txFQz9Uv6sSRECOa8RAvng}{数值分析课程介绍与函数的极限、连续、可微性质} %1
	\item \href{https://mp.weixin.qq.com/s/BuHnLOTA9n4k2t0ojFIJzw}{均值定理、中间值定理、函数的可积性、Lebesgue 定理与广义积分均值定理} %2
	\item \href{https://mp.weixin.qq.com/s/V7arbQphHjXyKOA3TIzmTw}{单变数函数的泰勒定理} %3
	\item \href{https://mp.weixin.qq.com/s/El2mhfgln6Bm6YEAlTzQHA}{多变数函数的泰勒定理} %4
	\item \href{https://mp.weixin.qq.com/s/uBixHedRMFbjLdzP9Uqvgg}{计算多变数函数泰勒多项式之实例, 与 Big O 符号} %5
\end{enumerate}

\vspace{-1cm}

\section*{\href{titlehttps://www.youtube.com/playlist?list=PLGwoNNTgFGegki7UM7c-jKpvbisfBmRGn}{2. Solutions of Nonlinear Equations}} %2

\vspace{-0.5cm}

\begin{enumerate}
	\item \href{https://mp.weixin.qq.com/s/6xS0HIZ5yhT6GkG2ZkonPg}{第二章的介绍, 与二分法求根} %6
	\item \href{https://mp.weixin.qq.com/s/WP8zPNNjeCUD3U4TxZzjSQ}{二分法误差估计与固定点迭代} %7
	\item \href{https://mp.weixin.qq.com/s/8wV18NXy7BJWM2IcTOfspA}{Banach 固定点定理的证明} %8
	\item \href{https://mp.weixin.qq.com/s/6TkANAVqRoM2SJOEXEhPPg}{使用固定点迭代找函数零根的实例, 以及牛顿法的进一步介绍} %9
	\item \href{https://mp.weixin.qq.com/s/A8n6ndiE4Ak4f_CaBDWWnA}{牛顿法二次收敛的证明以及割线法的介绍} %10
	\item \href{https://mp.weixin.qq.com/s/TSTofV8kOoEPSLO0GD_zLw}{割线法求根的复习、两个未知数与方程求根的牛顿法与matlab 实作} %11
	\item \href{https://mp.weixin.qq.com/s/o9Za8whZfVxK2256tnOMUA}{matlab 实作、多个未知数与方程求零根的牛顿法} %12
\end{enumerate}

\vspace{-1cm}

\section*{\href{https://www.youtube.com/playlist?list=PLGwoNNTgFGejoumJVBpp-V046IPxJqWUu}{3. Interpolation and Polynomial Approximation}} %3

\vspace{-0.5cm}

\begin{enumerate}
	\item \href{https://mp.weixin.qq.com/s/PLQKX2qhPnwlk1HUNizH5A}{第三章主要内容的介绍: 关于插值} %13
	\item \href{https://mp.weixin.qq.com/s/wu2pRrydW-SUvpQY8-x5wg}{Lagrange 插值多项式、一个函数与其 Lagrange 插值多项式的误差} %14
	\item \href{https://mp.weixin.qq.com/s/iMYF5UcoD65mAuAc3gUPSg}{Newton's divided difference 插值多项式} %15
	\item \href{https://mp.weixin.qq.com/s/kGiKqbzl1A8fTdq-HDnsZQ}{使用高低 divided difference 的关系求 Newton's 插值多项式的系数, 以及 Hermite 插值多项式的介绍} %16
	\item \href{https://mp.weixin.qq.com/s/P64oecp7mFGH7QsRlzX6_A}{Hermite 多项式的形式与余项的证明} %17
	\item \href{https://mp.weixin.qq.com/s/jMgpFqg1XSVp2-eTpZb7Vw}{Spline 插值的介绍, 以及决定 cubic spline 的条件} %18
	\item \href{https://mp.weixin.qq.com/s/bWMkLmoaJ-QsIaklfTtS-A}{Natural cubic spline 与 Clamped cubic spline 的唯一性} %19
\end{enumerate}

\vspace{-1cm}

\section*{\href{https://www.youtube.com/playlist?list=PLGwoNNTgFGehTrxUCQ47fvWbH5YqPViUB}{4. Numerical Differentiation and Integration}} %4

\vspace{-0.5cm}

\begin{enumerate}
	\item \href{https://mp.weixin.qq.com/s/j52k12G3aPAPHHsvUKhNDQ}{数值微分中的 central difference formula 与误差} %20
	\item \href{https://mp.weixin.qq.com/s/3TYsTAwIHfKJ2fFQh9dZOA}{使用 Richarson extrapolation 由低阶演算法得到高阶演算法} %21
	\item \href{https://mp.weixin.qq.com/s/7DRpu8YgUsOpr7GIo0ci1g}{Richardson extrapolation 的一般式、使用 Lagrange 插值多项式求一阶导数的演算法} %22
	\item \href{https://mp.weixin.qq.com/s/Q8IIc3Bvh67NRjlYBJ9SrQ}{数值积分: 以 Lagrange 插值多项式推导梯形法与 Simpson 法、以 Taylor 定理推导中点法} %23
	\item \href{https://mp.weixin.qq.com/s/MXBhKAsO7hfMbqCpacJQ5g}{使用泰勒定理与柯西均值定理推导 Simpson 法, 以及 composite 数值积分法} %24
	\item \href{https://mp.weixin.qq.com/s/p0FYIQoVicf9VPa5OZ5prg}{Legendre 多项式两个性质的证明, 与使用 Gauss quadrature 做积分的证明} %25
\end{enumerate}

\vspace{-1cm}

\section*{\href{https://www.youtube.com/playlist?list=PLGwoNNTgFGehnDSyUWtPPXhe-awbm-hz8}{5. Direct and Iterative Methods for Solving Linear Systems}} %5

\vspace{-0.5cm}

\begin{enumerate}
	\item \href{https://mp.weixin.qq.com/s/pVb-OJywvIz6iXVjzh6vig}{线性代数关于可逆矩阵的复习} %26
	\item \href{https://mp.weixin.qq.com/s/tVrhFFiSb3lMqxbi1NtuPQ}{线性代数关于基本列(行)运算与 elementary matrix 等价关系的复习} %27
	\item \href{https://mp.weixin.qq.com/s/_gMsv6Bg5yFdPgodceYzBw}{当 A 是对角矩阵、上下三角矩阵时, 解 Ax=b 的演算法与其计算量} %28
	\item \href{https://mp.weixin.qq.com/s/qJPYZxJ7dVLnKVa6IFK-Ig}{以高斯消去做矩阵的 LU 分解之演算法的 order, 与赋范向量空间} %29
	\item \href{https://mp.weixin.qq.com/s/ldnKvAuTIuIkU9ULvffhDw}{有限维向量空间上的任意两个 norms 都等价的证明} %30
	\item \href{https://mp.weixin.qq.com/s/uFkRKQZkgaxflq7bY7rzNg}{Induced matrix norm 与其性质, 以及一个  $m \times n$ 矩阵的 infinity norm} %31
	\item \href{https://mp.weixin.qq.com/s/bHVmE-3p2OKmpdgi9MwfJg}{矩阵的 1-norm, 2-norm 还有 spectral radius} %32
	\item \href{https://mp.weixin.qq.com/s/MrSYuZxh-j-vsC7faip1fg}{矩阵的 Frobenius norm 与 spectural radius 是所有 norm 中的 inf 的证明} %33
	\item \href{https://mp.weixin.qq.com/s/7PVforjtssKRIXcrG5MtCg}{Spectural radius 是所有 norm 中的 inf 的证明、收敛矩阵的定义, 与收敛矩阵的等价性质} %34
	\item \href{https://mp.weixin.qq.com/s/rOGghA6GGYK3NxfmQtgGYQ}{解 Ax=b 系统之迭代法初探: Jacobi method 与 Gauss-Seidel method 的介绍} %35
	\item \href{https://mp.weixin.qq.com/s/qqO3dIIXlthHbWmhx9JMPg}{x = Tx + c 迭代格式收敛之充分必要条件的证明、严格对角优势矩阵之 Jacobi method 与 Gauss-Seidel method 的收敛性, 以及 SOR 的概念} %36
	\item \href{https://mp.weixin.qq.com/s/-sqCogtHOl6XQFRcMIUDcw}{SOR 补完、绝对误差、相对误差与条件数} %37
\end{enumerate}

\vspace{-1cm}

\section*{\href{https://www.youtube.com/playlist?list=PLGwoNNTgFGeg8szzAhG-B2f3KgFpG5G8H}{6. Numerical Ordinary Differential Equations}} %6

\vspace{-0.5cm}

\begin{enumerate}
	\item \href{https://mp.weixin.qq.com/s/LsOSiuGwMfshC2-MLdtcTQ}{数值常微分方程的开头} %38
	\item \href{https://mp.weixin.qq.com/s/cFdYrdVeGDyufepzMWEQ6g}{一些与数值解初始值问题相关术语的介绍} %39
	\item \href{https://mp.weixin.qq.com/s/_-lV7xTYv-K41S7d1Vmfzg}{边界值问题与泰勒方法} %40
	\item \href{https://mp.weixin.qq.com/s/a5ATswmXkS09zUvdHNujKw}{泰勒方法的应用例、泰勒方法的优缺点、Euler 方法与 Runge-Kutta 方法的介绍} %41
	\item \href{https://mp.weixin.qq.com/s/gbq2xZo0jqZ0jQ-2vnOXfQ}{二次 Runge-Kutta 法的进一步说明} %42
	\item \href{https://mp.weixin.qq.com/s/ELvdgDQTA2dQeHpYWv_nkw}{四次 Runge-Kutta 法的说明, 以及 Collocation 方法解初始值问题} %43
	\item \href{https://mp.weixin.qq.com/s/gc6q0p2EACGvNzDXNklTxA}{使用 Collocation 法与有限差分法解边界值问题, 以及有限元素法解边界值问题的初步介绍} %44
	\item \href{https://mp.weixin.qq.com/s/59iaE1-wromc-jhfjmIG_w}{边界值问题之 variation (weak) form 与某最佳化问题之间的等价性} %45
	\item \href{https://mp.weixin.qq.com/s/3UN0pPcz7JIF1_nB5com7g}{变分形式 (Variational form) 有唯一解的证明, 以及有限元素法求解变分形式解的实作方法} %46
	\item \href{https://mp.weixin.qq.com/s/68LooYSsgRt48UIw3Ve0Yw}{有限元素法的具体实作方法, 以及有限元素法与有限差分法的比较} %47
	%\item \href{url}{Materials}
\end{enumerate}



%%%%%%%%%%%%%%%%%%%%%%%%%%%%%%%%%%%%%%%%%%%%%%%%%%%%%%%%%%%%%%%%%%%%%%%%%%%%%%%%%%%%%%%%%%%%%%%%%%%%%%%%%%%%%%%%%%%%%%%
%\bibliographystyle{ieeetr} % number
%%\bibliographystyle{unsrtnat} % author year
%\bibliography{HeBib}
%%%%%%%%%%%%%%%%%%%%%%%%%%%%%%%%%%%%%%%%%%%%%%%%%%%%%%%%%%%%%%%%%%%%%%%%%%%%%%%%%%%%%%%%%%%%%%%%%%%%%%%%%%%%%%%%%%%%%%%
\begin{flushright}
	\tiny \today 
\end{flushright}
%%%%%%%%%%%%%%%%%%%%%%%%%%%%%%%%%%%%%%%%%%%%%%%%%%%%%%%%%%%%%%%%%%%%%%%%%%%%%%%%%%%%%%%%%%%%%%%%%%%%%%%%%%%%%%%%%%%%%%%
\end{document}
%%%%%%%%%%%%%%%%%%%%%%%%%%%%%%%%%%%%%%%%%%%%%%%%%%%%%%%%%%%%%%%%%%%%%%%%%%%%%%%%%%%%%%%%%%%%%%%%%%%%%%%%%%%%%%%%%%%%%%%
              