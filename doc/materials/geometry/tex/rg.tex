%%%%%%%%%%%%%%%%%%%%%%%%%%%%%%%%%%%%%%%%%%%%%%%%%%%%%%%%%%%%%%%%%%%%%%%%%%%%%%%%%%%%%%%%%%%%%%%%%%%%%%%%%%%%%%%%%%%%%
%%%%%%%%%%%%%%%%%%%%%%%%%%%%%%%%%%%%%%%%%%%%%   Author:Yao Zhang  %%%%%%%%%%%%%%%%%%%%%%%%%%%%%%%%%%%%%%%%%%%%%%%%%%%
%%%%%%%%%%%%%%%%%%%%%%%%%%%%%%%%%%%%%%%%%%%%% Email: jaafar_zhang@163.com %%%%%%%%%%%%%%%%%%%%%%%%%%%%%%%%%%%%%%%%%%%
%%%%%%%%%%%%%%%%%%%%%%%%%%%%%%%%%%%%%%%%%%%%%%%%%%%%%%%%%%%%%%%%%%%%%%%%%%%%%%%%%%%%%%%%%%%%%%%%%%%%%%%%%%%%%%%%%%%%%
\documentclass[11pt]{article}
\usepackage{babel}
\usepackage[utf8]{inputenc} 
\usepackage[table]{xcolor}
\usepackage[most]{tcolorbox}
\usepackage[left=2.50cm, right=1.50cm, top=2.0cm, bottom=2.50cm]{geometry}
\usepackage{xcolor,url}
\usepackage{amsmath,amsthm,amsfonts,amssymb,amscd,multirow,booktabs,fullpage,calc,multicol}
\usepackage{lastpage,enumitem,fancyhdr,mathrsfs,wrapfig,setspace,cancel,amsmath,empheq,framed}
\usepackage[retainorgcmds]{IEEEtrantools}
\usepackage{subfig,graphicx,framed}
\usepackage{ctex}
\usepackage{txfonts}
\usepackage{bbm}
\usepackage{chngcntr}
\usepackage[colorlinks,linkcolor=blue,anchorcolor=green,citecolor=red,urlcolor=blue]{hyperref}
\usepackage{titlesec}
%%%%%%%%%%%%%%%%%%%%%%%%%%%%%%%%%%%%%%%%%%%%%%%%%%%%%%%%%%%%%%%%%%%%%%%%%%%%%%%%%%%%%%%%%%%%%%%%%%%%%%%%%%%%%%%%%%%%%%
\newtheorem{thm}{Theorem}[section]
\newtheorem{defi}{Definition}[subsection]
\newtheorem{exercise}{Exercise}[subsection]
\newtheorem{note}{Note}[subsection]
\newtheorem{notation}{Notation}
\newtheorem{lemma}{Lemma}[subsection]
\newtheorem{proposition}{Proposition}[subsection]
\newtheorem{example}{Example}[subsection]
\newtheorem{problem}{Problem}[section]
\newtheorem{homework}{Homework}[section]
\newtheorem{summary}{Summary}[subsection]
\newtheorem{corollary}{Corollary}[subsection]
\newtheorem{rmk}{Remark}[section]
\usepackage{romannum}
%%%%%%%%%%%%%%%%%%%%%%%%%%%%%%%%%%%%%%%%%%%%%%%%%%%%%%%%%%%%%%%%%%%%%%%%%%%%%%%%%%%%%%%%%%%%%%%%%%%%%%%%%%%%%%%%%%%%%
\newlength{\tabcont}
\setlength{\parindent}{0.0in}
\setlength{\parskip}{0.05in}
\colorlet{shadecolor}{orange!15}
\parindent 0in
\parskip 12pt
\geometry{margin=1in, headsep=0.25in}
%%%%%%%%%%%%%%%%%%%%%%%%%%%%%%%%%%%%%%%%%%%%%%%%%%%%%%%%%%%%%%%%%%%%%%%%%%%%%%%%%%%%%%%%%%%%%%%%%%%%%%%%%%%%%%%%%%%%%
\graphicspath{ {img/EoM/}}
%%%%%%%%%%%%%%%%%%%%%%%%%%%%%%%%%%%%%%%%%%%%%%%%%%%%%%%%%%%%%%%%%%%%%%%%%%%%%%%%%%%%%%%%%%%%%%%%%%%%%%%%%%%%%%%%%%%%%
%\renewcommand{\cite}[1]{[#1]}
\makeatletter
\@addtoreset{equation}{section}
\makeatother
\renewcommand{\theequation}{\arabic{section}.\arabic{equation}}
\renewcommand{\contentsname}{\centering \small \color{blue} Contents}
%\counterwithin{figure}{section}
\renewcommand{\figurename}{\textbf{Fig.}}
%\renewcommand{\refname}{\textbf{\kaishu 参考文献}}
\renewcommand{\refname}{\textbf{Bibliography}}
\setcounter{secnumdepth}{4}
\titleformat{\paragraph}
{\normalfont\normalsize\bfseries}{\theparagraph}{1em}{}
\titlespacing*{\paragraph}{0pt}{3.25ex plus 1ex minus .2ex}{1.5ex plus .2ex}
\def\beginrefs{\begin{list}%
		{[\arabic{equation}]}{\usecounter{equation}
			\setlength{\leftmargin}{0.8truecm}\setlength{\labelsep}{0.4truecm}%
			\setlength{\labelwidth}{1.6truecm}}}
	\def\endrefs{\end{list}}
\def\bibentry#1{\item[\hbox{[#1]}]}
%%%%%%%%%%%%%%%%%%%%%%%%%%%%%%%%%%%%%%%%%%%%%%%%%%%%%%%%%%%%%%%%%%%%%%%%%%%%%%%%%%%%%%%%%%%%%%%%%%%%%%%%%%%%%%%%%%%%%%
%\begin{figure}[!htb]
%	\centering
%	\subfloat[$A \cap B$]{%
%		\includegraphics[width=0.3\linewidth,height=0.2\linewidth]{img001.jpg}}
%	\label{img001}\qquad \qquad %\hfill
%	\subfloat[${A_1} \cap {A_2} \cap {A_3}$]{%
%		\includegraphics[width=0.3\linewidth,height=0.2\linewidth]{img002.jpg}}
%	\label{img002}
	%\caption{ Examples.}
%\end{figure}
%\begin{figure}[!htb]
%	\centering
%	\includegraphics[width=0.4\linewidth,height=0.3\linewidth]{img005.jpg}
%	\label{img005}
	%\caption{ illustration for $ 3 $}
%\end{figure}
%\={a}1 \'{a}2\v{a}3\.{a}4

\usepackage{datetime}
\renewcommand{\today}{\shortmonthname[\the\month] \the \day,  \the\year}
%%%%%%%%%%%%%%%%%%%%%%%%%%%%%%%%%%%%%%%%%%%%%%%%%%%%%%%%%%%%%%%%%%%%%%%%%%%%%%%%%%%%%%%%%%%%%%%%%%%%%%%%%%%%%%%%%%%%%%
\begin{document}
	\kaishu 
	%\thispagestyle{empty}
	\pagenumbering{arabic} 
	\setcounter{section}{0}
	\begin{center}
		{\LARGE  \href{https://www.bilibili.com/video/BV1Q6uuzJEAU/?spm_id_from=333.1387.homepage.video_card.click}{Riemannian Geometry}}
		
		\vspace{-0.25cm}
		
		{\large \href{https://space.bilibili.com/2049367688?spm_id_from=333.788.upinfo.detail.click}{Yu Xuan Zhou}}
	\end{center}
%%%%%%%%%%%%%%%%%%%%%%%%%%%%%%%%%%%%%%%%%%%%%%%%%%%%%%%%%%%%%%%%%%%%%%%%%%%%%%%%%%%%%%%%%%%%%%%%%%%%%%%%%%%%%%%%%%%%%%
%%\newpage 
%%\thispagestyle{empty}	
%%%%%%%%%%%%%%%%%%%%%%%%%%%%%%%%%%%%%%%%%%%%%%%%%%%%%%%%%%%%%%%%%%%%%%%%%%%%%%%%%%%%%%%%%%%%%%%%%%%%%%%%%%%%%%%%%%%%%%
%\tableofcontents	
%{\pagestyle{empty}\mbox{}\newpage\pagestyle{empty}}
%\newpage 
%{\pagestyle{empty}\mbox{}\newpage\pagestyle{empty}}
%%%%%%%%%%%%%%%%%%%%%%%%%%%%%%%%%%%%%%%%%%%%%%%%%%%%%%%%%%%%%%%%%%%%%%%%%%%%%%%%%%%%%%%%%%%%%%%%%%%%%%%%%%%%%%%%%%%%%%
%%\newpage 
\setcounter{page}{1}

%\vspace{1.5cm}


\begin{multicols}{3}
	\begin{enumerate}
		\item \href{https://mp.weixin.qq.com/s/5HeCArjxJe5baSnMOLDpIQ}{微分流形}	%1
		\item \href{https://mp.weixin.qq.com/s/1Cw2SDvNsz_OiYQ1Zrhq5Q}{光滑映射}	%2
		\item \href{https://mp.weixin.qq.com/s/RBd5140kNGevCDqM_aFdlA}{单位分解定理}	%3
		\item \href{https://mp.weixin.qq.com/s/8pek1FW0SGoqpZ0OvHtgrA}{切向量和切空间}	%4
		\item \href{https://mp.weixin.qq.com/s/RsNPI9TeLuvZ1tE9V4Jz-Q}{光滑向量场}	%5
		\item \href{https://mp.weixin.qq.com/s/Mf1UmG2yTctHO_5omjXMdw}{光滑张量场}	%6
		\item \href{https://mp.weixin.qq.com/s/AWfCLy5AspknoYfHwffOHA}{外微分式}	%7
		\item \href{https://mp.weixin.qq.com/s/shAEaltWTigfuK2UgOBfxQ}{外微分式的积分和Stokes定理}	%8
		\item \href{https://mp.weixin.qq.com/s/7hCaCkXp1_lv-vK_qaGlag}{切丛和向量丛}	%9
		\item \href{https://mp.weixin.qq.com/s/uyNVEKJfQq2Q-pVj7rrUFA}{黎曼度量}	%10
		\item \href{https://mp.weixin.qq.com/s/UKbSl6M1SBJnlQgOsPd4Gg}{黎曼流形的例子}	%11
		\item \href{https://mp.weixin.qq.com/s/-wWlt-VC_djBijyhE_UhMg}{切向量场的协变微分}	%12
		\item \href{https://mp.weixin.qq.com/s/ZA__7ZYVeJkg5FZ8mEobhQ}{联络和黎曼联络}	%13
		\item \href{https://mp.weixin.qq.com/s/beL1Ul7sXInzT9bzWE4xgA}{黎曼流形上的微分算子}	%14
		\item \href{https://mp.weixin.qq.com/s/iRgVB70TqEAOaPQY5N8zDw}{联络形式}	%15
		\item \href{https://mp.weixin.qq.com/s/JC7VMVj42WApRD0PsNkX2g}{平行移动}	%16
		\item \href{https://mp.weixin.qq.com/s/pqyu103inbNcrWURLCKLcA}{向量丛上的联络}	%17
		\item \href{https://mp.weixin.qq.com/s/O-9PSfjxJJB8wScvHzm6CQ}{测地线的概念}	%18
		\item \href{https://mp.weixin.qq.com/s/r_qXUEQhuf27U6nuXZVKpQ}{指数映射}	%19
		\item \href{https://mp.weixin.qq.com/s/8-sWMwcOHhqAqLb28WSHNg}{弧长的第一变分公式}	%20
		\item \href{https://mp.weixin.qq.com/s/NpRSyr7s6laHIuvnWSDy7A}{Gauss引理和法坐标系}	%21
		\item \href{https://mp.weixin.qq.com/s/QWloLbn7T69YxJU45hiuGQ}{测地凸邻域}	%22
		\item \href{https://mp.weixin.qq.com/s/tEeB2WBDMjOra2AbDeYT1w}{Hopf-Rinow定理}	%23
		\item \href{https://mp.weixin.qq.com/s/c7gX62AdOvyvW4ic-ODgvw}{曲率张量}	%24
		\item \href{https://mp.weixin.qq.com/s/BOVqEWHii-4JB41x-HTyCg}{曲率形式}	%25
		\item \href{https://mp.weixin.qq.com/s/Ls0ZhbLUAgiYJHx6A4TlBg}{截面曲率}	%26
		\item \href{https://mp.weixin.qq.com/s/xAVRUJ3yt8RV2YA6v4Wdqg}{Ricci曲率和数量曲率}	%27
		\item \href{https://mp.weixin.qq.com/s/j9sWIxbiJuXLUwgJAAk-Lw}{Ricci恒等式}	%28
		\item \href{https://mp.weixin.qq.com/s/EAsE58Zm95gAMJgkjBqsUw}{Jacobi场}	%29
		\item \href{https://mp.weixin.qq.com/s/X0DgxYeecyGBH_81aJ9Bag}{共轭点}	%30
		\item \href{https://mp.weixin.qq.com/s/gZ92KQgpCQos1je10pgk2g}{Cartan-Hadamard定理}	%31
		\item \href{https://mp.weixin.qq.com/s/13gl4CV7bH_OOUunnPR8tQ}{Cartan等距定理}	%32
		\item \href{https://mp.weixin.qq.com/s/i0oE4Jp4cSXP99QMBRzUIQ}{空间形式}	%33
		\item \href{https://mp.weixin.qq.com/s/jVWUQPehLKnGgI-I_Uw0YQ}{弧长的第二变分公式}	%34
		\item \href{https://mp.weixin.qq.com/s/bP6APZGDZhJZfvd7xxj_5Q}{Bonnet-Myers定理}	%35
		\item \href{https://mp.weixin.qq.com/s/ck6MUYcHATcx9CCXLY2_6w}{Synge定理}	%36
		\item \href{https://mp.weixin.qq.com/s/zM49jDOw-Ga2np-jNU68bw}{基本指标引理}	%37
		\item \href{https://mp.weixin.qq.com/s/rebKuD7MxtFdSPw3yfSMCA}{黎曼几何中的比较定理 1}	%38
		\item \href{https://mp.weixin.qq.com/s/96eKkEoj82zvnI8SVkZ8Dg}{黎曼几何中的比较定理 2}	%39
		\item \href{https://mp.weixin.qq.com/s/Hn4WzWrwfsTLtjFPR1b4AA}{子流形的基本公式}	%40
		\item \href{https://mp.weixin.qq.com/s/gtktCS-NlUR_8z9O_1vOig}{子流形的基本方程}	%41
		\item \href{https://mp.weixin.qq.com/s/-SlzkxKw7VlWJ61znoeXWA}{欧氏空间中的子流形}	%42
		\item \href{https://mp.weixin.qq.com/s/Hq0lOINQAnrXVHggXY8IEg}{极小子流形}	%43
		\item \href{https://mp.weixin.qq.com/s/0pjRjVILW3y4yEF2z1Y2iw}{体积的第二变分公式}	%44
		\item \href{https://mp.weixin.qq.com/s/BGz9e1L6U_tMFBfsDwdwDQ}{复向量空间}	%45
		\item \href{https://mp.weixin.qq.com/s/ZIpgC7iIWe4_6VKU4xt9Hw}{复流形和近复流形 1}	%46
		\item \href{https://mp.weixin.qq.com/s/v_ibVvAE8nS26AgN-7GEig}{复流形和近复流形 2}	%47
		\item \href{https://mp.weixin.qq.com/s/oaJkNQbgbH8tOjoauLR2zQ}{复流形和近复流形 3}	%48
		\item \href{https://mp.weixin.qq.com/s/W35l5QD3ZhT-DSJyNL0Dlw}{复向量丛上的联络 1}	%49
		\item \href{https://mp.weixin.qq.com/s/C7bsR1wIzHMMZbnvJgoREg}{复向量丛上的联络 2}	%50
		\item \href{https://mp.weixin.qq.com/s/f5uAKhq2K1lZf7avjKPOzA}{K{\"a}hler流形的几何}	%51
		\item \href{https://mp.weixin.qq.com/s/rdU07tBTTakN_RiHHpjq8Q}{全纯截面曲率}	%52
		\item \href{https://mp.weixin.qq.com/s/CQgUQqnCqGC7oB4G_RevnQ}{K{\"a}hler流形的例子 1}	%53
		\item \href{https://mp.weixin.qq.com/s/lrLQCykwfXYnG7u8g8ajkQ}{K{\"a}hler流形的例子 2}	%54
		\item \href{https://mp.weixin.qq.com/s/rAwaD-IU-KFehl8IUKpgWw}{陈示性类}	%55
		\item \href{https://mp.weixin.qq.com/s/4MQno-8YB9tenxKP0FvEgQ}{黎曼对称空间}	%56
		\item \href{https://mp.weixin.qq.com/s/3SdOkMuyQfvS_DO-URuqmw}{黎曼对称空间的性质}	%57
		\item \href{https://mp.weixin.qq.com/s/WKHsLsmp22V7gm6HB7c_3A}{黎曼对称对 1}	%58
		\item \href{https://mp.weixin.qq.com/s/GKUJXpLMN-m2ixhzxmmU6Q}{黎曼对称对 2}	%59
		\item \href{https://mp.weixin.qq.com/s/1zRR8RD8n3vpxPe7BJzPOw}{黎曼对称空间的例子 1}	%60
		\item \href{https://mp.weixin.qq.com/s/hzj-Y3Ds63ZhafXAcOdEMg}{黎曼对称空间的例子 2}	%61
		\item \href{https://mp.weixin.qq.com/s/oGCGXSI5oYWM5jH8bdUKoQ}{黎曼对称空间的例子 3}	%62
		\item \href{https://mp.weixin.qq.com/s/YZimzwB69zBSC5hMp0sBfg}{黎曼对称空间的例子 4}	%63
		\item \href{https://mp.weixin.qq.com/s/44S9p7su57X0DTlBG1d9SA}{黎曼对称空间的例子 5}	%64
		\item \href{https://mp.weixin.qq.com/s/uy0Xm1gWpOQpWSc-iJvxmQ}{向量丛上的联络和水平分布}	%65
		\item \href{https://mp.weixin.qq.com/s/QGRmm9bUen0p9-oR-O60nA}{标架丛和联络 1}	%66
		\item \href{https://mp.weixin.qq.com/s/4Lx2aorCmkqbzpRZbReivA}{标架丛和联络 2}	%67
		\item \href{https://mp.weixin.qq.com/s/BEFuPPiafaiV7oeGaSPwqA}{微分纤维丛 1}	%68
		\item \href{https://mp.weixin.qq.com/s/nijtgDkMCztFq1LaVY36wg}{微分纤维丛 2}	%69
		\item \href{https://mp.weixin.qq.com/s/aBLIF2JzNHQreKqeiySLag}{微分纤维丛 3}	%70
		\item \href{https://mp.weixin.qq.com/s/7RcwjnbzsbFNA1Z8sPAbjw}{主纤维丛上的联络 1}	%71
		\item \href{https://mp.weixin.qq.com/s/eyxUR5Q437VZhetZI9YAhg}{主纤维丛上的联络 2}	%72
		\item \href{https://mp.weixin.qq.com/s/WAEUGdnMs7OwlVx3pRKWUg}{主纤维丛上的联络 3}	%73
		\item \href{https://mp.weixin.qq.com/s/Ill0B0O1qHUkWtDgvIDM_g}{主丛上联络的曲率 1}	%74
		\item \href{https://mp.weixin.qq.com/s/jdysGnQ-gFhSNVQ9f83HAw}{主丛上联络的曲率 2}	%75
		\item \href{https://mp.weixin.qq.com/s/pQhfG8KOUKxls_OmNk-i9Q}{主丛上联络的曲率 3}	%76
		\item \href{https://mp.weixin.qq.com/s/aL_409sQ2ihgu26xpRehlQ}{Yang-Mills 场简介 1}	%77
		\item \href{https://mp.weixin.qq.com/s/aoxF3emuBn4VM_o3iE4nWw}{Yang-Mills 场简介 2}	%78
		\item \href{https://mp.weixin.qq.com/s/J-FPSwJ5H9BK9IFceB-3pw}{Yang-Mills 场简介 3}	%79
		%\item \href{url}{Materials}
	\end{enumerate}
\end{multicols}





%%%%%%%%%%%%%%%%%%%%%%%%%%%%%%%%%%%%%%%%%%%%%%%%%%%%%%%%%%%%%%%%%%%%%%%%%%%%%%%%%%%%%%%%%%%%%%%%%%%%%%%%%%%%%%%%%%%%%%%
%\bibliographystyle{ieeetr} % number
%%\bibliographystyle{unsrtnat} % author year
%\bibliography{HeBib}
%%%%%%%%%%%%%%%%%%%%%%%%%%%%%%%%%%%%%%%%%%%%%%%%%%%%%%%%%%%%%%%%%%%%%%%%%%%%%%%%%%%%%%%%%%%%%%%%%%%%%%%%%%%%%%%%%%%%%%%
\begin{flushright}
	\tiny \today 
\end{flushright}
%%%%%%%%%%%%%%%%%%%%%%%%%%%%%%%%%%%%%%%%%%%%%%%%%%%%%%%%%%%%%%%%%%%%%%%%%%%%%%%%%%%%%%%%%%%%%%%%%%%%%%%%%%%%%%%%%%%%%%%
\end{document}
%%%%%%%%%%%%%%%%%%%%%%%%%%%%%%%%%%%%%%%%%%%%%%%%%%%%%%%%%%%%%%%%%%%%%%%%%%%%%%%%%%%%%%%%%%%%%%%%%%%%%%%%%%%%%%%%%%%%%%%
              