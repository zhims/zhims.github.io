%%%%%%%%%%%%%%%%%%%%%%%%%%%%%%%%%%%%%%%%%%%%%%%%%%%%%%%%%%%%%%%%%%%%%%%%%%%%%%%%%%%%%%%%%%%%%%%%%%%%%%%%%%%%%%%%%%%%%
%%%%%%%%%%%%%%%%%%%%%%%%%%%%%%%%%%%%%%%%%%%%%   Author:Yao Zhang  %%%%%%%%%%%%%%%%%%%%%%%%%%%%%%%%%%%%%%%%%%%%%%%%%%%
%%%%%%%%%%%%%%%%%%%%%%%%%%%%%%%%%%%%%%%%%%%%% Email: jaafar_zhang@163.com %%%%%%%%%%%%%%%%%%%%%%%%%%%%%%%%%%%%%%%%%%%
%%%%%%%%%%%%%%%%%%%%%%%%%%%%%%%%%%%%%%%%%%%%%%%%%%%%%%%%%%%%%%%%%%%%%%%%%%%%%%%%%%%%%%%%%%%%%%%%%%%%%%%%%%%%%%%%%%%%%
\documentclass[11pt]{article}
\usepackage{babel}
\usepackage[utf8]{inputenc} 
\usepackage[table]{xcolor}
\usepackage[most]{tcolorbox}
\usepackage[left=2.50cm, right=1.50cm, top=2.0cm, bottom=2.50cm]{geometry}
\usepackage{xcolor,url}
\usepackage{amsmath,amsthm,amsfonts,amssymb,amscd,multirow,booktabs,fullpage,calc,multicol}
\usepackage{lastpage,enumitem,fancyhdr,mathrsfs,wrapfig,setspace,cancel,amsmath,empheq,framed}
\usepackage[retainorgcmds]{IEEEtrantools}
\usepackage{subfig,graphicx,framed}
\usepackage{ctex}
\usepackage{txfonts}
\usepackage{bbm}
\usepackage{chngcntr}
\usepackage[colorlinks,linkcolor=blue,anchorcolor=green,citecolor=red,urlcolor=blue]{hyperref}
\usepackage{titlesec}
%%%%%%%%%%%%%%%%%%%%%%%%%%%%%%%%%%%%%%%%%%%%%%%%%%%%%%%%%%%%%%%%%%%%%%%%%%%%%%%%%%%%%%%%%%%%%%%%%%%%%%%%%%%%%%%%%%%%%%
\newtheorem{thm}{Theorem}[section]
\newtheorem{defi}{Definition}[subsection]
\newtheorem{exercise}{Exercise}[subsection]
\newtheorem{note}{Note}[subsection]
\newtheorem{notation}{Notation}
\newtheorem{lemma}{Lemma}[subsection]
\newtheorem{proposition}{Proposition}[subsection]
\newtheorem{example}{Example}[subsection]
\newtheorem{problem}{Problem}[section]
\newtheorem{homework}{Homework}[section]
\newtheorem{summary}{Summary}[subsection]
\newtheorem{corollary}{Corollary}[subsection]
\newtheorem{rmk}{Remark}[section]
\usepackage{romannum}
%%%%%%%%%%%%%%%%%%%%%%%%%%%%%%%%%%%%%%%%%%%%%%%%%%%%%%%%%%%%%%%%%%%%%%%%%%%%%%%%%%%%%%%%%%%%%%%%%%%%%%%%%%%%%%%%%%%%%
\newlength{\tabcont}
\setlength{\parindent}{0.0in}
\setlength{\parskip}{0.05in}
\colorlet{shadecolor}{orange!15}
\parindent 0in
\parskip 12pt
\geometry{margin=1in, headsep=0.25in}
%%%%%%%%%%%%%%%%%%%%%%%%%%%%%%%%%%%%%%%%%%%%%%%%%%%%%%%%%%%%%%%%%%%%%%%%%%%%%%%%%%%%%%%%%%%%%%%%%%%%%%%%%%%%%%%%%%%%%
\graphicspath{ {img/EoM/}}
%%%%%%%%%%%%%%%%%%%%%%%%%%%%%%%%%%%%%%%%%%%%%%%%%%%%%%%%%%%%%%%%%%%%%%%%%%%%%%%%%%%%%%%%%%%%%%%%%%%%%%%%%%%%%%%%%%%%%
%\renewcommand{\cite}[1]{[#1]}
\makeatletter
\@addtoreset{equation}{section}
\makeatother
\renewcommand{\theequation}{\arabic{section}.\arabic{equation}}
\renewcommand{\contentsname}{\centering \small \color{blue} Contents}
%\counterwithin{figure}{section}
\renewcommand{\figurename}{\textbf{Fig.}}
%\renewcommand{\refname}{\textbf{\kaishu 参考文献}}
\renewcommand{\refname}{\textbf{Bibliography}}
\setcounter{secnumdepth}{4}
\titleformat{\paragraph}
{\normalfont\normalsize\bfseries}{\theparagraph}{1em}{}
\titlespacing*{\paragraph}{0pt}{3.25ex plus 1ex minus .2ex}{1.5ex plus .2ex}
\def\beginrefs{\begin{list}%
		{[\arabic{equation}]}{\usecounter{equation}
			\setlength{\leftmargin}{0.8truecm}\setlength{\labelsep}{0.4truecm}%
			\setlength{\labelwidth}{1.6truecm}}}
	\def\endrefs{\end{list}}
\def\bibentry#1{\item[\hbox{[#1]}]}
%%%%%%%%%%%%%%%%%%%%%%%%%%%%%%%%%%%%%%%%%%%%%%%%%%%%%%%%%%%%%%%%%%%%%%%%%%%%%%%%%%%%%%%%%%%%%%%%%%%%%%%%%%%%%%%%%%%%%%
%\begin{figure}[!htb]
%	\centering
%	\subfloat[$A \cap B$]{%
%		\includegraphics[width=0.3\linewidth,height=0.2\linewidth]{img001.jpg}}
%	\label{img001}\qquad \qquad %\hfill
%	\subfloat[${A_1} \cap {A_2} \cap {A_3}$]{%
%		\includegraphics[width=0.3\linewidth,height=0.2\linewidth]{img002.jpg}}
%	\label{img002}
	%\caption{ Examples.}
%\end{figure}
%\begin{figure}[!htb]
%	\centering
%	\includegraphics[width=0.4\linewidth,height=0.3\linewidth]{img005.jpg}
%	\label{img005}
	%\caption{ illustration for $ 3 $}
%\end{figure}
%\={a}1 \'{a}2\v{a}3\.{a}4

\usepackage{datetime}
\renewcommand{\today}{\shortmonthname[\the\month] \the \day,  \the\year}
%%%%%%%%%%%%%%%%%%%%%%%%%%%%%%%%%%%%%%%%%%%%%%%%%%%%%%%%%%%%%%%%%%%%%%%%%%%%%%%%%%%%%%%%%%%%%%%%%%%%%%%%%%%%%%%%%%%%%%
\begin{document}
	\kaishu 
	%\thispagestyle{empty}
	\pagenumbering{arabic} 
	\setcounter{section}{0}
	\begin{center}
		{\LARGE  \href{https://tysunseven.github.io/video/Foundations%20of%20Geometry%202022F.html}{Basic Geometry}}
		
		%\vspace{-0.25cm}
		
		{\large \href{http://staff.ustc.edu.cn/~wangzuoq/}{Zuoqin Wang}}
	\end{center}
%%%%%%%%%%%%%%%%%%%%%%%%%%%%%%%%%%%%%%%%%%%%%%%%%%%%%%%%%%%%%%%%%%%%%%%%%%%%%%%%%%%%%%%%%%%%%%%%%%%%%%%%%%%%%%%%%%%%%%
%%\newpage 
%%\thispagestyle{empty}	
%%%%%%%%%%%%%%%%%%%%%%%%%%%%%%%%%%%%%%%%%%%%%%%%%%%%%%%%%%%%%%%%%%%%%%%%%%%%%%%%%%%%%%%%%%%%%%%%%%%%%%%%%%%%%%%%%%%%%%
%\tableofcontents	
%{\pagestyle{empty}\mbox{}\newpage\pagestyle{empty}}
%\newpage 
%{\pagestyle{empty}\mbox{}\newpage\pagestyle{empty}}
%%%%%%%%%%%%%%%%%%%%%%%%%%%%%%%%%%%%%%%%%%%%%%%%%%%%%%%%%%%%%%%%%%%%%%%%%%%%%%%%%%%%%%%%%%%%%%%%%%%%%%%%%%%%%%%%%%%%%%
%%\newpage 
\setcounter{page}{1}

\vspace{0.25cm}


\begin{multicols}{2}
	\begin{enumerate}
		\item \href{https://mp.weixin.qq.com/s/eMWzqGr8L5K09U7eT9H8wg}{课程概况, 几何简史}	%1
		\item \href{https://mp.weixin.qq.com/s/LpbpnJFbX-0bQUCQDb6iMA}{几何简史续, 欧几里得公理体系}	%2
		\item \href{https://mp.weixin.qq.com/s/_tx57JHp7GVTrKZHpLkgqg}{希尔伯特公理体系}	%3
		\item \href{https://mp.weixin.qq.com/s/fxpSPNuB0jrVLlVMGcYYxw}{反射与平移}	%4
		\item \href{https://mp.weixin.qq.com/s/Vy4qh9BduTY0X6b2ZqZZ6g}{向量空间}	%5
		\item \href{https://mp.weixin.qq.com/s/Td3_DvXz91UlVmv_bAw8Vw}{欧氏空间}	%6
		\item \href{https://mp.weixin.qq.com/s/G28TEFlq845UVeyQHcpM1Q}{基, 定向}	%7
		\item \href{https://mp.weixin.qq.com/s/I6P7lMgbSoJief9u8Orhvg}{外积}	%8
		\item \href{https://mp.weixin.qq.com/s/3KMq71qte0Gxhi1_p5Hdng}{应用: 球面几何初探}	%9
		\item \href{https://mp.weixin.qq.com/s/Vt9vlWZfYTH1eeMCjBpoLg}{希尔伯特公理体系的模型}	%10
		\item \href{https://mp.weixin.qq.com/s/6w9-m4IL0hlkMSlQBztZDA}{平面与直线的方程}	%11
		\item \href{https://mp.weixin.qq.com/s/7kCTabCGba_Ss64MRQpEjA}{曲线与曲面}	%12
		\item \href{https://mp.weixin.qq.com/s/Z9LOUQegV2CUYkuD15IE-Q}{几何变换}	%13
		\item \href{https://mp.weixin.qq.com/s/JJCOBGIKbEYW9dntj5IUhw}{刚体变换}	%14
		\item \href{https://mp.weixin.qq.com/s/w6U0lwQDKqFccF9vF3B5ow}{等价类, 二次曲线分类}	%15
		\item \href{https://mp.weixin.qq.com/s/-1zWLh4Ik6fq94gcJv6Z_A}{二次曲面分类}	%16
		\item \href{https://mp.weixin.qq.com/s/-1zWLh4Ik6fq94gcJv6Z_A}{埃尔朗根纲领}	%17
		\item \href{https://mp.weixin.qq.com/s/AUV6LxdNBgEvMOzmyMxiBg}{平面不同几何之比较}	%18
		\item \href{https://mp.weixin.qq.com/s/A8eZB2NUvMhsJ4yXA9dFoQ}{射影平面}	%19
		\item \href{https://mp.weixin.qq.com/s/yXKBvb2sq5r56jdqbbivlw}{射影平面里的直线}	%20
		\item \href{https://mp.weixin.qq.com/s/x0MySIH7VtX65zYAyBr1fQ}{射影变换}	%21
		\item \href{https://mp.weixin.qq.com/s/Sq2oYGbCSrVqMz4Vlh8JkQ}{交比}	%22
		\item \href{https://mp.weixin.qq.com/s/FblyQMWxwPjBNwcjB2T6RA}{射影平面上射影变换的刻画}	%23
		\item \href{https://mp.weixin.qq.com/s/4t2fWh9d40ybYH9mUiLKag}{射影变换的应用}	%24
		\item \href{https://mp.weixin.qq.com/s/3X8B75TCqpWGQ-ezsryh9Q}{拓扑的直观概念}	%25
		\item \href{https://mp.weixin.qq.com/s/y9Wp7sjf34ZauXdrNtX5ww}{多面体的欧拉公式}	%26
		\item \href{https://mp.weixin.qq.com/s/956skwM3my-8HPkyMcHjOw}{拓扑曲面}	%27
		%\item \href{url}{Materials}
	\end{enumerate}
\end{multicols}


% [./doc/note/Mathematics/Geometry/BasicGeometry/notes.pdf Notes]

\vspace{0.5cm}

%%%%%%%%%%%%%%%%%%%%%%%%%%%%%%%%%%%%%%%%%%%%%%%%%%%%%%%%%%%%%%%%%%%%%%%%%%%%%%%%%%%%%%%%%%%%%%%%%%%%%%%%%%%%%%%%%%%%%%%
%\bibliographystyle{ieeetr} % number
%%\bibliographystyle{unsrtnat} % author year
%\bibliography{HeBib}
%%%%%%%%%%%%%%%%%%%%%%%%%%%%%%%%%%%%%%%%%%%%%%%%%%%%%%%%%%%%%%%%%%%%%%%%%%%%%%%%%%%%%%%%%%%%%%%%%%%%%%%%%%%%%%%%%%%%%%%
\begin{flushright}
	\tiny \today 
\end{flushright}
%%%%%%%%%%%%%%%%%%%%%%%%%%%%%%%%%%%%%%%%%%%%%%%%%%%%%%%%%%%%%%%%%%%%%%%%%%%%%%%%%%%%%%%%%%%%%%%%%%%%%%%%%%%%%%%%%%%%%%%
\end{document}
%%%%%%%%%%%%%%%%%%%%%%%%%%%%%%%%%%%%%%%%%%%%%%%%%%%%%%%%%%%%%%%%%%%%%%%%%%%%%%%%%%%%%%%%%%%%%%%%%%%%%%%%%%%%%%%%%%%%%%%
              